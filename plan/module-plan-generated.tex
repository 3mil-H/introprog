W01 & Introduktion & sekvens, alternativ, repetition, abstraktion, editera, kompilera, exekvera, datorns delar, virtuell maskin, litteral, värde, uttryck, identifierare, variabel, typ, tilldelning, namn, val, var, def, definera och anropa funktion, funktionshuvud, funktionskropp, procedur, inbyggda grundtyper, Int, Long, Short, Double, Float, Byte, Char, String, println, typen Unit, enhetsvärdet (), stränginterpolatorn s, if, else, true, false, MinValue, MaxValue, aritmetik, slumptal, math.random, logiska uttryck, de Morgans lagar, while-sats, for-sats \\
W02 & Program, kontrollstrukturer & kompilerad app, skript, main i Scala, scalac, utdata, println, indata, scala.io.StdIn.readLine, programargument, args i main, main i Java, javac, java.lang.System.out.println, iterera över element i samling, for-uttryck, yield, map, foreach, samling, sekvens, indexering, Array, Vector, intervall, Range, algoritm vs implementation, pseudokod, algoritm: SWAP, algoritm: SUM, algoritm: MIN/MAX, algoritm: MININDEX \\
W03 & Funktioner, abstraktion & abstraktionsmekanism, funktion, parameter, argument, returtyp, default-argument, namngivna argument, parameterlista, funktionshuvud, funktionskropp, applicera funktion på alla element i en samling, uppdelad parameterlista, skapa egen kontrollstruktur, funktionsvärde, funktionstyp, äkta funktion, stegad funktion, apply, anonyma funktioner, lambda, predikat, aktiveringspost, anropsstacken, objektheapen, stack trace, rekursion, scala.util.Random, slumptalsfrö \\
W04 & Objekt, inkapsling & modul, singelobjekt, paket, punktnotation, tillstånd, medlem, attribut, metod, paket, import, filstruktur, jar, dokumentation, programlayout, JDK, import, selektiv import, namnbyte vid import, tupel, multipla returvärden, block, lokal variabel, skuggning, lokal funktion, funktioner är objekt med apply-metod, namnrymd, synlighet, privat medlem, inkapsling, getter och setter, principen om enhetlig access, överlagring av metoder, introprog.PixelWindow, initialisering, lazy val, värdeandrop, namnanrop, typalias \\
W05 & Klasser, datamodeller & applikationsdomän, datamodell, objektorientering, klass, instans, Any, isInstanceOf, toString, new, null, this, accessregler, private, private[this], klassparameter, primär konstruktor, fabriksmetod, alternativ konstruktor, förändringsbar, oföränderlig, case-klass, kompanjonsobjekt, referenslikhet, innehållslikhet, eq, == \\
W06 & Mönster, felhantering & mönstermatchning, match, Option, throw, try, catch, Try, unapply, sealed, flatten, flatMap, partiella funktioner, collect, wildcard-mönster, variabelbindning i mönster, sekvens-wildcard, bokstavliga mönster, implementera equals, hashcode \\
W07 & Sekvenser, enumerationer & översikt av Scalas samlingsbibliotek och samlingsmetoder, klasshierarkin i scala.collection, Iterable, Seq, List, ListBuffer, ArrayBuffer, WrappedArray, sekvensalgoritm, algoritm: SEQ-COPY, in-place vs copy, algoritm: SEQ-REVERSE, registrering, algoritm: SEQ-REGISTER, linjärsökning, algoritm: LINEAR-SEARCH, tidskomplexitet, minneskomplexitet, sekvenser i Java vs Scala, for-sats i Java, java.util.Scanner, översikt strängmetoder, StringBuilder, ordning, inbyggda sökmetoder, find, indexOf, indexWhere, inbyggda sorteringsmetoder, sorted, sortWith, sortBy, repeterade parametrar \\
KS & \multicolumn{2}{l}{KONTROLLSKRIVN.} \\
W08 & Matriser, typparametrar & matris, nästlad samling, nästlad for-sats, typparameter, generisk funktion, generisk klass, fri vs bunden typparameter, generiska datastrukturer, generiska samlingar i Scala \\
W09 & Mängder, tabeller & innehållstest, mängd, Set, mutable.Set, nyckel-värde-tabell, Map, mutable.Map, hash code, java.util.HashMap, java.util.HashSet, persistens, serialisering, textfiler, Source.fromFile, java.nio.file, repetition inför kontrollskrivning \\
W10 & Arv, komposition & arv, polymorfism, trait, extends, asInstanceOf, with, inmixning, supertyp, subtyp, bastyp, override, Scalas typhierarki, Any, AnyRef, Object, AnyVal, Null, Nothing, topptyp, bottentyp, referenstyper, värdetyper, Shape som bastyp till Rectangle och Circle, accessregler vid arv, protected, final, case-object, typer med uppräknade värden, trait, abstrakt klass, inmixning, TODO: komposition (ä.k. aggregering) \\
W11 & Kontextparametrar, api-design & TODO, given, using, extension, ad hoc polymorfism, typklass, programmeringsgränssnitt (api), api, kodläsbarhet, idiom, kodningsstandard, granskningar, riktlinjer för projektredovisning \\
W12 & Fördjupning & TODO, TODO flytta ordningsgrejs till sekvensveckan!!! strängjämförelse, compareTo, implicit ordning, binärsökning, algoritm: BINARY-SEARCH, sortering till ny vektor, sortering på plats, insättningssortering, urvalssortering, algoritm: INSERTION-SORT, algoritm: SELECTION-SORT, Ordering[T], Ordered[T], Comparator[T], Comparable[T], tråd, jämlöpande exekvering, icke-blockerande anrop, callback, java.lang.Thread, java.util.concurrent.atomic.AtomicInteger, scala.concurrent.Future, kort om html+css+javascript+scala.js och webbprogrammering \\
W13 & Repetition & göra extenta, förbereda projektredovisning, skapa dokumentation med scaladoc och javadoc \\
W14 & \multicolumn{2}{l}{Muntlig examen} \\
T & \multicolumn{2}{l}{VALFRI TENTAMEN} \\