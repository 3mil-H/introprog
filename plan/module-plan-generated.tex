W01 & Introduktion & sekvens, alternativ, repetition, abstraktion, editera, kompilera, exekvera, datorns delar, virtuell maskin, litteral, värde, uttryck, identifierare, variabel, typ, tilldelning, namn, val, var, def, definera och anropa funktion, funktionshuvud, funktionskropp, procedur, inbyggda grundtyper, Int, Long, Short, Double, Float, Byte, Char, String, println, typen Unit, enhetsvärdet (), stränginterpolatorn s, if, else, true, false, MinValue, MaxValue, aritmetik, slumptal, math.random, logiska uttryck, de Morgans lagar, while-sats, for-sats \\
W02 & Program & kompilerad app, skript, main i Scala, scalac, utdata, println, indata, scala.io.StdIn.readLine, programargument, args i main, main i Java, javac, java.lang.System.out.println, iterera över element i samling, for-uttryck, yield, map, foreach, samling, sekvens, indexering, Array, Vector, intervall, Range, algoritm vs implementation, pseudokod, algoritm: SWAP, algoritm: SUM, algoritm: MIN/MAX, algoritm: MININDEX \\
W03 & Funktioner & parameter, argument, returtyp, default-argument, namngivna argument, parameterlista, funktionshuvud, funktionskropp, applicera funktion på alla element i en samling, uppdelad parameterlista, skapa egen kontrollstruktur, funktionsvärde, funktionstyp, äkta funktion, stegad funktion, apply, anonyma funktioner, lambda, aktiveringspost, anropsstacken, objektheapen, funktioner är objekt med apply-metod, rekursion, scala.util.Random, slumptalsfrö \\
W04 & Objekt & modul, singelobjekt, paket, punktnotation, tillstånd, medlem, attribut, metod, paket, import, filstruktur, jar, dokumentation, programlayout, JDK, import, selektiv import, namnbyte vid import, tupel, multipla returvärden, block, lokal variabel, skuggning, lokal funktion, namnrymd, synlighet, privat medlem, inkapsling, getter och setter, principen om uniform access, överlagring av metoder, cslib.window.SimpleWindow, initialisering, lazy val, värdeandrop, namnanrop, typalias \\
W05 & Klasser & objektorientering, klass, instans, Point, Square, Complex, Any, isInstanceOf, toString, new, null, this, accessregler, private, private[this], klassparameter, primär konstruktor, fabriksmetod, alternativ konstruktor, förändringsbar, oföränderlig, case-klass, kompanjonsobjekt, referenslikhet, innehållslikhet, eq, == \\
W06 & Sekvenser & översikt av Scalas samlingsbibliotek och samlingsmetoder, klasshierarkin i scala.collection, Traversable, Iterable, Seq, List, ListBuffer, ArrayBuffer, WrappedArray, sekvensalgoritm, algoritm: SEQ-COPY, in-place vs copy, algoritm: SEQ-REVERSE, registrering, algoritm: SEQ-REGISTER, linjärsökning, algoritm: LINEAR-SEARCH, tidskomplexitet, minneskomplexitet, sekvenser i Java vs Scala, for-sats i Java, java.util.Scanner, översikt strängmetoder, StringBuilder, ordning, inbyggda sökmetoder, find, indexOf, indexWhere, inbyggda sorteringsmetoder, sorted, sortWith, sortBy, variabelt argumentantal \\
W07 & Mängder, tabeller & innehållstest, mängd, Set, mutable.Set, nyckel-värde-tabell, Map, mutable.Map, hash code, java.util.HashMap, java.util.HashSet, persistens, serialisering, textfiler, Source.fromFile, java.nio.file, repetition inför kontrollskrivning \\
KS & \multicolumn{2}{l}{KONTROLLSKRIVN.} \\
W08 & Matriser, typparametrar & matris, nästlad samling, nästlad for-sats, typparameter, generisk funktion, generisk klass, fri vs bunden typparameter, generisk samling som typparameter, matriser i Java vs Scala, allokering av nästlade arrayer i Scala och Java \\
W09 & Arv & arv, polymorfism, trait, extends, asInstanceOf, with, inmixning, supertyp, subtyp, bastyp, override, Scalas typhierarki, Any, AnyRef, Object, AnyVal, Null, Nothing, topptyp, bottentyp, referenstyper, värdetyper, Shape som bastyp till Rectangle och Circle, accessregler vid arv, protected, final, case-object, typer med uppräknade värden, trait, abstrakt klass, inmixning, gränssnitt, interface i Java, programmeringsgränssnitt (api) \\
W10 & Mönstermatchning, undantag & mönstermatchning, match, Option, throw, try, catch, Try, unapply, sealed, flatten, flatMap, partiella funktioner, collect, wildcard-mönster, variabelbinding i mönster, sekvens-wildcard, bokstavliga mönster, equals, hashcode, exempel: equals för klassen Complex, switch-sats i Java \\
W11 & Språkskillnader & syntaxskillnader mellan Scala och Java, klasser i Scala och Java, referensvariabler i Java, enkla värden i Java, primitiva typer i Java, referenstilldelning och värdetilldelning i Java, alternativ konstruktor i Scala och Java, for-sats i Java, for-each-sats i Java, java.util.ArrayList, autoboxing i Java, wrapperklasser i Java, samlingar i Java, scala.collection.JavaConverters, namnkonventioner för konstanter i Scala och Java, kodläsbarhet, idiom, kodningsstandard \\
W12 & Sortering & strängjämförelse, compareTo, implicit ordning, binärsökning, algoritm: BINARY-SEARCH, sortering till ny vektor, sortering på plats, insättningssortering, urvalssortering, algoritm: INSERTION-SORT, algoritm: SELECTION-SORT, Ordering[T], Ordered[T], Comparator[T], Comparable[T], riktlinjer för projektredovisning \\
W13 & Repetition, tentaträning, projekt & göra extenta, förbereda projektredovisning, skapa dokumentation med scaladoc och javadoc \\
W14 & Extra & tråd, jämlöpande exekvering, icke-blockerande anrop, callback, java.lang.Thread, java.util.concurrent.atomic.AtomicInteger, scala.concurrent.Future, kort om html+css+javascript+scala.js och webbprogrammering \\
T & \multicolumn{2}{l}{TENTAMEN} \\