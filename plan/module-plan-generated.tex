W01 & Introduktion & sekvens, alternativ, repetition, abstraktion, editera, kompilera, exekvera, datorns delar, virtuell maskin, litteral, värde, uttryck, identifierare, variabel, typ, tilldelning, namn, val, var, def, definiera och anropa funktion, funktionshuvud, funktionskropp, procedur, inbyggda grundtyper, println, typen Unit, enhetsvärdet (), stränginterpolatorn s, aritmetik, slumptal, logiska uttryck, de Morgans lagar, if, true, false, while, for \\
W02 & Program, kontrollstrukturer & huvudprogram, program-argument, indata, scala.io.StdIn.readLine, kontrollstruktur, iterera över element i samling, for-uttryck, yield, map, foreach, samling, sekvens, indexering, Array, Vector, intervall, Range, algoritm, implementation, pseudokod, algoritmexempel: SWAP, SUM, MIN-MAX, MIN-INDEX \\
W03 & Funktioner, abstraktion & abstraktion, funktion, parameter, argument, returtyp, default-argument, namngivna argument, parameterlista, funktionshuvud, funktionskropp, applicera funktion på alla element i en samling, uppdelad parameterlista, skapa egen kontrollstruktur, funktionsvärde, funktionstyp, äkta funktion, stegad funktion, apply, anonyma funktioner, lambda, predikat, aktiveringspost, anropsstacken, objektheapen, stack trace, rekursion, scala.util.Random, slumptalsfrö \\
W04 & Objekt, inkapsling & modul, singelobjekt, punktnotation, tillstånd, medlem, attribut, metod, paket, filstruktur, jar, dokumentation, JDK, import, selektiv import, namnbyte vid import, tupel, multipla returvärden, block, lokal variabel, skuggning, lokal funktion, funktioner är objekt med apply-metod, namnrymd, synlighet, privat medlem, inkapsling, getter och setter, principen om enhetlig access, överlagring av metoder, introprog.PixelWindow, initialisering, lazy val, värdeandrop, namnanrop, typalias \\
W05 & Klasser, datamodellering & applikationsdomän, datamodell, objektorientering, klass, instans, Any, isInstanceOf, toString, new, null, this, accessregler, private, private[this], klassparameter, primär konstruktor, fabriksmetod, alternativ konstruktor, förändringsbar, oföränderlig, case-klass, kompanjonsobjekt, referenslikhet, innehållslikhet, eq, == \\
W06 & Mönster, felhantering & mönstermatchning, match, Option, throw, try, catch, Try, unapply, sealed, flatten, flatMap, partiella funktioner, collect, wildcard-mönster, variabelbindning i mönster, sekvens-wildcard, bokstavliga mönster, implementera equals, hashcode \\
W07 & Sekvenser, enumerationer & översikt av Scalas samlingsbibliotek och samlingsmetoder, klasshierarkin i scala.collection, Iterable, Seq, List, ListBuffer, ArrayBuffer, WrappedArray, sekvensalgoritm, algoritm: SEQ-COPY, in-place vs copy, algoritm: SEQ-REVERSE, registrering, algoritm: SEQ-REGISTER, linjärsökning, algoritm: LINEAR-SEARCH, tidskomplexitet, minneskomplexitet, sekvenser i Java vs Scala, for-sats i Java, java.util.Scanner, översikt strängmetoder, StringBuilder, ordning, inbyggda sökmetoder, find, indexOf, indexWhere, inbyggda sorteringsmetoder, sorted, sortWith, sortBy, repeterade parametrar \\
KS & \multicolumn{2}{l}{KONTROLLSKRIVN.} \\
W08 & Matriser, typparametrar & matris, nästlad samling, nästlad for-sats, typparameter, generisk funktion, generisk klass, fri och bunden typparameter, generiska datastrukturer, generiska samlingar i Scala \\
W09 & Mängder, tabeller & innehållstest, mängd, Set, mutable.Set, nyckel-värde-tabell, Map, mutable.Map, hash code, java.util.HashMap, java.util.HashSet, persistens, serialisering, textfiler, Source.fromFile, java.nio.file \\
W10 & Arv, komposition & arv, komposition, polymorfism, trait, extends, asInstanceOf, with, inmixning supertyp, subtyp, bastyp, override, Scalas typhierarki, Any, AnyRef, Object, AnyVal, Null, Nothing, topptyp, bottentyp, referenstyper, värdetyper, accessregler vid arv, protected, final, trait, abstrakt klass \\
W11 & Kontextparametrar, api & given, using, extension, ad hoc polymorfism, typklass, api, kodläsbarhet, granskningar \\
W12 & Valfri fördjupning, Projekt & välj valfritt fördjupningsområde, påbörja projekt \\
W13 & Repetition & träna på extentor, redovisa projekt \\
W14 & \multicolumn{2}{l}{Muntligt prov} \\
T & \multicolumn{2}{l}{VALFRI TENTAMEN} \\