\documentclass[a4paper,12pt,oneside]{memoir}
\usepackage[a4paper, total={16.2cm, 26.0cm}]{geometry}
\usepackage[utf8]{inputenc}
\usepackage{graphicx}
\usepackage[T1]{fontenc}
\usepackage[swedish]{babel}

\usepackage{microtype}
\usepackage{hyperref}
\hypersetup{hidelinks}
\usepackage{longtable}
\usepackage{booktabs}

% FONTS
\usepackage{tgtermes}

\usepackage{enumitem}
\setitemize{noitemsep,topsep=0pt,parsep=0pt,partopsep=0pt, leftmargin=*}

\pagenumbering{gobble}

\usepackage{url}
\usepackage{color}
%\newcommand{\TBD}{\colorbox{yellow}{\textbf{???}}}
%\newcommand{\TENTADATUM}{\colorbox{yellow}{8:e Januari, kl 08:00--13:00, se schema}}
%\newcommand{\KSDATUM}{\colorbox{yellow}{Tisdagen 24:e Oktober, kl 14:00--19:00, se schema}}
\newcommand{\YEAR}{2022}

\begin{document}

%% COMPATIBILITY PROBLEM In latex --version 2022 \bottomrule \addlinespace \midrule \toprule etc gives error ! Misplaced \noalign.
%% Here they are replaced by \hline and \\[1.2em]  etc
%% The commands from the booktab chapter are thus cancelled here; can the booktab package be removed?


\section*{EDAA45 Programmering, grundkurs  -- Kursprogram \YEAR}
\emph{Institutionen för Datavetenskap, LTH, Lunds Universitet.} Senast uppdaterad: \today\\

\begin{longtable}[l]{ll}
\hline\\[-0.75em]%\toprule

\textbf{EDAA45} & \textit {D1, C1, W3, 7,5 högskolepoäng, Läsperiod 1 \& 2} \\[-0.75em] \tabularnewline
\hline%\midrule
\endhead
\emph{Kursansvarig}   & Björn Regnell, rum E:2413,
                        \href{mailto:bjorn.regnell@cs.lth.se}
                        {\nolinkurl{bjorn.regnell@cs.lth.se}},
                        046--222 90 09\tabularnewline
\emph{Bitr. kursansv.}   & Marcus Klang, rum E:4133C,
                        \href{mailto:marcus.klang@cs.lth.se}
                        {\nolinkurl{marcus.klang@cs.lth.se}},
                        046--222 38 63\tabularnewline
                        \emph{Hemsida}        
                        & \url{http://cs.lth.se/pgk}\tabularnewline
\emph{Kurslitteratur} & Kompendium. Säljes på institutionens expedition efter förbeställning.\tabularnewline
\emph{Expedition}     & \url{http://cs.lth.se/kontakt/expedition/} Rum E:2179
                        \tabularnewline
                      %& Rum E:2179, expeditionstid Mån-Tor kl. 9.30--11.30, 12.45--13.30\tabularnewline

%\bottomrule
\hline
\end{longtable}

\subsection{Undervisning}\label{undervisning}

\begin{itemize}
\item
  \emph{Föreläsningar}. Föreläsningarna ger en översikt av
  kursinnehållet och åskådliggör teorin med praktiska
  programmeringsexempel. Föreläsningarna ger även utrymme för diskussion
  och frågor.
\item
  \emph{Resurstider}. I kursens schema finns särskilda resurstider
  där du kan få hjälp med övningar, laborationer och
  inlämningsuppgifter. Utnyttja dessa tillfällen!
\item
  \emph{Övningar}. I kursen ingår övningar som du arbetar med
  självständigt eller tillsammans med en kamrat.
  Du kan få hjälp med övningarna av handledare under resurstiderna.
  Övningarna är förberedelser inför laborationerna och den skriftliga tentamen.
  %Se anvisningar i kompendiet.
\item
  \emph{Laborationer}. I kursen ingår obligatoriska laborationer.
  Laborationerna redovisas för handledare.
  %Se anvisningar i kompendiet.
\item
  \emph{Projektuppgift}. Du ska självständigt arbeta med ett större
  program som redovisas för handledare. %Se anvisningar i kompendiet.
\end{itemize}
Se vidare anvisningar om olika undervisningsmoment i kompendiet och på kurshemsidan.

\subsection{Samarbetsgrupper}\label{samarbetsgrupper}

Kursdeltagarna indelas i \emph{samarbetsgrupper} baserat
på förkunskapsenkät, där studenter med olika förkunskapsnivåer
sammanförs. Målet är att deltagarna gemensamt ska
dela med sig av och träna på förklaringar av teori, begrepp och
programmeringspraktik. Kontrollskrivningen (se
nedan) kan ge samarbetsbonus  och en av laborationerna görs i grupp. Ni ska hjälpa varandra att
förstå, men \emph{inte} lösa uppgifterna åt varandra.

\subsection{Examination}\label{examination}

\begin{itemize}
\item
  \emph{Obligatoriska kursmoment:}

  \begin{itemize}
  \item
    \emph{Laborationer} (Delmoment 0221 i Ladok: 4.5 hp) bedöms av handledare på schemalagd tid.
  \item
    \emph{Kontrollskrivningen} är diagnostisk och visar ditt kunskapsläge efter
    halva kursen. Kontrollskrivningen görs individuellt och bedöms
    av studiekamrater vid skrivningstillfället.
    Kontrollskrivningen kan ge \emph{samarbetsbonus} som adderas till
    det skriftliga tentamensresultatet vid första ordinarie
    tentatillfälle med medelvärdet av gruppmedlemmarnas individuella
    kontrollskrivningspoäng. Det räcker att delta för att bli godkänd på kontrollskrivningen.%Datum: \KSDATUM
  \item
    \emph{Projektuppgift och teori} (Delmoment 0121 i Ladok: 3 hp) innefattar en större praktisk projektuppgift och ett muntligt teoriprov. Dessa redovisas för handledare på
    schemalagd tid. Alla laborationer ska vara godkända innan du får göra det muntliga provet.
  \end{itemize}

\item \emph{Betyg:} 
  Godkända obligatoriska moment krävs för betyg 3. Skriftlig tentamen är valfri och kan ge betyg 4 el. 5. Tillåtet hjälpmedel på tentamen:
  \href{http://cs.lth.se/pgk/quickref}{Snabbreferens}. 
  För att få tentera krävs att alla obligatoriska moment är godkända.\\
  % Ordinarie tentamen: \TENTADATUM
\end{itemize}

\clearpage

\subsection*{Veckoöversikt}

\resizebox{\columnwidth}{!}{%
{\fontsize{12pt}{20pt}\selectfont
%!TEX encoding = UTF-8 Unicode
\begin{tabular}{l|l|l|l|l|l|l}
\textit{W} & \textit{Datum} & \textit{Lp V} & \textit{Modul} & \textit{Förel} & \textit{Övn} & \textit{Lab} \\ \hline \hline
W01 & 29/8-2/9 & Lp1V1 & Introduktion & F01 F02 & expressions & kojo \\
W02 & 5/9-9/9 & Lp1V2 & Kodstrukturer & F03 F04 & programs & -- \\
W03 & 12/9-16/9 & Lp1V3 & Funktioner, objekt & F05 F06 & functions & blockmole \\
W04 & 19/9-23/9 & Lp1V4 & Datastrukturer & F07 F08 & data & pirates \\
W05 & 26/9-30/9 & Lp1V5 & Sekvensalgoritmer & F09 F10 & sequences & shuffle \\
W06 & 3/10-7/10 & Lp1V6 & Klasser & F11 F12 & classes & turtlegraphics \\
W07 & 10/10-14/10 & Lp1V7 & Arv & F13 F14 & traits & turtlerace-team \\
KS & 25/10 & TP1 & KONTROLLSKRIVN. & -- & -- & -- \\
W08 & 31/10-4/11 & Lp2V1 & Repetition, specialundervisning & F15 F16 & Repetera & Kom-i-kapp \\
W09 & 7/11-11/11 & Lp2V2 & Mönster, undantag & F17 F18 & matching & chords-team \\
W10 & 14/11-18/11 & Lp2V3 & Matriser, typparametrar & F19 F20 & matrices & maze \\
W11 & 21/11-25/11 & Lp2V4 & Sökning, sortering & F21 F22 & sorting & survey \\
W12 & 28/11-2/12 & Lp2V5 & Scala och Java & F23 F24 & scalajava & lthopoly-team \\
W13 & 5/12-9/12 & Lp2V6 & Extra: design, api, trådar, webb & F25 F26 & threads & Projekt \\
W14 & 12/12-16/12 & Lp2V7 & Tentaträning & F27 F28 & Extenta & -- \\
T & 9/1 & TP2 & TENTAMEN & -- & -- & -- \\
\end{tabular}

}
}

\vspace{1.1em}\noindent\hspace*{-2.0mm}%
\noindent\textit{Preliminärt innehåll per vecka}\\~\\
\noindent\resizebox{\columnwidth}{!}
{%
{
\fontsize{5.0pt}{6.0pt}\selectfont
\begin{tabular}{l|l|p{7.4cm}}
W01 & Introduktion & sekvens, alternativ, repetition, abstraktion, programmeringsspråk, programmeringsparadigmer, editera-kompilera-exekvera, datorns delar, virtuell maskin, REPL, literal, värde, uttryck, identifierare, variabel, typ, tilldelning, namn, val, var, def, inbyggda typer, Int, Long, Short, Double, Float, Byte, Char, String, println, typen Unit, enhetsvärdet (), stränginterpolatorn s, if, else, true, false, MinValue, MaxValue, aritmetik, slumptal, math.random, logiska uttryck, de Morgans lagar, while-sats, for-sats \\
W02 & Kodstrukturer & iterering, for-uttryck, map, foreach, Range, Array, Vector, algoritm vs implementation, pseudokod, algoritm: SWAP, algoritm: SUM, algoritm: MIN/MAX, algoritm: MININDEX, block, namnsynlighet, namnöverskuggning, lokala variabler, paket, import, filstruktur, jar, dokumentation, programlayout, JDK, main i Java vs Scala, java.lang.System.out.println \\
W03 & Funktioner, objekt & definera funktion, anropa funktion, parameter, returtyp, värdeandrop, namnanrop, default-argument, namngivna argument, applicera funktion på alla element i en samling, procedur, värdeanrop vs namnanrop, uppdelad parameterlista, skapa egen kontrollstruktur, objekt, modul, punktnotation, tillstånd, metod, medlem, funktionsvärde, funktionstyp, äkta funktion, stegad funktion, apply, lazy val, lokala funktioner, anonyma funktioner, lambda, aktiveringspost, anropsstacken, objektheapen, rekursion  cslib.window.SimpleWindow \\
W04 & Datastrukturer & attribut (fält), medlem, metod, tupel, klass, Any, isInstanceOf, toString, case-klass, samling, scala.collection, föränderlighet vs oföränderlighet, List, Vector, Set, Map, typparameter, generisk samling som parameter, översikt samlingsmetoder, översikt strängmetoder, läsa/skriva textfiler, Source.fromFile, java.nio.file \\
W05 & Sekvensalgoritmer & sekvensalgoritm, algoritm: SEQ-COPY, in-place vs copy, algoritm: SEQ-REVERSE, algoritm: SEQ-REGISTER, sekvenser i Java vs Scala, for-sats i Java, java.util.Scanner, scala.collection.mutable.ArrayBuffer, StringBuilder, java.util.Random, slumptalsfrö \\
W06 & Klasser & objektorientering, klass, Point, Square, Complex, new, null, this, inkapsling, accessregler, private, private[this], kompanjonsobjekt, getters och setters, klassparameter, primär konstruktor, objektfabriksmetod, överlagring av metoder, referenslikhet vs strukturlikhet, eq vs == \\
W07 & Arv & arv, polymorfism, trait, extends, asInstanceOf, with, inmixning, supertyp, subtyp, bastyp, override, klasshierarkin i Scala: Any AnyRef Object AnyVal Null Nothing, referenstyper vs värdetyper, klasshierarkin i scala.collection, Shape som bastyp till Point och Rectangle, accessregler vid arv, protected, final, klass vs trait, abstract class, case-object, typer med uppräknade värden \\
KS & \multicolumn{2}{l}{KONTROLLSKRIVN.} \\
W08 & Mönster, undantag & mönstermatchning, match, Option, throw, try, catch, Try, unapply, sealed, flatten, flatMap, partiella funktioner, collect, speciella matchningar: wildcard pattern; variable binding; sequence wildcard; back-ticks, equals, hashcode, exempel: equals för klassen Complex, switch-sats i Java \\
W09 & Matriser, typparametrar & matris, nästlad samling, nästlad for-sats, typparameter, generisk funktion, generisk klass, fri vs bunden typparameter, matriser i Java vs Scala, allokering av nästlade arrayer i Scala och Java \\
W10 & Sökning, sortering & strängjämförelse, compareTo, imlicit ordning, linjärsökning, binärsökning, algoritm: LINEAR-SEARCH, algortim: BINARY-SEARCH, algoritmisk komplexitet, sortering till ny vektor, sortering på plats, insättningssortering, urvalssortering, algoritm: INSERTION-SORT, algoritm: SELECTION-SORT, Ordering[T], Ordered[T], Comparator[T], Comparable[T] \\
W11 & Scala och Java & översikt av syntaxskillnader mellan Scala och Java, klasser i Scala vs Java, referensvariabler vs enkla värden i Java, referenstilldelning vs värdetilldelning i Java, alternativ konstruktor i Scala och Java, for-sats i Java, java for-each i Java, java.util.ArrayList, autoboxing i Java, primitiva typer i Java, wrapperklasser i Java, samlingar i Java vs Scala, scala.collection.JavaConverters, namnkonventioner för konstanter \\
W12 & Webb, trådar & översikt webbprogrammering, kort om html+css+javascript+scala.js, tråd, jämlöpande exekvering, icke-blockerande anrop, callback, java.lang.Thread, java.util.concurrent.atomic.AtomicInteger, scala.concurrent.Future \\
W13 & Design, api & utvecklingsprocessen, krav-design-implementation-test, gränssnitt, trait vs interface, programmeringsgränssnitt (api), designexempel \\
W14 & \multicolumn{2}{l}{Tentaträning} \\
T & \multicolumn{2}{l}{TENTAMEN} \\
\end{tabular}
}
}


\end{document}
