  bastyp & 1 & ~~\Large$\leadsto$~~ &  N & den mest generella typen i en arvshierarki \\ 
  supertyp & 2 & ~~\Large$\leadsto$~~ &  O & en typ som är mer generell \\ 
  subtyp & 3 & ~~\Large$\leadsto$~~ &  D & en typ som är mer specifik \\ 
  körtidstyp & 4 & ~~\Large$\leadsto$~~ &  J & kan vara mer specifik än den statiska typen \\ 
  dynamisk bindning & 5 & ~~\Large$\leadsto$~~ &  L & körtidstypen avgör vilken metod som körs \\ 
  polymorfism & 6 & ~~\Large$\leadsto$~~ &  B & kan ha många former, t.ex. en av flera subtyper \\ 
  trait & 7 & ~~\Large$\leadsto$~~ &  I & är abstrakt, kan mixas in, kan ej ha parametrar \\ 
  inmixning & 8 & ~~\Large$\leadsto$~~ &  H & tillföra egenskaper med \code|with| och en trait \\ 
  överskuggad medlem & 9 & ~~\Large$\leadsto$~~ &  M & medlem i subtyp ersätter medlem i supertyp \\ 
  anonym klass & 10 & ~~\Large$\leadsto$~~ &  C & klass utan namn, utvidgad med extra implementation \\ 
  skyddad medlem & 11 & ~~\Large$\leadsto$~~ &  K & är endast synlig i subtyper \\ 
  abstrakt medlem & 12 & ~~\Large$\leadsto$~~ &  F & saknar implementation \\ 
  abstrakt klass & 13 & ~~\Large$\leadsto$~~ &  E & kan ha parametrar, kan ej instansieras, kan ej mixas in \\ 
  förseglad typ & 14 & ~~\Large$\leadsto$~~ &  P & subtypning utanför denna kodfil är förhindrad \\ 
  referenstyp & 15 & ~~\Large$\leadsto$~~ &  A & har supertypen \code|AnyRef|, allokeras i heapen via referens \\ 
  värdetyp & 16 & ~~\Large$\leadsto$~~ &  G & har supertypen \code|AnyVal|, lagras direkt på stacken \\ 