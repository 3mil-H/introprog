  funktionshuvud & 1 & ~~\Large$\leadsto$~~ &  E & har parameterlista och eventuellt returtyp \\ 
  funktionskropp & 2 & ~~\Large$\leadsto$~~ &  A & koden som exekveras vid funktionsanrop \\ 
  parameterlista & 3 & ~~\Large$\leadsto$~~ &  H & beskriver namn och typ på parametrar \\ 
  parameter & 4 & ~~\Large$\leadsto$~~ &  B & namn i funktionshuvud; binds till argument \\ 
  argument & 5 & ~~\Large$\leadsto$~~ &  P & uttryck som är invärde vid funktionsanrop \\ 
  block & 6 & ~~\Large$\leadsto$~~ &  M & kan ha lokala namn; sista raden ger värdet \\ 
  namngivna argument & 7 & ~~\Large$\leadsto$~~ &  L & gör att argument kan ges i valfri ordning \\ 
  default-argument & 8 & ~~\Large$\leadsto$~~ &  J & gör att argument kan utelämnas \\ 
  värdeanrop & 9 & ~~\Large$\leadsto$~~ &  K & argumentet evalueras innan anrop \\ 
  namnanrop & 10 & ~~\Large$\leadsto$~~ &  G & fördröjd evaluering av argument \\ 
  tupel & 11 & ~~\Large$\leadsto$~~ &  F & lista med bestämt antal (heterogena) värden \\ 
  tupelreturtyp & 12 & ~~\Large$\leadsto$~~ &  N & gör att en funktion kan flera resultatvärden \\ 
  äkta funktion & 13 & ~~\Large$\leadsto$~~ &  I & ger alltid samma resultat om samma argument \\ 
  slumptalsfrö & 14 & ~~\Large$\leadsto$~~ &  C & om lika blir sekvensen av pseudoslumptal samma \\ 
  anonym funktion & 15 & ~~\Large$\leadsto$~~ &  D & funktion utan namn; kallas även lambda \\ 
  rekursiv funktion & 16 & ~~\Large$\leadsto$~~ &  O & en funktion som anropar sig själv \\ 