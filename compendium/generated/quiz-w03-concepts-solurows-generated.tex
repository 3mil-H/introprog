  funktionshuvud & 1 & ~~\Large$\leadsto$~~ &  G & har parameterlista och eventuellt en returtyp \\ 
  funktionskropp & 2 & ~~\Large$\leadsto$~~ &  O & koden som exekveras vid funktionsanrop \\ 
  parameterlista & 3 & ~~\Large$\leadsto$~~ &  J & beskriver namn och typ på parametrar \\ 
  block & 4 & ~~\Large$\leadsto$~~ &  B & kan ha lokala namn; sista raden ger värdet \\ 
  namngivna argument & 5 & ~~\Large$\leadsto$~~ &  L & gör att argument kan ges i valfri ordning \\ 
  defaultargument & 6 & ~~\Large$\leadsto$~~ &  D & gör att argument kan utelämnas \\ 
  värdeanrop & 7 & ~~\Large$\leadsto$~~ &  H & argumentet evalueras innan anrop \\ 
  namnanrop & 8 & ~~\Large$\leadsto$~~ &  E & fördröjd evaluering av argument \\ 
  tupel & 9 & ~~\Large$\leadsto$~~ &  I & lista med bestämt antal (heterogena) värden \\ 
  tupelreturtyp & 10 & ~~\Large$\leadsto$~~ &  F & gör att en funktion kan flera resultatvärden \\ 
  äkta funktion & 11 & ~~\Large$\leadsto$~~ &  C & ger alltid samma resultat om samma argument \\ 
  predikat & 12 & ~~\Large$\leadsto$~~ &  M & en funktion som ger ett booleskt värde \\ 
  slumptalsfrö & 13 & ~~\Large$\leadsto$~~ &  K & ger återupprepningsbar sekvens av pseudoslumptal \\ 
  anonym funktion & 14 & ~~\Large$\leadsto$~~ &  A & funktion utan namn; kallas även lambda \\ 
  rekursiv funktion & 15 & ~~\Large$\leadsto$~~ &  N & en funktion som anropar sig själv \\ 