  litteral & 1 & ~~\Large$\leadsto$~~ &  C & anger ett specifikt datavärde \\ 
  sträng & 2 & ~~\Large$\leadsto$~~ &  A & en sekvens av tecken \\ 
  sats & 3 & ~~\Large$\leadsto$~~ &  F & en kodrad som gör något; kan särskiljas med semikolon \\ 
  uttryck & 4 & ~~\Large$\leadsto$~~ &  O & kombinerar värden och funktioner till ett nytt värde \\ 
  funktion & 5 & ~~\Large$\leadsto$~~ &  H & vid anrop beräknas ett returvärde \\ 
  procedur & 6 & ~~\Large$\leadsto$~~ &  E & vid anrop sker (sido)effekt; returvärdet är tomt \\ 
  exekveringsfel & 7 & ~~\Large$\leadsto$~~ &  D & sker medan programmet kör \\ 
  kompileringsfel & 8 & ~~\Large$\leadsto$~~ &  P & sker innan exekveringen startat \\ 
  abstrahera & 9 & ~~\Large$\leadsto$~~ &  G & att införa nya begrepp som förenklar kodningen \\ 
  kompilera & 10 & ~~\Large$\leadsto$~~ &  M & att översätta kod till exekverbar form \\ 
  typ & 11 & ~~\Large$\leadsto$~~ &  B & beskriver vad data kan användas till \\ 
  for-sats & 12 & ~~\Large$\leadsto$~~ &  I & bra då antalet repetitioner är bestämt i förväg \\ 
  while-sats & 13 & ~~\Large$\leadsto$~~ &  K & bra då antalet repetitioner ej är bestämt i förväg \\ 
  tilldelning & 14 & ~~\Large$\leadsto$~~ &  N & för att ändra en variabels värde \\ 
  flyttal & 15 & ~~\Large$\leadsto$~~ &  J & decimaltal med begränsad noggrannhet \\ 
  boolesk & 16 & ~~\Large$\leadsto$~~ &  L & antingen sann eller falsk \\ 