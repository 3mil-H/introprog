  funktionshuvud & 1 & & A & har parameterlista och eventuellt returtyp \\ 
  funktionskropp & 2 & & B & en funktion som anropar sig själv \\ 
  parameterlista & 3 & & C & gör att en funktion kan flera resultatvärden \\ 
  parameter & 4 & & D & ger alltid samma resultat om samma argument \\ 
  argument & 5 & & E & funktion utan namn; kallas även lambda \\ 
  block & 6 & & F & uttryck som är invärde vid funktionsanrop \\ 
  namngivna argument & 7 & & G & koden som exekveras vid funktionsanrop \\ 
  default-argument & 8 & & H & kan ha lokala namn; sista raden ger värdet \\ 
  värdeanrop & 9 & & I & gör att argument kan utelämnas \\ 
  namnanrop & 10 & & J & argumentet evalueras innan anrop \\ 
  tupel & 11 & & K & lista med bestämt antal (heterogena) värden \\ 
  tupelreturtyp & 12 & & L & om lika blir sekvensen av pseudoslumptal samma \\ 
  äkta funktion & 13 & & M & gör att argument kan ges i valfri ordning \\ 
  slumptalsfrö & 14 & & N & fördröjd evaluering av argument \\ 
  anonym funktion & 15 & & O & beskriver namn och typ på parametrar \\ 
  rekursiv funktion & 16 & & P & namn i funktionshuvud; binds till argument \\ 