  funktionshuvud & 1 & & A & koden som exekveras vid funktionsanrop \\ 
  funktionskropp & 2 & & B & argumentet evalueras innan anrop \\ 
  parameterlista & 3 & & C & ger återupprepningsbar sekvens av pseudoslumptal \\ 
  block & 4 & & D & gör att argument kan ges i valfri ordning \\ 
  namngivna argument & 5 & & E & kan ha lokala namn; sista raden ger värdet \\ 
  defaultargument & 6 & & F & gör att argument kan utelämnas \\ 
  värdeanrop & 7 & & G & gör att en funktion kan flera resultatvärden \\ 
  namnanrop & 8 & & H & en funktion som anropar sig själv \\ 
  tupel & 9 & & I & en funktion som ger ett booleskt värde \\ 
  tupelreturtyp & 10 & & J & fördröjd evaluering av argument \\ 
  äkta funktion & 11 & & K & ger alltid samma resultat om samma argument \\ 
  predikat & 12 & & L & beskriver namn och typ på parametrar \\ 
  slumptalsfrö & 13 & & M & har parameterlista och eventuellt en returtyp \\ 
  anonym funktion & 14 & & N & lista med bestämt antal (heterogena) värden \\ 
  rekursiv funktion & 15 & & O & funktion utan namn; kallas även lambda \\ 