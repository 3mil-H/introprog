  parameter & 1 & & A & argumentet evalueras innan funktionen appliceras \\ 
  argument & 2 & & B & möjliggör att argument kan ges i valfri ordning \\ 
  block & 3 & & C & kan ha lokala namn; sista uttrycket blir returvärde \\ 
  värdeanrop & 4 & & D & namn i funktionshuvud; binds till argument vid anrop \\ 
  namnanrop & 5 & & E & fördröjd evaluering av argument \\ 
  namngivna argument & 6 & & F & uttryck som är invärde vid funktionsapplicering \\ 
  tupel & 7 & & G & koden som exekveras vid funktionsapplicering \\ 
  funktionshuvud & 8 & & H & en sekvens med ett visst antal värden, ev. av olika typ \\ 
  funktionskropp & 9 & & I & har en parameterlista och eventuellt en returtyp \\ 
  anonym funktion & 10 & & J & en funktion som anropar sig själv \\ 
  parameterlista & 11 & & K & beskriver namn och typ på parametrar om fler än noll \\ 
  rekursiv funktion & 12 & & L & funktion utan namn; kallas även lambda \\ 