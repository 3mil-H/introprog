%!TEX encoding = UTF-8 Unicode
\chapter{Funktioner, Objekt}\label{chapter:W03}
Begrepp du ska lära dig denna vecka:
\begin{multicols}{2}\begin{itemize}[nosep,label={$\square$},leftmargin=*]
\item definera funktion
\item anropa funktion
\item parameter
\item returtyp
\item värdeandrop
\item namnanrop
\item default-argument
\item namngivna argument
\item applicera funktion på alla element i en samling
\item procedur
\item värdeanrop vs namnanrop
\item uppdelad parameterlista
\item skapa egen kontrollstruktur
\item objekt
\item modul
\item punktnotation
\item tillstånd
\item metod
\item medlem
\item funktionsvärde
\item funktionstyp
\item äkta funktion
\item stegad funktion
\item apply
\item lazy val
\item lokala funktioner
\item anonyma funktioner
\item lambda
\item aktiveringspost
\item rekursion
\item basfall
\item anropsstacken
\item objektheapen
\item algoritm: GCD (största gemensamma delare)
\item cslib.window.SimpleWindow\end{itemize}\end{multicols}
