  modul & 1 & ~~\Large$\leadsto$~~ &  N & kodenhet med abstraktioner som kan återanvändas \\ 
  singelobjekt & 2 & ~~\Large$\leadsto$~~ &  E & modul som kan ha tillstånd; finns i en enda upplaga \\ 
  paket & 3 & ~~\Large$\leadsto$~~ &  A & modul som skapar namnrymd; maskinkod får egen katalog \\ 
  import & 4 & ~~\Large$\leadsto$~~ &  M & gör namn tillgängligt utan att hela sökvägen behövs \\ 
  lat initialisering & 5 & ~~\Large$\leadsto$~~ &  F & allokering sker först när namnet refereras \\ 
  medlem & 6 & ~~\Large$\leadsto$~~ &  H & tillhör ett objekt; nås med punktnotation om synlig \\ 
  attribut & 7 & ~~\Large$\leadsto$~~ &  K & variabel som utgör (del av) ett objekts tillstånd \\ 
  metod & 8 & ~~\Large$\leadsto$~~ &  I & funktion som är medlem av ett objekt \\ 
  privat & 9 & ~~\Large$\leadsto$~~ &  O & modifierar synligheten av en objektmedlem \\ 
  överlagring & 10 & ~~\Large$\leadsto$~~ &  D & metoder med samma namn men olika parametertyper \\ 
  namnskuggning & 11 & ~~\Large$\leadsto$~~ &  C & lokalt namn döljer samma namn i omgivande block \\ 
  namnrymd & 12 & ~~\Large$\leadsto$~~ &  G & omgivning där är alla namn är unika \\ 
  uniform access & 13 & ~~\Large$\leadsto$~~ &  L & ändring mellan def och val påverkar ej användning \\ 
  punktnotation & 14 & ~~\Large$\leadsto$~~ &  J & används för att komma åt icke-privata delar \\ 
  typalias & 15 & ~~\Large$\leadsto$~~ &  B & alternativt namn på typ som ofta ökar läsbarheten \\ 