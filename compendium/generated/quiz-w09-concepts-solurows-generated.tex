  bastyp & 1 & ~~\Large$\leadsto$~~ &  P & den mest generella typen i en arvshierarki \\ 
  supertyp & 2 & ~~\Large$\leadsto$~~ &  N & en typ som är mer generell \\ 
  subtyp & 3 & ~~\Large$\leadsto$~~ &  M & en typ som är mer specifik \\ 
  körtidstyp & 4 & ~~\Large$\leadsto$~~ &  H & kan vara mer specifik än den statiska typen \\ 
  dynamisk bindning & 5 & ~~\Large$\leadsto$~~ &  A & körtidstypen avgör vilken metod som körs \\ 
  polymorfism & 6 & ~~\Large$\leadsto$~~ &  J & kan ha många former, t.ex. en av flera subtyper \\ 
  trait & 7 & ~~\Large$\leadsto$~~ &  F & abstrakt klass, kan mixas in, kan ej ha parametrar \\ 
  inmixning & 8 & ~~\Large$\leadsto$~~ &  O & klass får nya egenskaper från trait \\ 
  överskuggad medlem & 9 & ~~\Large$\leadsto$~~ &  L & medlem i subtyp ersätter medlem i supertyp \\ 
  anonym klass & 10 & ~~\Large$\leadsto$~~ &  C & den mest generella typen i en arvshierarki \\ 
  skyddad medlem & 11 & ~~\Large$\leadsto$~~ &  D & är endast synlig i subtyper \\ 
  abstrakt medlem & 12 & ~~\Large$\leadsto$~~ &  G & saknar implementation \\ 
  abstrakt klass & 13 & ~~\Large$\leadsto$~~ &  E & kan ej instansieras \\ 
  referenstyp & 14 & ~~\Large$\leadsto$~~ &  I & ej värdetyp, har supertypen \code|AnyRef| \\ 
  förseglad typ & 15 & ~~\Large$\leadsto$~~ &  K & subtypning utanför denna kodfil är förhindrad \\ 
  värdetyp & 16 & ~~\Large$\leadsto$~~ &  B & minneslagring kan optimeras, har supertypen \code|AnyVal| \\ 