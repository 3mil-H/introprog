\documentclass[a4paper]{compendium}

\usepackage[swedish]{babel}

%\addto\captionsswedish{\renewcommand{\chaptername}{Vecka}} % http://tex.stackexchange.com/questions/30757/change-the-word-chapter-to-something-else

%TODO: Glossary
%http://tex.stackexchange.com/questions/5821/creating-a-standalone-glossary/5837#5837

\setlength{\columnsep}{16mm}

%\usepackage{chngcntr} %http://tex.stackexchange.com/questions/54383/how-to-reset-chapter-and-section-counter-with-part
%\counterwithin*{chapter}{part}

\title{
{\bf\Huge\sffamily  Programmering, grundkurs} 
\\ \vspace{2em}
{\sffamily  Kompendium}
}

%\author{Redaktör: Björn Regnell}
\date{EDAA45, Lp1-2, HT 2016}

% http://tex.stackexchange.com/questions/136527/section-numbering-without-numbers
\usepackage{titlesec}
\usepackage{titletoc}

\begin{document}

\maketitle

%%% 
\clearpage\null\thispagestyle{empty}
\vskip15cm

{\setlength{\parindent}{0pt}
\emph{Redaktör}: Björn Regnell, LTH, Lunds Universitet. \\ 

\emph{Bidragsgivare}: 
Björn Regnell,
Per Holm,
Sandra Nilsson,
Patrik Andersson,
Gustav Cedersjö,
Maj Stenmark,
Anna Axelsson,
Roy Andersson,
Torgny Roxå,
Markus Borg,
Anton Klarén.
\\ \newline

Detta manuskript är under pågående arbete. \\
Bidrag och förslag på förbättringar skickas till: 
\url{bjorn.regnell@cs.lth.se}
\\ \newline

LICENCE: CC BY-NC-SA 4.0 \\
\url{http://creativecommons.org/licenses/by-nc-sa/4.0/}
\\ \newline
Copyright \copyright~Datavetenskap, LTH \& Björn Regnell. 2016. Lund. Sweden.\\

}

%%% 
%\begin{multicols*}{2}
\tableofcontents
%\end{multicols*}

%%%%%%%%%%%%%%%%%%%%%%%%%%%%%%%%%%%%%%%%%%
\chapter*{Förord} 

\lipsum[1-3]

\mainmatter
%%%%%%%%%%%%%%%%%%%%%%%%%%%%%%%%%%%%%%%%%%

\part{Teori} 

\chapter{Introduktion}

\section{Vad är programmering?}

\lipsum
\lipsum

\section{Hur fungerar en dator?}

\lipsum

\lipsum

\chapter{Kodstrukturer}
\lipsum

\part{Verktyg}



\chapter{Terminalfönster}

\section{Vad är ett terminalfönster?}

I ett terminalfönster kan man skriva kommandon som till exempel kör program och hanterar filer på din dator. När man programmerar använder man ofta terminalkommando för att kompilera och exekvera sina program.   
 
\subsubsection{Terminal i Linux}

\subsubsection{PowerShell i Microsoft Windows}
Microsoft Windows är inte Unix-baserat, men i kommandotolken PowerShell finns alias definierat för en del vanliga unix-kommandon. Du startar Powershell t.ex. genom att genom att trycka på Windows-knappen och skriva \texttt{powershell}.

\subsubsection{Terminal i Apple OS X}
Apple OS X är ett Unix-baserat operativsystem. Många kommandon som fungerar under Linux fungerar också under Apple OS X.

\section{Några viktiga terminalkommando}

\chapter{Editera}
\section{Vad är en editor?}
\section{Välj editor}

\chapter{Kompilera och exekvera}
\section{Vad är en kompilator?}
\section{Java JDK}
\subsection{Installera Java JDK}
\section{Scala}
\subsection{Installera Scala-kompilatorn}
\section{Read-Evaluate-Print-Loop (REPL)}
För många språk, t.ex. Scala och Python, finns det en interaktiv tolk som gör det möjligt att exekvera enstaka programrader och direkt se effekte. En sådan tolk kallas Read-Evaluate-Print-Loop eftersom den läser en rad i taget och översätter till maskinkod som körs direkt.    
\subsection{Scala REPL}
\subsubsection{Kommandon i REPL}
:paste

Kortkommandon: Ctrl+K etc.

\chapter{Dokumentation}
\section{Vad gör ett dokumentationsverktyg?}
\section{scaladoc}
\section{javadoc}

\chapter{Integrerad utvecklingsmiljö}
\section{Vad är en IDE?}
\section{ScalaIDE och Eclipse}
\subsection{Installera ScalaIDE}
\section{Handledning ScalaIDE}

\chapter{Byggverktyg}
\section{Vad gör ett byggverktyg?}
\section{Byggverktyget sbt}
\subsubsection{Installera sbt}

\chapter{Versionshantering}
\section{Vad är versionshantering?}
\section{Versionshanteringsverktyget git}
\subsubsection{Installera git}


\part{Uppgifter}
%\renewcommand{\thesection}{\Alph{section}}
%\renewcommand{\thesubsection}{\thesection.\Roman{subsection}}
\renewcommand{\thesubsection}{\Alph{subsection}.}
\titleformat{\section}
  {\sffamily\Large\bfseries}   % The style of the section title
  {}                             % a prefix
  {0pt}                          % How much space exists between the prefix and the title
  {}    % How the section is represented

%\titleformat{name=\subsubsection,numberless}
%  {\sffamily\bfseries}
%  {}
%  {0pt}
%  {}  


\titlecontents{chapter}[0em]%
%http://compgroups.net/comp.text.tex/titletoc-right-justification-of-page-number-and/1910269
{\vskip 1em}%
{\bfseries\makebox[1.6em][l]{\enspace}} % before 
%{\bfseries}% numbered sections formattin
{}% unnumbered sections formatting
{\hfill\makebox[-0.5em][l]\thecontentspage}%

\titlecontents{section}[0em]
{\vskip 1ex}%
{\bfseries\makebox[3.7em][l]{\enspace}}% numbered sections formattin
{}% unnumbered sections formatting
{\hfill\makebox[-0.5em][l]\thecontentspage}%

%\setcounter{chapter}{0}

\renewcommand{\chaptername}{}% http://tex.stackexchange.com/questions/30757/change-the-word-chapter-to-something-else

\titleformat{\chapter}[hang] 
{\vskip -0.4em\sffamily\huge\bfseries}{}{0em}{} 

\newcounter{WeekCounter}
\newcommand{\Week}[1]{\refstepcounter{WeekCounter}\chapter{Vecka \arabic{WeekCounter}: #1}}

\newcounter{ExerciseCounter}
\newcommand{\Exercise}[1]{\refstepcounter{ExerciseCounter}\section{Övning \arabic{ExerciseCounter}: #1}}

\newcounter{LabCounter}
\newcommand{\Lab}[1]{\refstepcounter{LabCounter}\section{Laboration \arabic{LabCounter}: #1}}
   
\Week{Introduktion}

%!TEX root = ../compendium.tex

\Exercise{Hello}

\subsubsection{Mål}
\begin{itemize}[nosep]
\item Lär dig detta
\item Lär dig och detta
\end{itemize}

\subsubsection{Förberedelser}
\begin{itemize}[nosep]
\item Läs detta
\item Läs och detta
\end{itemize}

\subsection{Grundläggande uppgifter}

\subsubsection{Avdelning ditten}

\Task Gör detta och detta.

\Task Gör sedan detta och detta. 

\TaskComputer Gör sedan detta och detta. 

\TaskPencil Gör sedan detta och detta.

\subsubsection{Avdelning datten}
\lipsum[7]

\subsection{Extrauppgifter: öva mer på grunderna}
\lipsum[2]


\subsection{Fördjupningsuppgifter: överkurs}
\lipsum[2]
\newpage
%!TEX root = ../compendium.tex

\Lab{Quiz}

\subsubsection{Mål}
\begin{itemize}[nosep]
\item Lär dig detta
\item Lär dig och detta
\end{itemize}

\subsubsection{Förberedelser}
\begin{itemize}[nosep]
\item Läs detta
\item Läs och detta
\end{itemize}

\subsection{Obligatoriska uppgifter}


\Task Gör först detta

\Task Gör sedan detta

\subsection{Frivilliga extrauppgifter}

\Task Gör först detta

\Task Gör sedan detta



\Week{Kodstrukturer}

%!TEX root = ../compendium.tex

\Exercise{Hello}

\subsubsection{Mål}
\begin{itemize}[nosep]
\item Lär dig detta
\item Lär dig och detta
\end{itemize}

\subsubsection{Förberedelser}
\begin{itemize}[nosep]
\item Läs detta
\item Läs och detta
\end{itemize}

\subsection{Grundläggande uppgifter}

\subsubsection{Avdelning ditten}

\Task Gör detta och detta.

\Task Gör sedan detta och detta. 

\TaskComputer Gör sedan detta och detta. 

\TaskPencil Gör sedan detta och detta.

\subsubsection{Avdelning datten}
\lipsum[7]

\subsection{Extrauppgifter: öva mer på grunderna}
\lipsum[2]


\subsection{Fördjupningsuppgifter: överkurs}
\lipsum[2]
\newpage
%!TEX root = ../compendium.tex

\Lab{Quiz}

\subsubsection{Mål}
\begin{itemize}[nosep]
\item Lär dig detta
\item Lär dig och detta
\end{itemize}

\subsubsection{Förberedelser}
\begin{itemize}[nosep]
\item Läs detta
\item Läs och detta
\end{itemize}

\subsection{Obligatoriska uppgifter}


\Task Gör först detta

\Task Gör sedan detta

\subsection{Frivilliga extrauppgifter}

\Task Gör först detta

\Task Gör sedan detta





\end{document}

