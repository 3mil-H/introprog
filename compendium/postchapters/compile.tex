%!TEX encoding = UTF-8 Unicode
%!TEX root = ../compendium.tex

\chapter{Kompilera och exekvera}\label{appendix:compile}

\section{Vad är en kompilator?}

En \textbf{kompilator} \Eng{compiler} är ett program som läser programtext och översätter den till exekverbar maskinkod, så som visas i figur \ref{fig:appendix:compiler}. Programtexten som kompileras kallas källkod och utgörs av text som följer reglerna för ett programmeringsspråk, till exempel Scala eller Java. 

\begin{figure}[H]
\centering
\begin{tikzpicture}[node distance=1.8cm, scale=1.5]
\node (input) [startstop] {\bf\sffamily Källkod};
\node(inptext) [right of=input, text width=2cm, xshift=1.5cm]{För\\människor};
\node (compile) [process, below of=input] {\bf\sffamily Kompilator};
%\node(explain) [right of=compile, text width=5cm, xshift=3.0cm]{Översätter från källkod till maskinkod};
\node (output) [startstop, below of=compile] {\bf\sffamily Maskinkod};
\node(outtext) [right of=output, text width=2cm, xshift=1.5cm]{För\\maskiner};
\draw [arrow] (input) -- (compile);
\draw [arrow] (compile) -- (output);
\end{tikzpicture}
    \caption{En kompilator översätter från källkod till maskinkod.}
    \label{fig:appendix:compiler}
\end{figure}




Vissa kompilatorer genererar kod som kan köras av en processor direkt, medan andra kompilatorer genererar ett mellanformat som tolkas under exekveringen. Det senare är fallet med Java och Scala, vilket möjliggör att programmet kan kompileras en gång för alla plattformar och sedan kan programmet köras på all de processorer till vilka en s.k. virtuell maskin för Java finns \Eng{Java Virtual Machine, JVM}. Den kod som genereras av en kompilator som kodgenererar för JVM kallas \textbf{bytekod}.

Om kompileringen inte lyckas skriver kompilatorn ut ett felmeddelande och ingen maskinkod genereras. Det är inte lätt att bygga en kompilator som ger bra felmeddelanden i alla lägen, men felmeddelandet ger oftast goda ledtrådar till felorsaken efter att man lärt sig tolka det programmeringsspråkspecifika vokabulär som kompilatorn använder.

Även om programmet kompilerar utan felmeddelande och genererar exekverbar maskinkod, är det vanligt att programmet ändå inte fungerar som det är tänkt. Ibland är det mucket svårt att lista ut vad problemet beror på och man kan behöver göra omfattande undersökningar av vad som händer under körningen, genom att t.ex. skriva ut olika variablers värden eller på annat sätt ändra koden och se vad som händer. Denna process kallas felsökning eller avlusning \Eng{debugging}, och är en väsentlig del av all systemutveckling. 

En uttömmande testning av ett större program, som kör programmets \textit{alla} möjliga exekveringsvägar, är i praktiken omöjlig att genomföra inom rimlig tid, då antalet komnbinationsmöjligheter växer mycket snabbt med storleken på programmet. 
Därför är kompilatorn ett mycket viktigt hjälpmedel. Med hjälp av den analys och de kontroller som görs av kompilatorn kan många buggar, som annars vore mycket svåra att hitta, undvikas och åtgärdas i kopileringsfasen, redan \textit{innan} man exekverar programmet. 


\section{Java JDK}

Scala och flera andra språk använder Java-plattformen som exekveringsmiljö. Om man inte bara vill köra program som andra har utvecklat för denna exekveringsmiljö, utan även utveckla egna program som fungerar i denna miljö, behöver man installera Java Develpment Kit (JDK). Detta utvecklingskit innehåller flera delar, bland annat:
\begin{itemize}
\item Kompilatorn \texttt{javac} som kompilerar programtext i språket Java till bytekod som lagras i klassfiler med filnamnsändelsen \texttt{.class}.
\item Exekveringsmiljön Java Runtime Enviroment (JRE) med kommandot \texttt{java} som drar igång den virtuella javamaskinen (Java Virtual Machine) för exekvering av bytekod.
\item En mycket stor mängd färdiga programbibliotek med stöd för nätverkskommunikation, filhantering, grafik, kryptering och en massa annat som behövs när man bygger moderna system. 
\item Programmet \texttt{jar} som packar ihop många sammanhörande klassfiler till en enda jar-fil som lätt kan distribueras via nätet och sedan köras med \texttt{java}-kommandot på alla maskiner med JRE. 
\item Programmet \texttt{javap} som läser klassfiler och skriver ut vad de innehåller i ett format som kan läsas av människor (ett sådant program kallas disassembler).
\end{itemize}  

\subsection{Installera Java JDK}

Det finns flera JDK-distributioner att välja mellan, varav Oracle JDK och Azul Zulu OpenJDK är två exempel.   

\section{Scala}

\subsection{Installera Scala-kompilatorn}

\subsection{Scala Read-Evaluate-Print-Loop (REPL)}\label{appendix:compile:REPL}

För många språk, t.ex. Scala och Python, finns det en interaktiv tolk som gör det möjligt att exekvera enstaka programrader och direkt se effekten. En sådan tolk kallas Read-Evaluate-Print-Loop eftersom den läser en rad i taget och översätter till maskinkod som körs direkt.    

\TODO Kortkommandon: Ctrl+K etc.

\TODO :paste

