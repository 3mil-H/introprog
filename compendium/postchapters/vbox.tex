%!TEX encoding = UTF-8 Unicode
%!TEX root = ../compendium.tex

\chapter{Virtuell maskin}\label{appendix:vbox}

\section{Vad är en virtuell maskin?}

Du kan köra alla kursens verktyg i en så kallad \textbf{virtuell maskin} (förk. vm, eng. \textit{virtual machine}). 
Detta är ett enkelt och säkert sätt köra ett annat operativsystem i en ''sandlåda'' som inte påverkar din dators ursprungliga operativsystem. Figur \ref{fig:vm} visar kursens virtuella maskin \texttt{pgk-\VMName}. Exekveringen av en vm sker på en \textbf{värddator} \Eng{host}. I figur \ref{fig:vm} körs kursens vm i en Windows-värd med virtualiseringsapplikationen \textit{VirtualBox}\footnote{\href{https://en.wikipedia.org/wiki/VirtualBox}{/en.wikipedia.org/wiki/VirtualBox}}, som är öppen och gratis och finns för Linux-, Windows- och macOS-värdar. 



\begin{figure}[H]
\centering
\includegraphics[width=0.75\textwidth]{../img/\VMName.png}
\caption{Den virtuella maskinen pgk-\VMName~med Ubuntu \UbuntuVersion~under Windows \WindowsVersion~ i VirtualBox \VirtualBoxVersion.}
\label{fig:vm}
\end{figure}


\section{Vad innehåller kursens vm?}

Kursens virtuell maskin kör en minimal installation av Ubuntu Linux och har verktyg för programmering med Scala förinstallerade. Detta ingår i kursens vm:

\begin{itemize}
\item \texttt{java} och \texttt{javac} med OpenJDK \JDKVersion
\item \texttt{scala} och \texttt{scalac} version \ScalaVersion
\item Kojo \KojoVersion
\item VS \texttt{code} version \VSCodeVersion~med Scala (Metals) version \MetalsVersion
\item \texttt{sbt} version \SbtVersion
\end{itemize}

Du kan själv installera fler applikationer i maskinen. Fråga någon som vet hur man installerar saker i Ubuntu eller sök på nätet.

\section{Installera kursens vm}

Det går lite långsammare att köra i en virtuell maskin jämfört med att köra direkt ''på metallen'', då det sker vissa översättningar och kontroller under virtualiseringsprocessen som annars är onödiga. Och den virtuella maskinen behöver få en rejäl andel av din dators minne. Så för att köra en virtuell maskin utan att det ska bli segt behövs en ganska snabb processor, gärna över 2 GHz, och ganska mycket minne, gärna mer än 4~GB. 

Även om det går lite segt är en virtuell maskin ett utmärkt sätt att prova på Linux och Ubuntu. Eftersom man lätt kan spara undan en hel maskin är det ett bra sätt att experimentera med olika inställningar och installationer utan att ens normala miljö påverkas. Du kan lätt klona maskinen för att spara undan den i ett visst läge. Och kör du terminalfönster och en enkel editor brukar svag prestanda och lite minne inte vara ett stort problem. Om du tycker det går alltför segt kan du istället installera Linux direkt på din dator jämsides ditt andra operativsystem -- fråga någon som vet om hur man gör detta. 

Gör så här för att installera VirtualBox och köra kursens virtuella maskin:
\begin{enumerate}
\item  Ladda ner VirtualBox \VirtualBoxVersion~(eller senare version) för ditt operativsystem (t.ex. '''Platform Packages for Windows Hosts'') här och installera: \\ \url{https://www.virtualbox.org/wiki/Downloads}

%\item Ladda även ner \textit{''VirtualBox Oracle VM VirtualBox Extension Pack''}  och installera enligt instruktionerna här:\\ \url{https://www.virtualbox.org/wiki/Downloads} \\ Om du stöter på problem eller undrar hur, fråga någon om hjälp.

\item     Ladda ner filen \texttt{pgk-\VMName.ova} här: \\ \url{http://cs.lth.se/pgk/vm/} \\ OBS! Då filen är mer än 5~GB kan nedladdningen ta \textit{mycket} lång tid, kanske flera timmar beroende på din internetuppkoppling. Har du problem med nedladdningstider kan du prova att ladda ner filen till ett USB-minne på skolans datorer, så att överföringen sker lokalt i E-huset.

\item     Öppna VirtualBox och välj \MenuArrow{File}\Menu{Import appliance...} och välj filen \texttt{pgk-\VMName.ova} och klicka \Button{Next} och sedan \Button{Import}. Själva importen kan ta lång tid, kanske flera tiotals minuter beroende på hur snabbt din dator läser från disk.

\item Starta maskinen \textbf{pgk-\VMName} med ett dubbelklick. Ha lite tålamod innan maskinen är igång. Du kan behöva justera skärmstorleken i värdmaskinsmenyn \Menu{View}. Du  behöver lösenordet~\textbf{\texttt{\VMPassword}} för att logga in och för att installera nya program. 

\item Det kan hända att du får ett felmeddelande som innehåller något som liknar ''Intel VT-x'' eller ''Hyper-V'', så som beskrivs här:
\\ \href{http://www.howtogeek.com/213795/how-to-enable-intel-vt-x-in-your-computers-bios-or-uefi-firmware/}{www.howtogeek.com/213795/how-to-enable-intel-vt-x-in-your-computers-bios-or-uefi-firmware}\\
Då behöver du tillåta virtualiseringsfunktioner i BIOS på din dator. Om du inte vet hur du ska göra detta, be någon som vet om hjälp.

\item Börja med att öppna ett terminalfönster och uppdatera mjukvaran på din virtuella maskin med detta terminalkommando:
\begin{REPLnonum}
sudo apt update && sudo apt dist-upgrade
\end{REPLnonum}

\item Byt lösenord genom att trycka på windowsknappen och skriva ''users'', klicka på \textit{Users}-ikonen och trycka \Button{Password}.

\item Skriv \texttt{scala} i ett terminalfönster och du är igång och kan börja göra övningarna i detta kompendium!

%\item För att dra ner de senaste inlämningarna i kursrepot och uppdatera kompendiet och workspace, kör följande %terminalkommando:
%\begin{REPLnonum}
%$ cd ~/git/lunduniversity/introprog
%$ git pull
%$ sbt build
%$ sbt eclipse
%\end{REPLnonum}

\item Om allt verkar fungera fint kan du nu prova att öka minnet och även sätta på 3D-accelereringen för snabbare grafikrendering så här: 

\begin{enumerate}
\item Stäng maskinen genom att välja \Menu{Shut Down...} i systemmenyn. 

\item Markera maskinen \textbf{pgk-\VMName} och välj menyn \MenuArrow{Machine}\Menu{Settings...} (eller tryck Ctrl+S) och undersök inställningarna. Se speciellt under fliken \textbf{System} och \textbf{Motherboard} där det står hur mycket \textbf{Base memory} du tilldelar. Om din värddator har gott om minne (undersök exakt hur mycket) så kan du med fördel öka minnet till minst 4096~MB, speciellt om du tänker köra IntelliJ. I din virtuella maskin kan du undersöka hur stor andel av maskinens minne som är ockuperat genom att trycka på windowsknappen, skriva ''syst'', klicka på \Button{System Monitor} och välja fliken \Button{Resources}.

\item Ändra inställningar i menyn \MenuArrow{Settings...}\Menu{Display} genom att i fliken \textbf{Acceleration} under \textbf{Screen} markera \FramedCheckmark{Enable 3D acceleration}. Stara maskinen. Om det fungerar så blir animeringar avsevärt snyggare och smidigare. Om det inte fungerar, stäng av maskinen med \Menu{Power off} och avmarkera \FramedUnchecked{Enable 3D acceleration} igen.%\footnote{Du kan också prova att genomföra stegen som visas här, för att ominstallera vissa saker som kan ha uppdaterats sedan detta skrevs: \url{https://www.linuxbabe.com/virtualbox/how-to-install-virtualbox-guest-additions-on-ubuntu-step-by-step}}


\end{enumerate}

\end{enumerate}