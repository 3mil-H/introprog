%!TEX encoding = UTF-8 Unicode
%!TEX root = ../compendium.tex

\chapter{Editera}\label{appendix:edit}
\section{Vad är en editor?}

En editor används för att redigera programkod. Det finns många olika editorer att välja på. Erfarna utvecklare lägger ofta mycket energi på att lära sig att använda favoriteditorns kortkommandon och specialfunktioner, eftersom detta påverkar stort hur snabbt kodredigeringen kan göras. 

En bra editor har \textbf{syntaxfärgning} för språket du använder, så att olika delar av koden visas i olika färger. Då går det lättare att läsa och hitta i koden. 

I en integrerad utvecklingsmiljö (se appendix \ref{appendix:ide}) finns en inbyggd editor som, förutom syntaxfärgning, har fler avancerade funktioner. 

\section{Välj editor}

I tabell \ref{editor:popular-editors} visas en lista med några populära editorer. Det är en stor fördel om en editor finns på flera plattformar så att du har nytta av dina inövade färdigheter när du behöver växla mellan olika operativsystem. 

Om du inte vet vilken du ska välja, börja med \textit{gedit}, som inte är så avancerad, men därför lätt att kommna igång med. När du sedan är redo att investera din lärtid i en mer avancerad editor rekommenderas \textit{Atom}, eftersom den är öppen, gratis och finns för Linux, Windows och macOS. 

Det är är också bra att lära sig åtminståne de mest basala kommandona i editorn \textit{vim} eftersom denna  editor kan köras direkt i terminalen, även vid fjärrinloggning, och finns förinstallerad i de flesta Linux-system.