%!TEX encoding = UTF-8 Unicode
%!TEX root = ../compendium2.tex


\ChapterUnnum{Förord}

Detta kompendium innehåller övningar och laborationer och övningslösningar för andra läsperioden i LTH:s grundkurs i programmering för civilingenjörsprogrammet Datateknik.


Vi avslutar första läsperioden med en diagnostisk kontrollskrivning där du får återkoppling på vad du lärt dig hittills. Det är viktigt att du använder dina lärdomar om vad du behöver träna mer på och direkt gör upp en plan för hur du kan befästa din förståelse för begreppen i första läsperioden, så att du hänger med under kommande läsperiod.

Det övergripande målet för den andra läsperioden är att du ska kunna skapa egna program som löser mer omfattande problem än tidigare, genom att kombinera flera abstraktionsmekanismer och begrepp från läsperiod 1. Vi inför även nya abstraktionsmekanismer (t.ex. arv), nya språkkonstruktioner (t.ex. mönstermatching), samt jämför och kombinerar Scala och Java. Läsperiod 2 avslutas med ett individuellt projektarbete där du får möjlighet att fördjupa dig enligt dina egna intressen och önskemål.

Kompendiet distribueras som öppen källkod. Det får användas fritt så länge erkännande ges och eventuella ändringar publiceras under samma licens som ursprungsmaterialet. 

I kursens repo \href{http://github.com/lunduniversity/introprog}{github.com/lunduniversity/introprog} finns instruktioner om hur du kan bidra till kursmaterialet.

Välkommen till andra halvlek!

\vspace{1em}\noindent \textit{\hfill Lund, \today, Björn Regnell}
