%!TEX root = ../compendium.tex


\ChapterUnnum{Förord} 

Programmering är inte bara ett sätt att ta makten över systemen som styr vårt samhälle. Det är också ett kraftfullt verktyg för tanken. Att lära sig programmering och systemutveckling är första steget på en livslång resa av kontinuerligt lärande. Programmeringsspråk och utvecklingsverktyg kommer och går, men de grundläggande koncepten sekvens, alternativ, repetition och abstraktion som ligger bakom all mjukvara består. 

Detta kompendium utgör kursmaterial för studier i grundläggande programmering, med syfte att ge en solid bas för ingenjörsstudenter och andra som utvecklar system som innehåller mjukvara. 

Kompendiet är framtaget av, med och för studenter och lärare på universitetsnivå, och distribueras som öppen källkod. Det får användas fritt så länge erkännande ges och eventuella ändringar också publiceras som öppen källkod under samma licens som ursprungsmaterialet. På kursens hemsida \href{http://cs.lth.se/pgk}{cs.lth.se/pgk} och repo \href{http://github.com/lunduniversity/introprog}{github.com/lunduniversity/introprog} finns instruktioner om hur du kan bidra till kursmaterialet.

Läromaterialet fokuserar på lärande genom eget arbete och innehåller övningar och laborationer som är organiserade i moduler. Varje modul har ett tema och tillhörande föreläsningsanteckningar.

I kursen används språken Scala och Java för att illustrera grunderna i imperativ och objektorienterad programmering, tillsammans med elementär funktionsprogrammering. Mer avancerad objektorientering och funktionsprogrammering och  lämnas till fortsättningskurser. 



Den kanske viktigaste framgångsfaktorn vid studier i programmering är att bejaka din egen upptäckarglädje och experimentlusta. Det fantastiska med programmering är att dina egna intellektuella konstruktioner faktiskt \emph{gör} något som just \emph{du} har bestämt! Ta vara på det och prova dig fram genom att koda egna idéer -- det är kul när det funkar men minst lika lärorikt är felsökning, buggrättande och alla misslyckade försök som efter hårt arbete vänds till lyckade lösningar och bestående lärdomar. 

Välkommen i programmeringens fascinerande värld och hjärtligt lycka till med dina studier!


