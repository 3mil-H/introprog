%!TEX root = ../compendium.tex


\ChapterUnnum{Förord} 

Programmering är inte bara ett sätt att ta makten över de människoskapade system som är förutsättningen för vårt moderna samhälle. Programmering är också ett kraftfullt verktyg för tanken. Med kunskap i programmeringens grunder kan du påbörja den livslånga läranderesa som det innebär att vara systemutvecklare och abstraktionskonstnär. Programmeringsspråk och utvecklingsverktyg kommer och går, men de grundläggande koncepten bakom \emph{all} mjukvara består: sekvens, alternativ, repetition och abstraktion. 

Detta kompendium utgör kursmaterial för en grundkurs i programmering, som syftar till att ge en solid bas för ingenjörsstudenter och andra som vill utveckla system med mjukvara. Materialet omfattar en termins studier på kvartsfart och förutsätter kunskaper motsvarande gymnasienivå i svenska, matematik och engelska. 

Kompendiet är framtaget för och av studenter och lärare, och distribueras som öppen källkod. Det får användas fritt så länge erkännande ges och eventuella ändringar publiceras under samma licens som ursprungsmaterialet. På kurshemsidan \href{http://cs.lth.se/pgk}{cs.lth.se/pgk} och i kursrepot \href{http://github.com/lunduniversity/introprog}{github.com/lunduniversity/introprog} finns instruktioner om hur du kan bidra till kursmaterialet.

Läromaterialet fokuserar på lärande genom praktiskt programmeringsarbete och innehåller övningar och laborationer som är organiserade i moduler. Varje modul har ett tema och en teoridel i form av föreläsningsbilder med tillhörande anteckningar. 

I kursen använder vi språken Scala och Java för att illustrera grunderna i imperativ och objektorienterad programmering, tillsammans med elementär funktionsprogrammering. Mer avancerad objektorientering och funktionsprogrammering lämnas till efterföljande fördjupningskurser. 

Den kanske viktigaste framgångsfaktorn vid studier i programmering är att bejaka din egen upptäckarglädje och experimentlusta. Det fantastiska med programmering är att dina egna intellektuella konstruktioner faktiskt \emph{gör} något som just \emph{du} har bestämt! Ta vara på det och prova dig fram genom att koda egna idéer -- det är kul när det funkar men minst lika lärorikt är felsökning, buggrättande och alla misslyckade försök som efter hårt arbete vänds till lyckade lösningar och/eller bestående lärdomar. 

Välkommen i programmeringens fascinerande värld och hjärtligt lycka till med dina studier!

\vspace{2em}\noindent\emph{LTH, Lund 2016}


