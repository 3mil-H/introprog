På schemalagd tid senast sista läsveckan i december ska du avlägga ett obligatoriskt muntligt prov för handledare. Du måste vara godkänt på alla laborationer för att få göra det muntliga provet. Syftet med provet är att kontrollera att du har godkänd förståelse för de begrepp som ingår i kursen. Du rekommenderas att förbereda dig noga inför provet, t.ex. genom att gå igenom grundläggande begrepp för varje kursmodul och repetera grundövningar och laborationer.

Provet sker som ett stickprov ur kursens innehåll. Du kommer att få några slumpvis valda frågor där du ombeds förklara några av de begrepp som ingår i kursen. Du får även uppdrag att skriva kod som liknar kursens övningar och förklara hur koden fungerar. Du kan träna på typiska frågor här: \url{https://cs.lth.se/pgk/muntabot/}

Om det visar sig oklart huruvida du uppnått godkänd förståelse kan du behöva komplettera ditt muntliga prov. Kontakta kursansvarig för information om omprov.  
