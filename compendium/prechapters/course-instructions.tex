%!TEX encoding = UTF-8 Unicode
%!TEX root = ../compendium.tex

\chapter{Anvisningar}

Detta kapitel innehåller anvisningar och riktlinjer för kursens olika delar. Läs noga så att du inte missar viktig information om syftet bakom kursmomenten och vad som förväntas av dig.

\section{Samarbetsgrupper}

Ditt lärande i allmänhet, och ditt programmeringslärande i synnerhet, fördjupas om det sker i dialog med andra. Dessutom är din samarbetsförmåga och din pedagogiska förmåga avgörande för din framgång som professionell systemutvecklare. Därför är kursdeltagarna indelade i \emph{samarbetsgrupper} om 4-6 personer där medlemmarna samverkar för att alla i gruppen ska nå så långt som möjligt i sina studier.

För att hantera och dra nytta av skillnader i förkunskaper är samarbetsgrupperna indelade så att deltagarnas har \emph{varierande förkunskaper} baserat på en förkunskapsenkät. De som redan har provat på att programmera får då chansen att träna på sin pedagogiska förmåga som är så viktig för systemutvecklare, medan de som ännu inte kommit lika långt kan dra nytta av gruppmedlemmarnas samlade kompetens i sitt lärande. Kompetensvariationen i gruppen kommer att förändras under kursens gång då olika individer lär sig olika snabbt i olika skeden av sitt lärande; de som till att börja med har ett försprång kanske senare får kämpa för att komma över en viss lärandetröskel.

Samarbetsgrupperna organiserar själva sitt arbete och varje grupp får finna de samarbetsformer som passar medlemmarna bäst. Här följer några erfarenhetsbaserade tips:

\begin{enumerate}
\item Träffas så fort som möjligt i hela gruppen och lär känna varandra. Ju snabbare ni kommer samman som grupp och får den sociala interaktionen att fungera desto bättre. Ni kommer att ha nytta av denna investering under hela terminen och kanske under resten av er studietid.
\item Kom överens om stående mötestider och mötesplatser. Det är viktigt med kontinuiteten i arbetet för att samarbetet i gruppen ska utvecklas och fördjupas. Träffas minst en gång i veckan. Ha en stående agenda, t.ex. en runda runt bordet där var och en berättar hur långt hen kommit och listar de begreppen som hen för tillfället behöver fokusera på.
\item Hjälps åt att tillsammans identifiera och diskutera era olika individuella studiebehov och studieambitioner. När man ska lära sig att programmera stöter man på olika lärandetrösklar som man kan få hjälp att ta sig över av någon som redan är förbi tröskeln. Men det gäller då för den som hjälper att först förstå exakt vad det är som är svårt, eller vilka specifika pusselbitar som saknas, för att på bästa sätt kunna underlätta för en medstudent att ta sig över tröskeln. Det gäller att hjälpa \emph{lagom} mycket så att var och en självständigt får chansen att skriva sin egen kod.
\item Var en schysst kamrat och agera professionellt, speciellt i situationer där gruppdeltagarna vill olika. Kommunicera på ett respektfullt sätt och sök konstruktiva kompromisser. Att utvecklas socialt är viktigt för din framtida yrkesutövning som systemutvecklare och i samarbetsgruppen kan du träna och utveckla din samarbetsförmåga.
\end{enumerate}

\subsection{Samarbetskontrakt}

Ni ska upprätta ett samarbetskontrakt redan under första veckan och visa för en handledare. Alla gruppmedlemmarna ska skriva under kontraktet. Handledaren ska också skriva under som bekräftelse på att ni visat kontraktet.

Syftet med kontraktet är att ni ska diskutera igenom i gruppen hur ni vill arbeta och vilka regler ni tycker är rimliga. Ni bestämmer själva vad kontraktet ska innehålla. Nedan finns förslag på punkter som kan ingå i ert kontrakt. En kontraktsmall finns här: \url{https://github.com/lunduniversity/introprog/blob/master/admin/collaboration-contract.tex}

\begin{tcolorbox}%[width=1.05\textwidth,  grow to right by=0.03\textwidth,grow to left by=0.03\textwidth,breakable, enhanced]
\subsubsection*{Samarbetskontrakt}
Vi som skrivit under detta kontrakt lovar att göra vårt bästa för att följa samarbetsreglerna nedan, så att alla ska lära sig så mycket som möjligt.
\begin{enumerate}
\item Komma i tid till gruppmöten.
\item Vara väl förberedda genom självstudier inför gruppmöten.
\item Hjälpa varandra att förstå, men inte ta över och lösa allt åt någon annan.
\item Ha ett respektfullt bemötande även om vi har olika åsikter.
\item Inkludera alla i gemenskapen.
\item ...
\end{enumerate}
\end{tcolorbox}
\subsection{Grupplaborationer}\label{subsection:grouplabs}



Det finns två typer av laborationer: individuella laborationer och grupplaborationer. Det flesta av kursens laborationer är individuella, medan laborationerna i veckorna W07, W08 och W11 genomförs av respektive samarbetsgrupp gemensamt. Följande anvisningar gäller speciellt för grupplaborationer. (Allmänna anvisningar för laborationer finns i avsnitt \ref{section:labs}.)

\begin{enumerate}
%!TEX encoding = UTF-8 Unicode
%!TEX root = compendium.tex
\item 
Diskutera i din samarbetsgrupp hur ni ska dela upp koden mellan er i flera olika delar, som ni kan arbeta med var för sig. En sådan del kan vara en klass, en trait, ett objekt, ett paket, eller en funktion. 
\item
Varje del ska ha en \emph{huvudansvarig} individ. 
\item
Arbetsfördelningen ska vara någorlunda jämnt fördelad mellan gruppmedlemmarna.
\item
När ni redovisar er lösning ska ni börja med att redogöra för handledaren hur ni delat upp koden och vem som är huvudansvarig för vad. 
\item
Den som är huvudansvarig för en viss del redovisar den delen.
\item 
Grupplaborationer görs i huvudsak som hemuppgift. Salstiden används primärt för redovisning.
\end{enumerate}

\subsection{Samarbetsbonus}\label{section:bonus}

Alla tjänar på att samarbeta och hjälpa varandra i lärandet. Som extra incitament för grupplärande utdelas \emph{samarbetsbonus} baserat på resultatet från den diagnostiska kontrollskrivningen efter halva kursen (se avsnitt \ref{section:diagnostic-test}). Bonus ges till varje student enligt gruppmedelvärdet av kontrollskrivningspoängen och räknas ut med funktionen \code{collaborationBonus} nedan, där \code{points} är en sekvens med heltal som utgör gruppmedlemmars individuella poäng från kontrollskrivningen.

\begin{Code}
  def collaborationBonus(points: Seq[Int]): Int =
    (points.sum / points.size.toDouble).round.toInt
\end{Code}

Samarbetsbonusen viktas så att den högsta möjliga bonusen maximalt utgör $5\%$ av maxpoängen på tentan och adderas till det individuella tentaresultatet om du är godkänd på kursens sluttentamen. Samarbetsbonusen kan alltså påverka om du når högre betyg, men den påverkar \emph{inte} om du får godkänt eller ej. Detta gör att alla i gruppen gynnas av att så många som möjligt lär sig på djupet inför kontrollskrivningen. Din eventuella samarbetsbonusen räknas dig tillgodo endast vid det första, ordinarie tentamenstillfället.

\section{Föreläsningar}

En normal läsperiodsvecka börjar med två föreläsningspass om $2$ timmar vardera. Föreläsningarna ger en översikt av kursens teoretiska innehåll och går igenom innebörden av de begrepp du ska lära dig. Föreläsningarna innehåller många programmeringsexempel och föreläsaren ''lajvkodar'' då och då för att illustrera den kreativa problemlösningsprocess som ingår i all programmering. Föreläsningarna berör även kursens organisation och olika praktiska detaljer.

På föreläsningarna ges goda möjligheter att ställa allmänna frågor om teorin och att i plenum diskutera specifika svårigheter (individuell lärarhjälp ges på resurstider, se avsnitt \ref{section:tutorials}, och på laborationer, se avsnitt \ref{section:labs}). Även om det är många i föreläsningssalen, \emph{tveka inte att ställa frågor} -- det är säkert fler som undrar samma sak som du!

Föreläsningarna är inte obligatoriska, men det är mycket viktigt att du går dit, även om du i perioder känner att du har bra koll på all teori. På föreläsningarna får du en övergripande ämnesstruktur och en konkret programmeringsupplevelse, som du delar med dina kursare och kan diskutera i samarbetsgrupperna. Föreläsningarna ger också en prioritering av materialet och förbereder dig inför examinationen med praktiska råd och tips om hur du bör fokusera dina studier.

\section{Övningar}

I en normal läsperiodsvecka ingår en övning med flera uppgifter och deluppgifter.
Övningarna utgör basen för dina programmeringsstudier och erbjuder en systematisk genomgång av kursteorins alla delar genom praktiska kodexempel som du genomför steg för steg vid datorn med hjälp av ett interaktivt verktyg som kallas Read-Evaluate-Print-Loop (REPL). Om du gör övningarna i REPL säkerställer du att du skaffar dig tillräcklig förståelse för alla begrepp som ingår i kursen och att du inte missar någon viktigt pusselbit.

Övningarna utgör också förberedelse inför laborationerna. Om du inte gör veckans övning är det inte troligt att du kommer att klara veckans laboration inom rimlig tid.

Dessa två punkter är speciellt viktiga när du ska lära sig att programmera:
\begin{itemize}
\item \textbf{Programmera!} Det räcker inte med att bara passivt läsa om programmering; du måste \emph{aktivt} själv skriva mycket kod och genomföra egna programmeringsexperiment. Det underlättar stort om du bejakar din nyfikenhet och experimentlusta. Alla programmeringsfel som du gör och alla dina misstag, som i efterhand verkar enkla, är i själva verket oumbärliga steg på vägen och ger avgörande \emph{''Aha!''}-upplevelser. Kursens övningarna är grunden för denna form av lärande.

\item \textbf{Ha tålamod!} Det är först när du har förmågan att aktivt kombinera \emph{många} olika programmeringskoncept som du själv kan lösa lite större programmeringsuppgifter. Det kan vara frustrerande i början innan du når så långt att din verktygslåda med begrepp är tillräckligt stor för att du ska kunna skapa den kod du vill. Ibland krävs det extra tålamod innan allt plötslig lossnar. Många programmeringslärare och -studenter vittnar om att ''polletten plötsligt trillar ner'' och allt faller på plats. Övningarna syftar till att, steg för steg, bygga din verktygslåda så att den till slut blir tillräckligt kraftfull för mer avancerad problemlösning.
\end{itemize}

Olika studenter har olika ambitionsnivå, olika arbetskapacitet, mer eller mindre välutvecklad studieteknik och olika lätt för att lära sig att programmera. För att hantera denna variation erbjuds övningsuppgifter av tre olika typer:

\begin{itemize}
\item \textbf{Grunduppgifter}. Varje veckas grunduppgifter täcker basteorin och hjälper dig att säkerställa att du kan gå vidare utan kunskapsluckor. Grunduppgifterna utgör även basen för laborationerna. Alla studenter bör göra alla grunduppgifter. En bra förståelse för innehållet i grunduppgifterna ger goda förutsättningar att klara godkänt betyg på sluttentamen.

\item \textbf{Extrauppgifter}. Om du upplever att grunduppgifterna är svåra och du vill öva mer, eller om du vill vara säker på att du verkligen befäster dina grundkunskaper, då ska du göra extrauppgifterna. Dessa är på samma nivå som grunduppgifterna och ger extra träning.

\item \textbf{Fördjupningsuppgifter}. Om du vill gå djupare och har kapacitet att lära dig ännu mer, gör då fördjupningsuppgifterna. Dessa kompletterar grunduppgifterna med mer avancerade exempel och går utöver vad som krävs för godkänt på kursen. Om du satsar på något av de högre betygen ska du göra fördjupningsuppgifterna.
\end{itemize}

\Pen Vissa uppgifter har en penna i marginalen. Denna symbol indikerar att du ska räkna ut något, rita en figur över minnessituationen, söka information på nätet eller på annat sätt komma fram till ett resultat och gärna skriva ner resultatet (snarare än att ''bara'' köra kodexempel i REPL).

Till varje övning finns lösningar som du hittar i Appendix \ref{chapter:solutions}. Titta \emph{inte} på lösningen innan du själv först försökt lösa uppgiften. Ofta innehåller lösningarna kommentarer och tips så glöm inte att kolla igenom veckans lösningar innan du börjar förbereda dig inför veckans  laboration.

Tänk på att det ofta finns \emph{många olika lösningar} på samma programmeringsproblem, som kan vara likvärdiga eller ha olika fördelar och nackdelar beroende på sammanhanget. Diskutera gärna olika lösningsvarianter med dina kursare och handledare -- att prova många olika sätt att lösa en uppgift fördjupar ditt lärande avsevärt!

Många uppgifter lyder ''testa detta i REPL och förklara vad som händer'' och svårigheten ligger ofta inte i att skapa själva koden utan att förstå hur den fungerar och \emph{varför}. På detta sätt tränar du ditt programmeringstänkande med hjälp av en växande begreppsapparat. Syftet är ofta att illustrera ett allmängiltigt koncept och det är därför extra bra om du skapar egna övningsuppgifter på samma tema och experimenterar med nya varianter som ger dig ytterligare förståelse.

Övningsuppgifterna innehåller ofta färdiga kodsnuttar som du ska skriva in i REPL medan den kör i ett terminalfönster. REPL-kod visas i övningsuppgifterna med ljus text på mörk bakgrund, så här:
\begin{REPL}
scala> val msg = "Hello world!"
scala> println(msg)
\end{REPL}
Prompten \code{scala>} indikerar att REPL är igång och väntar på indata. Du ska skriva den kod som står \emph{efter} prompten. Mer information om hur du använder REPL hittar du i appendix \ref{appendix:compile:REPL}.

Även om kompendiet finns tillgängligt för nedladdning, frestas \emph{inte} att klippa ut och klistra in alla kodsnuttar i REPL. Ta dig istället den ringa tiden det tar att skriva in koden rad för rad. Medan du själv skriver hinner du tänka efter, och det egna, aktiva skrivandet främjar ditt lärande och gör det lättare att komma ihåg och förstå.


\section{Resurstider}\label{section:tutorials}

Under varje läsperiodsvecka finns ett flertal resurstider i schemat. Det finns minst en tid som passar din schemagrupp, men du får gärna gå på andra och/eller flera tider i mån av plats. Resurstiderna är schemalagda i datorsal med Linuxdatorer och i varje sal finns en handledare som är redo att svara på dina frågor.

Följande riktlinjer gäller för resurstiderna:

\begin{enumerate}
\item Resurstiderna är primärt till för att hjälpa dig vidare om du kör fast med övningarna eller laborationsförberedelserna, men du får fråga om vad som helst som rör kursen i den mån handledaren kan svara och hinner med.

\item Hjälp gärna varandra under resurstiderna. Om någon kursare kör fast är det utvecklande och lärorikt att hjälpa till. Om schema och plats tillåter kan du gärna gå på samma resurstidstillfälle som någon medlem i din samarbetsgrupp, men ni kan också lika gärna hjälpas åt tvärs över gruppgränserna.

\item När du hjälper andra, tänk på att prata riktigt tyst så att du inte stör andras koncentration. Tänk också på att alla behöver träna mycket själv utan att bli alltför styrda av en ''baksätesförare''. Ta inte över tangentbordet från någon annan; ge hellre välgenomtänkta tips på vägen och låt din kursare behålla kontrollen över uppgiftslösningen.


\item Du ska \emph{inte} göra och redovisa laborationen på resurstiderna; dessa ska göras och redovisas på laborationstid. Men om du varit sjuk eller ej blivit godkänd på någon enstaka laborationerna kan du, om handledaren så hinner, be att få redovisa din restlaboration på en resurstid.

\item På sidan \pageref{progress-protocoll} finns ett framstegsprotokoll för övningarna. Håll detta uppdaterat allteftersom du genomför övningarna och visa protokollet när du frågar om hjälp av handledare. Då blir det lättare för handledaren att se vilka kunskaper du förvärvat hittills och anpassa dialogen därefter.


\end{enumerate}
\section{Laborationer}\label{section:labs}

En normal läsperiodsvecka avslutas med en lärarhandledd laboration. Medan övningar tränar teorins olika delar i många mindre uppgifter, syftar laborationerna till träning i att kombinera flera begrepp och applicera dessa tillsammans i ett större program med flera samverkande delar.


En laboration varar i  $2$ timmar och är schemalagd i salar med datorer som kör Linux. Följande anvisningar gäller för laborationerna:

\begin{enumerate}

\item \textbf{Obligatorium}. Laborationerna är obligatoriska och en viktig del av kursens examination. Godkända laborationer visar att du kan tillämpa den teori som ingår i kursen och att du har tillgodogjort dig en grundläggande förmåga att självständigt, och i grupp, utveckla större program med många delar.  \emph{Observera att samtliga laborationer måste vara godkända innan du får tentera!}

\item  \textbf{Individuellt arbete.} Du ska lösa de individuella laborationerna \emph{självständigt} genom eget, enskilt arbete. Det är tillåtet att under förberedelserna diskutera övergripande principer för laborationernas lösningar i samarbetsgruppen, men var och en måste skapa sin egen lösning. (Speciella anvisningar för grupplaborationer finns i avsnitt \ref{subsection:grouplabs}.) \emph{Du ska absolut \textbf{inte} lägga ut laborationslösningar på nätet}. Läs noga på denna webbsida om var gränsen går mellan samarbete och fusk: \url{http://cs.lth.se/utbildning/samarbete-eller-fusk/}

\item \textbf{Förberedelser}. Till varje laboration finns förberedelser som du ska göra \emph{före} laborationen. Detta är helt avgörande för att du ska hinna göra laborationen inom $2$ timmar. Ta hjälp av en kamrat eller en handledare under resurstiderna om det dyker upp några frågor under ditt förberedelsearbete. Innan varje laboration skall du ha:

\begin{enumerate}
\item studerat relevanta delar av kompendiet;
\item gjort grunduppgifterna som ingår i veckans övning, och gärna även (några) extraövningar och/eller fördjupningsövningar;
\item läst igenom \emph{hela} laborationen noggrant;
\item löst förberedelseuppgifterna. I labbförberedelserna ska du i förekommande fall skriva delar av den kod som ingår i laborationen. Det krävs inte att allt du skrivit är helt korrekt, men du ska ha gjort ett rimligt försök. Ta hjälp om du får problem med uppgifterna, men låt inte någon annan lösa uppgiften åt dig.
\end{enumerate}

Om du inte hinner med alla obligatoriska labbuppgifter, får du göra de återstående uppgifterna på egen hand och redovisa dem vid påföljande labbtillfälle eller resurstid, och förbereda dig \emph{ännu} bättre till nästa laboration...

\item \textbf{Sjukanmälan}. Om du är sjuk vid något laborationstillfälle måste du anmäla detta till \emph{kursansvarig} via mejl \emph{före} laborationen. Om du varit sjuk ska du försöka göra uppgiften på egen hand och sedan redovisa den vid nästa labbtillfälle eller resurstid. Om du behöver hjälp att komma ikapp efter sjukdom, kom till en eller flera resurstider och prata med en handledare. Om du uteblir utan att ha anmält sjukdom kan kursansvarig besluta att du får vänta till nästa läsår med redovisningen, och då får du inte något slutbetyg i kursen under innevarande läsår.

\item\Pen \textbf{Skriftliga svar}. Vid några laborationsuppgifter finns en penna i marginalen. Denna symbol indikerar att du ska skriva ner och spara ett resultat som du behöver senare, och/eller som du ska visa upp för labbhandledaren vid en efterföljande kontrollpunkt.

\item\Checkpoint \textbf{Kontrollpunkter}. Vid några laborationsuppgifter finns en ögonsymbol med en bock i marginalen. Detta innebär att du nått en kontrollpunkt där du ska diskutera dina resultat med en handledare. Räck upp handen och visa vad du gjort innan du fortsätter. Om det är lång väntan innan  handledaren kan komma så är det ok att ändå gå vidare, men glöm inte att senare diskutera med handledaren så att ni gemensamt säkerställer att du förstått alla delresultat. Dialogen med din handledare är en viktig chans till återkoppling på din kod -- ta vara på den!

\end{enumerate}

\section{Kontrollskrivning}\label{section:diagnostic-test}

Efter första halvan av kursen ska du göra en obligatoriska kontrollskrivning, som genomförs individuellt på papper och penna och liknar till formen den ordinarie tentan. Kontrollskrivningen är \emph{diagnostisk} och syftar till att hjälpa dig att avgöra ditt kunskapsläge när halva kursen återstår. Ett annat syfte är att ge träning i att lösa skrivningsuppgifter med papper och penna utan datorhjälpmedel.

Kontrollskrivningen rättas med \emph{kamratbedömning} under själva skrivningstillfället. Du och en kurskamrat får efter att skrivningstiden är ute två andra skrivningar att poängbedöma i enlighet med en bedömningsmall. Syftet med detta är att du ska få träning i att bedöma kod som andra skrivit och att resonera kring kodkvalitet. När rättningen är klar får du se poängsättningen av din skrivning och kan i händelse av avgörande felaktigheter överklaga bedömningen till kursansvarig.

Den diagnostiska kontrollskrivningen påverkar inte om du blir godkänd eller ej på kursen, men det samlade poängresultatet för din samarbetsgrupp ger möjlighet till samarbetsbonus som kan påverka ditt betyg på kursen (se stycke \ref{section:bonus}).

\section{Projektuppgift}

Efter avslutad labbserie följer en projektuppgift där du på egen hand ska skapa ett stort program med många olika samverkande delar. Det är först när mängden kod blir riktigt stor som du verkligen har nytta av de olika abstraktionsmekanismer du lärt dig under kursens gång och din felsökningsförmåga sätts på prov. Följande anvisningar gäller för projektuppgiften:

\begin{enumerate}
\item \textbf{Val av projektuppgift}.
Du väljer själv projektuppgift. I kapitel \ref{chapter:W13} finns flera förslag att välja bland. Läs igenom alla uppgiftsalternativ innan du väljer vilken du vill göra. Du kan också i samråd med en handledare definiera en egen projektuppgift, men innan du börjar på en egendefinierad projektuppgift ska en skriftlig beskrivning av uppgiften godkännas av handledare, senast två veckor innan redovisningstillfället.

\item
Anvisningarna 1 och 2 som gäller för laborationer (se stycke \ref{section:labs}) gäller också för projektuppgiften: den är \textbf{obligatorisk} och arbetet ska ske \textbf{individuellt}.
Du får diskutera din projektuppgift på ett övergripande plan med andra och du kan be om hjälp av handledare på resurstid med enskilda detaljer om du kör fast, men lösningen ska vara din och du ska ha skrivit hela programmet själv.


\item \textbf{Omfattning}.
Skillnaden mellan projektuppgiften och labbarna är att den ska vara \emph{väsentligt} mer omfattande än de största laborationerna och att du färdigställer den kompletta lösningen  \emph{innan} redovisningstillfället. Du behöver därför börja i god tid, förslagsvis två veckor innan redovisningstillfället, för att säkert hinna klart.

\item \textbf{Redovisning}.
Vid redovisningen använder du tiden med handledaren till att gå igenom din lösning och redogöra för hur din kod fungerar och diskutera för- och nackdelar med ditt angreppssätt. Du ska också beskriva framväxten av ditt program och hur du stegvis har avlusat och förbättrat implementationen.

\end{enumerate}


\section{Tentamen}

Kursen avslutas med en skriftlig tentamen med snabbreferens i appendix \ref{chapter:quickref} som enda tillåtna hjälpmedel. Tentamensuppgifterna är uppdelade i två delar, del A och del B. Följande preliminära gränser gäller:

\begin{itemize}[noitemsep]
\item Du måste ha totalt minst $50\%$ av maxpoängen, exkl. ev. samarbetsbonus, för att bli godkänd.
\item Del A omfattar $20\%$ av den maximala poängsumman.
\item  Om du efter bedömning av del A erhållit färre än $80\%$ av A-delens maxpoäng underkänns din tentamen utan att del B bedöms.
\item  För betyg 4 krävs minst $70\%$ av maxpoängen, inkl. ev. samarbetsbonus.
\item  För betyg 5 krävs minst $85\%$ av maxpoängen, inkl. ev. samarbetsbonus.
\end{itemize}
