%!TEX root = ../compendium.tex

\Exercise{\ExeWeekTWO}

\begin{Goals}
\item 
\end{Goals}

\begin{Preparations}
\item Studera teorin i kapitel~\ref{chapter:W02}.
\item Bekanta dig med grundläggande terminalkommandon; se appendix~\ref{appendix:terminal}. 
\item Bekanta dig med den editor du vill använda; se appendix~\ref{appendix:edit}.
\end{Preparations}

\BasicTasks %%%%%%%%%%%%%%%%

\Task \emph{Datastrukturen \code+Range+.} Evaluera nedan uttryck i Scala REPL. Vad har respektive uttryck för värde och typ?

\Subtask \code{Range(1, 10)}

\Subtask \code{Range(1, 10).inclusive}

\Subtask \code{Range(0, 50, 5)}

\Subtask \code{Range(0, 50, 5).size}

\Subtask \code{Range(0, 50, 5).inclusive}

\Subtask \code{Range(0, 50, 5).inclusive.size}

\Subtask \code{0.until(10)}

\Subtask \code{0 until (10)}

\Subtask \code{0 until 10}

\Subtask \code{0.to(10)}

\Subtask \code{0 to 10}

\Subtask \code{0.until(50).by(5)}

\Subtask \code{0 to 50 by 5}

\Subtask \code{(0 to 50 by 5).size}

\Subtask \code{(1 to 1000).sum}

\Task \emph{Datastrukturen \code+Array+.} Kör nedan kodrader i Scala REPL. Beskriv vad som händer.

\Subtask \code{val xs = Array("hej","på","dej", "!")}

\Subtask \code{xs(0)}

\Subtask \code{xs(3)}

\Subtask \code{xs(4)}

\Subtask \code{xs(1) + " " + xs(2)}

\Subtask \code{xs.mkString}

\Subtask \code{xs.mkString(" ")}

\Subtask \code{xs.mkString("(", ",", ")")}

\Subtask \code{xs.mkString("Array(", ", ", ")")}

\Subtask \code{xs(0) = 42}

\Subtask \code{xs(0) = "42"; println(x(0))}

\Subtask \code{val ys = Array(42, 7, 3, 8)}

\Subtask \code{ys.sum}

\Subtask \code{val zs = Array.fill(10)(42)}

\Subtask \code{zs.sum}

\Task \emph{Datastrukturen \code+Vector+.} Kör nedan kodrader i Scala REPL. Beskriv vad som händer.

\Subtask \code{val words = Vector("hej","på","dej", "!")}

\Subtask \code{words(0)}

\Subtask \code{words(3)}

\Subtask \code{words.mkString}

\Subtask \code{words.mkString(" ")}

\Subtask \code{words.mkString("(", ",", ")")}

\Subtask \code{words.mkString("Ord(", ", ", ")")}

\Subtask \code{words(0) = "42"}

\Subtask \code{val numbers = Vector(42, 7, 3, 8)}

\Subtask \code{numbers.sum}

\Subtask \code{val moreNumbers = Vector.fill(10000)(42)}

\Subtask \code{moreNumbers.sum}

\Subtask\Pen Jämför med föregående uppgift. Vad kan man göra med en \code{Array} som man inte kan göra med en \code{Vector}?

\Task \emph{\code+for+-uttryck}. Evaluera nedan uttryck i Scala REPL. Vad har respektive uttryck för värde och typ?

\Subtask \code{for (i <- Range(1,10)) yield i}

\Subtask \code{for (i <- 1 until 10) yield i}

\Subtask \code{for (i <- 1 until 10) yield i + 1}

\Subtask \code{for (i <- Range(1,10).inclusice) yield i}

\Subtask \code{for (i <- 1 to 10) yield i}

\Subtask \code{for (i <- 1 to 10) yield i + 1}

\Subtask \code{(for (i <- 1 to 10) yield i + 1).sum}

\Subtask \code{for (x <- 0.0 to 2 * math.Pi by math.Pi/4) yield math.sin(x)}


\Task \emph{Metoden \code+map+ på en samling.}

\Task \emph{Metoden \code+foreach+ på en samling.}

\Task \emph{Skript.} Skapa med hjälp av en editor en fil med namn \texttt{hello-script.scala} som innehåller denna enda rad:
\begin{Code}
println("hej skript")
\end{Code}
Spara filen och kör kommandot \code{scala hello-script.scala} i terminalen:
\begin{REPL}
> scala hello-script.scala
\end{REPL}

\Subtask Vad händer?

\Subtask Ändra i filen så att högerparentesen saknas. Spara och kör skriptfilen igen. Vad händer?

\Subtask Lägg till en sats sist i skriptet som skriver ut summan av de ett tusen stycken heltalen från och med 2 till och med 1001, så som visas nedan.
\begin{REPL}
> scala hello-script.scala
hej skript
501500
\end{REPL}

\Task \emph{Applikation med \code+main+-metod.} Skapa med hjälp av en editor en fil med namn \texttt{hello-app.scala}.
\begin{REPL}
> gedit hello-app.scala
\end{REPL}
Skriv dessa rader i filen:


\scalainputlisting{examples/hello-app.scala}

\Subtask Kompilera med \code{scalac hello-app.scala} och kör koden med \code{scala Hello}.
\begin{REPL}
> scalac hello-app.scala
> ls
> scala Hello
\end{REPL}
Vad heter filerna som kompilatorn skapar?

\Subtask Ändra i din kod så att kompilatorn ger följande felmeddelande: \\
\texttt{Missing closing brace}

\Subtask\Pen Varför behövs \code{main}-metoden?

\Subtask\Pen Vilket alternativ går snabbast att köra igång, ett skript eller en kompilerad applikation? Varför? Vilket alternativ kör snabbast när väl exekveringen är igång?


\Task \emph{Java-applikation.} Skapa med hjälp av en editor en fil med namn \texttt{Hi.java}.
\begin{REPL}
> gedit Hi.java
\end{REPL}
Skriv dessa rader i filen:


\javainputlisting{examples/Hi.java}

Kompilera med \code{javac Hi.java} och kör koden med \code{java Hi}.
\begin{REPL}
> javac Hi.java
> ls
> java Hi
\end{REPL}

\Subtask\Pen Vad heter filen som kompilatorn skapat?

\Subtask\Pen Jämför signaturen för Java-programmets main-metod med signaturen för Scala-programmets main-metod. De betyder samma sak men syntaxen är olika. Beskriv skillnader och likheter i syntaxen.

\Subtask\Pen Vad blir det för felmeddelande om källkodsfilen och klassnamnet inte överensstämmer?

\ExtraTasks %%%%%%%%%%%%%%%%%%%

\Task 

\AdvancedTasks %%%%%%%%%%%%%%%%%

\Task ArrayBuffer vs Vector vs Array och metoden append