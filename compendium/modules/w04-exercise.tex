%!TEX encoding = UTF-8 Unicode
%!TEX root = ../compendium.tex

\Exercise{\ExeWeekFOUR}\label{exe:W04}

\begin{Goals}
\item Kunna skapa och använda tupler, som variabelvärden, parametrar och returvärden.

\item Förstå skillnaden mellan ett objekt och en klass och kunna förklara betydelsen av begreppet instans.

\item Kunna skapa och använda attribut som medlemmar i objekt och klasser och som som klassparametrar.

\item Beskriva innebörden av och syftet med att ett attribut är privat.

\item Kunna byta ut implementationen av metoden \code{toString}.

\item Kunna skapa och använda en objektfabrik med metoden \code{apply}.

\item Kunna skapa och använda en enkel case-klass.

\item Kunna använda operatornotation och förklara relationen till punktnotation.

\item Förstå konsekvensen av uppdatering av föränderlig data i samband med multipla referenser.

\item Känna till och kunna använda några grundläggande metoder på samlingar.

\item Känna till den principiella skillnaden mellan \code{List} och \code{Vector}.

\item Kunna skapa och använda en oföränderlig mängd med klassen \code{Set}.

\item Förstå skillnaden mellan en mängd och en sekvens.

\item Kunna skapa och använda en nyckel-värde-tabell, \code{Map}.

\item Förstå likheter och skillnader mellan en \code{Map} och en \code{Vektor}.
\end{Goals}

\begin{Preparations}
\item \StudyTheory{04}
\end{Preparations}

\BasicTasks %%%%%%%%%%%%%%%%

\Task \emph{En enkel datastruktur: tupel.} Du kan samla olika data i en tupel. Du kommer åt värdena med en metod som har namnet understreck följt av ordningsnumret.
\begin{REPL}
scala> val namn = ("Pippi", "Långstrump")
scala> namn._1
scala> namn._2
scala> println("Förnamn: " + namn._1 + "\nEfternamn:" + namn._2)
\end{REPL}

\Subtask Definiera en oföränderlig variabel med namnet \code{pt} som representerar en punkt med x-koordinaten 15.9 och y-koordinaten 28.9. Använd sedan \code{math.hypot} för att ta reda på avståndet från origo till punkten. Vad blir svaret?

\Subtask Du kan dela upp en tupel i sina beståndsdelar så här:
\begin{REPLnonum}
scala> val (förnamn, efternamn) = ("Ronja", "Rövardotter")
\end{REPLnonum}
Dela upp din punkt \code{pt} i sina beståndsdelar och kalla delarna \code{x} och \code{y}

\Subtask Värdena i en tupel kan ha olika typ. 
\begin{REPLnonum}
scala> val creature = ("Doktor", "Krokodil", 65.0, false)
scala> val (title, name, weight, isHuman)  = creature
\end{REPLnonum}
Vilken typ har 4-tupeln \code{creature} ovan?

\Subtask \label{subtask:tuplecoll} Tupler kan ingå i samlingar.
\begin{REPLnonum}
scala> val pts = Vector((0.0, 0.0), (1.0, 0.0), (1.0, 1.0), (0.0, 1.0)) 
scala> pts.foreach(println)
\end{REPLnonum}
Vilken typ har vektorn \code{pts} ovan?


\Subtask För 2-tupler finns ett kortare skrivsätt:
\begin{REPLnonum}
scala> ("Skåne", "Malmö")
scala> "Skåne -> "Malmö"
scala> val huvudstäder = Vector("Sverige" -> "Stockholm", "Norge" -> "Oslo") 
\end{REPLnonum}
Lägg till fler huvudstäder i vektorn ovan.

\Subtask Funktioner kan ta tupler som parametrar.
\begin{REPL}
scala> def length(pt: (Double, Double)) = math.hypot(pt._1, pt._2) 
scala> length((3.0, 4.0))
scala> length(3.0, 4.0)    //kompilatorn lägger till parenteserna innan anrop
\end{REPL}
Applicera funktionen \code{length} ovan på alla tupler i samlingen \code{pts} från uppgift \ref{subtask:tuplecoll} med \code{map}. Vad får resultatet för värde och typ?

\Subtask Funktioner kan ge tupler som resultat.
\begin{REPL}
scala> def div(a: Int, b: Int) = (a / b, a % b)
scala> div(10, 3)
scala> (div(9,2), div(10,2))
scala> (div(9,2)._2, div(10,2)._2)
scala> val nOdd = (1 to 10).map(i => div(i, 2)._2).sum
\end{REPL}
Förklara vad som händer ovan. Använd \code{div} ovan för att ta reda på hur många udda tal finns det i intervallet $[1234, 3456]$.

\Subtask En tupel med $n$ värden kallas $n$-tupel. Om man betraktar enhetsvärdet \code{()} som en tupel, vad kan man då kalla detta värde?

\Task \emph{Objekt med attribut (fält).} Ett objekt kan samla data som hör ihop och på så sätt skapa en datastruktur. Data i ett objekt kallas \emph{attribut} eller \emph{fält}, \Eng{field}. Objekt som samlar enbart data kallas även \emph{post} \Eng{record}.
\begin{REPLnonum}
scala> object mittKonto { var saldo = 0; val nummer = 12345L }
\end{REPLnonum}
\Subtask Skriv en sats som sätter in ett slumpmässigt belopp mellan 0 och en miljon på \code{mittKonto} ovan med hjälp av punktnotation och tilldelning. 

\Subtask Vad händer om du försöker ändra attributet \code{nummer}?

\Task \emph{Klass med attribut.} Om du vill ha många objekt av samma typ, kan du använda en \textbf{klass}. På så sätt kan man skapa många datastrukturer av samma typ men med olika innehåll. Man skapar nya objekt med nyckelordet \code{new} följt av klassens namn. Klassen utgör en ''mall'' för objektet som skapas. Ett objekt som skapas med \code{new Klassnamn} kallas även en \textbf{instans} av klassen \code{Klassnamn}. Nedan skapas en datastruktur \code{Konto} som samlar data om ett bankonto. Instanser av typen \code{Konto} håller reda på hur mycket pengar det finns på kontot och vilket kontonumret är. Datavärden som sparas i varje objektinstans, så som \code{saldo} och \code{nummer}, kallas \textbf{attribut} \Eng{attribute} eller \textbf{fält} \Eng{field}. 

\begin{REPL}
scala> class Konto {
         var saldo = 0
         var nummer = 0L
       }
scala> val k1 = new Konto
scala> val k2 = new Konto
scala> k1.saldo = 1000
scala> k1.nummer = 12345L
scala> k2.saldo = 2000
scala> k2.nummer = 67890L
scala> println("Konto: " + k1.nummer + " Saldo:" + k1.saldo)
scala> println("Konto: " + k2.nummer + " Saldo:" + k2.saldo)
\end{REPL}

\Subtask\Pen Rita hur minnessituationen ser ut efter att ovan rader har exekverats.

\Subtask\Pen Vad hade det fått för konsekvenser om attributet \code{nummer} vore oföränderligt i klassen ovan? (Jämför med objektet \code{mittKonto}.)


\Task \emph{Klass med attribut som parametrar.} Om man vill ge attributen initialvärden när objektet skapas med \code{new}, kan man placera attributen i en parameterlista till klassen. Koden som körs när objektet skapas och attributen tilldelas sina initialvärden, kallas \textbf{konstruktor} \Eng{constructor}.

\begin{REPL}
scala> class Konto(var saldo: Int, val nummer: Long)
scala> val k = new Konto(0, 12345L)
scala> println("Konto: " + k.nummer + " Saldo:" + k.saldo)
scala> println(k)
scala> k.toString
\end{REPL}

\Subtask Den två sista raderna ovan skriver ut den identifierare som JVM använder för att hålla reda på objektet i sina interna datastrukturer. Vad skrivs ut?

\Subtask Skapa ännu en instans av klassen Konto  med samma saldo och nummer som \code{k} ovan och spara den i \code{val k2} och undersök dess objektidentifierare. Får objekten \code{k} och \code{k2} olika objektidentifierare?

\Subtask Sätt in olika belopp på respektive konto.

\Subtask Vad händer om du försöker ändra attributet \code{nummer}?

\Subtask\Pen Ibland räcker det fint med en tupel, men ofta vill man ha en klass istället. Beskriv några fördelar med en Konto-klassen ovan jämfört med en tupel av typen \code{(Int, Long)}.

\begin{REPLnonum}
scala> var k3 = (0, 12345L)
scala> k3 = (k3._1 + 100, k3._2)
\end{REPLnonum}

\Task \emph{Publikt eller privat attribut?} Man kan förhindra att ett attribut syns utanför klassen med hjälp av nyckelordet \code{private}.  

\begin{REPL}
scala> class Konto1(val nummer: Long){ var saldo = 0 }
scala> val k1 = new Konto1(12345678901L)
scala> k1.nummer
scala> k1.saldo += 1000
scala> class Konto2(val nummer: Long){ private var saldo = 0 }
scala> val k2 = new Konto2(12345678901L)
scala> k2.nummer
scala> k2.saldo += 1000
\end{REPL}

\Subtask Vad händer ovan?

\Subtask Gör en ny version av klassen \code{Konto} enligt nedan:

\begin{Code}
class Konto(val nummer: Long){ 
  private var saldo = 0
  def in(belopp: Int): Unit = {saldo += belopp}
  def ut(belopp: Int): Unit = {saldo -= belopp}
  def show: Unit = 
    println("Konto Nr: " + nummer + " saldo: " + saldo) 
}

object Main {
  def main(args: Array[String]): Unit = {
    val k = new Konto(1234L)
    k.show
    k.in(1000)
    println("Uttag: " + k.ut(500))
    println("Uttag: " + k.ut(1000))
    k.show
  }
}
\end{Code}

\Subtask Spara koden i en fil, kompilera med \code{scalac} och kör. Testa även vad som händer om du försöker komma åt attributet \code{saldo} i main-metoden med t.ex. \code{println(k.saldo)} eller \code{k.saldo += 1000}. 

\Subtask Vi ska nu förhindra överuttag. Ändra i metoden \code{ut} så att den får signaturen \code{ut(belopp: Int): (Int, Int) = ???} och implementera \code{ut} så att den returnerar både beloppet man verkligen kan ta ut och kvarvarande saldo. Om man försöker ta ut mer än det finns på kontot så ska saldot bli 0 och man får bara ut det som finns kvar. Spara, kompilera, kör. 

\Subtask Förbättra metoderna \code{in} och \code{ut} så att man inte kan sätta in eller ta ut negativa belopp.

\Subtask Vad är fördelen med att göra föränderliga attribut privata och bara påverka deras värden indirekt via metoder?

\Task \emph{Vilken typ har ett objekt?} Objektets typ bestäms av klassen. Vid tilldelning måste typerna passa ihop.

\Subtask Vilka rader nedan ger felmeddelande? Hur lyder felmeddelandet?
\begin{REPL}
scala> class Punkt(val x: Double, val y: Double)
scala> val pt: Punkt = new Punkt(10.0, 10.0)
scala> val i: Int = pt.x
scala> val (x: Double, y: Double) = (pt.x, pt.y)
scala> val p: Double = new Punkt(5.0, 5.0)
scala> val p = new Punkt(5.0, 5.0): Double
scala> val p = new Punkt(5.0, 5.0): Punkt
scala> pt: Punkt
\end{REPL}


\Subtask Man kan undersöka om ett objekt är av en viss typ med metoden \\ \code{isInstanceOf[Typnamn]}. Vad ger nedan anrop av metoden \code{isInstanceOf} för värde?
\begin{REPL}
scala> class Punkt(val x: Double, val y: Double)
scala> val pt: Punkt = new Punkt(1.0, 2.0)
scala> pt.isInstanceOf[Punkt]
scala> pt.isInstanceOf[Double]
scala> pt.x.isInstanceOf[Punkt]
scala> pt.x.isInstanceOf[Double]
scala> pt.x.isInstanceOf[Int]
\end{REPL}

\Task \emph{\code{Any}}. Alla klasser är också av typen \code{Any}. Alla klasser får därmed med sig några gemensamma metoder som finns i den fördefinierade klassen \code{Any}, däribland metoderna  \code{isInstanceOf} och \code{toString}.  Vad blir resultatet av respektive rad nedan? Vilken rad ger ett felmeddelande?

\begin{REPL}
scala> class Punkt(val x: Double, val y: Double)
scala> val pt: Punkt = new Punkt(1.0, 2.0)
scala> pt.isInstanceOf[Punkt]
scala> pt.isInstanceOf[Any]
scala> pt.x.toString
scala> println(pt.x)
scala> val a: Any = pt
scala> println(a.x)
scala> a.toString
scala> pt.y.toString
scala> a.y.toString
\end{REPL}

\Task \emph{Byta ut metoden \code{toString}}. I klassen \code{Any} finns metoden \code{toString} som skapar en strängrepresentation av objektet. Du kan byta ut metoden \code{toString} i klassen \code{Any} mot din egen implementation. Man använder nyckelordet \code{override} när man vill byta ut en metodimplementation.

\begin{REPL}
scala> class Punkt(val x: Double, val y: Double) {
         override def toString: String = "[x=" + x + ",y=" + y + "]"
       }
scala> val pt = new Punkt(1.0, 42.0)
scala> pt.toString
scala> println(pt)
\end{REPL}

\Subtask Vad händer egentligen på sista raden ovan?

\Subtask Omdefiniera toString så att den ger en sträng på formen \code{Punkt(1.0, 42.0)}.

\Subtask Vad händer om du utelämnar nyckelordet \code{override} vid omdefiniering?

\Task \emph{Objektfabrik med \code{apply}-metod.} Man kan ordna så att man slipper skriva \code{new} med ett s.k. \emph{fabriksobjekt} \Eng{factory object}. 
\begin{Code}
class Pt(val x: Double, y: Double) {
  override def toString: String = "Pt(x=" + x + ",y=" + y + ")"
}
object Pt { 
  def apply(x: Double, y: Double): Pt = new Pt(x, y)
}
\end{Code}

\Subtask Skriv satser som använder metoden \code{apply} i fabriksobjektet \code{object Pt} för att skapa flera olika punkter.

\Subtask Ge applymetoden default-argument 0.0 för både x och y så att \code{Pt()} skapar en punkt i origo.

\Subtask Skapa en klass \code{Rational} som representerar rationellt tal som en kvot mellan två heltal. Ge klassen två oföränderliga, publika klassparameterattribut med namnen \code{nom} för täljaren och \code{denom} för nämnaren. 

\Subtask Skapa ett fabriksobjekt med en \code{apply}-metod som tar två heltalsparametrar och skapar en instans av klassen \code{Rational}.

\Subtask Skapa olika instanser av din klass \code{Rational} ovan med hjälp av fabriksobjektet.


\Task \emph{Skapa en case-klass.} Med en case-klass får man \code{toString} och fabriksobjekt på köpet. Man behöver inte skriva \code{val} framför klassparametrar i case-klasser; klassparametrar blir publika, oföränderliga attribut automatiskt när man deklarerar en case-klass.

\begin{REPL}
scala> case class Pt(x: Double, y: Double) 
scala> val p = Pt(1.0, 42.0)
scala> p.toString
scala> println(p)
scala> println(Pt(5,6))
\end{REPL}

\Subtask Implementera din klass \code{Rational} från föregående uppgift, men nu som en case-klass.
	
\Task \label{task:point} \emph{Metoder på datastrukturer.} En datastruktur blir mer användbar om det finns metoder som kan användas på datastrukturen. Metoder i Scala kan även ha (vissa) specialtecken som namn, t.ex. \code{+} enligt nedan.  
\begin{REPL}
scala> case class Point(x: Double, y: Double) {
         def distToOrigin: Double = math.hypot(x, y)   
         def add(p: Point): Point = Point(x + p.x, y + p.y)
         def +(p: Point): Point = add(p)
       }
\end{REPL}

\Subtask Använd metoden \code{distToOrigin} för att ta reda på vad punkten med koordinaterna (3, 4) har för avstånd till origo?

\Subtask Skriv satser som skapar två punkter (3,4) och (5, 6) och låt variablerna p1 och p2 referera till respektive punkt. Låt variabeln p3 bli summan av p1 och p2 med hjälp av metoden \code{add}. Vad får uttrycken \code{p3.x} resp. \code{p3.y} för värden?



\Task \emph{Operatornotation.} Vid punktnotation på formen: \\ \code{objekt.metod(argument)} \\ kan man skippa punkten och parenteserna och skriva:\\ \code{objekt metod argument}  \\
Detta förenklade skrivsätt kallas \textbf{operatornotation}.

\Subtask Använd klassen \code{Point} från uppgift \ref{task:point} och prova nedan satser. Vilka rader använder operatortnotation och vilka rader använder punktnotation? Vilka rader ger felmeddelande?
\begin{REPL}
scala> val p1 = Point(3,4)
scala> val p2 = Point(3,4)
scala> p1.add(p2)
scala> p1 add p2
scala> p1.+(p2)
scala> p1 + p2
scala> 42 + 1
scala> 42.+(1)
scala> 42.+ 1
scala> 42 +(1)
scala> 1.to(42)
scala> 1 to 42
scala> 1.to(42)
\end{REPL}

\Subtask Implementera metoderna \code{sub} och \code{-} i klassen \code{Point} och skriv uttryck som kombinerar add och sub, samt + och - i både punktnotation och operatornotation.

\Subtask Operatornotation fungerar även med flera argument. Man använder då parenteser om listan med argumenten: 
\code{ objekt metod (arg1, arg2)}  \\
Definiera en metod \\
\code{def scale(a: Double, b: Double) = Point(x * a, y * b)} \\ 
i klassen \code{Point} och skriv satser som använder metoden med punktnotation och operatornotation. 





\Task \emph{Föränderlighet och oföränderlighet.} Oföränderliga och föränderliga objekt beter sig olika vid tilldelning.  

\Subtask\Pen Innan du kör nedan kod: Försök lista ut vad som kommer att skrivas ut. Rita minnessituationen efter varje tilldelning. 

\begin{Code}
println("\n--- Example 1: mutable value assigmnent")
var x1 = 42
var y1 = x1
x1 = x1 + 42
println(x1)
println(y1)
\end{Code}

\Subtask\Pen Innan du kör nedan kod: Försök lista ut vad som kommer att skrivas ut. Rita minnessituationen efter varje tilldelning.

\begin{Code}
println("\n--- Example 2: mutable object reference assignment")
class MutableInt(private var i: Int) {
  def +(a: Int): MutableInt = { i = i + a; this }
  override def toString: String = i.toString
}
var x2 = new MutableInt(42)
var y2 = x2
x2 = x2 + 42
println(x2)
println(y2)
\end{Code}

\Subtask\Pen Innan du kör nedan kod: Försök lista ut vad som kommer att skrivas ut. Rita minnessituationen efter varje tilldelning.

\begin{Code}
println("\n--- Example 3: immutable object reference assignment")
class ImmutableInt(val i: Int) {
  def +(a: Int): ImmutableInt = new ImmutableInt(i + a) 
  override def toString: String = i.toString
}
var x3 = new ImmutableInt(42)
var y3 = x3
x3 = x3 + 42
println(x3)
println(y3)
\end{Code}

\Subtask\Pen Vad finns det för fördelar med oföränderliga datastrukturer?


\Task \emph{Några användbara samlingar.} En \textbf{samling} \Eng{collection} är en datastruktur som samlar många objekt av samma typ. I \code{scala.collection} och \code{java.util} finns många olika samlingar med en uppsjö användbara metoder. De olika samlingarna i \code{scala.collection} är ordnade i en gemensam hierarki med många gemensamma metoder; därför har man nytta av det man lär sig om metoderna i en Scala-samling när man använder en annan samling. Vi har redan tidigare sett samlingen \code{Vector}:

\begin{REPL}
scala> val tärningskast = Vector.fill(10000)((math.random * 6 + 1).toInt)
scala> tä   // tryck TAB
scala> tärningskast.  // tryck TAB
\end{REPL}

\Subtask Ungefär hur många metoder finns det som man kan göra på objekt av typen \code{Vector}? Det är svårt att lära sig alla dessa på en gång, så vi väljer ut några få i kommande uppgifter.

\Subtask Jämför överlappet mellan metoderna i \code{Vector} och \code{List} och uppskatta hur stor andel av metoderna som är gemensamma: 
\begin{REPL}
scala> val myntkast = 
         List.fill(10000)(if (math.random < 0.5) "krona" else "klave")
scala> my   // tryck TAB
scala> myntkast.  // tryck TAB
\end{REPL}

\Task \emph{Typparameter.} Vissa funktioner är generella för många typer och tar en så kallad \textbf{typparameter} inom hakparenteser. Ofta slipper man skriva typparametrar, då kompilatorn kan härleda typen utifrån argumenten. Om man anger typparametrar explicit så hjälper kompilatorn dig med att kolla att det verkligen är rätt typ i samlingen. 

\Subtask Vad händer nedan?
\begin{REPL}
scala> var xs = Vector.empty[Int]
scala> xs = xs :+ "42" 
scala> xs = xs :+ 43 :+ 64 :+ 46
scala> xs
scala> xs :+= "42".toInt 
scala> var ys = Vector[Int]("ett", "två", "tre")
scala> var ingenting = Vector.empty 
scala> ingenting = Vector(1,2,3)
\end{REPL}

\Subtask Samlingar är mer användbara om de är \emph{generiska}, vilket innebär att elementens typ avgörs av en typparameter och därför kan vara av vilken typ som helst. Man kan definiera egna funktioner som tar generiska samlingar som parametrar. Förklara vad som händer här:
\begin{REPL}
scala> val vego = Vector("gurka", "tomat", "apelsin", "banan")
scala> val prim = Vector(2, 3, 5, 7, 11, 13)
scala> def först[T](xs: Vector[T]): T = xs.head
scala> def sist[T](xs: Vector[T]) = xs.last
scala> def förstOchSist[T](xs: Vector[T]): (T, T) = (xs.head, xs.last)
scala> först(vego)
scala> sist(prim)
scala> förstOchSist(vego)
scala> förstOchSist(prim)
scala> def wrap[T](pair: (T, T))(xs: Vector[T]) = pair._1 +: xs :+ pair._2
scala> wrap("Odla", "och ät!")(vego)
scala> wrap("Odla", "och ät!")(vego).mkString(" ")
\end{REPL}





\Task \emph{Några viktiga samlingsmetoder.} Deklarera följande vektorer i REPL. 
\begin{REPL}
scala> val xs = (1 to 10).toVector
scala> val a = Vector("abra", "ka", "dabra")
scala> val b = Vector( "sim", "sala", "bim", "sala", "bim")
scala> val stor = Vector.fill(100000)(math.random)
\end{REPL}
Undersök i REPL vad som händer nedan. Alla dessa metoder fungerar på alla samlingar som är indexerbara sekvenser. Givet deklarationerna ovan: vad har uttrycken nedan för värde och typ? Förklara vad som händer hälp av denna  översikt: \href{http://docs.scala-lang.org/overviews/collections/seqs}{docs.scala-lang.org/overviews/collections/seqs}

\Subtask \code{a(1) + xs(1)}

\Subtask \code{a apply 0}

\Subtask \code{a.isDefinedAt(3)}

\Subtask \code{a.isDefinedAt(100)}

\Subtask \code{stor.length}

\Subtask \code{stor.size}

\Subtask \code{stor.min}

\Subtask \code{stor.max}

\Subtask \code{a indexOf "ka"}

\Subtask \code{b.lastIndexOf("sala")}

\Subtask \code{"först" +: b   //minnesregel: colon on the collection side}

\Subtask \code{a :+ "sist"    //minnesregel: colon on the collection side}

\Subtask \code{xs.updated(2,42)}

\Subtask \code{a.padTo(10, "!")}

\Subtask \code{b.sorted}

\Subtask \code{b.reverse}

\Subtask \code{a.startsWith(Vector("abra", "ka"))}

\Subtask \code{"hejsan".endsWith("san")}

\Subtask \code{b.distinct}



\Task \emph{Några generella samlingsmetoder.} Det finns metoder som går att köra på \emph{alla} samlingar även om de inte är indexerbara. Givet deklarationerna i föregående uppgift: vad har uttrycken nedan för värde och typ? Förklara vad som händer med hjälp av dessa översikter: \\ \href{http://docs.scala-lang.org/overviews/collections/trait-traversable}{docs.scala-lang.org/overviews/collections/trait-traversable} \\ \href{http://docs.scala-lang.org/overviews/collections/trait-iterable}{docs.scala-lang.org/overviews/collections/trait-iterable}

\Subtask \code{a ++ b}

\Subtask \code{a ++ stor}

\Subtask \code{val ys = xs.map(_ * 5)}

\Subtask \code{b.toSet     // En mängd har inga dubletter}

\Subtask \code{a.head + b.last}

\Subtask \code{a.tail}

\Subtask \code{a.head +: a.tail == a}

\Subtask \code{Vector(a.head) ++ Vector(b.last)}

\Subtask \code{a.take(1) ++ b.takeRight(1)}	

\Subtask \code{a.drop(2) ++ b.drop(1).dropRight(2)}

\Subtask \code{a.drop(100)}

\Subtask \code{val e = Vector.empty[String]; e.take(100)}

\Subtask \code{Vector(e.isEmpty, e.nonEmpty)}

\Subtask \code{a.contains("ka")}

\Subtask \code{"ka" contains "a"}

\Subtask \code{a.filter(s => s.contains("k")) }	

\Subtask \code{a.filter(_.contains("k")) }	

\Subtask \code{a.map(_.toUpperCase).filterNot(_.contains("K")) }	

\Subtask \code{xs.filter(x => x % 2 == 0)}

\Subtask \code{xs.filter(_ % 2 == 0)}


\Task De olika samlingarna i \code{scala.collection} används flitigt i andra paket, exempelvis \code{scala.util} och \code{scala.io}. 

\Subtask Vad händer här? (Metoden \code{shuffle} skapar en ny samling med elementen i slumpvis ordning.)
\begin{REPL}
val xs = Vector(1,2,3)
def blandat = scala.util.Random.shuffle(xs)
def test = if (xs == blandat) "lika" else "olika"
(for(i <- 1 to 100) yield test).count(_ == "lika")
\end{REPL}


\Subtask Skapa en textfil med namnet \code{fil.txt} som innehåller lite text och läs in den med: \\\code{scala.io.Source.fromFile("fil.txt", "UTF-8").getLines.toVector}
\begin{REPL}
> cat > fil.txt
hejsan
svejsan
> scala
scala> val xs = scala.io.Source.fromFile("fil.txt", "UTF-8").getLines.toVector
scala> xs.foreach(println)
\end{REPL}


\Subtask Vad händer här? (Metoden \code{trim} på värden av typen \code{String} ger en ny sträng med blanktecken i början och slutet borttagna.) 
\begin{REPL}
scala> val pgk = 
  scala.io.Source.fromURL("http://cs.lth.se/pgk/","UTF-8").getLines.toVector
scala> pgk.foreach(println)
scala> pgk.map(_.trim).
         filterNot(_.startsWith("<")).
         filterNot(_.isEmpty).
         foreach(println)
\end{REPL}



\Task \emph{Jämföra List och Vector.} En indexerbar sekvens av värden kallas vektor eller lista. I Scala finns flera klasser som kan kan indexeras, däribland klasserna \code{Vector} och \code{List}. 

\Subtask \emph{Likheter mellan \code{Vector} och \code{List}.} Kör nedan rader i REPL. Prova indexera i båda och studera hur stor andel av metoderna som är gemensamma.
\begin{REPL}
scala> val sv = Vector("en", "två", "tre", "fyra")
scala> val en = List("one", "two", "three", "four")
scala> sv(0) + sv(3)
scala> en(0) + en(3)
scala> sv. //tryck TAB 
scala> en. //tryck TAB
\end{REPL}

\Subtask \emph{Skillnader mellan \code{Vector} och \code{List}.} Klassen \code{Vector} i Scala har ''under huven'' en avancerad datastruktur i form av ett s.k. självbalanserande träd, vilket gör att \code{Vector} är snabbare än \code{List} på nästan allt, \emph{utom} att bearbeta elementen i \emph{början} av sekvensen; vill man lägga till och ta bort i början av en \code{List} så kan det ibland gå ungefär dubbelt så fort jämfört med \code{Vector}, medan alla andra operationer är lika snabba eller snabbare med \code{Vector}. Det finns ett fåtal speciella metoder, som bara finns i \code{List}, för att skapa en lista och lägga till i början av en lista. Vad händer nedan?

\begin{REPL}
scala> var xs = "one" :: "two" :: "three" :: "four" :: Nil
scala> xs = "zero" :: xs
scala> val ys = xs.reverse ::: xs
\end{REPL}


\Task \emph{Mängd.} En mängd är en samling som garanterar att det inte finns några dubbletter. Det går dessutom väligt snabbt, även i stora mängder, att kolla om ett element finns eller inte i mängden. Elementen i samlingen \code{Set} hamnar ibland, av effektivitetsskäl, i en förvånande ordning.
\begin{REPL}
scala> val s = Set("Malmö", "Stockholm", "Göteborg", "Köpenhamn", "Oslo")
s: scala.collection.immutable.Set[String] = 
     Set(Oslo, Malmö, Köpenhamn, Stockholm, Göteborg)

scala> val t = Set("Sverige", "Sverige", "Sverige", "Danmark", "Norge")
t: scala.collection.immutable.Set[String] = Set(Sverige, Danmark, Norge)
\end{REPL}
Givet ovan deklarationer: vad blir värde och typ av nedan uttryck?

\Subtask \code{s + "Malmö" == s}

\Subtask \code{s ++ t}

\Subtask \code{Set("Malmö", "Oslo").subsetOf(s)}

\Subtask \code{s subsetOf Set("Malmö", "Oslo")}

\Subtask \code{s contains "Lund"}

\Subtask \code{s apply "Lund"}

\Subtask \code{s("Malmö")}

\Subtask \code{s - "Stockholm"}

\Subtask \code{t - ("Norge", "Danmark", "Tyskland")}

\Subtask \code{s -- t}

\Subtask \code{s -- Set("Malmö", "Oslo")}

\Subtask \code{Set(1,2,3) intersect Set(2,3,4)}

\Subtask \code{Set(1,2,3) & Set(2,3,4)}

\Subtask \code{Set(1,2,3) union Set(2,3,4)}

\Subtask \code{Set(1,2,3) | Set(2,3,4)}


\Task \emph{Slå upp värden från nycklar med \code{Map}.} Samlingen \code{Map} är mycket användbar. Med den kan man snabbt leta upp ett värde om man har en nyckel. Samlingen \code{Map} är en generalisering av en vektor, där man kan ''indexera'', inte bara med ett heltal, utan med vilken typ av värde som helst, t.ex. en sträng. Datastrukturen \code{Map} är en s.k. \emph{associativ array}\footnote{\href{https://en.wikipedia.org/wiki/Associative_array}{https://en.wikipedia.org/wiki/Associative\_array}}, implementerad som en s.k. \emph{hashtabell}\footnote{\href{https://en.wikipedia.org/wiki/Hash_table}{https://en.wikipedia.org/wiki/Hash\_table}}. 
\begin{REPL}
scala> var huvudstad = 
  Map("Sverige" -> "Stockholm", "Norge" -> "Oslo", "Skåne" -> "Malmö") 
\end{REPL}
Givet ovan variabel \code{huvudstad}, förklara vad som händer nedan?

\Subtask \code{huvudstad apply "Skåne"}

\Subtask \code{huvudstad("Sverige")}

\Subtask \code{huvudstad.contains("Skåne")}

\Subtask \code{huvudstad.contains("Malmö")}

\Subtask \code{huvudstad += "Danmark" -> "Köpenhamn"}

\Subtask \code{huvudstad.foreach(println)} 

\Subtask \code{huvudstad getOrElse ("Norge", "???") }

\Subtask \code{huvudstad getOrElse ("Finland", "???") }

\Subtask \code{huvudstad.keys.toVector.sorted}

\Subtask \code{huvudstad.values.toVector.sorted}

\Subtask \code{huvudstad - "Skåne"}

\Subtask \code{huvudstad - "Jylland"}

\Subtask \code{huvudstad = huvudstad.updated("Skåne","Lund") }



\Task \emph{Skapa Map från en samling.}

\Subtask Definiera denna vektor och undersök dess typ:
\begin{Code}
val pairs = Vector(
  ("Björn", 46462229009L), 
  ("Maj", 46462221667L), 
  ("Gustav", 46462224906L))
\end{Code}

\Subtask Vad har variablen \code{telnr} nedan för typ: \\ \code{var telnr = pairs.toMap}

\Subtask Använd \code{telnr} för att slå upp telefonnummer för Maj och Kim med hjälp av metoderna \code{apply} och \code{get}.

\Subtask Använd metoden \code{getOrElse} vid upplagningar av \code{telnr} och ge \code{-1L} som telefonnummer i händelse av att ett nummer inte finns. 

\Subtask Lägg till \code{("Fröken Ur", 464690510L)} i \code{telnr}-mappen.

\Subtask Skapa en \code{Vector[(String, String)]} enligt nedan, så att telefonnumret blir en sträng utan inledande landsnummer men med en nolla i riktnumret. Byt ut \code{???} mot lämpligt uttryck.
\begin{REPL}
scala> telnr.toVector.map(p => ???) 
res85: Vector[(String, String)] = Vector(("Björn", "0462229009"), ("Maj", 
"0462221667"), ("Gustav", "0462224906"), ("Fröken Ur", 04690510"))

\end{REPL}

\Subtask Använd vektorn i resultatet ovan för att skapa en ny \code{Map[String, String]} med nationella telefonnumer. Slå upp numret till Fröken Ur.

\Task \emph{Samlingsmetoden \code{maxBy}.} Med samlingsmetoden \code{maxBy} kan man själv definiera vad som ska maximeras. (Denna metod kommer du att behöva i veckans laboration.)

\Subtask Förklara vad som händer nedan.
\begin{REPL}
scala> val xs = Vector((2,3), (1,5), (-1, 1), (7, 2))
scala> xs.maxBy(x => x._1)
scala> xs.maxBy(x => x._2)
\end{REPL}

\Subtask Om man bara använder en parameter i en anonym funktion, till exempel parametern \code{x} i lambdauttrycket \code{x => x + 1} \emph{en enda} gång, och kompilatorn kan gissa alla typer, kan man använda understreck som ''platshållare'' för att förkorta lambdauttrycket så här: \code{ _ + 1} 

Skriv uttrycken på raderna 2 och 3 i föregående deluppgift på ett kortare sätt med hjälp platshållarsyntax \Eng{place holder syntax}.  

\Subtask På motsvarande sätt kan man använda \code{minBy} för att välja vilket funktion som definierar  minimum. Prova \code{minBy} på motsvarande sätt som i föregående deluppgifter.

\Task \emph{Samlingsmetoden \code{sliding}.} I veckans labb kommer du att ha nytta av samlingsmetoden \code{sliding}, som ger en iterator för speciella delsekvenser av en sekvens, vilka kan liknas vid ''utsikten'' i ett ''glidande fönster''. Kör nedan i REPL och beskriv vad som händer.

\begin{REPL}
scala> val xs = Vector("fem", "gurkor", "är", "fler", "än", "fyra", "tomater")
scala> xs.sliding(2).toVector
scala> xs.sliding(3).toVector
scala> xs.sliding(4).toVector
scala> xs.sliding(7).toVector
scala> xs.sliding(10).toVector
\end{REPL}

\clearpage

\ExtraTasks %%%%%%%%%%%%%%%%%%%


\Task Skriv nedan program med en editor och kompilera från terminalen. Lägg till kod i huvudprogrammet som testar klassen \code{Account} och kompilera och kör. Utvidga sedan klassen \code{Account} med fler attribut och funktioner som du väljer själv.

\begin{Code}
class Account(val number: Long, val maxCredit: Int){ 
  private var balance = 0
  
  def deposit(amount: Int): Int = { 
    if (amount > 0) {balance += amount}
    balance
  }
  
  def withdraw(amount: Int): (Int, Int) = if (amount > 0) { 
    val allowedWithdrawal = 
      if (amount < balance + maxCredit) amount 
      else balance + maxCredit 
    balance = balance - allowedWithdrawal
    (allowedWithdrawal, balance)
  } else (0, balance)
  
  def show: Unit = 
    println("Account Nbr: " + number + " balance: " + balance) 
}

object Main {
  def main(args: Array[String]): Unit = {
    ???
  }
}
\end{Code}



\Task \label{task:keno-set} Läs om reglerna för spelet Keno här: \\ \url{https://sv.wikipedia.org/wiki/Keno} och gör deluppgifterna nedan. 

\Subtask Skapa en klass \code{Keno} som kan användas för att genomföra en Kenodragning. Låt klassen ha ett privat attribut \code{balls} som är en föränderlig mängd med heltal och som från början är tom. Implementera lämpliga metoder i klassen för att användaren av klassen ska kunna dra nya slumpmässiga bollar som inte redan är dragna.  

\Subtask Skapa en \code{case class KenoBet(bet: Set[Int])} för att hålla reda vilka 11 bollar en viss person satsar på. Definiera en metod \\ \code{def numberOfHits(keno: Keno): Int = ???}\\ i case-klassen \code{KenoBet} som givet en kenodragning räknar ut hur många bollar som satsats rätt. 

\Subtask Skriv ett huvudprogram som simulerar en enkel Kenodragning. Låt två personer satsa på 11 slumpmässiga bollar, genomför en dragning av 20 bollar ur 70 möjliga och kontrollera sedan hur många bollar som personerna har prickat rätt.




\AdvancedTasks %%%%%%%%%%%%%%%%%

\Task \emph{Dokumentationen för \code{Any}.} Undersök vilka metoder som finns i klassen Any här: \href{http://www.scala-lang.org/api/current/\#scala.Any}{http://www.scala-lang.org/api/current/\#scala.Any}. Prova några av metoderna i REPL.

\Task \emph{Dokumentationen för samlingar.} Leta upp metoden \code{tabulate} i dokumentationen för objektet \code{Vector} nästan längst ner i listan här: \\ \href{http://www.scala-lang.org/api/current/#scala.collection.immutable.Vector$}{http://www.scala-lang.org/api/current/\#scala.collection.immutable.Vector\$} \\Leta upp den variant av \code{tabulate} som har signaturen:\\ \code{def tabulate[A](n: Int)(f: (Int) => A): Vector[A] }\\ Klicka på den gråfyllda trekanten till vänster om signaturen som fäller ut beskrivningen

\Subtask Förklara vad som händer här:
\begin{REPLnonum}
scala> Vector.tabulate(10)(i => i % 3)
\end{REPLnonum}

\Subtask Klicka på det blåa stora o-et överst på sidan, för att växla till klass-vyn och studera listan med alla metoder  i klassen \code{Vector}. 


\Task \emph{Fler metoder på indexerbara sekvenser.} Deklarera följande vektorer i REPL. 
\begin{REPL}
scala> val xs = (1 to 10).toVector
scala> val a = Vector("abra", "ka", "dabra")
scala> val b = Vector( "sim", "sala", "bim", "sala", "bim")
\end{REPL}
Undersök i REPL vad som händer nedan. Alla dessa metoder fungerar på alla samlingar som är indexerbara sekvenser. Vad har uttrycken för värde och typ? Förklara vad metoden gör. Studera även denna  översikt: \href{http://docs.scala-lang.org/overviews/collections/seqs}{docs.scala-lang.org/overviews/collections/seqs}

\Subtask \code{b.indexWhere(s => s.startsWith("b"))}  % advanced

\Subtask \code{a.indices}  % advanced

\Subtask \code{xs.patch(1, Vector(42,43,44), 7)} % advanced

\Subtask \code{xs.segmentLength(_ < 8, 2)} % advanced

\Subtask \code{b.sortBy(_.reverse)}  % advanced

\Subtask \code{b.sortWith((s1, s2) => s1.size < s2.size)} % advanced

\Subtask \code{a.reverseMap(_.size)}	% advanced

\Subtask \code{a intersect Vector("ka", "boom", "pow")} % advanced

\Subtask \code{a diff Vector("ka")} % advanced

\Subtask \code{a union Vector("ka", "boom", "pow")} % advanced



\Task Jämför tidsprestanda mellan List och Vector vid hantering i början och i slutet. 

\Subtask Hur snabbt går nedan på din dator? (Exemplet nedan är exekverat på en Intel i7-4790K CPU @ 4.00GHz.) %sudo lshw -class processor

\begin{REPLnonum}
$scala
Welcome to Scala 2.11.8 (Java HotSpot(TM) 64-Bit Server VM, Java 1.8.0_66).
Type in expressions for evaluation. Or try :help.

scala> :pa
// Entering paste mode (ctrl-D to finish)

def time(n: Int)(block: => Unit): Double =  {
  def now = System.nanoTime
  var timestamp = now
  var sum = 0L
  var i = 0
  while (i < n) {
    block
    sum = sum + (now - timestamp)
    timestamp = now
    i = i + 1
  }
  val average = sum.toDouble / n
  println("Average time: " + average + " ns")
  average
}

// Exiting paste mode, now interpreting.

time: (n: Int)(block: => Unit)Double

scala> val n = 100000
scala> val l = List.fill(n)(math.random)
scala> val v = Vector.fill(n)(math.random)

scala> (for(i <- 1 to 20) yield time(n){l.take(10)}).min
Average time: 476.85004 ns
Average time: 52.29291 ns
Average time: 221.50289 ns
Average time: 218.60302 ns
Average time: 45.01888 ns
Average time: 243.7818 ns
Average time: 45.02228 ns
Average time: 2132.03995 ns
Average time: 52.83995 ns
Average time: 46.7478 ns
Average time: 51.8753 ns
Average time: 70.57658 ns
Average time: 45.26142 ns
Average time: 95.16307 ns
Average time: 43.84092 ns
Average time: 55.24695 ns
Average time: 84.06113 ns
Average time: 42.04872 ns
Average time: 50.9871 ns
Average time: 122.80649 ns
res0: Double = 42.04872


scala> (for(i <- 1 to 20) yield time(n){v.take(10)}).min
Average time: 429.23201 ns
Average time: 257.70543 ns
Average time: 271.02261 ns
Average time: 198.8826 ns
Average time: 161.67466 ns
Average time: 190.5253 ns
Average time: 112.82044 ns
Average time: 82.45798 ns
Average time: 81.17192 ns
Average time: 129.28968 ns
Average time: 104.86973 ns
Average time: 80.33942 ns
Average time: 81.64533 ns
Average time: 92.22053 ns
Average time: 78.5791 ns
Average time: 84.55555 ns
Average time: 78.88382 ns
Average time: 77.25284 ns
Average time: 83.62473 ns
Average time: 72.39703 ns
res1: Double = 72.39703

scala> (for(i <- 1 to 20) yield time(1000){l.takeRight(10)}).min
Average time: 264902.261 ns
Average time: 225706.676 ns
Average time: 228625.873 ns
Average time: 230358.379 ns
Average time: 229971.679 ns
Average time: 237404.948 ns
Average time: 242580.96 ns
Average time: 242455.325 ns
Average time: 242316.002 ns
Average time: 242046.311 ns
Average time: 242378.896 ns
Average time: 242740.221 ns
Average time: 242131.301 ns
Average time: 242466.169 ns
Average time: 242075.599 ns
Average time: 242247.534 ns
Average time: 242739.886 ns
Average time: 241982.93 ns
Average time: 242118.373 ns
Average time: 241941.998 ns
res2: Double = 225706.676

scala> (for(i <- 1 to 20) yield time(1000){v.takeRight(10)}).min
Average time: 661.737 ns
Average time: 420.765 ns
Average time: 225.867 ns
Average time: 256.524 ns
Average time: 235.596 ns
Average time: 231.764 ns
Average time: 154.279 ns
Average time: 139.37 ns
Average time: 139.183 ns
Average time: 153.957 ns
Average time: 142.883 ns
Average time: 140.837 ns
Average time: 154.178 ns
Average time: 138.72 ns
Average time: 202.93 ns
Average time: 174.179 ns
Average time: 175.98 ns
Average time: 171.658 ns
Average time: 177.097 ns
Average time: 173.1 ns
res3: Double = 138.72

\end{REPLnonum}

\Subtask Varför går det olika snabbt olika körningar?

\Task Studera skillnader i prestanda mellan olika samlingar här: \\ \href{http://docs.scala-lang.org/overviews/collections/performance-characteristics.html}{docs.scala-lang.org/overviews/collections/performance-characteristics.html} \\
(Mer om detta i kommande kurser.)

\Task För samlingen \code{List} finns en alternativ metod till \code{+:} som heter \code{::} och kallas ''cons'' och som i kombination med objektet \code{Nil} kan användas för att med alternativ syntax bygga listor. Läs om detta här: \\ \href{http://alvinalexander.com/scala/how-create-scala-list-range-fill-tabulate-constructors}{alvinalexander.com/scala/how-create-scala-list-range-fill-tabulate-constructors} \\ och hitta på några egna övningar för att undersöka hur cons och Nil fungerar. Metoder som slutar med kolon är högerassociativa. Läs mer om detta här: \href{http://www.artima.com/pins1ed/basic-types-and-operations.html#5.8}{http://www.artima.com/pins1ed/basic-types-and-operations.html\#5.8}
