%!TEX encoding = UTF-8 Unicode
%!TEX root = ../compendium.tex

\Assignment{tictactoe}
I detta projektet ska du implementera din egen version av spelet tic-tac-toe (eller som vi på svenska kallar det, tre i rad)! Du kommer börja med att implementera en version där du kan spela mot en kursare och sen gå vidare till att implementera en datorspelare som lägger sin pjäs slumpmässigt och till slut en som inte kan förlora!

\subsection{Regler}
Om du känner dig säker på hur reglerna i tic-tac-toe funkar kan du skippa detta. 
\begin{itemize}
	\item Spelplanen består av ett rutnät av storlek 3x3.
	\item Det finns två spelare: \texttt{x} och \texttt{o}.
	\item Spelarna placerar ut en pjäs var i växlande ordning där \texttt{x} börjar.
	\item Spelet tar slut om en spelare har fått antingen en rad, diagonal eller kolumn ifylld av sin spelpjäs eller om spelplanen är fylld.
\end{itemize}
\textit{Notera att pjäserna INTE får flyttas när de väl ligger på spelplanen.}

\subsection{Teori}
Representationen är vald till en endimensionell vektor av typen Int av storlek 9 där element $[0,2]$(note: intervall mellan 0 och 2) representerar den första raden $[3,5]$ andra och $[6,8]$ den tredje. Anledningen till detta är att vi vill ha en representation så att spelaren kan svara vilket drag den vill göra med ett heltal.
Varje element i vektorn ska kunna representera en tom plats, en plats allokerad av \texttt{x} och en plats allokerad av \texttt{o}. Detta innebär att en vektor av typen Boolean inte räcker till. Istället väjs den (kanske lite minnesöverflödiga) typen Int. Smidigast(note: detta kommer visa sig senare) är att välja representationen där 0 representerar tom plats, 1 representerar \texttt{x} och -1 representerar \texttt{o}. 
 
\subsection{Obligatoriska uppgifter}

\Task Implementera ett fungerande Spel.

\Subtask Implementera funktionen gameWon.

\Subtask Implementera en Human player.

\Subtask Implementera första version av Game.

\Task Randomized player

\Subtask Implementera en spelare som väljer ett slumpmässigt giltigt drag.

\Subtask Ändra Game så att användaren tillåts stänga av ritfunktionen och tillåts spela mågna spel.

\Subtask Vad är sannolikheterna för att \texttt{x} vinner, \texttt{o} vinner och att det blir oavgjort om två randomized players spelar mot varandra?

Hamnar man i närheten av dessa resultat tror vi på er randomized player.
\begin{itemize}
	\item P(\texttt{x} vinner) = 0.586
	\item P(\texttt{o} vinner) = 0.288
	\item P(lika) = 0.126
\end{itemize}


\Subtask Varför är det större sannolikhet för \texttt{x} att vinna än \texttt{o}?

\Task Optimal Player

\Subtask Läs igenom eval-funktionen och Appendix om max-min-evaluering.

\Subtask Implementera Optimal Players move-funktion.

\Subtask testa att spela mot din Optimal player med en human player, kan du spela lika? Kan du vinna?

\Subtask Vad händer om du sätter en random player mot Optimal player? Blir det någonsin oavgjort, hur ofta?

\subsection{Frivilliga extrauppgifter}

\Task Hashning.

\Subtask En underuppgift.

\Subtask En underuppgift.