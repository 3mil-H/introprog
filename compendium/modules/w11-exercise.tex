%!TEX encoding = UTF-8 Unicode

%!TEX root = ../compendium.tex

\Exercise{\ExeWeekELEVEN}\label{exe:W11}

\begin{Goals}
\item \TODO
\end{Goals}

\begin{Preparations}
\item \StudyTheory{11} 
\end{Preparations}

\BasicTasks %%%%%%%%%%%%%%%%

\Task \emph{Grundläggande skillnader och likheter mellan Scala och Java.} 

\Subtask Översätt nedan kod från Java till Scala. Skriv Scala-koden i en fil som heter \texttt{hangman.scala} och kalla Scala-objektet med \code{main}-metoden för \code{hangman}. 

\javainputlisting[numbers=left]{examples/scalajava/Hangman.java}


\Subtask Översätt nedan kod från Scala till Java.
\TODO
\begin{Code}

\end{Code}




\Task \emph{Översätta mellan klasser i Scala och klasser i Java.}  

\Subtask Översätt nedan klass från Scala till Java. Klassen är en model av en punkt som kan sparas på begäran i en lista. Listan är privat för kompanjonsobjektet men kan skrivas ut med en metod \code{showSaved}. 

I framtiden vill man kanske ändra från \code{ArrayBuffer} till en annan sekvenssamlingsimplementation, t.ex. \code{ListBuffer}, som uppfyller egenskaperna hos supertypen \code{Buffer} men har andra prestandaegenskaper för olika operationer, varför attributet \code{saved} i kompanjonsobjektet är deklarerat med den mer generella typen.

Skriv koden för respektive klass i en editor i två filer, en för Scala och en för Java. Vi ska i nästa deluppgift kompilera båda programmen i terminalen och testa att klasserna har motsvarande funktionalitet. 

\scalainputlisting[numbers=left]{examples/scalajava/Point.scala}

\emph{Tips och riktlinjer:}
\begin{itemize}[nolistsep, noitemsep]
\item För att namnen inte ska krocka i våra kommande tester, kalla Javatypen för \code{JPoint}. 
\item  I ställert för Scalas \code{ArrayBuffer} och \code{Buffer}, använd Javas \code{ArrayList} och \code{List} som båda ligger i paketet \code{java.util}. 
\item Undersök dokumentationen för \code{java.util.List} för att hitta en motsvarighet till \code{prepend} för att lägga till i början av listan.
\item I stället för default-argumentet i Scalas primärkonstruktor, använd en extra Java-konstruktor. 
\item Det finns inga singelobjekt och inga kompanjonsobjekt i Java; istället kan man använda statiska klassmedlemmar. Placera Scala-kompanjonsobjektets medlemmars motsvarigheter \emph{innuti} Java-klassen och gör dem till \jcode{static}-medlemmar.
\item Kod i klasskroppen i Scalaklassen, så som if-satsen på rad 4, placeras i lämplig konstruktor i Javaklassen.
\item Utskrifter med \code{print} och \code{println} behöver i Java föregås av \code{System.out.}
\item Det finns inget nyckelord \code{override} i Java, men en s.k. annotering som ger samma kompilatorhjälp. Den skrivs med ett snabel-a och stor begynnelsebokstav, så här: \jcode{ @Override }  före metoddeklarationen.
\item I Java används konventionen att börja getter-metoder med ordet \code{get}, t.ex. \code{getX()}.
\item Det finns ingen motsvarighet till \code{mkString} för \code{List} så du behöver själv gå igenom listan och hämta elementreferenser för utskrift med en \jcode{for}-loop. Notera att efter sista elementet ska radbrytning göras i utskriften och att inget komma ska skrivas ut efter sista elementet.
\item I Java behövs en ny \jcode{import}-deklaration om man vill importera ännu en typ från samma paket.
\item Metoder i Java slutar med \code{()} om de saknar parametrar. 
\item Alla satser slutar med lättglömda semikolon.
\end{itemize}


\TODO Flytta nedan kod till facit:
\javainputlisting[numbers=left]{examples/scalajava/JPoint.java}


\Subtask Starta REPL i samma bibliotek som du kompilerat kodfilerna. Testa så att klasserna \code{Point} och \code{JPoint} beter sig på samma vis enligt nedan. Skriv även testkod i REPL för att avläsa de attributvärden som har getters och undersök att allt funkar som det ska.
\begin{REPLnonum}
$ scalac Point.scala
$ javac JPoint.java
$ scala
scala> val (p, jp) = (new Point, new JPoint)
scala> p.distanceTo(new Point(3, 4))
scala> Point.showSaved
scala> jp.distanceTo(new JPoint(3, 4))
scala> JPoint.showSaved
scala> for (i <- 1 to 10) { new Point(i, i, true) }
scala> Point.showSaved
scala> for (i <- 1 to 10) { new JPoint(i, i, true) }
scala> JPoint.showSaved
\end{REPLnonum}


\Subtask Översätt nedan klass \code{JPerson} från Java till en case-klass \code{Person} i Scala med  motsvarande funktionalitet. 


\javainputlisting[numbers=left]{examples/scalajava/JPerson.java}


\TODO flytta dena kod till FACIT:
\begin{Code}
case class Person(name: String, age: Int = 0)
\end{Code}

\Subtask\Pen Undersök i REPL vilken funktionalitet i Scala-case-klassen \code{Person} som \emph{inte} är implementerad i Java-klassen \code{JPerson} ovan. Skriv upp namnen på några av case-klassens extra metoder samt deras signatur genom att för en \code{Person}-instans, och för kompansjonsobjektet \code{Person}, trycka på TAB-tangenten. Prova några av de extra metoderna i REPL och förklara vad de gör.

\begin{REPL}
scala> val p = Person("Björn", 49)
scala> p.      // tryck TAB en gång
scala> Person. // tryck TAB en gång
scala> p.copy  // tryck TAB en gång
scala> p.copy()
scala> p.copy(age = p.age + 1)
scala> Person.unapply(p)
\end{REPL}


\Task \emph{Auto(un)boxing i JVM.} I JVM måste typparametern för generiska klasser vara av referenstyp. I Scala löser kompilatorn detta åt oss så att vi ändå kan ha t.ex. \code{Int} som argument till en typparameter i Scala, medan man i Java \emph{inte} direkt kan ha den primitiva typen \jcode{int} som typparameter till t.ex. \code{ArrayList}.

I Java och i den underliggande plattformen JVM används s.k. wrapper-klasser för att lösa detta, t.ex. genom wrapper-klassen \code{Integer} som boxar den primitiva typen \jcode{int}. Java-kompilatorn har stöd för att automatiskt packa in värden av primitiv typ i sådana wrapper-klasser för att skapa referenstyper och kan även automatiskt packa upp dem. 

\Subtask Studera hur Scala-kompilatorn låter oss arbeta med en \code{Cell[Int]} även om det underliggande JVM:ens körtidstyp \Eng{runtime type} är en wrapper-klass. Man kan se JVM-körtidstypen med metoderna \code{getClass} och \code{getTypeName} enligt nedan.
\begin{REPL}
scala> class Cell[T](var value: T){
         val typeName: String = value.getClass.getTypeName
         override def toString = "Cell[" + typeName + "](" + value + ")"
       }
scala> val c = new Cell[Int](42)
scala> c.value.getClass.getTypeName
\end{REPL}

\Subtask Vad är körtidstypen för \code{c.value} ovan? Förklara hur det kan komma sig trots att vi deklarerade med typargumentet \code{Int}? 


\Subtask \TODO auto(un)boxing in Java och fallgropar

\ExtraTasks %%%%%%%%%%%%%%%%%%%

\Subtask Översätt nedan kod från Java till Scala. Skriv koden i en fil som heter \texttt{showInt.scala} och kalla Scala-objektet med \code{main}-metoden för \code{showInt}.

\emph{Tips:}
\begin{itemize}[nolistsep, noitemsep]
\item En Javaklass med bara statiska medlemmar motsvaras av ett singeltonobjekt i Scala, alltså en \code{object}-deklaration. Scala har därför inte nyckelordet \jcode{static}.
\item Typen \jcode{Object} i Java motsvaras av Scalas \code{Any}.
\item Du kan använda Scalas möjlighet med default-argument (som saknas i Java) för att bara definiera en enda \code{show}-metod med en tom sträng som default \code{msg}-argument.
\item I Scala har objekt av typen \code{Char} en metod \code{def *(n: Int): String} som skapar en sträng med tecknet repeterat \code{n} gånger. Men du kan ju välja att ändå implementera metoden \code{repeatChar} med \code{StringBuilder} som nedan om du vill träna på att översätta en \code{for}-loop från Java till Scala.
\item I stället för \code{Scanner.nextLine} kan du i Scala använda \code{scala.io.StdIn.readLine} som tar en prompt som parameter, men du kan också använda \code{Scanner} i Scala om du vill träna på det.
\item I Java \emph{måste} man använda nyckelordet \jcode{return} om metoden inte är en \jcode{void}-metod, medan man i Scala faktiskt \emph{får} använda \code{return} även om man brukar undvika det och utnyttja att satser i Scala också är uttryck.
\end{itemize}
\javainputlisting[numbers=left]{examples/scalajava/JShowInt.java}

\Subtask Kompilera din Scala-kod och kör i terminalen och testa så att allt funkar. Vill du även kompilera Java-koden finns den i kursens repo i biblioteket \texttt{compendium/examples/scalajava}.  


\AdvancedTasks %%%%%%%%%%%%%%%%%

\TODO 
\Task \emph{Gränssnitt i Scala och Java.} 

\Task\Pen Studera fallgropar för hur man skriver en \code{equals}-metod i Java här:
\url{http://www.artima.com/lejava/articles/equality.html}\\
Vilken fallgrop trillar man \emph{inte} i om man endast överskuggar \code{equals} i finala klasser som inte har några superklasser?

\Task\Pen Studera det fullständiga receptet för hur man skriver en välfungerande \code{equals} och \code{hashcode} i Scala här: \url{http://www.artima.com/pins1ed/object-equality.html} 


    