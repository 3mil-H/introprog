%!TEX encoding = UTF-8 Unicode

%!TEX root = ../compendium.tex

\Lab{\LabWeekTWELVE}

\begin{Goals}
\item Att lära sig om hur man separerar beteende från vy med hjälp av Model-View uppdelningen.
\item Att lära sig om grundläggande cellulära automata \eng{cellular automata}.
\item Att lära sig om trådar, dvs. hur man gör för att köra kod \emph{"samtidigt"} som annan kod.
\item Att lära sig om hur man kan använda matriser för att lösa problem.
\item Att lära sig...
\item Att lära sig...
\end{Goals}

\begin{Preparations}
\item Att göra.
\end{Preparations}

\subsection{Obligatoriska uppgifter}

% Gör denna tasken till en preparation?
\Task Skapa en model som kan visas i vyn.

\Subtask En underuppgift.

\Subtask En underuppgift.


\Task Implementera modellens beteende enligt reglerna för Life.

\Subtask En underuppgift.

\Subtask En underuppgift.


\subsection{Frivilliga extrauppgifter}

\Task Implementera andra regler för cellulära automata.

    Det finns massor med regler för cellulära automata med sina egna intressanta beteenden och tillstånd.
    Gör den eller de du tycker verkar mest intressant!

    Fler regler kan finnas här: \url{https://en.wikipedia.org/wiki/Category:Cellular_automaton_rules}

    Nedan följer några roliga exempel som valts ut och anses lämpliga.

    \Subtask Implementera cyklisk cellulär automata.

        Denna typ av automata kallar cyklisk just för att det finns $N$ möjliga tillstånd och när tillståndet N nås så är "nästa" tillstånd $0$.

        Regeln för att en cell byter tillstånd ges av att om en granne har tillståndet exakt ett över cellens tillstånd så får cellen sin grannes tillstånd.

        För att få intressant beteende brukar man initialisera hela brädet så att varje cell får ett slumpvalt tillstånd.

        \url{https://en.wikipedia.org/wiki/Cyclic_cellular_automaton}

    \Subtask Implementera Wireworld.

        Wireworld är ett lite annorlunda då man i Wireworld designar "kretsar" inte helt olika de som finns i moderna datorer.

        I Wireworld kan man skapa komponenter som fungerar som dioder samt transistorer, och med dessa bygga logiska grindar.

        \url{https://en.wikipedia.org/wiki/Wireworld}


\Task En labbuppgiftsbeskrivning.

    \Subtask En underuppgift.

    \Subtask En underuppgift.


\Task Implementera spara och ladda.

    \Subtask Spara brädets tillstånd till ett format som kan både exporteras/sparas och importeras/laddas.

    \Subtask Ladda in det exporterade tillståndet.



\Task Alternativ vy: Kör programmet i webbläsaren med Scala.js

% Tidsuppskattning fungerar nog inte, varierar stort
%Tidsuppskattning: ?h

\Task Alternativ vy: Kör programmet på Android


