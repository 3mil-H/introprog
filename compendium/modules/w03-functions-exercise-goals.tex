%!TEX encoding = UTF-8 Unicode
%!TEX root = ../exercises.tex

\item Kunna skapa och använda funktioner med en eller flera parametrar, default-argument, och namngivna argument.
\item Kunna förklara nästlade funktionsanrop med aktiveringsposter på stacken.
\item Kunna förklara skillnaden mellan äkta och ''oäkta'' funktioner.
\item Kunna applicera en funktion på alla element i en samling.

\item Kunna använda funktioner som äkta värden.
\item Kunna skapa och använda anonyma funktioner (ä.k. lambda-funktioner).

\item Känna till att funktioner kan ha uppdelad parameterlista.
\item Känna till att det går att partiellt applicera argument på funktioner med uppdelad parameterlista för att skapa s.k. stegade funktioner (ä.k. curry-funktioner).

\item Känna till rekursion och kunna beskriva vad som kännetecknar en rekursiv funktion.
%\item Känna till att man kan loopa med rekursion och att svansrekursiva funktioner kan optimeras till while-loopar.

\item Känna till att det går att skapa egna kontrollstrukturer med hjälp av namnanrop.
\item Känna till skillnaden mellan värdeanrop och namnanrop.
\item Kunna tolka en stack trace.
