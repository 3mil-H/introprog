%!TEX encoding = UTF-8 Unicode
%!TEX root = ../exercises.tex

\item Kunna skapa och använda funktioner med en eller flera parametrar, default-argument, namngivna argument, och uppdelad parameterlista.
\item Kunna använda funktioner som äkta värden.
\item Kunna skapa och använda anonyma funktioner (s.k. lambda-funktioner).
\item Kunna applicera en funktion på element i en samling.
\item Förstå skillnader och likheter mellan en funktion och en procedur.
\item Förstå skillnader och likheter mellan en värde-anrop och namnanrop.
\item Kunna skapa en procedur i form av en enkel kontrollstruktur med fördröjd evaluering av ett block.
\item Förstå skillnaden mellan äkta funktioner och funktioner med sidoeffekter.
\item Kunna skapa och använda variabler med fördröjd initialisering och förstå när de är användbara.
\item Kunna förklara hur nästlade funktionsanrop fungerar med hjälp av begreppet aktiveringspost.
\item Kunna skapa och använda lokala funktioner, samt förstå nyttan med lokala funktioner.
\item Känna till stegade funktioner och kunna använda partiellt applicerade argument.
\item Känna till rekursion och kunna förklara hur rekursiva funktioner fungerar.

%\item Känna till svansrekursion och att svansrekursiva funktioner kan optimeras till loopar.
