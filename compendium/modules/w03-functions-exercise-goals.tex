%!TEX encoding = UTF-8 Unicode
%!TEX root = ../exercises.tex

\item Kunna skapa och använda funktioner med en eller flera parametrar, default-argument, namngivna argument, och uppdelad parameterlista.
\item Kunna använda funktioner som äkta värden.
\item Kunna skapa och använda anonyma funktioner (ä.k. lambda-funktioner).
\item Kunna applicera en funktion på element i en samling.
\item Förstå skillnader och likheter mellan en funktion och en procedur.
\item Förstå vad ett block och en lokal variabel är.
\item Kunna skapa och använda lokala funktioner och förklara nyttan med dessa.
\item Förstå skillnader och likheter mellan värdeanrop och namnanrop.
\item Kunna skapa en enkel kontrollstruktur med fördröjd evaluering av ett block.
\item Förstå skillnaden mellan äkta funktioner och funktioner med sidoeffekter.
%\item Kunna skapa och använda variabler med fördröjd initialisering och förstå när de är användbara.
\item Kunna förklara hur nästlade funktionsanrop sker med   aktiveringsposter.
\item Känna till rekursion och kunna förklara hur rekursiva funktioner fungerar.
\item Känna till att det går att partiellt applicera argument på funktioner med uppdelad parameterlista för att skapa s.k. stegade funktioner (ä.k. curry-funktioner).

%\item Känna till svansrekursion och att svansrekursiva funktioner kan optimeras till loopar.
