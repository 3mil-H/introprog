%!TEX encoding = UTF-8 Unicode
%!TEX root = ../compendium.tex

\ExerciseSolution{\ExeWeekONE}

\BasicTasks %%%%%%%%%%%

\Task % uppgift 1

\Subtask REPL skriver ut \code{"hejsan REPL"}

\Subtask Man får fortsätta på nästa rad

\Subtask Värde: gurkatomat. Typ:  \code{String}

\Subtask Värde: gurkatomat upprepat 10 gånger. Typ: \code{String}

% uppgift 2
\Task Bestämda värden man skriver i koden. T.ex.  \code{1},  \code{"hej"},  \code{3.0f}
 
\Task % uppgift 3

\Subtask  \code{Int}

\Subtask  \code{Long}

\Subtask  \code{char}

\Subtask  \code{String}

\Subtask  \code{Double}

\Subtask  \code{Double}

\Subtask  \code{Double}

\Subtask  \code{Float}

\Subtask  \code{Float}

\Subtask  \code{Boolean}

\Subtask  \code{Boolean}

% uppgift 4 
\Task REPLn skriver ut 
\begin{REPLnonum}
hejsan
42
gurka
\end{REPLnonum}
klammer definierar funktionen och semikolon tilåter en att skriva flera satser på samma rad

\Task % uppgift 5

\Subtask  utryck är evaluering utav matematiska operationer. satser är kommandon

\Subtask \code{println()}

\Subtask 

 Värdesaknas inehåller Unit

 Skriver ut \code{Unit}

 Skriver ut \code{"()"}

 Skriver ut \code{"()"}

 Skriver först ut hej med det innersta anropet och sen \code{()} med det yttre anropet

\Subtask  \code{Unit}

\Subtask  \code{Unit}

\Task % uppgift 6

\Subtask  \code{Int, 42}

\Subtask \code{Float,19}

\Subtask \code{Double,42}

\Subtask \code{Double,42}

\Subtask \code{Float,1.042E42}

\Subtask \code{Long, 12E6}

\Subtask \code{String, gurka}

\Subtask \code{Char, 'A'}

\Subtask \code{Int,65}

\Subtask \code{Int,48}

\Subtask \code{Int,49}

\Subtask \code{Int,57}

\Subtask \code{Int, 113}

\Subtask \code{ Char, 'q'}

\Subtask \code{ Char, '*'}

\Task % uppgift 7

\Subtask \code{Int, 84}

\Subtask \code{Float, 21}

\Subtask \code{Float, 41.8}

\Subtask \code{Double, 12}

\Task % uppgift 8

\Subtask \code{Int,27}

\Subtask \code{Int,50}

\Subtask \code{Double, 13.3}

\Subtask \code{Int, 13}

\Task % uppgift 9

\Subtask  \code{Int, 21}

\Subtask  \code{Int, 10}

\Subtask \code{Float,10.5}

\Subtask \code{Int, 0}

\Subtask \code{Int, 1}

\Subtask \code{Int,3}

\Subtask \code{Int, 0}


\Task % uppgift 10

\Subtask 127,-128

\Subtask 32767, -32768

\Subtask 2147483647,-2147483648

\Subtask 9223372036854775807,-9223372036854775808

\Task % uppgift 11

\Subtask 
java: PI scala: Pi

\Subtask andvänder sig utav pythagoras sats

\Subtask \code{scalb()}

\Task % uppgift 12

\Subtask den blir \code{Int.MinValue}

\Subtask kastar exeption

\Subtask \code{1.0000000000000001E8}

\Subtask avrundas till \code{1E8}

\Subtask \code{45.00000000000001}

\Subtask returnerar en double som är oändlig

\Subtask \code{Int.MaxValue}

\Subtask \code{NaN}

\Subtask \code{NaN}

\Subtask Man kastar ett nytt exception.

\Task % uppgift 13

\Subtask \code{true}

\Subtask \code{false}

\Subtask \code{false}

\Subtask \code{false}

\Subtask \code{true}

\Subtask \code{true}

\Subtask \code{true}

\Subtask \code{false}

\Subtask \code{true}

\Subtask \code{false}

\Subtask \code{false}

\Subtask \code{true}

\Subtask \code{true}

\Subtask \code{false}

\Subtask \code{true}

\Subtask \code{true}

\Subtask \code{true}

\Subtask \code{false}

\Subtask \code{true}

\Subtask \code{false}

\Subtask \code{true}

\Subtask \code{true}

\Task % uppgift 14

\code{a = 13}

\code{b = 14}

\code{c = Double 54}

\code{b = 0}

\code{a = 0}

\code{c = Double 55}


\Task % uppgift 15

\Subtask 

x blir 30

x blir 31

skriv ut x

x = 32

skriv ut x

false

constant värde y blir 20

fungerar ej

skriv ut gurka och z blir 10

funktionen w blir det inom måsvingarna

skriv ut z

skriv ut z

z blir 11

anropa w

anropa w

fungerar ej

\Subtask  Rad 8 och 16. y är konstant och kan ej modifieras. kan ej modifiera en funktion

\Subtask 

var är en modifierbar variabel

val är ett konstant värde

def är en funktion

\Task % uppgift 16

a blir 40

b blir 80

a blir 50

b blir 70

a blir 160

b blir 35

\Task % uppgift 17

\Subtask Namnet Kim Robinsson har 12 bokstäver.

\Subtask

\begin{REPLnonum}
val fTot = f.size
val eTot = e.size
println(s"$f har  $fTot bokstäver.")
println(s"$e har  $eTot bokstäver.")
\end{REPLnonum}

\Task % uppgift 18

skriver ut sant eftersom den hoppar över andra kod blocket

skriver ut falsk för den hoppar över det första kodblocket

skriver ut sant eftersom den hoppar över andra kod blocket

skriver ut falsk för den hoppar över det första kodblocket

definerar en funktion som 50% utav gångerna skriver ut krona, hälften klave

kastar kronan tre gånger

\Task % uppgift 19

\Subtask \code{String}, inte gott

\Subtask \code{String}, gott

\Subtask \code{String}, likastora

\Subtask \code{String}, gurka

\Subtask \code{String}, banan

\Task % uppgift 20

\Subtask 

1, 2, 3, 4, 5, 6, 7, 8, 9, 10,

1, 2, 3, 4, 5, 6, 7, 8, 9,

2, 4, 6, 8, 10,

1, 11, 21, 31, 41, 51, 61, 71, 81, 91,

inget

\Subtask 

\begin{REPLnonum}
scala> for(i <- 1 to 43 by 3) print("A" + i + ", ")
\end{REPLnonum}

\Task % uppgift 21

\Subtask 

9, 10, 11, 12, 13, 14, 15, 16, 17, 18, 19,

1, 2, 3, 4, 5, 6, 7, 8, 9, 10, 11, 12, 13, 14, 15, 16, 17, 18, 19,

0, 3, 6, 9, 12, 15, 18, 21, 24, 27, 30, 33,

\Subtask 

B33, B30, B27, B24, B21, B18, B15, B12, B9, B6, B3, B0,

\Task % uppgift 22

\Subtask 

0 till 9
0, 2, 4, 6, 8, 10, 12

\Subtask 

\begin{REPLnonum}
var k = 0
while(k <= 43)
{
print("A" + k + ", ")
k = k + 3
}
\end{REPLnonum}

\Subtask \code{foreach}

\Task % uppgift 23

\Subtask  \code{Double}

\Subtask  0, less than 1.0

\Subtask  Nej

\Subtask Man får olika slumpmässiga tal

\Subtask \code{for for (i <- 1 to 100) println((math.random * 9).toInt)}

\Subtask \code{for (i <- 1 to 100) println((math.random * 5 + 1).toInt)}

\Subtask \code{for (i <- 1 to 100) println((math.random * 6).toInt + 1 )}

\Subtask  gurka skrivs ut olika antal gånger

\Subtask \code{while (math.random > 0.01) println("gurka")}

\Subtask  Samma sak som i dem förra fast man skriver ut slumptalet

\Task % uppgift 24

\Subtask \code{poäng > 1000}

\Subtask\code{poäng > 100}

\Subtask \code{poäng < highscore}

\Subtask\code{poäng < 0 || poäng > highscore }

\Subtask \code{poäng > 0 \&\& poäng < highscore}

\Subtask \code{klar}

\Subtask \code{!klar}




\ExtraTasks %%%%%%%%%%%%

\Task 

\Subtask 

\Subtask Lösningstext.


\AdvancedTasks %%%%%%%%%

\Task 

\Subtask \code{42}

\Subtask Lösningstext.
