%!TEX encoding = UTF-8 Unicode
%!TEX root = ../compendium.tex

\ExerciseSolution{\ExeWeekONE}

\BasicTasks %%%%%%%%%%%

\Task % uppgift 1

\Subtask REPL skriver ut "hejsan REPL"

\Subtask man får fortsätta på nästa rad

\Subtask Värde: gurkatomat. Typ: String

\Subtask värde: gurkatomat upprepat 42 gånger. Typ:String

\Task % uppgift 2
 
\Task % uppgift 3

\Subtask Int

\Subtask Long

\Subtask char

\Subtask String

\Subtask Double

\Subtask Double

\Subtask Double

\Subtask Float

\Subtask Float

\Subtask Boolean

\Subtask Boolean


\Task % uppgift 4 
 skriver ut hejsan42
gurka

 klammer definierar funktionen och semikolon tilåter en att skriva flera satser på samma rad

\Task % uppgift 5
\Subtask  utryck är evaluering utav matematiska operationer. satser är kommandon

\Subtask println()

\Subtask 

 värdesaknas inehåller Unit

 skriver ut Unit

 skriver ut "()"

 skriver ut "()"

 skriver först ut hej med det innersta anropet och sen () med det yttre anropet

\Subtask  Unit

\Subtask  Unit

\Task % uppgift 6

\Subtask  Int, 42

\Subtask Float,42

\Subtask Double,42

\Subtask Double,42

\Subtask Float,1.042E42

\Subtask Long, 42E6

\Subtask String, gurka

\Subtask Char, 'A'

\Subtask Int,65

\Subtask Int,48

\Subtask Int,49

\Subtask Int,57

\Subtask  Char, 'q'

\Subtask  Char, '*'

\Task % uppgift 7
\Subtask Iint, 84

\Subtask Float, 21

\Subtask Float, 41.8

\Subtask Double, 44

\Task % uppgift 8
\Subtask Int,46

\Subtask Int,88

\Subtask Double, 13.3

\Subtask Int, 13

\Task % uppgift 9
\Subtask  int, 21

\Subtask  int, 10

\Subtask float,10.5

\Subtask int, 0

\Subtask int, 1

\Subtask int,3

\Subtask int, 0


\Task % uppgift 10
\Subtask 127,-128

\Subtask 32767, -32768

\Subtask 2147483647,-2147483648

\Subtask 9223372036854775807,-9223372036854775808

\Task % uppgift 11
\Subtask 
java: PI scala: Pi

\Subtask andvänder sig utav pythagoras sats

\Subtask 

\Task % uppgift 12
\Subtask den blir Int.MinValue

\Subtask kastar exeption

\Subtask 1.0000000000000001E8

\Subtask avrundas till 1E8

\Subtask 45.00000000000001

\Subtask returnerar en double som är oändlig

\Subtask Int.MaxValue

\Subtask NaN

\Subtask NaN

\Subtask 

\Task % uppgift 13
\Subtask true

\Subtask false

\Subtask false

\Subtask false

\Subtask true

\Subtask true

\Subtask true

\Subtask false

\Subtask true

\Subtask false

\Subtask false

\Subtask true

\Subtask true

\Subtask false

\Subtask true

\Subtask true

\Subtask true

\Subtask false

\Subtask true

\Subtask false

\Subtask true

\Subtask true

\Task % uppgift 14
a = 42
b = 43
c = Double 170
b = 0
a = 0
c = Double 171

\Task % uppgift 15

\Subtask 

x blir 42
x blir 43
skriv ut x
x = 44
skriv ut x
false
constant värde y blir 42
fel
skriv ut gurka och z blir 42
funktionen w blir det inom måsvingarna
skriv ut z
skriv ut z
z blir 43
anropa w
anropa w
fel

\Subtask 

rad 8 och 16 
y är konstant och kan ej modifieras
kan ej modifiera en funktion

\Subtask 

var är en modifierbar variabel
val är ett konstant värde
def är en funktion

\Task % uppgift 16

a blir 42
b blir 84
a blir 52
b blir 74
a blir 104
b blir 37

\Task % uppgift 17

\Subtask Namnet Kim Robinsson har 12 bokstäver.

\Subtask

val fTot = f.size
val eTot = e.size
println(s"$f har  $fTot bokstäver.")
println(s"$e har  $eTot bokstäver.")

\Task % uppgift 18

skriver ut sant eftersom den hoppar över andra kod blocket
skriver ut falsk för den hoppar över det första kodblocket
skriver ut sant eftersom den hoppar över andra kod blocket
skriver ut falsk för den hoppar över det första kodblocket
definerar en funktion som 50% utav gångerna skriver ut krona, hälften klave
kastar kronan tre gånger

\Task % uppgift 19

\Subtask String, inte gott
\Subtask string, gott
\Subtask string, likastora
\Subtask string, gurka
\Subtask string, banan

\Task % uppgift 20

\Subtask 

1, 2, 3, 4, 5, 6, 7, 8, 9, 10,
1, 2, 3, 4, 5, 6, 7, 8, 9,
2, 4, 6, 8, 10,
1, 11, 21, 31, 41, 51, 61, 71, 81, 91,
inget

\Subtask 

scala> for(i <- 1 to 43 by 3) print("A" + i + ", ")

\Task % uppgift 21

\Subtask 

9, 10, 11, 12, 13, 14, 15, 16, 17, 18, 19,
1, 2, 3, 4, 5, 6, 7, 8, 9, 10, 11, 12, 13, 14, 15, 16, 17, 18, 19,
0, 3, 6, 9, 12, 15, 18, 21, 24, 27, 30, 33,

\Subtask 

B33, B30, B27, B24, B21, B18, B15, B12, B9, B6, B3, B0,

\Task % uppgift 22

\Subtask 

0 till 9
0, 2, 4, 6, 8, 10, 12

\Subtask 

var k = 0
while(k <= 43)
{
print("A" + k + ", ")
k = k + 3
}

\Subtask 

foreach

\Task % uppgift 23

\Subtask  Double

\Subtask  0, less than 1.0

\Subtask  

\Subtask man får olika slumpmässiga tal

\Subtask den skriver ut flera slumptal

\Subtask for for (i <- 1 to 100) println((math.random * 9).toInt)

\Subtask for (i <- 1 to 100) println((math.random * 5 + 1).toInt)

\Subtask  gurka skrivs ut olika antal gånger

\Subtask while (math.random > 0.01) println("gurka")

\Subtask  sama sak som i dem förra fast man skriver ut slumptalet

\Task % uppgift 24

\Subtask  poäng > 1000

\Subtask poäng > 100

\Subtask  poäng < highscore

\Subtask poäng < 0 || poäng > highscore 

\Subtask  poäng > 0 \&\& poäng < highscore

\Subtask  klar

\Subtask  !klar




\ExtraTasks %%%%%%%%%%%%

\Task 

\Subtask 

\Subtask Lösningstext.


\AdvancedTasks %%%%%%%%%

\Task 

\Subtask \code{42}

\Subtask Lösningstext.
