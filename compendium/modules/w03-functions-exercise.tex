%!TEX encoding = UTF-8 Unicode
%!TEX root = ../exercises.tex

\ifPreSolution

\Exercise{\ExeWeekTHREE}\label{exe:W03}
\begin{Goals}
%!TEX encoding = UTF-8 Unicode
%!TEX root = ../exercises.tex

\item Kunna skapa och använda funktioner med en eller flera parametrar, default-argument, och namngivna argument.
\item Kunna förklara nästlade funktionsanrop med aktiveringsposter på stacken.
\item Kunna förklara skillnaden mellan äkta och ''oäkta'' funktioner.
\item Kunna applicera en funktion på alla element i en samling.

\item Kunna använda funktioner som äkta värden.
\item Kunna skapa och använda anonyma funktioner (ä.k. lambda-funktioner).

\item Känna till att funktioner kan ha uppdelad parameterlista.
\item Känna till att det går att partiellt applicera argument på funktioner med uppdelad parameterlista för att skapa s.k. stegade funktioner (ä.k. curry-funktioner).

\item Känna till rekursion och kunna beskriva vad som kännetecknar en rekursiv funktion.
%\item Känna till att man kan loopa med rekursion och att svansrekursiva funktioner kan optimeras till while-loopar.

\item Känna till att det går att skapa egna kontrollstrukturer med hjälp av namnanrop.
\item Känna till skillnaden mellan värdeanrop och namnanrop.
\item Kunna tolka en stack trace.

\end{Goals}

\begin{Preparations}
\item \StudyTheory{03}
\end{Preparations}

\BasicTasks %%%%%%%%%%%%%%%%

\else

\ExerciseSolution{\ExeWeekTHREE}

\fi





\WHAT{Para ihop begrepp med beskrivning.}

\QUESTBEGIN

\Task \what~Koppla varje begrepp med den (förenklade) beskrivning som passar bäst:

\begin{ConceptConnections}
  funktionshuvud & 1 & & A & fördröjd evaluering av argument \\ 
  funktionskropp & 2 & & B & koden som exekveras vid funktionsanrop \\ 
  parameterlista & 3 & & C & funktion utan namn; kallas även lambda \\ 
  block & 4 & & D & gör att argument kan ges i valfri ordning \\ 
  namngivna argument & 5 & & E & en funktion som anropar sig själv \\ 
  defaultargument & 6 & & F & gör att en funktion kan flera resultatvärden \\ 
  värdeanrop & 7 & & G & har parameterlista och eventuellt returtyp \\ 
  namnanrop & 8 & & H & kan ha lokala namn; sista raden ger värdet \\ 
  tupel & 9 & & I & lista med bestämt antal (heterogena) värden \\ 
  tupelreturtyp & 10 & & J & argumentet evalueras innan anrop \\ 
  äkta funktion & 11 & & K & ger alltid samma resultat om samma argument \\ 
  slumptalsfrö & 12 & & L & beskriver namn och typ på parametrar \\ 
  anonym funktion & 13 & & M & gör att argument kan utelämnas \\ 
  rekursiv funktion & 14 & & N & om lika blir sekvensen av pseudoslumptal samma \\ 
\end{ConceptConnections}

\SOLUTION

\TaskSolved \what

\begin{ConceptConnections}
  funktionshuvud & 1 & ~~\Large$\leadsto$~~ &  M & har parameterlista och eventuellt en returtyp \\ 
  funktionskropp & 2 & ~~\Large$\leadsto$~~ &  L & koden som exekveras vid funktionsanrop \\ 
  parameterlista & 3 & ~~\Large$\leadsto$~~ &  I & beskriver namn och typ på parametrar \\ 
  block & 4 & ~~\Large$\leadsto$~~ &  E & kan ha lokala namn; sista raden ger värdet \\ 
  namngivna argument & 5 & ~~\Large$\leadsto$~~ &  J & gör att argument kan ges i valfri ordning \\ 
  defaultargument & 6 & ~~\Large$\leadsto$~~ &  F & gör att argument kan utelämnas \\ 
  värdeanrop & 7 & ~~\Large$\leadsto$~~ &  C & argumentet evalueras innan anrop \\ 
  namnanrop & 8 & ~~\Large$\leadsto$~~ &  K & fördröjd evaluering av argument \\ 
  äkta funktion & 9 & ~~\Large$\leadsto$~~ &  A & ger alltid samma resultat om samma argument \\ 
  predikat & 10 & ~~\Large$\leadsto$~~ &  G & en funktion som ger ett booleskt värde \\ 
  slumptalsfrö & 11 & ~~\Large$\leadsto$~~ &  B & ger återupprepningsbar sekvens av pseudoslumptal \\ 
  anonym funktion & 12 & ~~\Large$\leadsto$~~ &  D & funktion utan namn; kallas även lambda \\ 
  rekursiv funktion & 13 & ~~\Large$\leadsto$~~ &  H & en funktion som anropar sig själv \\ 
\end{ConceptConnections}

\QUESTEND





\WHAT{Definiera och anropa funktioner.}

\QUESTBEGIN

\Task \label{task:funcall} \what~
En funktion med en parameter definieras med följande syntax i Scala:
\vspace{0.5em} \\
\texttt{\code{def} \textit{namn}(\textit{parameter}: \textit{Typ} = \textit{defaultArgument}): \textit{Returtyp} = \textit{returvärde}}

% En funktion med två parametrar definieras med följande syntax i Scala: \vspace{0.5em} \\  \texttt{\code{def} \textit{namn}(\textit{parameter1}: \textit{Typ1}, \textit{parameter2}: \textit{Typ2}): \textit{Returtyp} = \textit{returvärde}}

\Subtask Definiera en funktionen \code{öka} som har en heltalsparameter \code{x} och vars returvärde är argumentet plus 1. Defaultargument ska vara 1. Ange returtypen explicit.

\Subtask Vad har uttrycket \code{öka(öka(öka(öka())))} för värde?

\Subtask Definiera funktionen \code{minska} som har en heltalsparameter \code{x} och vars returvärde är argumentet minus 1. Defaultargument ska vara 1. Ange returtypen explicit.

\Subtask Vad är värdet av uttrycket \code{öka(minska(öka(öka(minska(minska())))))}

\Subtask Vad är det för skillnad mellan parameter och argument?

\SOLUTION

\TaskSolved \what

\SubtaskSolved
\begin{Code}
def öka(x: Int = 1): Int = x + 1
\end{Code}

\SubtaskSolved  \code{5}

\SubtaskSolved
\begin{Code}
def minska(x: Int = 1): Int = x - 1
\end{Code}

\SubtaskSolved  \code{1}

\SubtaskSolved
\begin{itemize}
  \item \emph{Kort, förenklad förklaring:} Parametern i funktionshuvudet är ett lokalt namn på indata som kan användas i funktionskroppen, medan argumentet är själva värdet på parametern som skickas med vid anrop.
  \item \emph{Längre, mer exakt förklaring:} En \textbf{parameter} är en deklaration av en oföränderlig variabel i ett funktionshuvud vars namn finns tillgängligt lokalt i funktionskroppen. Vid anrop \emph{binds} parameternamnet till ett specifikt argument. Ett \textbf{argument} är ett uttryck som  appliceras på en funktion vid anrop. Normalt evalueras argumentet innan anropet sker, men om parametertypen föregås av \code{=>} fördröjs evalueringen av argumentet och sker i stället \emph{varje gång} parameternamnet förekommer i funktionskroppen.
\end{itemize}

\QUESTEND




\WHAT{Textspelet AliensOnEarth.}

\QUESTBEGIN

\Task  \what~Ladda ner spelet nedan \footnote{
\url{https://raw.githubusercontent.com/lunduniversity/introprog/master/compendium/examples/AliensOnEarth.scala}} och studera koden.

\scalainputlisting[basicstyle=\ttfamily\fontsize{10.5}{12.5}\selectfont,numbers=left]{examples/AliensOnEarth.scala}

% def randomDistribution(weights: Vector[Int]): Int = {
%   require(weights.size > 0)
%   require(weights.forall(_ >= 0))
%
%   val probabilities = for (w <- weights) yield w / weights.sum.toDouble
%   val rnd = math.random
%   var i = 0
%   var sum = probabilities(i)
%   while (i < probabilities.size - 1 && rnd > sum) {
%     i += 1
%     sum += probabilities(i)
%   }
%   i
% }

\Subtask Medan du läser koden, försök lista ut vilket som är bästa strategin för att få så mycket poäng som möjligt. Kompilera och kör spelet i terminalen med ditt favoritnamn som argument. Vilket av de tre objekten på planeten jorden har störst sannolikhet att vara bästa alternativet?

\Subtask Para ihop kodsnuttarna nedan med bästa beskrivningen.\footnote{Gör så gott du kan även om allt inte är solklart. Vissa saker kommer vi att gå igenom i detalj först under senare kursmoduler.}

\begin{ConceptConnections}
  \code|options.indices| & 1 & & A & slumptal i intervallet \code|0 until n| \\ 
  \code|"1X2".toLowercase| & 2 & & B & fångar undantag för att förhindra krasch \\ 
  \code|Random.nextInt(n)| & 3 & & C & gör om en sträng till små bokstäver \\ 
  \code|try { } catch { }| & 4 & & D & skriver ut information om ett undantag \\ 
  \code|""" ... """| & 5 & & E & heltalssekvens med alla index i en sekvens \\ 
  \code|s.stripMargin| & 6 & & F & sträng som kan sträcka sig över flera kodrader \\ 
  \code|e.printStackTrace| & 7 & & G & tar bort marginal till och med vertikalstreck \\ 
\end{ConceptConnections}

\noindent\emph{Tips:} Med hjälp av REPL kan du ta reda på hur olika delar fungerar, t.ex.:

\begin{REPL}
scala> val os = Vector("p", "w", "a")
scala> os.indices
scala> os.indices.foreach(i => println(i))
scala> os.indexOf("w")
scala> os.indexOf("gurka")
scala> Vector("hej", "hejsan", "hej").indexOf("hej")
scala> try { 1 / 0 } catch { case e: Exception=> println(e) }
\end{REPL}
Kolla även dokumentationen för \code{nextInt}, \code{readLine}, m.fl genom att söka
här: \\ \url{http://www.scala-lang.org/api/current/index.html}


\begin{framed}
\noindent\emph{Tips inför fortsättningen:}

\begin{itemize}%[nolistsep]
  \item När jag hittade på \code{AliensOnEarth} började jag med ett mycket litet program med en enkel \code{main}-funktion som bara skrev ut något kul. Sedan byggde jag vidare på programmet steg för steg och kompilerade och testade efter varje liten ändring.

  \item När jag kodar har jag REPL igång i ett eget terminalfönster och api-dokumentationen för Scala i en webbläsare redo för sökningar. Jag återanvänder också användbara snuttar från kod jag gjort tidigare och inspireras ofta av lösningar från \url{https://stackoverflow.com} (om jag kan begripa dem och de verkar rimliga).

  \item Detta arbetssätt tar ett tag att komma in i, men är ett bra sätt att uppfinna allt större och bättre program. Ett stort program byggs lättast i små inkrement och felsökning blir mycket lättare om man bara gör små tillägg åt gången.

  \item Du får också det mycket lättare att förstå ditt program om du delar upp koden i många korta funktioner med bra namn. Du kan sedan lättare hitta på mer avancerade funktioner genom att återanvända befintliga.

  \item Under veckans laboration ska du utveckla ditt eget textspel. Då har du nytta av att återanvända funktionerna för indata och slumpdragning från \code{AliensOnEarth}.
\end{itemize}

\end{framed}


\SOLUTION

\TaskSolved \what~

\SubtaskSolved \code{"penguin"} är bästa alternativ med sannolikheten $\frac{1}{2} + \frac{1}{2}\cdot\frac{1}{3} = \frac{2}{3}$

\SubtaskSolved

\begin{ConceptConnections}
    \code|options.indices| & 1 & ~~\Large$\leadsto$~~ &  D & heltalssekvens med alla index i en sekvens \\ 
  \code|"1X2".toLowercase| & 2 & ~~\Large$\leadsto$~~ &  C & gör om en sträng till små bokstäver \\ 
  \code|Random.nextInt(n)| & 3 & ~~\Large$\leadsto$~~ &  G & slumptal i intervallet \code|0 until n| \\ 
  \code|try { } catch { }| & 4 & ~~\Large$\leadsto$~~ &  A & fångar undantag för att förhindra krasch \\ 
  \code|""" ... """| & 5 & ~~\Large$\leadsto$~~ &  B & sträng som kan sträcka sig över flera kodrader \\ 
  \code|s.stripMargin| & 6 & ~~\Large$\leadsto$~~ &  F & tar bort marginal till och med vertikalstreck \\ 
  \code|e.printStackTrace| & 7 & ~~\Large$\leadsto$~~ &  E & skriver ut information om ett undantag \\ 
\end{ConceptConnections}

\QUESTEND



\WHAT{Äkta funktioner.}

\QUESTBEGIN

\Task  \what~  En äkta funktion%
\footnote{Äkta funktioner uppfyller per definition  \textit{referentiell transparens} \Eng{referential transparency} som du kan läsa mer om här:  \href{https://en.wikipedia.org/wiki/Referential_transparency}{en.wikipedia.org/wiki/Referential\_transparency}}
\Eng{pure function} ger alltid samma resultat med samma argument (så som vi är vana vid inom matematiken) och har inga externt observerbara sidoeffekter (till exempel utskrifter).

Vilka funktioner i objektet \code{inSearchOfPurity} nedan är äkta funktioner?
\begin{Code}
object inSearchOfPurity {
  var x = 0
  val y = x
  def inc(i: Int): Int = i + 1
  def oink(i: Int): String = { x = x + i; "Pig says " + ("oink " * x) }
  def addX(i: Int): Int = x + i
  def addY(i: Int): Int = y + i
  def isPalindrome(s: String): Boolean = s == s.reverse
  def rnd(min: Int, max: Int): Double = math.random * max + min
}
\end{Code}


\noindent\emph{Tips:} Klistra in hela singelobjektet i REPL och testa att anropa funktionerna om du är osäker på vad som händer. Om du gör \code{import inSearchOfPurity._} kommer du åt namnen i singelobjektet direkt och kan lätt undersöka variablernas värden.

\SOLUTION

\TaskSolved \what

\begin{itemize}
  \item Funktionerna  \code{inc}, \code{addY} och \code{isPalindrome} är äkta. Notera att \code{y}-variablen initialiseras till \code{0} och kan sedan inte ändras eftersom den är deklarerad med nyckelordet \code{val}.
\end{itemize}

\QUESTEND




\WHAT{Funktioner är objekt med en \code{apply}-metod.}

\QUESTBEGIN

\Task  \what~ \\
\noindent Vilka rader nedan ger kompileringsfel?

\begin{REPL}
scala> object plus { def apply(x: Int, y: Int) = x + y }
scala> plus.apply(42, 43)
scala> plus.apply(42)
scala> plus(42, 43)
scala> plus(42)
\end{REPL}


\SOLUTION

\TaskSolved \what

\begin{itemize}
  \item Rad 3 och 5 ger kompileringsfelet\\
  \texttt{error: not enough arguments for method apply}
\end{itemize}

\QUESTEND



\WHAT{Applicera funktion på varje element i en samling. Funktion som argument.}

\QUESTBEGIN

\Task  \what~

\noindent Deklarera funktionen \code{öka} och variabeln \code{xs} enligt nedan i REPL:
\begin{REPL}
scala> def öka(x: Int) = x + 1
scala> val xs = Vector(3, 4, 5)
\end{REPL}
\noindent Para ihop nedan uttryck till vänster med det uttryck till höger som har samma värde. Om du undrar något, testa uttrycken och olika varianter av dem i REPL.

\begin{ConceptConnections}
  \code|for (i <- 1 to 3) yield öka(i)| & 1 & & A & \code|Vector(5, 6, 7)| \\ 
  \code|Vector(2, 3, 4).map(i => öka(i))| & 2 & & B & \code|()| \\ 
  \code|xs.map(öka)| & 3 & & C & \code|xs| \\ 
  \code|xs.map(öka).map(öka)| & 4 & & D & \code|Vector(2, 3, 4)| \\ 
  \code|xs.foreach(öka)| & 5 & & E & \code|Vector(4, 5, 6)| \\ 
\end{ConceptConnections}

\SOLUTION

\TaskSolved \what

\begin{ConceptConnections}
    \code|for (i <- 1 to 3) yield öka(i)| & 1 & ~~\Large$\leadsto$~~ &  C & \code|Vector(2, 3, 4)| \\ 
  \code|Vector(2, 3, 4).map(i => öka(i))| & 2 & ~~\Large$\leadsto$~~ &  E & \code|xs| \\ 
  \code|xs.map(öka)| & 3 & ~~\Large$\leadsto$~~ &  B & \code|Vector(4, 5, 6)| \\ 
  \code|xs.map(öka).map(öka)| & 4 & ~~\Large$\leadsto$~~ &  D & \code|Vector(5, 6, 7)| \\ 
  \code|xs.foreach(öka)| & 5 & ~~\Large$\leadsto$~~ &  A & \code|()| \\ 
\end{ConceptConnections}

\QUESTEND



\WHAT{Funktion som äkta värde.}

\QUESTBEGIN

\Task  \what~  Funktioner är \emph{äkta värden} i Scala\footnote{I likhet med t.ex. Javascript, men till skillnad från t.ex. Java.}. Det betyder att variabler kan ha funktioner som värden och funktionsvärden kan vara argument till funktioner som har funktionsparametrar\footnote{Funktioner som tar funktioner som argument kallas \emph{högre ordningens funktioner}}.

  En funktion som har en heltalsparameter och ett heltalsresultat är av funktionstypen \code{Int => Int} (uttalas \emph{int-till-int}) och värdet av funktionen utgör ett objekt som har en metod som heter \code{apply} med motsvarande funktionstyp.

\Subtask \label{subtask:funcval} Deklarera nedan funktioner och variabler i REPL. Notera understrecket på rad 3. Para sedan ihop nedan uttryck till vänster med det uttryck till höger som har samma värde. Om du undrar något, testa uttrycken och olika varianter av dem i REPL.

\begin{REPL}
scala> def öka(x: Int): Int = x + 1
scala> def app(x: Int, f: Int => Int): Int = f(x)
scala> val f1 = öka _
scala> var f2 = (x: Int) => x - 1
\end{REPL}

\begin{ConceptConnections}
  \code| öka(-1)     | & 1 & & A & \code| 1     | \\ 
  \code| app(1, öka) | & 2 & & B & \code| 0     | \\ 
  \code| app(5, f2)  | & 3 & & C & \code| 3     | \\ 
  \code| f1(2)       | & 4 & & D & \code| öka(1)| \\ 
  \code| f2(2)       | & 5 & & E & \code| 4     | \\ 
\end{ConceptConnections}


\Subtask Vilka typer har variablerna \code{f1} och \code{f2}?

\Subtask Går det att ge variabeln \code{f2} funktionsvärdet \code{öka} genom tilldelning?

\Subtask Går det bra att skriva \code{val f3 = öka} utan understreck?

\Subtask Går det bra att skriva \code{val f3: Int => Int = öka} utan understreck?

\SOLUTION

\TaskSolved \what

\SubtaskSolved

\begin{ConceptConnections}
    \code| öka(-1)     | & 1 & ~~\Large$\leadsto$~~ &  C & \code| 0     | \\ 
  \code| app(1, öka) | & 2 & ~~\Large$\leadsto$~~ &  A & \code| öka(1)| \\ 
  \code| app(5, f2)  | & 3 & ~~\Large$\leadsto$~~ &  E & \code| 4     | \\ 
  \code| f1(2)       | & 4 & ~~\Large$\leadsto$~~ &  B & \code| 3     | \\ 
  \code| f2(2)       | & 5 & ~~\Large$\leadsto$~~ &  D & \code| 1     | \\ 
\end{ConceptConnections}

\SubtaskSolved Båda har typen \code{Int => Int}

\SubtaskSolved  Ja, det går fint.

\SubtaskSolved  Nej det blir kompileringsfel: \\
\begin{REPL}
scala> val f3 = öka
<console>:12: error: missing argument list for method öka
Unapplied methods are only converted to functions when
a function type is expected. You can make this conversion
explicit by writing 'öka _' or 'öka(_)' instead of 'öka'.
\end{REPL}

\SubtaskSolved  Ja, det går fint. Nu med typinformationen på plats är kompilatorn säker på vad du vill göra.

\QUESTEND




\WHAT{Anonyma funktioner.}

\QUESTBEGIN

\Task  \what~  Vi har flera gånger sett syntaxen \code{i => i + 1}, till exempel i en loop \code{(1 to 10).map(i => i + 1)} där funktionen \code{i => i + 1} appliceras på alla heltal från 1 till och med 10 och resultatet blir en ny sekvenssamling.

Syntaxen \code{(i: Int) => i + 1} är en litteral för att skapa ett funktionsvärde. Syntaxen liknar den för funktionsdeklarationer, men nyckelordet \code{def} saknas i funktionshuvudet och i stället för likhetstecken används \code{=>} för att avskilja parameterlistan från funktionskroppen.
Om kompilatorn kan härleda typen ur sammanhanget kan kortformen \code{i => i + 1} användas.

Det finns ett \emph{ännu} kortare sätt att skriva en anonym funktion \emph{om} typen kan härledas \emph{och} den bara använder sin parameter \emph{en enda gång}; då går funktionslitteraler att skriva med s.k. \emph{platshållarsyntax} som använder understreck, till exempel \code{ _ + 1} och som automatiskt expanderas av kompilatorn till \code{ngtnamn => ngtnamn + 1} (namnet på parametern spelar ingen roll; kompilatorn väljer något eget, internt namn).

Para ihop uttryck till vänster med uttryck till höger som har samma värde:

\begin{ConceptConnections}
  \code|(0 to 2).map(i => i + 1)           | & 1 & & A & \code|Vector(9.0, 16.0, 25.0) | \\ 
  \code|(1 to 3).map(_ + 1)                | & 2 & & B & \code|Vector(4.0, 8.0, 16.0)  | \\ 
  \code|(2 to 4).map(math.pow(2, _))       | & 3 & & C & \code|Vector(2.0, 2.5, 3.0)   | \\ 
  \code|(3 to 5).map(math.pow(_, 2))       | & 4 & & D & \code|Vector(2, 3, 4)         | \\ 
  \code|(4 to 6).map(_.toDouble).map(_ / 2)| & 5 & & E & \code|(2 to 4).map(i => i - 1)| \\ 
\end{ConceptConnections}

\noindent
Funktionslitteraler kallas även \textit{anonyma funktioner}\footnote{Ett annat vanligt namn är \textit{lambda} efter det datalogiska matematikverktyget lambdakalkyl:  \\\url{https://sv.wikipedia.org/wiki/Lambdakalkyl}}, eftersom de inte har något namn, till skillnad från t.ex. \code{def öka(i: Int): Int = i + 1}, som ju heter \code{öka}.

\SOLUTION

\TaskSolved \what

\begin{ConceptConnections}
    \code|(0 to 2).map(i => i + 1)           | & 1 & ~~\Large$\leadsto$~~ &  C & \code|(2 to 4).map(i => i - 1)| \\ 
  \code|(1 to 3).map(_ + 1)                | & 2 & ~~\Large$\leadsto$~~ &  B & \code|Vector(2, 3, 4)         | \\ 
  \code|(2 to 4).map(math.pow(2, _))       | & 3 & ~~\Large$\leadsto$~~ &  E & \code|Vector(4.0, 8.0, 16.0)  | \\ 
  \code|(3 to 5).map(math.pow(_, 2))       | & 4 & ~~\Large$\leadsto$~~ &  A & \code|Vector(9.0, 16.0, 25.0) | \\ 
  \code|(4 to 6).map(_.toDouble).map(_ / 2)| & 5 & ~~\Large$\leadsto$~~ &  D & \code|Vector(2.0, 2.5, 3.0)   | \\ 
\end{ConceptConnections}

\QUESTEND




\ExtraTasks %%%%%%%%%%%%%%%%%%%%%%%%%%%%%%%%%%%%%%%%%%%%%%%%%%%%%%%%%%



\WHAT{Funktion med flera parametrar.}

\QUESTBEGIN

\Task  \what~  Definiera i REPL två funktioner \code{sum} och \code{diff} med två heltalsparametrar som returnerar summan respektive differensen av argumenten: \\
\code{def sum(x: Int, y: Int): Int = x + y} \\
\code{def diff(x: Int, y: Int): Int = x - y} \\
Vad har nedan uttryck för värden? Förklara vad som händer.

\Subtask \code{diff(0, 100)}

\Subtask \code{diff(100, sum(42, 43))}

\Subtask \code{sum(sum(42, 43), diff(100, sum(0, 0)))}

\Subtask \code{sum(diff(Byte.MaxValue, Byte.MinValue),1)}

\SOLUTION

\TaskSolved \what

\SubtaskSolved  \code{-100}

\SubtaskSolved  \code{15}

\SubtaskSolved  \code{185}

\SubtaskSolved  \code{256}

\QUESTEND



\WHAT{Medelvärde.}

\QUESTBEGIN

\Task  \what~ Skriv och testa en funktion \code{avg} som räknar ut medelvärdet mellan två heltal och returnerar en \code{Double}.

\SOLUTION

\TaskSolved \what

\begin{Code}
def avg(x: Int, y: Int): Double = (x + y) / 2.0
\end{Code}

\QUESTEND




\WHAT{Funktionsanrop med namngivna argument.}

\QUESTBEGIN

\Task  \what~
\begin{REPL}
scala> def skrivNamn(efternamn: String, förnamn: String) =
         println(s"Namn: $efternamn, $förnamn")
scala> skrivNamn(förnamn = "Stina", efternamn = "Triangelsson")
scala> skrivNamn(efternamn = "Oval", "Viktor")

\end{REPL}

\Subtask Vad skrivs ut efter rad 3 resp. rad 4 ovan?

\Subtask Nämn tre fördelar med namngivna argument.

\SOLUTION

\TaskSolved \what~

\SubtaskSolved
\begin{REPL}
Namn: Triangelsson, Stina
Namn: Oval, Viktor
\end{REPL}

\SubtaskSolved
\begin{itemize}
  \item Anroparen kan själv välja ordning.
  \item Koden blir lättare att begripa om parameternamnen är självbeskrivande.
  \item Hjälper till att förhindra buggar som beror på förväxlade parametrar.
\end{itemize}

\QUESTEND




\WHAT{Bortkastade resultatvärden och returtypen \code{Unit}.}

\QUESTBEGIN

\Task  \what~ Undersök nedan kod i REPL och förklara vad som händer.

\Subtask
\begin{REPL}
scala> def tom = println("")
scala> println(tom)
\end{REPL}

\Subtask
\begin{REPL}
scala> def bortkastad: Unit = 1 + 1
scala> println(bortkastad)
\end{REPL}

\Subtask
\begin{REPL}
scala> def bortkastad2 = { val x = 1 + 1 }
scala> println(bortkastad2)
\end{REPL}

\Subtask Varför är det bra att explicit ange \code{Unit} som returtyp för procedurer?

\SOLUTION

\TaskSolved \what

\SubtaskSolved Procedurer returnerar tomma värdet och \code{println} är en procedur. När tomma värdet skrivs ut visas \code{()}.

\SubtaskSolved Procedurer returnerar tomma värdet. Om du anger returtyp \code{Unit} explicit, har du bättre chans att kompilatorn kan ge varning då uträkningar kommer att kastas bort. En varning avbryter inte exekveringen, utan är ett sätt för kompilatorn att ge dig tips om saker som kan behöva fixas till i din kod.

\SubtaskSolved I Scala är variabeldeklaration, precis som en tilldelningssats, och inte ett uttryck och saknar värde.

\SubtaskSolved  Koden blir lättare att läsa och kompilatorn får bättre möjlighet att hjälpa till med varningar om resultatvärden riskerar att bli bortkastade.

\QUESTEND


\clearpage

\AdvancedTasks %%%%%%%%%%%%%%%%%%%%%%%%%%%%%%%%%%%%%%%%%%%%%%%%%%%%%%%%%%%




\WHAT{Föränderlighet av parametrar.}

\QUESTBEGIN

\Task \what~Är en parameter förändringsbar i funktionskroppen ...

\Subtask ... i Scala?  (Ja/Nej)

\Subtask ... i Java?  (Ja/Nej)

\SOLUTION

\TaskSolved \what~

\Subtask Nej, i Scala är parametern oföränderlig och det blir kompileringsfel om man försöker tilldela den ett nytt värde i funktionskroppen.

\Subtask Ja det går utmärkt i Java att ändra värdet på parametern i funktionskroppen med tilldelning, men koden riskerar att bli förvirrande.\\
\url{https://stackoverflow.com/questions/2970984}

\QUESTEND




\WHAT{Uppdelad parameterlista och stegade funktioner.}

\QUESTBEGIN

\Task \what~Man kan dela upp parametrarna till en funktion i flera parameterlistor. Funktionen \code{add1} nedan har en parameterlista med två parametrar medan \code{add2} har två parameterlistor med en parameter vardera:
\begin{Code}
  def add1(a: Int, b: Int) = a + b
  def add2(a: Int)(b: Int) = a + b
\end{Code}

\Subtask  När man anropar funktionen \code{add2} ska argumenten skrivas inom två olika parentespar. Hur kan du använda \code{add2} för att räkna ut \code{1 + 1}?

\Subtask En fördel med uppdelade parameterlistor är att man kan skapa s.k. \emph{stegade funktioner}\footnote{Kallas även Curry-funktioner efter matematikern och logikern Haskell Brooks Curry.} där argumenten är partiellt applicerade. Prova det stegade funktionsvärdet \code{singLa} nedan. Vad skrivs ut på efter raderna 3 och 5?

\begin{REPL}
scala> def repeat(s: String)(n: Int): String = s * n
scala> val song = repeat("doremi ")(3)
scala> println(song)
scala> val singLa = repeat("la") _
scala> println(singLa(7))
\end{REPL}

\SOLUTION

\TaskSolved \what

\SubtaskSolved
\begin{REPL}
scala> def add2(a: Int)(b: Int) = a + b
add2: (a: Int)(b: Int)Int

scala> add2(1)(1)
res0: Int = 2
\end{REPL}

\SubtaskSolved
\begin{itemize}

\item Rad 3:
\begin{REPLnonum}
doremi doremi doremi
\end{REPLnonum}

\item Rad 5:
\begin{REPLnonum}
lalalalalalala
\end{REPLnonum}

\end{itemize}


\QUESTEND




\WHAT{Rekursion.}

\QUESTBEGIN

\Task\Uberkurs  \what~  En rekursiv funktion anropar sig själv.

\Subtask Förklara vad som händer nedan.

\begin{REPL}
scala> def countdown(x: Int): Unit = if (x > 0) {println(x); countdown(x -1)}
scala> countdown(10)
scala> countdown(-1)
scala> def finalCountdown(x: Byte): Unit =
         {println(x); Thread.sleep(100); finalCountdown((x-1).toByte); 1 / x}
scala> finalCountdown(Byte.MaxValue)
\end{REPL}

\Subtask Vad händer om du gör satsen som riskerar division med noll \emph{före} det rekursiva anropet i funktionen \code{finalCountdown} ovan?

\Subtask Förklara vad som händer nedan. Varför tar sista raden längre tid än näst sista raden?
\begin{REPL}
scala> def signum(a: Int): Int = if (a >= 0) 1 else -1
scala> def add(x: Int, y: Int): Int =
         if (y == 0) x else add(x + 1, y - signum(y))
scala> add(100,100)
scala> add(Int.MaxValue, 0)
scala> add(0, Int.MaxValue)
\end{REPL}

\SOLUTION

\TaskSolved \what

\SubtaskSolved
\code{countdown} skriver ut x och gör ett rekursivt anrop med \code{x-1} som argument, men bara om basvillkoret \code{(x > 0)} är uppfyllt. Resultatet blir en ändlig  repetition.
\code{finalCountdown} anropar sig själv rekursivt men saknar ett basvillkor som kan avbryta rekursionen, vilket genererar en oändlig repetition. Vid -128 blir det \emph{overflow} eftersom bitarna inte räcker till för större negativa tal och räkningen börjar om på 127. (Om minskar fördröjningen till \code{Thread.sleep(1)} blir det ganska snabbt \emph{stack overflow})

\SubtaskSolved
Eftersom vi hade \code{1/x} \emph{efter} det rekursiva anropet i föregående deluppgift, så kom vi aldrig till denna (potentiellt ödesdigra) beräkning, utan lade bara aktiveringsposter på hög på stacken vid varje anrop. Om vi placerar \code{1/x} \emph{före} det rekursiva anropet, så når vi detta uttryck direkt och det kastas ett undantag p.g.a. division med noll.

\SubtaskSolved
Den sista raden leder till många fler rekursiva anrop, så som basvillkoret och det rekursiva anropet är konstruerade. Lägg gärna in en \code{println}-sats före det rekursiva anropet och undersök i detalj vad som sker.

\QUESTEND



\WHAT{Undersök svansrekursion genom att kasta undantag.}

\QUESTBEGIN

\Task\Uberkurs  \what~  Förklara vad som händer. Kan du hitta bevis för att kompilatorn kan optimera rekursionen till en vanlig loop?

\begin{REPL}
scala> def explode = throw new Exception("BANG!!!")
scala> explode
scala> lastException.printStackTrace
scala> def countdown(n: Int): Unit =
         if (n == 0) explode else countdown(n-1)
scala> countdown(10)
scala> countdown(10000)
scala> lastException.printStackTrace
scala> def countdown2(n: Int): Unit =
         if (n == 0) explode else {countdown2(n-1); print("no tailrec")}
scala> countdown2(10)
scala> countdown2(10000)
scala> lastException
scala> lastException.getStackTrace.size
scala> :javap countdown
scala> :javap countdown2
\end{REPL}

\SOLUTION

\TaskSolved \what~\code{countdown} är svansrekursiv eftersom det rekursiva anropet står \emph{sist} och kan då optimeras till en \code{while}-loop av kompilatorn. Det går fint att köra ända till det exploderar, även med 10000 anrop.

\code{countdown2} är inte svansrekursiv eftersom den har ett uttryck \code{efter} det rekursiva anropet. Med \code{:javap} får man se maskinkoden och i den ser man ett \code{goto} i \code{countdown} som är den maskinkod som generas vid \code{while}-loopar, medan \code{countdown2} har maskinkodsinstruktionen \code{invokevirtual} och inget \code{goto} vilket indikerar att kompilatorn aldrig lyckades omforma den rekursiva repetitionen till en \code{while}-sats.

\QUESTEND



\WHAT{\code{@tailrec}-annotering.}

\QUESTBEGIN

\Task\Uberkurs  \what~  Du kan be kompilatorn att ge felmeddelande om den inte kan optimera koden till motsvarande en while-loop. Detta kan användas i de fall man vill vara helt säker på att kompilatorn kan optimera koden och det inte kan finnas risk för en överfull stack \Eng{stack overflow} på grund av för djup anropsnästling.

Prova nedan rader i REPL och förklara vad som händer.
\begin{REPL}
scala> def countNoTailrec(n: Long): Unit =
         if (n <= 0L) println("Klar! " + n) else {countNoTailrec(n-1L); ()}
scala> countNoTailrec(1000L)
scala> countNoTailrec(100000L)
scala> import scala.annotation.tailrec
scala> @tailrec def countNoTailrec(n: Long): Unit =
         if (n <= 0L) println("Klar! " + n) else {countNoTailrec(n-1L); ()}
scala> @tailrec def countTailrec(n: Long): Unit =
         if (n <= 0L) println("Klar! " + n) else countTailrec(n-1L)
scala> countTailrec(1000L)
scala> countTailrec(100000L)
scala> countTailrec(Int.MaxValue.toLong * 2L)
\end{REPL}

\SOLUTION

\TaskSolved \what~Första gången \code{countNoTailrec(100000L)}  anropas blir det \code{StackOverflowError}. Med annoteringen \code{@tailrec} får vi ett kompileringsfel eftersom kompilatorn inte kan optimera en icke svansrekursiv funktion. Om funktionen skrivs om kan kompilatorn optimera funktionen så att rekursionen byts ut mot en \code{while}-loop och vi kan köra så länge vi orkar utan att stacken flödar över. Och himla snabbt går det!!

\QUESTEND
