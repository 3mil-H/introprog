%!TEX encoding = UTF-8 Unicode

%!TEX root = ../compendium.tex

\ExerciseSolution{\ExeWeekSEVEN}

\Task 

\Subtask \code{Vector[Object]}.

\Subtask Det beror på att vektorns element är av typen \code{Object}. \code{vikt} är inte definierat för denna typ.

\Subtask -.

\Subtask \code{Vector[Grönsak]}.

\Subtask Ja.

\Subtask -.

\Subtask \code{Grönsak}. \$anon\$1@88dfbe.

\Task 

\Subtask 
\begin{Code}
def skapaDjur: Djur = 
   {if(math.random > 0.5) new Ko else new Gris} 
\end{Code}

\Subtask 
\begin{Code}
class Häst extends Djur{ def väsnas = println("Gnääääägg") }
def skapaDjur: Djur = {val r = math.random; 
   if(r < 0.33) new Ko else if(r < 0.67) new Gris else new Häst}
\end{Code}

\Task 

\Subtask 
\begin{Code}
val c1 = Circle(Point(1, 1), 42)
val r1 = Rectangle(Point(3, 3), 20, 30)
c1.move(2, 3)
r1.move(3, 2)
\end{Code}
   
\Subtask För \code{Point}: \code{def moveTo(dx: Double, dy: Double): Point = Point(dx, dy)}. \\
För \code{Shape}: \code{def moveTo(dx: Double, dy: Double): Shape}. \\
För \code{Rectangle}: \code{override def moveTo(dx: Double, dy: Double): Rectangle = } \\
\code{Rectangle(pos.moveTo(dx, dy), this.dx, this.dy)}. \\
För \code{Circle}: \code{override def moveTo(dx: Double, dy: Double): Circle =} \\
\code{Circle(pos.moveTo(dx, dy), radius)}. 
            
\Subtask \code{def distanceTo(that: Point): Double = math.hypot(that.x - x, that.y - y)}.

\Subtask \code{def distanceTo(that: Shape): Double = pos.distanceTo(that.pos)}.
         
\Task

\Subtask 
\begin{Code}
fyle.filter(f => f.isInstanceOf[Ånka] && f.ärFlygkunnig).size 
\end{Code}
      
\Subtask 
\begin{Code}
val antalKrax: Int = fyle.filter(f => !f.ärSimkunnig).size * 2
val antalKvack: Int = fyle.filter(f => f.ärSimkunnig).size * 4 
\end{Code}
      
\Task

\Subtask Sätt \code{final} framför \code{class} i klasserna.

\Subtask error: illegal inheritance from final class Kråga.

\Task

\Subtask error: not found: value minHemlis.

\Subtask error: value vårHemlis in class Super\$class cannot be accessed in Sub.

\Subtask Ja.

\Task

\Subtask I Fyle: 
\begin{Code}
protected var räknaLäte: Int = 0
def väsnas: Unit = { print(läte * 2); räknaLäte += 2 }
\end{Code}

I Ånka: \code| override def väsnas = { print(läte * 4); räknaLäte += 4 }|
      
\Subtask \code{ def antalLäten: Int = räknaLäte }

\Subtask Om en klass som representerar en fågel som skulle ge ifrån sig fler/färre läten än en vanlig \code{Fyle}, behöver \code{väsnas} ändras. Denna metod behöver tillgång till \code{räknaLäte}, vilken inte får vara \code{private}.

\Subtask Räknar-variabeln ska inte kunna påverkas i någon annan del av programmet.

\Task 

\Subtask B ärver A. C och D ärver B.

\Subtask 1. True eftersom c är av typen C. \\
2. False eftersom c inte är av typen D. \\
3. True eftersom d är av typen D som är en subtyp av B. \\
4. True eftersom c är av typen C som är en subtyp av B, som i sin tur är en subtyp av A. \\
5. True eftersom b är av typen D, som är en subtyp av B, som i sin tur är en subtyp av A. \\
6. True eftersom b är av typen D. \\
7. True eftersom a är av typen C som är en subtyp av B. \\
8. True eftersom c är av typen C som är en subtyp av AnyRef. \\
9. True eftersom c är av typen C som är en subtyp av Any. \\
10. Error eftersom \code{isInstanceOf} inte kan använda sig av \code{AnyVal}.  \\
11. True eftersom c är av typen C som är en subtyp av Object (Object är java-representationen av AnyRef). \\
12. Error eftersom \code{isInstanceOf} inte kan testa om värdetyper (i detta fallet \code{42}) är referenstyper. \\
13. True eftersom \code{42} är av typen \code{Int} som är en subtyp av Any. \\

\Subtask 3. Går inte eftersom c inte är av typen D, utan typen C. \\
6. Går inte eftersom a inte är av typen D, utan typen C. \\
7. Går inte eftersom typen E inte finns. \\

\Task

\Subtask 2. Måste ha \code{override} framför \code{b} för att kunna ändra på metoden. \\
4. \code{c} är \code{private}, vilket betyder att den är gömd för subklasserna. Därför kan den inte överskuggas. Genom att ta bort \code{override} fungerar klassen. \\
5. En \code{final}-medlem måste ha ett bestämt värde. Kan lösas genom att tilldela \code{final a} ett värde eller ta bort \code{final}. \\
6. En \code{final}-medlem kan inte överskuggas, varken med eller utan \code{override}. Här får konflikterna tas bort.  \\
7. Se 6. \\
8. Eftersom \code{c} inte finns i \code{Super5} kan den inte överskuggas. Genom att ta bort \code{override} fungerar klassen. \\
10. Överskuggningen av \code{val} måste vara oföränderlig (immutable); detta är inte nödvändigtvis \code{def}. Löses genom att byta ut \code{def a} mot \code{val a} hos \code{Sub10}.  \\
11. Samma problem som i 10.; \code{lazy val} kan vara föränderlig. Löses genom att ta bort \code{lazy}. \\
12. Samma problem igen! \code{var} är föränderlig, vilket bryter mot typsäkerheten när man försöker överskugga en \code{val}. Löses genom att ändra \code{var} till \code{val}. \\
15.\code{def a = 43} och \code{val c = "?"} täcker inte allt som \code{var} kräver. Det behövs en setter för att kunna uppfylla kraven för överskuggning för en \code{var}. Dessutom finns det ingen anledning för en \code{val} att överskuggas; man kan ju ändra på den lite hur man vill!

\Subtask Sub3: a = 43, b = 43 eftersom medlemmen är överskuggad. c hittas inte eftersom den är \code{private}.

Sub13: a = 43, b = 42, c = "still lazy" eftersom medlemmen överskuggas.

SubSub: a = 44 eftersom medlemmen överskuggas, b = 42, c = "still lazy".

\Subtask -.
 
\Task
 
\Subtask
\begin{Code}
val person = new Person("Person1")
val akademiker = new Akademiker("Person2", "LTH")
val student = new Student("Person3", "LTH", "D")
val forskare = new Forskare("Person4", "LTH", "Doktorand")
\end{Code}

\Subtask
\begin{Code}
val vec = Vector(person, akademiker, student, forskare)
for(i <- vec){ print(i.toString + i.namn) }
\end{Code}

\Subtask error: class Person is abstract; cannot be instantiated.

\Subtask error: overriding value namn in class Person of type String; value namn needs `override' modifier.\\ 
toString för Student: Student(Person3,LTH,D). \\
toString för Forskare: Student(Person4,LTH,Doktorand). 

\Subtask
\begin{Code}
trait Person {val namn: String; val nbr: Int}
trait Akademiker extends Person {val universitet: String} 
case class Student(
  namn: String,
  nbr: Int,
  universitet: String,
  program: String) extends Akademiker
case class Forskare(
  namn: String,
  nbr: Int,
  universitet: String,
  titel: String) extends Akademiker
case class IckeAkademiker(
    namn: String,
    nbr: Int) extends Person
\end{Code}

\Subtask Man måste använda en klass om man behöver klassparametrar. Man måste använda en trait om man vill göra in-mixning med \code{with}. \\
 Se \href{http://www.artima.com/pins1ed/traits.html\#12.7}{http://www.artima.com/pins1ed/traits.html\#12.7}.

\Task

\Subtask Sättet är säkrare då man inte kan tilldela korten en färg som inte finns. Med heltalskonstanterna kan man till exempel ge ett kort färgen 5, vilken inte korresponderar till någon riktig färg.

\Subtask \code{for (f <- Färg.values; v <- 1 to 13) yield Kort(f,v)}

\Subtask
\begin{Code}
def blandadKortlek: Vector[Kort] = {
  val kortlek = 
    for (f <- Färg.values; v <- 1 to 13) yield Kort(f,v)
  scala.util.Random.shuffle(kortlek)
}
\end{Code}

\Subtask \code{def färgPoäng(xs: Vector[Kort]): Int = xs.map(_.färg.toInt).sum}
