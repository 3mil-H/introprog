%!TEX encoding = UTF-8 Unicode

%!TEX root = ../compendium.tex

\Lab{\LabWeekNINE}

\begin{Goals}
\item Kunna skapa och iterera över matriser med nästlade for-loopar.
\item Kunna använda sig av och förstå arv.
\item Förstå och träna på olika villkor med if-satser.
\item Känna till algoritmer för att lösa problem så som att ta sig igenom en labyrint eller slumpmässigt skapa en labyrint
\end{Goals}

\begin{Preparations}
\item Läs om matriser.
\item Läs om arv.
\item Läs om olika algoritmer för att ta sig igenom en labyrint \href{https://en.wikipedia.org/wiki/Maze_solving_algorithm}{en.wikipedia.org/wiki/Maze\_solving\_algorithm} 
och skapa en labyrint 
\href{https://en.wikipedia.org/wiki/Maze_generation_algorithm}{en.wikipedia.org/wiki/Maze\_generation\_algorithm}.
\end{Preparations}

\subsection{Bakgrund}
I denna laboration kommer du att få skapa labyrinter och sedan implementera algoritmer för att ta dig igenom dessa. En labyrint är ett rum som har en ingång och en utgång. Ingången är i de här fallen alltid längst ner i labyrinten, och utgången högst upp. Alla väggar är också parallella med antingen x-axeln eller y-axeln. Ett sätt att beskriva en sådan labyrint i kod är med hjälp av en matris. Varje element i matrisern motsvarar då en "ruta" (ordval? steg??) i labyrinten. Om ett element på en viss plats i matrisen är TRUE betyder det att på motsvarande plats i labyrinten finns en vägg, och om ett element är FALSE att det här i labyrinten finns en gång.

Det finns många olika sätt att ta sig igenom en labyrint men ett av de vanligaste och enklaste sätten att konstruera en algoritm är att hålla sin vänstra eller högra hand mot motsvarande vägg och följa väggen utan att släppa den med handen tills man når slutet av labyrinten. Detta funkar för alla labyrinter där väggarna från ingången till utgången är sammankopplade.

Man kan även skapa slumpmässiga labyrinter med hjälp av algoritmer. Det går då till så att ....


\subsection{Obligatoriska uppgifter}

\Task I denna uppgift ska du implementera en metod som kan rita upp en labyrint i SimpleWindow.

\Subtask Läs igenom klassen Maze och se till att du förstår det mesta. Vad Maze gör är att den läser in rader med strängar, antingen från en fil eller direkt som argument, där tecknet '\#' representerar en vägg och ' ' representerar en gång. Utifrån detta skapar Maze en Boolean-matris "mapMatrix" som representerar labyrinten, där TRUE står för vägg och FALSE står för gång. Fråga om något är oklart!

\Subtask Implementera metoden DRAW i Maze som ritar upp labyrinten i SimpleWindow. För att göra detta behöver du gå ingenom matrisen "mapMatrix" i Maze och undersöka elementen på varje plats. Om ett element är TRUE så betyder det att det här ska finnas (det vill säga ritas upp) en vägg i labyrinten, om FALSE att det här ska finnas en gång. Ta hjälp av metoden brickInTheWall som finns tillgänglig i Maze-klassen.

\Task En labbuppgiftsbeskrivning.

\Subtask Skapa en ny klass AMazeIngRace genom att välja File -> New -> Scala Class. I denna klass ska du skriva en main-metod där du skapar ett objekt av klassen Maze genom att läsa in från fil (börja exempelvis med filen maze1.txt). Du måste även skapa ett objekt av SimpleWindow för att skicka med i draw-metoden. Anropa sedan metoden draw på Maze-objektet och kolla att labyrinten ritas upp som den ska. Gör samma sak för resterande av filerna mazeX.txt, alla ska kunna ritas upp korrekt. Testa att rita en egen labyrint genom att skapa en textfil och lägg i samma mapp som övriga maze-filer. Kontrollera så att även denna ritas upp som den ska, och fixa annars till metoden draw så att den fungerar som tänkt.

\Task I den här uppgiften ska du implementera en algoritm för att få en sköldpadda att ta sig genom en labyrint med hjälp av att alltid hålla i väggen med vänster hand (eller kanske fot i det här fallet!).

\Subtask Skapa en ny klass MazeTurtle som ärver från klassen ColourTurtle. MazeTurtle ska ta ett extra argument, nämligen ett av typen Maze som är den labyrint som sköldpaddan ska gå i.

\Subtask Definiera i MazeTurtle en ny metod "walk". Implementera sedan denna metod. I metoden ska sköldpaddan med hjälp av tekniken "hålla vänster hand i väggen" ta sig genom labyrinten, från början till slut. Varje steg motsvarar att flytta sig från en ruta till en annan i Boolean-matrisen i Maze. Sköldpaddan kommer alltså ta sig fram genom att undersöka för varje steg om den borde svänga vänster, det vill säga om den inte har någon vägg till vänster om sig längre, om den borde gå rakt fram eller om den borde svänga höger.

\Subtask Lägg till kod i AMazeIngRace som skapar en sköldpadda och sedan låter denna gå igenom labyrinten med metoden walk. Testa att din MazeTurtle fungerar som den ska! Sköldpaddan ska klara att ta sig igenom alla labyrinter i filerna maze1.txt - maze3.txt samt din egna labyrint.

\subsection{Frivilliga extrauppgifter}

(Skriv om till riktig pseudoko??)

Börja med att välja en slumpmässig kolumn som startkolumn och den nedersta raden som startrad ur matrisen
Markera i matrisen att platsen (startrad)(startkolumn) och platsen (startrad - 1)(startkolumn) i "brädet" inte är en vägg (då det är ingången!)
Lägg till platsen (startrad - 1)(startkolumn) i listan med väggar
Så länge som listan med väggar inte är tom:
    Välj ett slumpmässigt tal (index) mellan 0 och storleken på listan med väggar
    Hämta det element (row, col) som finns på platsen index i listan med väggar
    Om det finns väggar runt omkring platsen (row, col):
        Markera i matrisen att platsen (row, col) inte är en vägg
        Lägg till platsen (row)(col) i listan med väggar
    Ta bort elementet på plats index i listan med väggar
Sätt en variabel check till false
Så länge check är sann
    Välj ett slumpmässigt tal X mellan 0 och antalet kolumner
    Om elementet på plats i matrisen (1)(X) inte är en vägg
        Markera i matrisen att (0)(X) inte är en vägg
        Sätt check till att inte vara sann
Skapa en ny labyrint med matrisen som inparameter


%    Start with a grid full of walls.
%    Pick a cell, mark it as part of the maze. Add the walls of the cell to the wall list.
%    While there are walls in the list:
%        Pick a random wall from the list. We now have two cases: either there exists exactly one unvisited cell on one of the two sides of the chosen wall, or there does not. If it is the former:
%            Make the wall a passage and mark the unvisited cell as part of the maze.
%            Add the neighboring walls of the cell to the wall list.
%        Remove the wall from the list.


\Task I den här uppgiften ska du implementera en algoritm för att skapa en slumpmässig labyrint.

\Subtask Inspektera ovanstående "pseudokod" och försök förstå den. Fråga gärna om något är oklart! Läs också de färdigskrivna metoderna i metoden "random" i Maze och se om du kan förstå vad de gör.

\Subtask Implementera metoden random i Maze som skapar och returnerar en slumpmässigt utformad labyrint med hjälp av "pseudokoden" ovan (eller på egen hand för den modiga/nyfikna!). Ta även hjälp av de färdigskrivna metoderna addWallsToBuffer och checkWallsAround.

\Subtask Skapa en ny labyrint i AMazeIngRace genom att anropa din random-metod! Vad får du för resultat? Ett bra värde att använda på storleken när du anropar metoden är mellan 50 och 100. Dvs anropa metoden med till exempel random(50, 50). Om du lyckas få fram en labyrint och din random-metod är korrekt, testa att låta sköldpaddan gå igenom labyrinten och se vad som händer!
    
