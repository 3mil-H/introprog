%!TEX encoding = UTF-8 Unicode

%!TEX root = ../compendium.tex

\Lab{\LabWeekNINE}

\begin{Goals}
\item Skapa och iterera över matriser med nästlade for-loopar.
\item Använda sig av och förstå arv.
\item Träna på villkor med if-satser.
\item Få en inblick i algoritmer för att lösa problem så som att ta sig igenom en labyrint (?)
\end{Goals}

\begin{Preparations}
\item Läs om matriser.
\item Läs om arv.
\item Läs om olika algoritmer för att ta sig igenom en labyrint (https://en.wikipedia.org/wiki/Maze_solving_algorithm).
\end{Preparations}

\subsection{Bakgrund}
I denna laboration kommer du att få skapa labyrinter och sedan implementera algoritmer för att ta dig igenom dessa. En labyrint är ett rum som har en ingång och en utgång. Ingången är i de här fallen alltid längst ner i labyrinten, och utgången högst upp. Alla väggar är också parallella med antingen x-axeln eller y-axeln. Ett sätt att beskriva en sådan labyrint i kod är med hjälp av en matris. Varje element i matrisern motsvarar då en "ruta" (ordval? steg??) i labyrinten. Om ett element på en viss plats i matrisen är TRUE betyder det att på motsvarande plats i labyrinten finns en vägg, och om ett element är FALSE att det här i labyrinten finns en gång.

Det finns många olika sätt att ta sig igenom en labyrint men ett av de vanligaste och enklaste sätten att konstruera en algoritm är att hålla sin vänstra eller högra hand mot motsvarande vägg och följa väggen utan att släppa den med handen tills man når slutet av labyrinten. Detta funkar för alla labyrinter där väggarna från ingången till utgången är sammankopplade.

Det finns även flertalet algoritmer för att hitta den kortaste vägen igenom en labyrint. En av dessa är med hjälp av så kallad Bredth First Search. (Lägg till kort beskrivning av BFS!)


\subsection{Obligatoriska uppgifter}

\Task I denna uppgift ska du implementera en metod för att rita upp en labyrint i SimpleWindow.

\Subtask Läs igenom klassen Maze och se till att du förstår det mesta. Vad Maze gör är att den läser in rader med strängar, antingen från en fil eller direkt som argument, där tecknet '#' representerar en vägg och ' ' (eller vad för tecken ska väljas??) representerar en gång. Utifrån detta skapar Maze en Boolean-matris som representerar labyrinten. Fråga om något är oklart!

\Subtask Implementera metoden DRAW i Maze som ritar upp labyrinten i SimpleWindow. För att göra detta behöver du gå ingenom matrisen i Maze och undersöka elementen på varje plats. Om ett element är TRUE så betyder det att det här ska finnas en vägg i labyrinten, om FALSE att det här ska finnas en gång. Ta hjälp av metoderna buildWall och moveRight/moveUp som finns implementerade i Maze.

\Task En labbuppgiftsbeskrivning.

\Subtask Skapa en ny klass MazeMain (namn?) med en main-metod där du skapar ett objekt av Maze och sedan anropar metoden draw på detta. Läs in labyrinten från någon av filerna maze1.txt - maze5.txt eller skapa en egen labyrint. Båda alternativen ska fungera. Kolla att din labyrint ser ut som du tänkt dig, och fixa annars till den.

\Task I den här uppgiften ska du implementera en algoritm för att få en sköldpadda att ta sig genom en labyrint med hjälp av att alltid hålla i väggen med vänster hand (eller fot i det här fallet!).

\Subtask Skapa en ny klass MazeTurtle som ärver från klassen Turtle.

\Subtask Definiera i MazeTurtle en ny metod walk. Implementera sedan denna metod. I metoden ska sköldpaddan med hjälp av tekniken "hålla vänster hand i väggen" ta sig genom labyrinten, från början till slut. Varje steg motsvarar att flytta sig från en ruta till en annan i Boolean-matrisen i Maze.

\Subtask Lägg till kod i MazeMain som skapar en sköldpadda och sedan låter denna gå igenom labyrinten med metoden walk. Testa att din MazeTurtle fungerar som den ska! Sköldpaddan ska klara att ta sig igenom alla labyrinter i filerna maze1.txt - maze5.txt.

\subsection{Frivilliga extrauppgifter}

(Lägg till pseudokod för BFS-algoritm för att hitta "shortest path"!)

\Task I den här uppgiften ska du implementera en algoritm för att hitta den kortaste vägen genom en labyrint.

\Subtask Inspektera nedanstående pseudokod och försök förstå den. Fråga gärna om något är oklart!

\Subtask Lägg till en ny metod walkShortestPath i klassen MazeTurtle där sköldpaddan ska hitta och sedan gå den kortaste vägen genom en labyrint. Implementera denna metod med hjälp av den givna pseudokoden (eller på egen hand för den modiga/nyfikna!).

\Subtask Anropa metoden walkShortestPath i MazeMain! Jämför med algoritmen att hålla i väggen med vänster hand. Vilken sköldpadda är snabbast?
    
