%!TEX root = ../compendium.tex

\Exercise{\tt{expressions}}

\begin{Goals}
\item Lär dig detta
\item Lär dig och detta
\end{Goals}

\begin{Preparations}
\item Läs kap.~\ref{intro}
\item Säkerställ att du kan avända de grundläggande terminalkommandona \code{ls} och \code{cd} för att inspektera och navigera i filträdet, se kap.~\ref{terminal}. 
\item Du behöver en dator med scala installerad. Om du inte har Scala installerad på din maskin, se installationsanvisningar i kap.~\ref{installScala}
\item Starta den editor du vill använda under övningarna, se kap.~\ref{editor}.
\end{Preparations}

\BasicTasks

\TaskSection{Exekvera kod i REPL, som skript och som app.}

\TaskPen Starta Scala REPL och skriv satsen \code{println("hej")} och tryck på \textit{Enter}. 

\begin{REPL}
$ scala
Welcome to Scala version 2.11.7 (Java HotSpot(TM) 64-Bit Server VM, Java 1.8.0_66).
Type in expressions to have them evaluated.
Type :help for more information.

scala> println("hejsan REPL")
\end{REPL}

\Subtask Vad händer? 

\Subtask Skriv samma sats igen men ''glöm bort'' att skriva högerparentesen innan du trycker på \textit{Enter}. Vad händer?

\Subtask Evaulera uttrycket \code{"gurka" + "tomat"} i REPL. Vad har uttrycket för värde och typ? 
\begin{REPL}
scala> "gur" + "ka"   //tryck ENTER
\end{REPL}

\Subtask Evaluera uttrycket \code{res0 * 42} men byt ut \code{0}:an mot siffran efter \code{res} i utskriften vid förra evalueringen. Vad har uttrycket för värde och typ?
\begin{REPL}
scala> res1 * 42
\end{REPL}

\TaskPen Vad är det för skillnad på en sats och ett uttryck?

\Task Skapa med hjälp av en editor en fil med namn \texttt{hello-script.scala} som innehåller denna enda rad:
\begin{Code}
println("hej skript")
\end{Code}
Spara filen och kör detta kommandot \code{scala hello-script.scala} i terminalen:
\begin{REPL}
$ scala hello-script.scala
\end{REPL}


\Subtask Vad händer?

\Subtask Ändra i filen så att högerparentesen saknas. Spara och kör skriptfilen igen. Vad händer?

\Task Skapa med hjälp av en editor en fil med namn \texttt{hello-app.scala} som innehåller dessa rader:
\scalainputlisting{examples/hello-app.scala}

Kompilera och kör koden.

\Subtask Vilket alternativ går snabbast att köra igång, ett skript eller en kompilerad applikation? Varför?

\Subtask Hur kan koden se ut då kompilatorn ger följande felmeddelande: \\
\texttt{Missing closing brace}

\TaskSection{Aritmetriska uttryck}
\lipsum[7]

\ExtraTasks
\lipsum[2]

\AdvancedTasks
\lipsum[2]