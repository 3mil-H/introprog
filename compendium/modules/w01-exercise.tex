%!TEX encoding = UTF-8 Unicode
%!TEX root = ../compendium.tex

\Exercise{\ExeWeekONE}\label{exe:W01}

\begin{Goals}
\item Förstå vad som händer när satser exekveras och uttryck evalueras.
\item Förstå sekvens, alternativ och repetition.
\item Känna till literalerna för enkla värden, deras typer och omfång.
\item Kunna deklarera och använda variabler och tilldelning, samt kunna rita bilder av minnessituationen då variablers värden förändras.
\item Förstå skillnaden mellan olika numeriska typer, kunna omvandla mellan dessa och vara medveten om noggranhetsproblem som kan uppstå.
\item Förstå booelska uttryck och värdena \code{true} och \code{false}, samt kunna förenkla booelska uttryck.
\item Förstå skillnaden mellan heltalsdivision och flyttalsdivision, samt använding av rest vid heltalsdivision.
\item Förstå precedensregler och användning av parenteser i uttryck.
\item Kunna använda \code{if}-satser och \code{if}-uttryck.
\item Kunna anvädna \code{for}-satser och \code{while}-satser.
\item Kunna använda \code{math.random} för att generera slumptal i olika interval.
\end{Goals}

\begin{Preparations}
\item Studera teorin i kapitel~\ref{chapter:W01}.
\item Du behöver en dator med Scala installerad, se appendix~\ref{appendix:compile}.
\end{Preparations}

\BasicTasks

\Task Starta Scala REPL \Eng{Read-Evaluate-Print-Loop} och skriv satsen \code{println("hejsan REPL")} och tryck på \textit{Enter}. 

\begin{REPLnonum}
> scala
Welcome to Scala version 2.11.7 (Java HotSpot(TM) 64-Bit Server VM, Java 1.8).
Type in expressions to have them evaluated.
Type :help for more information.

scala> println("hejsan REPL")
\end{REPLnonum}

\Subtask Vad händer? 

\Subtask Skriv samma sats igen men ''glöm bort'' att skriva högerparentesen innan du trycker på \textit{Enter}. Vad händer?

\Subtask Evaulera uttrycket \code{"gurka" + "tomat"} i REPL. Vad har uttrycket för värde och typ? Vilken siffra står efter ordet \code{res} i variabeln som lagrar resultatet?

\begin{REPLnonum}
scala> "gurka" + "tomat"   
\end{REPLnonum}

\Subtask Evaluera uttrycket \code{res0 * 42} men byt ut \code{0}:an mot siffran efter \code{res} i utskriften från förra evalueringen. Vad har uttrycket för värde och typ?
\begin{REPLnonum}
scala> res0 * 42
\end{REPLnonum}


\Task\Pen Vad är en \textit{literal}? \\ \href{https://en.wikipedia.org/wiki/Literal\_\%28computer_programming\%29}{en.wikipedia.org/wiki/Literal\_(computer programming)}

\Task Vilken typ har följande literaler?

\Subtask \code{42} 

\Subtask \code{42L}

\Subtask \code{'*'}

\Subtask \code{"*"}

\Subtask \code{42.0}

\Subtask \code{42D}

\Subtask \code{42d}

\Subtask \code{42F}

\Subtask \code{42f}

\Subtask \code{true}

\Subtask \code{false}


\Task\Pen Vad gör dessa satser? Till vad används klammer och semikolon?
\begin{REPLnonum}
scala> def p = { print("hej"); print("san"); println(42); println("gurka") }
scala> p;p;p;p
\end{REPLnonum}

\Task\Pen Satser versus uttryck. 

\Subtask Vad är det för skillnad på en sats och ett uttryck?

\Subtask Ge exempel på satser som inte är uttryck?

\Subtask Förklara vad som händer för varje evaluerad rad:
\begin{REPL}
scala> def värdeSaknas = ()
scala> värdeSaknas
scala> värdeSaknas.toString
scala> println(värdeSaknas)
scala> println(println("hej"))
\end{REPL}

\Subtask Vilken typ har literalen \code{()}?

\Subtask Vilken returtyp har \code{println}?

\Task Vilken typ och vilket värde har följande uttryck? 

\Subtask \code{1 + 41}

\Subtask \code{1.0 + 41}

\Subtask \code{42.toDouble}

\Subtask \code{(41 + 1).toDouble}

\Subtask \code{1.042e42}

\Subtask \code{42E6.toLong}

\Subtask \code{"gurk" + 'a'}

\Subtask \code{'A'}

\Subtask \code{'A'.toInt}

\Subtask \linebreak[0] \code{'0'.toInt}

\Subtask \code{'1'.toInt}

\Subtask \code{'9'.toInt}

\Subtask \code{('A' + '0').toChar}

\Subtask \code{"*!%#".charAt(0)}

\Task \textit{De fyra räknesätten}. Vilket värde och vilken typ har följande uttryck?

\Subtask \code{42 * 2}

\Subtask \code{42.0 / 2}

\Subtask \code{42 - 0.2}

\Subtask \code{42L + 2d}

\Task \textit{Precedensregler}. Evalueringsordningen kan styras med parenteser. Vilket värde och vilken typ har följande uttryck?

\Subtask \code{42 + 2 * 2}

\Subtask \code{(42 + 2) * 2}

\Subtask \code{(-(2 - 42)) / (1 + 1 + 1).toDouble}

\Subtask \code{((-(2 - 42)) / (1 + 1 + 1).toDouble).toInt}


\Task \textit{Heltalsdivision}. Vilket värde och vilken typ har följande uttryck?

\Subtask \code{42 / 2}

\Subtask \code{42 / 4}

\Subtask \code{42.0 / 4}

\Subtask \code{1 / 4}

\Subtask \code{1 % 4}

\Subtask \code{2 % 42}

\Subtask \code{42 % 2}


\Task \textit{Hetalsomfång}. För var och en av heltalstyperna i deluppgifterna nedan: undersök i REPL med operationen \code{MaxValue} resp. \code{MinValue}, vad som är största och minsta värde, till exempel \code{Int.MaxValue} etc. 

\Subtask \code{Byte}

\Subtask \code{Short}

\Subtask \code{Int}

\Subtask \code{Long}

\Task Klassen \code{java.lang.Math} och paketobjektet \code{scala.math}.

\begin{REPL}
scala> java.lang.Math.    //tryck TAB
scala> scala.math.        //tryck TAB
\end{REPL}

\Subtask Undersök genom att trycka på Tab-tangenten, vilka funktioner som finns i \code{Math} och \code{math}. Vad heter konstanten $\pi$ i java.lang.Math respektive scala.math?


\Subtask Undersök dokumentationen för klassen \code{java.lang.Math} här: \\ \url{https://docs.oracle.com/javase/8/docs/api/java/lang/Math.html} \\
Vad gör \code{java.lang.Math.hypot}?

\Subtask Undersök dokumentationen för pakobjektet \code{scala.math} här: \\
\url{http://www.scala-lang.org/api/current/#scala.math.package} \\
Ge exempel på någon funktion i \code{java.lang.Math} som inte finns i \code{scala.math}.

%\TaskSection{Noggranhet och undantag i aritmetiska uttryck}

\Task Vad händer här? Notera undantag \Eng{exceptions} och nogranhetsproblem. 

\Subtask \code{Int.MaxValue} + 1

\Subtask \code{1 / 0}

\Subtask \code{1E8 + 1E-8}

\Subtask \code{1E9 + 1E-9}

\Subtask \code{math.pow(math.hypot(3,6), 2)}

\Subtask \code{1.0 / 0}

\Subtask \code{(1.0 / 0).toInt}

\Subtask \code{math.sqrt(-1)}

\Subtask \code{math.sqrt(Double.NaN)}

\Subtask \code{throw new Exception("PANG!!!")}


\Task \textit{Booelska uttryck}. Vilket värde och vilken typ har följande uttryck?

\Subtask \code{true && true}

\Subtask \code{false && true}

\Subtask \code{true && false}

\Subtask \code{false && false}

\Subtask \code{true || true}

\Subtask \code{false || true}

\Subtask \code{true || false}

\Subtask \code{false || false}

\Subtask \code{42 == 42}

\Subtask \code{42 != 42}

\Subtask \code{42.0001 == 42}

\Subtask \code{42.0000000000000001 == 42}

\Subtask \code{42.0001 > 42}

\Subtask \code{42.0000000000000001 > 42}

\Subtask \code{42.0001 >= 42}

\Subtask \code{42.0000000000000001 <= 42}

\Subtask \code{true == true}

\Subtask \code{true != true}

\Subtask \code{true > false}

\Subtask \code{true < false}

\Subtask \code{'A' == 65}

\Subtask \code{'S' != 66}


\Task\Pen \textit {Variabler och tilldelning}. Rita en ny bild av datorns minne efter varje evaluerad rad nedan. Bilderna ska visa variablers namn, typ och värde. 
\begin{REPL}[numbers=left, numberstyle=\color{black}\ttfamily\scriptsize\selectfont]
scala> var a = 42
scala> var b = a + 1
scala> var c = (a + b) * 2.0
scala> b = 0
scala> a = 0
scala> c = c + 1
\end{REPL}
Efter första raden ser minnessituationen ut så här:

\vspace{0.5em}
\begin{tikzpicture}[font=\ttfamily]
\matrix [matrix of nodes, row sep=0, column 2/.style={nodes={rectangle,draw,minimum width=4em}}] (mat)
{
a: Int   &  \makebox(16,12){42}\\
};
\end{tikzpicture}

\Task \textit{Deklarationer: \code{var}, \code{val}, \code{def}}. Evaluera varje rad nedan i tur och ordning i Scala REPL. 
\begin{REPL}[numbers=left, numberstyle=\color{black}\ttfamily\scriptsize\selectfont]
scala> var x = 42
scala> x + 1
scala> x
scala> x = x + 1
scala> x
scala> x == x + 1
scala> val y = 42
scala> y = y + 1
scala> var z = {println("gurka"); 42}
scala> def w = {println("gurka"); 42}
scala> z
scala> z
scala> z = z + 1
scala> w
scala> w
scala> w = w + 1
\end{REPL}

\Subtask För varje rad ovan: förklara för vad som händer. 

\Subtask Vilka rader ger kompileringsfel och i så fall vilket och varför?

\Subtask\Pen Vad är det för skillnad på \code{var}, \code{val} och \code{def}?

\Task\Pen \emph{Tilldelningsoperatorer.} Man kan förkorta en tilldelningssats som förändrar en variabel, t.ex. \code{x = x + 1}, genom att använda så kallade tilldelningsoperatorer och skriva \code{x += 1} som betyder samma sak. Rita en ny bild av datorns minne efter varje evaluerad rad nedan. Bilderna ska visa variablers namn, typ och värde. 

\begin{REPL}
scala> var a = 42
scala> var b = a + 42
scala> a += 10
scala> b -= 10
scala> a *= 2
scala> b /= 2
\end{REPL}

\Task \emph{Stränginterpolatorn \code{s}.} Man behöver ofta skapa strängar som innehåller variabelvärden. Med ett \code{s} framför en strängliteral får man hjälp av kompilatorn att infoga variabelvärden i en sträng. Variablernas namn ska föregås med ett \code{$}-tecken, t.ex. \code{s"Hej $namn"}. 

\begin{REPL}
scala> val f = "Kim"
scala> val e = "Robinsson"
scala> val tot = f.size + e.size
scala> println(s"Namnet '$f $e' har $tot bokstäver.") 
\end{REPL}

\Subtask Vad skrivs ut ovan?

\Subtask Skapa följande utskrifter med hjälp av stränginterpolatorn \code{s} och lämpliga variabler.
\begin{REPL}
Namnet 'Kim' har 3 bokstäver.
Namnet 'Robinsson' har 9 bokstäver.
\end{REPL}



\Task \code{if}\textit{-sats}.För varje rad nedan; förklara vad som händer.
\begin{REPL}
scala> if (true) println("sant") else println("falskt")
scala> if (false) println("sant") else println("falskt")
scala> if (!true) println("sant") else println("falskt")
scala> if (!false) println("sant") else println("falskt")
scala> def kasta = if (math.random > 0.5) print(" krona") else print(" klave")
scala> kasta; kasta; kasta
\end{REPL}


\Task \code{if}\textit{-uttryck}. Deklarera följande variabler med nedan initialvärden:

\begin{REPLnonum}
scala> var grönsak = "gurka"
scala> var frukt = "banan"
\end{REPLnonum}

Vad har följande uttryck för värden och typ?

\Subtask \code{if (grönsak == "tomat") "gott" else "inte gott" }

\Subtask \code{if (frukt == "banan") "gott" else "inte gott" }

\Subtask \code{if (frukt.size == grönsak.size) "lika stora" else "olika stora" }

\Subtask \code{if (true) grönsak else frukt }

\Subtask \code{if (false) grönsak else frukt }


\Task \code{for}\textit{-sats}.

\Subtask Vad ger nedan \code{for}-satser för utskrift?

\begin{REPL}
scala> for (i <- 1 to 10) print(i + ", ")
scala> for (i <- 1 until 10) print(i + ", ")
scala> for (i <- 1 to 5) print((i * 2) + ", ")
scala> for (i <- 1 to 92 by 10) print(i + ", ")
scala> for (i <- 10 to 1 by -1) print(i + ", ")
\end{REPL}

\Subtask Skriv en \code{for}-sats som ger följande utskrift:
\begin{REPLnonum}
A1, A4, A7, A10, A13, A16, A19, A22, A25, A28, A31, A34, A37, A40, A43, 
\end{REPLnonum}

\Task Repetition med \code{foreach}.

\Subtask Vad ger nedan satser för utskrifter?

\begin{REPL}
scala> (9 to 19).foreach{i => print(i + ", ")}
scala> (1 until 20).foreach{i => print(i + ", ")}
scala> (0 to 33 by 3).foreach{i => print(i + ", ")}
\end{REPL}

\Subtask Använd \code{foreach} och skriv ut följande:
\begin{REPLnonum}
B33, B30, B27, B24, B21, B18, B15, B12, B9, B6, B3, B0, 
\end{REPLnonum}

\Task \code{while}\textit{-sats}. 

\Subtask Vad ger nedan satser för utskrifter?
\begin{REPL}
scala> var i = 0
scala> while (i < 10) { println(i); i = i + 1 }
scala> var j = 0; while (j <= 10) { println(j); j = j + 2 }; println(j)
\end{REPL}

\Subtask Skriv en \code{while}-sats som ger följande utskrift. Använd en variabel \code{k} som initialiseras till 1.
\begin{REPLnonum}
A1, A4, A7, A10, A13, A16, A19, A22, A25, A28, A31, A34, A37, A40, A43, 
\end{REPLnonum}

\Subtask\Pen Vilken av \code{for}, \code{while} och \code{foreach} är kortast att skriva om man vill repetera mer än en sats 100 gånger? Vilken tycker du är lättast att läsa? 

\Task \textit{Slumptal}. Undersök vad dokumentationen säger om funktionen \code{scala.math.random}:\\
\url{http://www.scala-lang.org/api/current/#scala.math.package} 

\Subtask\Pen Vilken typ har värdet som returneras av funktionen \code{random}? 
 
\Subtask\Pen Vilket är det minsta respektive största värde som kan returneras? 

\Subtask\Pen Är \code{random} en \textit{äkta} funktion \Eng{pure function} i matematisk mening?

\Subtask Anropa funktionen \code{math.random} upprepade gånger och notera vad som händer. Använd pil-upp-tangenten.
\begin{REPLnonum}
scala> math.random
\end{REPLnonum}


\Subtask Vad händer? Använd \textit{pil-upp} och kör nedan \code{for}-sats flera gånger. Förklara vad som sker.

\begin{REPLnonum}
scala> for (i <- 1 to 10) println(math.random)
\end{REPLnonum}

\Subtask Skriv en for-sats som skriver ut 100 slumpmässiga heltal från 0 till och med 9 på var sin rad. 

\begin{REPLnonum}
scala> for (i <- 1 to 100) println(???)
\end{REPLnonum}

\Subtask Skriv en for-sats som skriver ut 100 slumpmässiga heltal från 1 till och med 6 på samm rad. 

\begin{REPLnonum}
scala> for (i <- 1 to 100) print(???)
\end{REPLnonum}


\Subtask Använd \textit{pil-upp} och kör nedan \code{while}-sats flera gånger. Förklara vad som sker.

\begin{REPLnonum}
scala> while (math.random > 0.2) println("gurka")
\end{REPLnonum}

\Subtask Ändra i \code{while}-satsen ovan så att sannolikheten ökar att riktigt många  strängar ska skrivs ut.

\Subtask Förklara vad som händer nedan.
\begin{REPL}
scala> var slumptal = math.random
scala> while (slumptal > 0.2) { println(slumptal); slumptal = math.random }
\end{REPL}

\Task\Pen \textit{Logik och De Morgans Lagar}. Förenkla följande uttryck. Antag att \code{poäng} och \code{highscore} är heltalsvariabler medan \code{klar} är av typen \code{Boolean}. 

\Subtask \code{poäng > 100 && poäng > 1000}

\Subtask \code{poäng > 100 || poäng > 1000}

\Subtask \code{!(poäng > highscore)}

\Subtask \code{!(poäng > 0 && poäng < highscore) }

\Subtask \code{!(poäng < 0 || poäng > highscore) }

\Subtask \code{klar == true}

\Subtask \code{klar == false}


\ExtraTasks

\Task \textit{Slumptal}.

\Subtask Ersätt \code{???} nedan med literaler så att \code{tärning} returnerar ett slumpmässigt heltal mellan 1 och 6.
\begin{REPLnonum}
scala> def tärning = (math.random * ??? + ???).toInt 
\end{REPLnonum}

\Subtask Ersätt \code{???} med literaler så att rnd blir ett decimaltal med max en decimal mellan 0.0 och 1.0.
\begin{REPLnonum}
scala> def rnd = math.round(math.random * ???) / ??? 
\end{REPLnonum}

\Subtask Vad blir det för skillnad om \code{math.round} ersätts med \code{math.floor} ovan? (Se dokumentationen av \code{java.lang.Math.round} och \code{java.lang.Math.floor}.)

\AdvancedTasks

\Task Läs om moduloräkning här \href{https://en.wikipedia.org/wiki/Modulo\_operation}{en.wikipedia.org/wiki/Modulo\_operation} och undersök hur tecknet blir med olika tecken på divisor och dividend.

\Task Integer.toBinaryString, Integer.toHexString

\Task Typannoteringar.

\Task 0x2a

\Task \code{i += 1; i *= 1; i /= 2}

\Task BigInt, BigDecimal

\Task Vad händer här? 
\begin{REPLnonum}
scala> Math.multiplyExact(2, 42)
scala> Math.multiplyExact(Int.MaxValue, Int.MaxValue)
\end{REPLnonum}

\Task Sök reda på dokumentationen i javadoc för klassen \code{java.lang.Math} i JDK 8. Tryck Ctrl+F i webbläsaren och sök efter förekomster av texten \textit{''overflow''}. Vad är ''overflow''? Vilka metoder finns i \code{java.lang.Math} som hjälper dig att upptäcka om det blir overflow?

\Task Använda Scala REPL för att undersöka konstanterna nedan. Vilket av dessa värden är negativt? Vad kan man ha för praktisk nytta av dessa värden i ett program som gör flyttalsberäkningar?

\Subtask \code{java.lang.Double.MIN_VALUE}

\Subtask \code{scala.Double.MinValue} 

\Subtask \code{scala.Double.MinPositiveValue}

\Task För typerna \code{Byte}, \code{Short}, \code{Char}, \code{Int}, \code{Long}, \code{Float}, \code{Double}: Undersök hur många bitar som behövs för att representera varje typs omfång? \\*
\textit{Tips:} Några användbara uttryck: \\*
 \code{Integer.toBinaryString(Int.MaxValue + 1).size} \\*
 \code{Integer.toBinaryString((math.pow(2,16) - 1).toInt).size} \\*
 \code{1 + math.log(Long.MaxValue)/math.log(2)}
Se även språkspecifikationen för Scala, kapitlet om heltalsliteraler: \\
\url{http://www.scala-lang.org/files/archive/spec/2.11/01-lexical-syntax.html#integer-literals}

\Subtask Undersök källkoden för pakobjektet \code{scala.math} här: \\
\url{https://github.com/scala/scala/blob/v2.11.7/src/library/scala/math/package.scala} \\
Hur många olika överlagrade varianter av funktionen \code{abs} finns det och för vilka parametertyper är den definierad?

\Task Läs mer om stränginterpolatorer här:\\ \href{http://docs.scala-lang.org/overviews/core/string-interpolation.html}{docs.scala-lang.org/overviews/core/string-interpolation.html} \\ Hur kan du använda \code{f}-interpolatorn för att göra följande utskrift i REPL? Byt ut \code{???} mot lämpliga tecken.
\begin{REPLnonum}
scala> val g: Double = 1 / 3.0
scala> val s: String = f"Gurkan är ??? meter lång"
scala> println(s)
Gurkan är 0.333 meter lång
\end{REPLnonum}


