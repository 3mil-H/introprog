%!TEX encoding = UTF-8 Unicode
%!TEX root = ../solutions.tex


%\BasicTasks %%%%%%%%%%%
% uppgift 2

\noindent\TODO{HÄREFTER KOMMER GAMLA LÖSNINGAR FRÅN \code{-solutions.tex}}


\AdvancedTasks %%%%%%%%%

\Task % Uppgift 32

\Subtask
Variabeln får namnet "konstig val", backtick gör att man kan namge variabler till annars otilåtna namn t.ex. med mellanrum eller nyckelord i sig.
 
\Subtask
Backticks tillåter en att anropa metoder som heter samma som nyckelord i scala. I java får man döpa en metod till t.ex. yield men ska man anropa metoden i scala krävs då backticks för att yield är ett nyckel ord. java.Thread.`yield`()

\Task % Uppgift 33
\Subtask
\code{toBinaryString} gör om heltalet till en sträng med ettor och nollor som är den binära versionen utav talet. \code{toHexString} gör sama sak fast till ett hexadecimalt tal.
\Subtask
42

\Task % Uppgift 34

\Task % Uppgift 35
 först blir \code{i 42}. \code{i} blir sedan 43 och multipliceras med 2 och blir 86. Efter den delas med 3 blir den 28 eftersom \code{Int} inte har några decimaler.

\Task % Uppgift 36

\Task % Uppgift 37

\Subtask 
Den första raden returnerar 84. Den andra kastar ett exception.

\Subtask
För att kunna hantera situationer när bydelängden på variabler inte är lång nog för värden.

\Subtask
Overflow är när en variabel inte kan inehålla ett värde då det är för stort och istället  blir ett värde som variabeln egentligen inte ska få.

\Task % Uppgift 38

\Subtask
\code{4.9E-3240}

\Subtask
\code{-1.7976931348623157E308}

\Subtask
\code{4.9E-324}

\Task % Uppgift 39

\Task % Uppgift 40

\begin{Code}
val s = f"Gurkan är $g meter lång"
\end{Code}






