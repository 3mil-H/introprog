
%!TEX encoding = UTF-8 Unicode
%!TEX root = ../exercises.tex

\ifPreSolution

\Exercise{\ExeWeekTWO}\label{exe:W02}
\begin{Goals}
%!TEX encoding = UTF-8 Unicode
%!TEX root = ../compendium2.tex

\item Kunna skapa samlingarna Range, Array och Vector med heltals- och strängvärden.
\item Kunna indexera i en indexerbar samling, t.ex. Array och Vector.
\item Kunna anropa operationerna size, mkString, sum, min, max på samlingar som innehåller heltal.
\item Känna till grundläggande skillnader och likheter mellan samlingarna Range, Array och Vector.
\item Förstå skillnaden mellan en for-sats och ett for-uttryck.
\item Kunna skapa samlingar med heltalsvärden som resultat av enkla for-uttryck.
\item Förstå skillnaden mellan en algoritm i pseudo-kod och dess implementation.
\item Kunna implementera algoritmerna SUM, MIN/MAX på en indexerbar samling med en \code{while}-sats.
\item Kunna köra igång enkel Scala-kod i REPL, som skript och som applikation.
\item Kunna skriva och köra igång ett enkelt Java-program.
\item Känna till några grundläggande syntaxskillnader mellan Scala och Java, speciellt variabeldeklarationer och indexering i Array.
\item Förstå vad ett block och en lokal variabel är.
\item Förstå hur nästlade block påverkar namnsynlighet och namnöverskuggning.
\item Förstå kopplingen mellan paketstruktur och kodfilstruktur.
\item Kunna skapa en jar-fil.
\item Kunna skapa dokumentation med scaladoc.

\end{Goals}

\begin{Preparations}
\item \StudyTheory{02}
\item Bekanta dig med grundläggande terminalkommandon, se appendix~\ref{appendix:terminal}.
\item Bekanta dig med den editor du vill använda, se appendix~\ref{appendix:compile}.
\end{Preparations}

\else

\ExerciseSolution{\ExeWeekTWO}

\fi


% TODO fundera på detta:
% terminalkommando
% scalac -> hello world; scala som script; javac
% paket, import, jar, main,



\BasicTasksNoLab %%%%%%%%%%%%%%%%




\WHAT{Para ihop begrepp med beskrivning.}

\QUESTBEGIN

\Task \what

\vspace{1em}\noindent Koppla varje begrepp med den (förenklade) beskrivning som passar bäst:

\begin{ConceptConnections}
  kompilerad & 1 & & A & datastruktur med element av samma typ \\ 
  skript & 2 & & B & en oföränderlig, indexerbar sekvenssamling \\ 
  objekt & 3 & & C & en specifik realisering av en algoritm \\ 
  main & 4 & & D & maskinkod sparas ej utan skapas vid varje körning \\ 
  programargument & 5 & & E & applicerar en funktion på varje element i en samling \\ 
  datastruktur & 6 & & F & datastruktur med element i en viss ordning \\ 
  samling & 7 & & G & där exekveringen av kompilerad app startar \\ 
  sekvenssamling & 8 & & H & maskinkod sparad och kan köras igen utan kompilering \\ 
  Array & 9 & & I & en förändringsbar, indexerbar sekvenssamling \\ 
  Vector & 10 & & J & stegvis beskrivning av en lösning på ett problem \\ 
  Range & 11 & & K & en samling som representerar ett intervall av heltal \\ 
  yield & 12 & & L & samlar variabler och funktioner \\ 
  map & 13 & & M & många olika element i en helhet; elementvis åtkomst \\ 
  algoritm & 14 & & N & överförs via parametern args i main \\ 
  implementation & 15 & & O & används i for-uttryck för att skapa ny samling \\ 
\end{ConceptConnections}

\SOLUTION

\TaskSolved \what

\begin{ConceptConnections}
  kompilerad & 1 & ~~\Large$\leadsto$~~ &  L & maskinkod sparad och kan köras igen utan kompilering \\ 
  skript & 2 & ~~\Large$\leadsto$~~ &  F & maskinkod sparas ej utan skapas vid varje körning \\ 
  objekt & 3 & ~~\Large$\leadsto$~~ &  N & samlar variabler och funktioner \\ 
  main & 4 & ~~\Large$\leadsto$~~ &  G & där exekveringen av kompilerad app startar \\ 
  programargument & 5 & ~~\Large$\leadsto$~~ &  I & överförs via parametern args i main \\ 
  datastruktur & 6 & ~~\Large$\leadsto$~~ &  E & många olika element i en helhet; elementvis åtkomst \\ 
  samling & 7 & ~~\Large$\leadsto$~~ &  O & datastruktur med element av samma typ \\ 
  sekvenssamling & 8 & ~~\Large$\leadsto$~~ &  K & datastruktur med element i en viss ordning \\ 
  Array & 9 & ~~\Large$\leadsto$~~ &  B & en förändringsbar, indexerbar sekvenssamling \\ 
  Vector & 10 & ~~\Large$\leadsto$~~ &  D & en oföränderlig, indexerbar sekvenssamling \\ 
  Range & 11 & ~~\Large$\leadsto$~~ &  J & en samling som representerar ett intervall av heltal \\ 
  yield & 12 & ~~\Large$\leadsto$~~ &  C & används i for-uttryck för att skapa ny samling \\ 
  map & 13 & ~~\Large$\leadsto$~~ &  A & applicerar en funktion på varje element i en samling \\ 
  algoritm & 14 & ~~\Large$\leadsto$~~ &  M & stegvis beskrivning av en lösning på ett problem \\ 
  implementation & 15 & ~~\Large$\leadsto$~~ &  H & en specifik realisering av en algoritm \\ 
\end{ConceptConnections}

\QUESTEND





\WHAT{Använda terminalen.}

\QUESTBEGIN

\Task \what~Läs om terminalen i appendix \ref{appendix:terminal}.

\Subtask Vilka tre kommando ska du köra för att 1) skapa en katalog med namnet \code{hello} och 2)  navigera till katalogen och 3) visa namnet på ut aktuell katalog? Öppna ett teminalfönster och kör dessa tre kommando.

\Subtask Vilka två kommando ska du köra för att 1) navigera tillbaka ''upp'' ett steg i filträdet och 2) lista alla filer och kataloger på denna plats? Kör dessa två kommando i terminalen.

\SOLUTION

\TaskSolved \what

\SubtaskSolved

\begin{REPL}
> mkdir hello
> cd hello
> pwd
\end{REPL}

\SubtaskSolved

\begin{REPL}
> cd ..
> ls
\end{REPL}


\QUESTEND









\WHAT{Skapa och köra ett Scala-skript.}

\QUESTBEGIN

\Task  \what~

\Subtask Skapa en fil med namn \texttt{sum.scala} i katalogen \code{hello} som du skapade i föregående uppgift med hjälp av en editor, t.ex. \code{atom}.
\begin{REPLnonum}
> cd hello
> atom sum.scala
\end{REPLnonum}

\noindent Filen ska innehålla dessa tre rader:
\scalainputlisting{examples/sum.scala}

\noindent Spara filen och kör kommandot \code{scala sum.scala} i terminalen:
\begin{REPLnonum}
> scala sum.scala
\end{REPLnonum}

\noindent Vad blir summan av de $1000$ första talen?

\Subtask Ändra i filen \code{sum.scala} så att högerparentesen på sista raden saknas. Spara filen (Ctrl+S) och kör skriptfilen igen i terminalen (pil-upp). Hur lyder felmeddelandet? Är det ett körtidsfel eller ett kompileringsfel?

\Subtask Ändra i \code{sum.scala} så att det i stället för \code{1000} står \code{args(0).toInt} efter \code{val n =} och spara och kör om ditt program med argumentet 5001 så här:
\begin{REPL}
> scala sum.scala 5001
\end{REPL}
\noindent Vad blir summan av de $5001$ första talen?

\Subtask Vad blir det för felmeddelande om du glömmer ge skriptet ett argument? Är det ett körtidsfel eller ett kompileringsfel?

\SOLUTION

\TaskSolved \what

\SubtaskSolved
\begin{REPL}
Summan av de 1000 första talen är: 500500
\end{REPL}

\SubtaskSolved  Kompileringsfelet blir: \code{error: ')' expected but eof found}

\SubtaskSolved  Filen ska se ut så här:
\begin{Code}
val n = args(0).toInt
val summa = (1 to n).sum
println(s"Summan av de $n första talen är: $summa")
\end{Code}

Utskriften blir så här:
\begin{REPL}
Summan av de 5001 första talen är: 12507501
\end{REPL}

\SubtaskSolved Körtidsfelet blir: \code{java.lang.ArrayIndexOutOfBoundsException: 0}\\(Anledningen är att arrayen \code{args} blir tom om programargument saknas och platsen med index $0$ därmed inte finns.)

\QUESTEND





\WHAT{Scala-applikation med \code+main+-metod.}

\QUESTBEGIN

\Task  \what~  Skapa med hjälp av en editor en fil med namn \texttt{hello.scala}.
\begin{REPLnonum}
> atom hello.scala
\end{REPLnonum}
Skriv nedan kod i filen:


\scalainputlisting{examples/hello.scala}

\Subtask Kompilera med \code{scalac hello.scala} och kör koden med \code{scala Hello}. Notera stor bokstav i klassnamnet. Vad heter filerna som kompilatorn skapar?
\begin{REPLnonum}
> scalac hello.scala
> ls
> scala Hello
\end{REPLnonum}

\Subtask Hur ska du ändra i din kod så att kompilatorn ger följande felmeddelande: \\
\texttt{Missing closing brace}

\Subtask Varför behövs \code{main}-metoden?

\Subtask Vilket alternativ går snabbast att köra igång, ett skript eller en kompilerad applikation? Varför? Vilket alternativ kör snabbast när väl exekveringen är igång?


\SOLUTION


\TaskSolved \what


\SubtaskSolved  Filerna som kompilatorn skapat heter \code{Hello.class} och \verb+Hello\$.class+

\SubtaskSolved  Felmeddelandet får du om du tar bort den sista krullparentesen.

\SubtaskSolved Ett kompilerat program måste ha ett objekt med en \code{main}-metod eftersom exekveringsmiljön JVM förutsätter detta och anropar \code{main}-metoden när exekveringen startar.

\SubtaskSolved
  Det går snabbare att göra i gång ett kompilerat program eftersom maskinkoden är sparad i en fil som kan köras igång direkt.  Ett kompilerat program kan bestå av många filer som samkompileras.

  När ett skript kör kompileras koden i skriptfilen före \emph{varje} körning och maskinkoden sparas inte mellan körningarna. Ett skript består bara av en enda text-fil som körs från början. Ingen main-metod behövs.

  När väl exekveringen är igång sker exekveringen av maskinkoden exakt lika snabbt oberoende av om koden är genererad ur ett skript eller en förkompilerad app.


\QUESTEND







\WHAT{Java-applikation.}

\QUESTBEGIN

\Task \label{task:java} \what~   Skapa med hjälp av en editor en fil med namn \texttt{Hi.java}. Notera stor bokstav. I ett Java-program måste namnet före \code{.java} stämma överens exakt med klassnamnet.
\begin{REPLnonum}
> atom Hi.java
\end{REPLnonum}
Skriv dessa rader i filen:
\javainputlisting{examples/Hi.java}
\noindent Kompilera med \code{javac Hi.java} och kör koden med \code{java Hi}.
\begin{REPLnonum}
> javac Hi.java
> ls
> java Hi
\end{REPLnonum}

\Subtask Vad heter filen som kompilatorn skapat?

\Subtask Jämför Java-programmet ovan med Scala-programmet i föregående uppgift. Programmen gör samma sak men syntaxen (hur koden ska skrivas) skiljer sig åt och det finns vissa skillnader i semantiken (vad koden betyder). Vi ska senare i kursen gå igenom \emph{exakt} vad varje fragment nedan betyder, men försök redan nu para ihop de Scala-delar till vänster som (ungefär) motsvarar de Java-delar som finns till höger.

\begin{ConceptConnections}
  \code|object| & 1 & & A & \jcode|public static main| \\ 
  \code|def main| & 2 & & B & \jcode|public class| \\ 
  \code|Array[String]| & 3 & & C & \jcode|void| \\ 
  \code|: Unit| & 4 & & D & \jcode|) {| \\ 
  \code|=| & 5 & & E & \jcode|System.out.println| \\ 
  \code|println| & 6 & & F & \jcode|String[]| \\ 
\end{ConceptConnections}

\SOLUTION


\TaskSolved \what


\SubtaskSolved  Hi.class

\SubtaskSolved

\begin{ConceptConnections}
  \code|object| & 1 & ~~\Large$\leadsto$~~ &  E & \jcode|public class| \\ 
  \code|def main| & 2 & ~~\Large$\leadsto$~~ &  B & \jcode|public static main| \\ 
  \code|Array[String]| & 3 & ~~\Large$\leadsto$~~ &  F & \jcode|String[]| \\ 
  \code|: Unit| & 4 & ~~\Large$\leadsto$~~ &  A & \jcode|void| \\ 
  \code|=| & 5 & ~~\Large$\leadsto$~~ &  D & \jcode|) {| \\ 
  \code|println| & 6 & ~~\Large$\leadsto$~~ &  C & \jcode|System.out.println| \\ 
\end{ConceptConnections}


\QUESTEND




\WHAT{Skapa och använda samlingar.}

\QUESTBEGIN

\Task \what~I Scalas standardbibliotek finns många olika samlingar som går att använda på ett enhetligt sätt (med vissa undantag för \code{Array}). Para ihop uttrycken som skapar eller använder samlingar med förklaringarna, så att alla kopplingar blir korrekta (minst en förklaring passar med mer än ett uttryck, men det finns bara en lösning där alla kopplingar blir parvis korrekta):

\begin{ConceptConnections}
  \code|val xs = Vector(2) | & 1 & & A & ny samling med en nolla tillagd i början \\ 
  \code|Array.fill(9)(0)   | & 2 & & B & ny samling med en nolla tillagd på slutet \\ 
  \code|Vector.fill(9)(' ')| & 3 & & C & förkortad skrivning av \code|apply(0)| \\ 
  \code|xs(0)              | & 4 & & D & ny förändringsbar sekvens med nollor \\ 
  \code|xs.apply(0)        | & 5 & & E & ny samling, elementen omgjorda till heltal \\ 
  \code|xs :+ 0            | & 6 & & F & ny samling, elementen omgjorda till strängar \\ 
  \code|0 +: xs            | & 7 & & G & ny sträng med alla element intill varandra \\ 
  \code|xs.mkString        | & 8 & & H & ny sträng med komma mellan elementen \\ 
  \code|xs.mkString(",") | & 9 & & I & indexering, ger första elementet \\ 
  \code|xs.map(_.toString) | & 10 & & J & ny oföränderlig sekvens med blanktecken \\ 
  \code|xs map (_.toInt)   | & 11 & & K & ny referens till sekvens av längd 1 \\ 
\end{ConceptConnections}

\noindent Träna med dina egna varianter i REPL tills du lärt dig använda uttryck som ovan utantill. Då har du lättare att komma igång med kommande laborationer.

\SOLUTION

\TaskSolved \what

\begin{ConceptConnections}
  \code|val xs = Vector(2) | & 1 & ~~\Large$\leadsto$~~ &  J & ny referens till sekvens av längd 1 \\ 
  \code|Array.fill(9)(0)   | & 2 & ~~\Large$\leadsto$~~ &  E & ny förändringsbar sekvens med nollor \\ 
  \code|Vector.fill(9)(' ')| & 3 & ~~\Large$\leadsto$~~ &  D & ny oföränderlig sekvens med blanktecken \\ 
  \code|xs(0)              | & 4 & ~~\Large$\leadsto$~~ &  C & förkortad skrivning av \code|apply(0)| \\ 
  \code|xs.apply(0)        | & 5 & ~~\Large$\leadsto$~~ &  H & indexering, ger första elementet \\ 
  \code|xs :+ 0            | & 6 & ~~\Large$\leadsto$~~ &  B & ny samling med en nolla tillagd på slutet \\ 
  \code|0 +: xs            | & 7 & ~~\Large$\leadsto$~~ &  F & ny samling med en nolla tillagd i början \\ 
  \code|xs.mkString        | & 8 & ~~\Large$\leadsto$~~ &  A & ny sträng med alla element intill varandra \\ 
  \code|xs.mkString(",") | & 9 & ~~\Large$\leadsto$~~ &  G & ny sträng med komma mellan elementen \\ 
  \code|xs.map(_.toString) | & 10 & ~~\Large$\leadsto$~~ &  K & ny samling, elementen omgjorda till strängar \\ 
  \code|xs map (_.toInt)   | & 11 & ~~\Large$\leadsto$~~ &  I & ny samling, elementen omgjorda till heltal \\ 
\end{ConceptConnections}

\QUESTEND





\WHAT{Jämför \code{Array} och \code{Vector}.}

\QUESTBEGIN

\Task \what~Para ihop varje samlingstyp med den beskrivning som passar bäst:

\Subtask Vad gäller angående föränderlighet \Eng{mutability}?

\begin{ConceptConnections}
  Vector & 1 & & A & förändringsbar \\ 
  Array & 2 & & B & oföränderlig \\ 
\end{ConceptConnections}

\Subtask Vad gäller vid tillägg av element i början \Eng{prepend} och slutet \Eng{append}, eller förändring av delsekvens på godtycklig plats (eng. \emph{to patch}, även på svenska: \emph{att patcha})?

\begin{ConceptConnections}
  Vector & 1 & & A & långsam vid ändring av storlek (kopiering av rubbet krävs) \\ 
  Array & 2 & & B & varianter med fler/andra element skapas snabbt ur befintlig \\ 
\end{ConceptConnections}

\Subtask Vad gäller vid likhetstest \Eng{equality test}.

\begin{ConceptConnections}
  Vector & 1 & & A & olikt andra Scala-samlingar kollar \code|==| ej innehållslikhet \\ 
  Array & 2 & & B & \code|xs == ys| är \code|true| om alla element lika \\ 
\end{ConceptConnections}


\SOLUTION

\TaskSolved \what

\Subtask

\begin{ConceptConnections}
  Vector & 1 & ~~\Large$\leadsto$~~ &  A & oföränderlig \\ 
  Array & 2 & ~~\Large$\leadsto$~~ &  B & förändringsbar \\ 
\end{ConceptConnections}

\Subtask

\begin{ConceptConnections}
  Vector & 1 & ~~\Large$\leadsto$~~ &  B & varianter med fler/andra element skapas snabbt ur befintlig \\ 
  Array & 2 & ~~\Large$\leadsto$~~ &  A & långsam vid ändring av storlek (kopiering av rubbet krävs) \\ 
\end{ConceptConnections}

\Subtask

\begin{ConceptConnections}
  Vector & 1 & ~~\Large$\leadsto$~~ &  B & \code|xs == ys| är \code|true| om alla element lika \\ 
  Array & 2 & ~~\Large$\leadsto$~~ &  A & olikt andra samlingar kollar \code|==| ej innehållslikhet \\ 
\end{ConceptConnections}

\QUESTEND







\WHAT{Räkna ut summa, min och max i \code{args}.}

\QUESTBEGIN

\Task \what~Skriv ett program som skriver ut summa, min och max för en sekvens av heltal i \code{args}. Du kan förutsätta att programmet bara körs med heltal som programparametrar. \emph{Tips:} Med uttrycken \code{xs.sum} och \code{xs.min} och \code{xs.max} ges summan, minsta resp. största värde.
%Med uttrycket \code{xs.map(_.toInt)} ges en ny samling med alla element omgjorda till heltal.

Exempel på körning i terminalen:
\begin{REPL}
> atom sum-min-max.scala
> scalac sum-min-max.scala
> scala SumMinMax 1 2 42 3 4
52 1 42
\end{REPL}

\SOLUTION

\TaskSolved \what~

\scalainputlisting{examples/sum-min-max.scala}

\QUESTEND






\WHAT{Algoritm: SWAP.}

\QUESTBEGIN

\Task  \what~Det är vanligt när man arbetar med förändringsbara datastrukturer att man kan behöva byta plats mellan element och då behövs algoritmen SWAP, som här illustreras genom platsbyte mellan värden:
\\ \emph{Problem:} Byta plats på två variablers värden. \\\emph{Lösningsidé:} Använd temporär variabel för mellanlagring.

\Subtask Skriv med \emph{pseudo-kod} (steg för steg på vanlig svenska) algoritmen SWAP nedan.

\emph{Indata:} två heltalsvariabler $x$ och $y$

\textbf{???}

\emph{Utdata:} variablerna $x$ och $y$ vars värden har bytt plats.

\Subtask Implementerar algoritmen SWAP. Ersätt \code{???} nedan med kod som byter plats på värdena i variablerna \code{x} och \code{y}:

\begin{REPL}
scala> var x = 42; var y = 43
scala> ???
scala> println("x är " + x + ", y är " + y)
x är 43, y är 42
\end{REPL}

\SOLUTION

\TaskSolved \what

\SubtaskSolved  Pseudokoden kan se ut såhär:
\begin{Code}
Deklarera heltalsvariabel temp.
Kopiera värdet från x till temp.
Kopiera värdet från y till x.
Kopiera värdet från temp till y.
\end{Code}

\SubtaskSolved
\begin{Code}
var temp = x
x = y
y = temp
\end{Code}

\QUESTEND




\WHAT{Indexering och tilldelning i Array med SWAP.}

\QUESTBEGIN

\Task \what~Skriva ett program som byter plats på första och sista elementet i \code{main}-parametern \code{args}. Bytet ska bara ske om det är minst två element i \code{args}. Oavsett om förändring skedde eller ej ska \code{args} sedan skrivas ut med blanktecken mellan argumenten.
  \emph{Tips:} Du kan komma åt sista elementet med \code{args(args.size - 1)}

Exempel på körning i terminalen:
\begin{REPL}
> atom swap-args.scala
> scalac swap-args.scala
> scala SwapFirstLastArg hej alla barn
barn alla hej
\end{REPL}

\SOLUTION

\TaskSolved \what~

\scalainputlisting{examples/swap-args.scala}

\QUESTEND



\WHAT{\code|for|-uttryck och \code|map|-uttryck.}

\QUESTBEGIN

\Task \what~Variabeln \code{xs} nedan refererar till samlingen \code{Vector(1, 2, 3)}. Para ihop uttrycken till vänster med rätt värde till höger.

\begin{ConceptConnections}
  \code|for (x <- xs) yield x * 2| & 1 & & A & \code|Vector(2, 3, 4)| \\ 
  \code|for (i <- xs.indices) yield i| & 2 & & B & \code|Vector(2, 4, 6)| \\ 
  \code|xs.map(x => x + 1)    | & 3 & & C & \code|Vector(0, 1, 2)| \\ 
  \code|for (i <- 0 to 1) yield xs(i)| & 4 & & D & \code|Vector(1, 2)| \\ 
  \code|(1 to 3).map(i => i)| & 5 & & E & \code|Vector(1, 2, 3)| \\ 
  \code|(1 until 3).map(i => xs(i))| & 6 & & F & \code|Vector(2, 3)| \\ 
\end{ConceptConnections}

\noindent Träna med dina egna varianter i REPL tills du lärt dig använda uttryck som ovan utantill. Då har du lättare att komma igång med kommande laborationer.

\SOLUTION

\TaskSolved \what

\begin{ConceptConnections}
  \code|for (x <- xs) yield x * 2| & 1 & ~~\Large$\leadsto$~~ &  F & \code|Vector(1, 2, 4)| \\ 
  \code|for (i <- xs.indices) yield i| & 2 & ~~\Large$\leadsto$~~ &  C & \code|Vector(0, 1, 2)| \\ 
  \code|xs.map(x => x + 1)    | & 3 & ~~\Large$\leadsto$~~ &  D & \code|Vector(2, 3, 4)| \\ 
  \code|for (i <- 0 to 1) yield xs(i)| & 4 & ~~\Large$\leadsto$~~ &  B & \code|Vector(1, 2)| \\ 
  \code|(1 to 3).map(i => i)| & 5 & ~~\Large$\leadsto$~~ &  A & \code|Vector(1, 2, 3)| \\ 
  \code|(1 until 3).map(i => xs(i))| & 6 & ~~\Large$\leadsto$~~ &  E & \code|Vector(2, 3)| \\ 
\end{ConceptConnections}

\QUESTEND






\WHAT{Algoritm: SUMBUG}

\QUESTBEGIN

\Task  \what~ . Nedan återfinns pseudo-koden för SUMBUG.

\begin{algorithm}[H]
 \SetKwInOut{Input}{Indata}\SetKwInOut{Output}{Resultat}

 \Input{heltalet $n$}
 \Output{utskrift av summan av heltalen $1$ till och med $n$ }
 $sum \leftarrow 0$ \\
 $i \leftarrow 1$  \\
 \While{$i \leq n$}{
  $sum \leftarrow sum + 1$
 }
 skriv ut $sum$
\end{algorithm}

\Subtask Kör algoritmen steg för steg med penna och papper, där du skriver upp hur värdena för respektive variabel ändras. Det finns två buggar i algoritmen. Vilka? Rätta buggarna och test igen genom att ''köra'' algoritmen med penna på papper och kontrollera så att algoritmen fungerar för $n=0$, $n=1$, och $n=5$. Vad händer om $n=-1$?

\Subtask Skapa med hjälp av en editor filen \code{sumn.scala}. Implementera algoritmen SUM enligt den rättade pseudokoden och placera implementationen i en main-metod i ett objekt med namnet \code{sumn}. Du kan skapa indata \code{n} till algoritmen med denna deklaration i början av din main-metod: \\ \code{val n = args(0).toInt} \\ Vad ger applikationen för utskrift om du kör den med argumentet 8888?

\begin{REPLnonum}
> scalac sumn.scala
> scala sumn 8888
\end{REPLnonum}

\noindent Kontrollera att din implementation räknar rätt genom att jämföra svaret med detta uttrycks värde, evaluerat i Scala REPL:
\begin{REPLnonum}
scala> (1 to 8888).sum
\end{REPLnonum}

\Subtask Implementera algoritmen SUM enligt pseudokoden ovan, men nu i Java. Skapa filen \code{SumN.java} och använd koden från uppgift \ref{task:java} som mall för att deklarera den publika klassen \code{SumN} med en main-metod. Några tips om Java-syntax och standarfunktioner i Java:

\begin{itemize}%[noitemsep, nolistsep]
\item Alla satser i Java måste avslutas med semikolon.
\item Heltalsvariabler deklareras med nyckelordet \lstinline[language=Java]{int} (litet i).
\item Typnamnet ska stå \emph{före} namnet på variabeln. Exempel: \\ \lstinline[language=Java]{int sum = 0;}
\item Indexering i en array görs i Java med hakparenteser: \code{args[0]}
\item I stället för Scala-uttrycket \code{args(0).toInt}, använd Java-uttrycket: \\ \code{Integer.parseInt(args[0])}
\item \code{while}-satser i Scala och Java har samma syntax.
\item Utskrift i Java görs med \code{System.out.println}
\end{itemize}


\SOLUTION


\TaskSolved \what


\SubtaskSolved  Bugg: Eftersom \code{i} inte inkrementeras, fastnar programmet i en oändlig loop. Fix: Lägg till en sats i slutet av while-blocket som ökar värdet på i med 1.
Bugg: Eftersom man bara ökar summan med 1 varje gång, kommer resultatet att bli summan av n stycken 1or, inte de n första heltalen. Fix: Ändra så att summan ökar med \code{i} varje gång, istället för 1.
För -1, blir resultatet 0. Förklaring: i börjar på 1 och är alltså aldrig mindre än n som ju är -1. while-blocket genomförs alltså noll gånger, och efter att \code{sum} får sitt ursprungsvärde förändras den aldrig.

\SubtaskSolved  39502716

\SubtaskSolved  Såhär kan implementationen se ut:
\begin{Code}[language=Java]
public class SumN {
  public static void main(String[] args) {
    int n = Integer.parseInt(args[0]);
    int sum = 0;
    int i = 1;
    while(i <= n){
      sum = sum + i;
      i = i + 1;
      }
    }
    System.out.println(sum);
}
\end{Code}

\QUESTEND



\clearpage

\ExtraTasks %%%%%%%%%%%%%%%%%%%




\WHAT{Algoritm: MAXBUG}

\QUESTBEGIN

\Task  \what~ . Nedan återfinns pseudo-koden för MAXBUG.

\begin{algorithm}[H]
 \SetKwInOut{Input}{Indata}\SetKwInOut{Output}{Resultat}

 \Input{Array $args$ med strängar som alla innehåller heltal}
 \Output{utskrift av största heltalet }
 $max \leftarrow$ det minsta heltalet som kan uppkomma  \\
 $n \leftarrow $ antalet heltal \\
 $i \leftarrow 0$ \\
 \While{$i < n$}{
   $x \leftarrow args(i).toInt$ \\
   \If{( x > $max$)}{$max \leftarrow x$}
  % $i \leftarrow i + 1$
 }
 skriv ut $max$
\end{algorithm}

\Subtask Kör med penna och papper. Det finns en bugg i algoritmen ovan. Vilken? Rätta buggen.

\Subtask Implementera algoritmen MAX (utan bugg) som en Scala-applikation. Tips:
\begin{itemize}[noitemsep, nolistsep]
\item Det minsta \code{Int}-värdet som någonsin kan uppkomma: \code{Int.MinValue}
\item Antalet element i $args$ ges av: \code{args.size}
\end{itemize}

\begin{REPL}
> atom maxn.scala
> scalac maxn.scala
> scala maxn 7 42 1 -5 9
42
\end{REPL}

\Subtask \label{subtask:arg0} Skriv om algoritmen så att variabeln $max$ initialiseras med det första talet i sekvensen.

\Subtask Implementera den nya algoritmvarianten från uppgift \ref{subtask:arg0} och prova programmet. Se till att programmet fungerar även om $args$ är tom.

\SOLUTION


\TaskSolved \what


\SubtaskSolved  Bugg: \code{i} inkrementeras aldrig. Programmet fastnar i en oändlig loop. Fix: Lägg till en sats som ökar i med 1, i slutet av while-blocket.

\SubtaskSolved  Så här kan implementationen se ut:
\begin{Code}
object Max {
  def main(args: Array[String]): Unit = {
    var max = Int.MinValue
    val n = args.size
    var i = 0
    while(i < n) {
      val x = args(i).toInt
      if(x > max)  max = x
      i += 1
    }
    println(max)
  }
}
\end{Code}

\SubtaskSolved  Raden där max initieras ändras till \code{var max = args(0).toInt}

\SubtaskSolved  För att inte få \code{java.lang.ArrayIndexOutOfBoundsException: 0} behövs en kontroll som säkerstället att inget görs om samlingen \code{args} är tom:
object Max {
  def main(args: Array[String]): Unit = if (args.size > 0) {
    var max = args(0).toInt
    val n = args.size
    var i = 0
    while(i < n) {
      val x = args(i).toInt
      if(x > max) {
        max = x
      }
      i += 1
    }
    println(max)
  } else println("Empty.")
}


\QUESTEND





\WHAT{Algoritm MINDEX.}

\QUESTBEGIN

\Task \label{task:minindex} \what~  Implementera algoritmen MININDEX som söker index för minsta heltalet i en sekvens. Pseudokod för algoritmen MININDEX:

\begin{algorithm}[H]
 \SetKwInOut{Input}{Indata}\SetKwInOut{Output}{Utdata}

 \Input{Sekvens $xs$ med $n$ st heltal.}
 \Output{Index för det minsta talet eller $-1$ om $xs$ är tom.  }
 $minPos \leftarrow 0 $\\
 $i \leftarrow 1$ \\
 \While{$i < n$}{
   \If{xs(i) < $xs(minPos)$}{$minPos \leftarrow i$}
   $i \leftarrow i + 1$
 }
 \eIf{$n > 0$}{\Return{$minPos$}}{\Return{$-1$}}
\end{algorithm}

\Subtask Prova algoritmen med penna och papper på sekvensen $(1, 2, -1, 4)$ och rita minnessituationen efter varje runda i loopen. Vad blir skillnaden i exekveringsförloppet om loopvariablen $i$  initialiserats till $0$ i stället för $1$?

\Subtask Implementera algoritmen MININDEX i ett Scala-program med nedan funktion:
\begin{Code}
def indexOfMin(xs: Array[Int]): Int = ???
\end{Code}
\begin{itemize}
  \item Låt programmet ha en \code{main}-funktion som ur \code{args} skapar en ny array med heltal som skickas till \code{indexOfMin} och sedan gör en utskrift av resultatet.
  \item Testa för olika fall:
  \begin{itemize}
    \item tom sekvenser
    \item sekvens med endast ett tal
    \item lång sekvens med det minsta talet först, någonstans mitt i, samt sist.
  \end{itemize}
\end{itemize}


\SOLUTION

\TaskSolved \what~

\SubtaskSolved En onödig jämförelse sker, men resultatet påverkas ej.

\SubtaskSolved

\begin{Code}
def indexOfMin(xs: Array[Int]): Int = {
  var minPos = 0
  var i = 1
  while (i < xs.size) {
    if (xs(i) < xs(minPos)) minPos = i
    i += 1
  }
  if (xs.size > 0) minPos else -1
}
\end{Code}


\QUESTEND





\WHAT{Datastrukturen \code+Range+.}

\QUESTBEGIN

\Task  \what~Evaluera nedan uttryck i Scala REPL. Vad har respektive uttryck för värde och typ?

\Subtask \code{Range(1, 10)}

\Subtask \code{Range(1, 10).inclusive}

\Subtask \code{Range(0, 50, 5)}

\Subtask \code{Range(0, 50, 5).size}

\Subtask \code{Range(0, 50, 5).inclusive}

\Subtask \code{Range(0, 50, 5).inclusive.size}

\Subtask \code{0.until(10)}

\Subtask \code{0 until (10)}

\Subtask \code{0 until 10}

\Subtask \code{0.to(10)}

\Subtask \code{0 to 10}

\Subtask \code{0.until(50).by(5)}

\Subtask \code{0 to 50 by 5}

\Subtask \code{(0 to 50 by 5).size}

\Subtask \code{(1 to 1000).sum}


\SOLUTION


\TaskSolved \what


\SubtaskSolved  värde: \code{Range(1,2,3,4,5,6,7,8,9)}

typ: \code{scala.collection.immutable.Range}

\SubtaskSolved  värde: \code{Range(1,2,3,4,5,6,7,8,9,10)}

typ: \code{scala.collection.immutable.Range}

\SubtaskSolved  värde: \code{Range(0,5,10,15,20,25,30,35,40,45)}

 typ: \code{scala.collection.immutable.Range}

\SubtaskSolved  värde: \code{10}, typ: \code{Int}

\SubtaskSolved  värde: \code{Range(0,5,10,15,20,25,30,35,40,45,50)}

typ: \code{scala.collection.immutable.Range}

\SubtaskSolved  värde: \code{11}, typ: \code{Int}

\SubtaskSolved  värde: \code{Range(0,1,2,3,4,5,6,7,8,9)}

typ: \code{scala.collection.immutable.Range}

\SubtaskSolved  värde: \code{Range(0,1,2,3,4,5,6,7,8,9)}

typ: \code{scala.collection.immutable.Range}

\SubtaskSolved  värde: \code{Range(0,1,2,3,4,5,6,7,8,9)}

typ: \code{scala.collection.immutable.Range}

\SubtaskSolved  värde: \code{Range(0,1,2,3,4,5,6,7,8,9,10)}

typ: \code{scala.collection.immutable.Range.Inclusive}

\SubtaskSolved  värde: \code{Range(0,1,2,3,4,5,6,7,8,9,10)}

typ: \code{scala.collection.immutable.Range.Inclusive}

\SubtaskSolved  värde: \code{Range(0,5,10,15,20,25,30,35,40,45)}

typ: \code{scala.collection.immutable.Range}

\SubtaskSolved  värde: \code{Range(0,5,10,15,20,25,30,35,40,45,50)}

typ: \code{scala.collection.immutable.Range}

\SubtaskSolved  värde: \code{11}, typ: \code{Int}

\SubtaskSolved  värde: \code{500500}, typ: \code{Int}




\QUESTEND






% %TODO Flytta några av nedan till extra uppgifter
%
%
% \WHAT{Datastrukturen \code+Array+.}
%
% \QUESTBEGIN
%
% \Task \label{task:array} \what~   Kör nedan kodrader i Scala REPL. Beskriv vad som händer.
%
% \Subtask \code{val xs = Array("hej","på","dej", "!")}
%
% \Subtask \code{xs(0)}
%
% \Subtask \code{xs(3)}
%
% \Subtask \code{xs(4)}
%
% \Subtask \code{xs(1) + " " + xs(2)}
%
% \Subtask \code{xs.mkString}
%
% \Subtask \code{xs.mkString(" ")}
%
% \Subtask \code{xs.mkString("(", ",", ")")}
%
% \Subtask \code{xs.mkString("Array(", ", ", ")")}
%
% \Subtask \code{xs(0) = 42}
%
% \Subtask \code{xs(0) = "42"; println(xs(0))}
%
% \Subtask \code{val ys = Array(42, 7, 3, 8)}
%
% \Subtask \code{ys.sum}
%
% \Subtask \code{ys.min}
%
% \Subtask \code{ys.max}
%
% \Subtask \code{val zs = Array.fill(10)(42)}
%
% \Subtask \code{zs.sum}
%
%
%
% \SOLUTION
%
%
% \TaskSolved \what
%
%
% \SubtaskSolved  Ett objekt av typen \code{Array[String]} skapas med värdet
%
% \code{Array(hej, på, dej, !)} och med namnet \code{xs}.
%
% \SubtaskSolved  Returnerar en sträng med värdet \code{hej}.
%
% \SubtaskSolved  Returnerar en sträng med värdet \code{!}.
%
% \SubtaskSolved  Ett exception genereras. Skriver ut:
%
% \code{java.lang.ArrayIndexOutOfBoundsException: 4}
%
% \SubtaskSolved  Returnerar en sträng med värdet \code{på dej}.
%
% \SubtaskSolved  Returnerar en sträng med värdet \code{hejpådej!}.
%
% \SubtaskSolved  Returnerar en sträng med värdet \code{hej på dej !}.
%
% \SubtaskSolved  Returnerar en sträng med värdet \code{(hej,på,dej,!)}.
%
% \SubtaskSolved  Returnerar en sträng med värdet \code{Array(hej,på,dej,!)}.
%
% \SubtaskSolved  Ett fel uppstår av typen \code{type mismatch}. Konsollen talar om för oss vad den fick, dvs värdet \code{42} av typen \code{Int}. Den talar även om för oss vad den ville ha, dvs något värde av typen \code{String}. Till sist skriver den ut vår kodrad och pekar ut felet.
%
% \SubtaskSolved  Det första elementet i \code{xs} ändras till värdet \code{42}. Därefter skrivs det första värdet i \code{xs} ut.
%
% \SubtaskSolved  Ett objekt av typen \code{Array[Int]} skapas med värdet \code{Array(42, 7, 3, 8)} och med namnet \code{ys}.
%
% \SubtaskSolved  Returnerar summan av elementen i \code{ys}. Resultatet är \code{60}.
%
% \SubtaskSolved  Returnerar det minsta värdet i \code{ys}. Resultatet är \code{3}.
%
% \SubtaskSolved  Returnerar det största värdet i \code{ys}. Resultatet är \code{42}.
%
% \SubtaskSolved  Ett nytt värde av typen \code{Array[Int]} skapas med \code{10} stycken element, alla med värdet \code{42}.
%
% \SubtaskSolved  Returnerar summan av elementen i \code{zs}. Resultatet blir 420 (42 multiplicerat med 10).
%
%
% \QUESTEND
%
%
%
%
% %%%%%%%%%%%%%%%%%%% SKA FIXAS:
%
%
%
%
%
%
%
% \WHAT{Datastrukturen \code+Vector+.}
%
% \QUESTBEGIN
%
% \Task  \what~  Kör nedan kodrader i Scala REPL. Beskriv vad som händer.
%
% \Subtask \code{val words = Vector("hej","på","dej", "!")}
%
% \Subtask \code{words(0)}
%
% \Subtask \code{words(3)}
%
% \Subtask \code{words.mkString}
%
% \Subtask \code{words.mkString(" ")}
%
% \Subtask \code{words.mkString("(", ",", ")")}
%
% \Subtask \code{words.mkString("Ord(", ", ", ")")}
%
% \Subtask \code{words(0) = "42"}
%
% \Subtask \code{val numbers = Vector(42, 7, 3, 8)}
%
% \Subtask \code{numbers.sum}
%
% \Subtask \code{numbers.min}
%
% \Subtask \code{numbers.max}
%
% \Subtask \code{val moreNumbers = Vector.fill(10000)(42)}
%
% \Subtask \code{moreNumbers.sum}
%
% \Subtask Jämför med uppgift \ref{task:array}. Vad kan man göra med en \code{Array} som man inte kan göra med en \code{Vector}?
%
% \SOLUTION
%
%
% \TaskSolved \what
%
%
% \SubtaskSolved  Ett objekt av typen \code{scala.collection.immutable.Vector[String]} initieras med värdet \code{Vector(hej, på dej, !)}.
%
% \SubtaskSolved  Returnerar det nollte elementet i \code{words}, dvs strängen \code{hej}.
%
% \SubtaskSolved  Returnerar det tredje elementet i \code{words}, dvs strängen \code{!}.
%
% \SubtaskSolved  Omvandlar vektorn till en Sträng.
%
% \SubtaskSolved  Samma som ovan, fast den här gången används mellanrum för att seperera elementen.
%
% \SubtaskSolved  Samma som ovan, fast den här gången sepereras elementen av kommatecken istället för mellanrum och dessutom börjar och slutar den resulterande strängen med parenteser.
%
% \SubtaskSolved  Samma som ovan, fast med ordet \code{Ord} tillagt i början av den resulterande strängen.
%
% \SubtaskSolved  Ett fel uppstår. Typen \code{Vector} är immutable. Dess element kan alltså inte bytas ut.
%
% \SubtaskSolved  En ny \code{Vector[Int]} skapas med värdet \code{Vector(42, 7, 3, 8)}.
%
% \SubtaskSolved  Returnerar summan av vektorn \code{numbers}.
%
% \SubtaskSolved  Returnerar vektorns minsta element.
%
% \SubtaskSolved  Returnerar vektorns största element.
%
% \SubtaskSolved  En ny vektor skapas innehållandes tiotusen 42or.
%
% \SubtaskSolved  Returnerar summan av vektorns element.
%
% \SubtaskSolved  Byta ut element.
%
%
%
% \QUESTEND
%
%
%
%
% %%<AUTOEXTRACTED by mergesolu>%%      %Uppgift 4
%
%
%
%
% \WHAT{\code+for+-uttryck}
%
% \QUESTBEGIN
%
% \Task  \what~ . Evaluera nedan uttryck i Scala REPL. Vad har respektive uttryck för värde och typ?
%
% \Subtask \code{for (i <- Range(1,10)) yield i}
%
% \Subtask \code{for (i <- 1 until 10) yield i}
%
% \Subtask \code{for (i <- 1 until 10) yield i + 1}
%
% \Subtask \code{for (i <- Range(1,10).inclusive) yield i}
%
% \Subtask \code{for (i <- 1 to 10) yield i}
%
% \Subtask \code{for (i <- 1 to 10) yield i + 1}
%
% \Subtask \code{(for (i <- 1 to 10) yield i + 1).sum}
%
% \Subtask \code{for (x <- 0.0 to 2 * math.Pi by math.Pi/4) yield math.sin(x)}
%
%
% \SOLUTION
%
%
% \TaskSolved \what
%
%
% \SubtaskSolved  typ: \code{scala.collection.immutable.IndexedSeq[Int]}
%
% värde: \code{Vector(1, 2, 3, 4, 5, 6, 7, 8, 9)}
%
% \SubtaskSolved  typ: \code{scala.collection.immutable.IndexedSeq[Int]}
%
% värde: \code{Vector(1, 2, 3, 4, 5, 6, 7, 8, 9)}
%
% \SubtaskSolved  typ: \code{scala.collection.immutable.IndexedSeq[Int]}
%
% värde: \code{Vector(2, 3, 4, 5, 6, 7, 8, 9, 10)}
%
% \SubtaskSolved  typ: \code{scala.collection.immutable.IndexedSeq[Int]}
%
% värde: \code{Vector(1, 2, 3, 4, 5, 6, 7, 8, 9, 10)}
%
% \SubtaskSolved  typ: \code{scala.collection.immutable.IndexedSeq[Int]}
%
% värde: \code{Vector(1, 2, 3, 4, 5, 6, 7, 8, 9, 10)}
%
% \SubtaskSolved  typ: \code{scala.collection.immutable.IndexedSeq[Int]}
%
% värde: \code{Vector(2, 3, 4, 5, 6, 7, 8, 9, 10, 11)}
%
% \SubtaskSolved  typ: \code{Int}, värde: \code{Vector(65)}
%
% \SubtaskSolved  typ: \code{scala.collection.immutable.IndexedSeq[Int]}
%
% värde: \code{Vector(0.0, 0.707, 1.0, 0.707, 0.0, -0.707, -1.0, -0.707)}
%
%
%
% \QUESTEND
%
%
%
%
% %%<AUTOEXTRACTED by mergesolu>%%      %Uppgift 5
%
%
%
%
% \WHAT{Metoden \code+map+ på en samling.}
%
% \QUESTBEGIN
%
% \Task  \what~  Evaluera nedan uttryck i Scala REPL. Vad har respektive uttryck för värde och typ?
%
% \Subtask \code{Range(0,10).map(i => i + 1)}
%
% \Subtask \code{(0 until 10).map(i => i + 1)}
%
% \Subtask \code{(1 to 10).map(i => i * 2)}
%
% \Subtask \code{(1 to 10).map(_ * 2)}
%
% \Subtask \code{Vector.fill(10000)(42).map(_ + 43)}
%
% \SOLUTION
%
%
% \TaskSolved \what
%
%
% \SubtaskSolved  typ: \code{scala.collection.immutable.IndexedSeq[Int]}
%
% värde: \code{Vector(1, 2, 3, 4, 5, 6, 7, 8, 9, 10)}
%
% \SubtaskSolved  typ: \code{scala.collection.immutable.IndexedSeq[Int]}
%
% värde: \code{Vector(1, 2, 3, 4, 5, 6, 7, 8, 9, 10)}
%
% \SubtaskSolved  typ: \code{scala.collection.immutable.IndexedSeq[Int]}
%
% värde: \code{Vector(2, 4, 6, 8, 10, 12, 14, 16, 18, 20)}
%
% \SubtaskSolved  typ: \code{scala.collection.immutable.IndexedSeq[Int]}
%
% värde: \code{Vector(2, 4, 6, 8, 10, 12, 14, 16, 18, 20)}
%
% \SubtaskSolved  typ: \code{scala.collection.immutable.Vector[Int]}
%
% värde: En vector av tiotusen 85or (85 = 42 + 43).
%
%
%
% \QUESTEND
%
%
%
%
% %%<AUTOEXTRACTED by mergesolu>%%      %Uppgift 6
%
%
%
%
% \WHAT{Metoden \code+foreach+ på en samling.}
%
% \QUESTBEGIN
%
% \Task  \what~  Kör nedan satser i Scala REPL. Vad händer?
%
% \Subtask \code{Range(0,10).foreach(i => println(i))}
%
% \Subtask \code{(0 until 10).foreach(i => println(i))}
%
% \Subtask \code|(1 to 10).foreach{i => print("hej"); println(i * 2)}|
%
% \Subtask \code{(1 to 10).foreach(println)}
%
% \Subtask \code{Vector.fill(10000)(math.random).foreach(r => }\\
%            \code{      if (r > 0.99) print("pling!"))}
%
%
% \SOLUTION
%
%
% \TaskSolved \what
%
%
% \SubtaskSolved  En \code{Range} skapas och dess element skrivs ut ett och ett.
%
% \SubtaskSolved  Samma sak händer.
%
% \SubtaskSolved  De tio första jämna talen (noll ej inräknat) skrivs ut med ett "hej" framför.
%
% \SubtaskSolved  Talen 1 till 10 skrivs ut.
%
% \SubtaskSolved  Tiotusen slumptal mellan 0 och 1 genereras. Varje gång ett tal är större än 0.99 kommer det ett pling.
%
%
%
% \QUESTEND
%
%
%
%
% %%<AUTOEXTRACTED by mergesolu>%%      %Uppgift 7
%
%
%
%


\newpage

\AdvancedTasks %%%%%%%%%%%%%%%%%




\WHAT{Sten-Sax-Påse-spel.}

\QUESTBEGIN

\Task  \what~ Bygg vidare på koden nedan och gör ett Sten-Sax-Påse-spel\footnote{\url{https://sv.wikipedia.org/wiki/Sten,_sax,_påse}}. Koden fungerar som den ska, förutom funktionen \code{winner} som fuskar till datorns fördel. Lägg även till en main-funktion så att programmet kan kompileras och köras i terminalen. Spelet blir roligare om du räknar antalet vinster och förluster. Du kan också göra så att datorn inte väljer med jämn fördelning.

\begin{Code}
object Game {
  val choices = Vector("Sten", "Påse", "Sax")

  def printChoices(): Unit =
    for (i <- 1 to choices.size) println(i + ": " + choices(i - 1))

  def userChoice(): Int = {
    printChoices()
    scala.io.StdIn.readLine("Vad väljer du? [1|2|3]<ENTER>:").toInt - 1
  }

  def computerChoice(): Int = (math.random * 3).toInt

  /** Ska returnera "Du", "Datorn", eller "Ingen" */
  def winner(user: Int, computer: Int): String = "Datorn"

  def play(): Unit = {
    val u = userChoice()
    val c = computerChoice()
    println("Du valde " + choices(u))
    println("Datorn valde " + choices(c))
    val w = winner(u, c)
    println(w + " är vinnare!")
    if (w == "Ingen") play()
  }
}
\end{Code}

% \begin{Code}[basicstyle=\ttfamily\footnotesize\selectfont]]
% object Game {
%   import javax.swing.JOptionPane
%   import JOptionPane.{showOptionDialog => optDlg}
%
%   def inputOption(msg: String, opt: Vector[String]) =
%     optDlg(null, msg, "Option", 0, 0, null, opt.toArray[Object], opt(0))
%
%   def msg(s: String) = JOptionPane.showMessageDialog(null, s)
%
%   val opt =  Vector("Sten", "Sax", "Påse")
%
%   def userChoice = inputOption("Vad väljer du?", opt)
%
%   def computerChoice = (math.random * 3).toInt
%
%   def winnerMsg(user: Int, computer: Int) = "??? vann!"
%
%   def main(args: Array[String]): Unit = {
%     var keepPlaying = true
%     while (keepPlaying) {
%       val u = userChoice
%       val c = computerChoice
%       msg("Du valde " + opt(u) + "\n" +
%           "Datorn valde " + opt(c) + "\n" +
%           winnerMsg(u, c))
%       if (u != c) keepPlaying = false
%     }
%   }
% }
% \end{Code}



\SOLUTION

\TaskSolved \what~ En (lättbegriplig?) lösning som provar alla kombinationer:

\begin{CodeSmall}
  def winner(user: Int, computer: Int): String =
    if      (choices(user) == "Sten" && choices(computer) == "Påse") "Datorn"
    else if (choices(user) == "Sten" && choices(computer) == "Sax")  "Du"
    else if (choices(user) == "Påse" && choices(computer) == "Sten") "Du"
    else if (choices(user) == "Påse" && choices(computer) == "Sax")  "Datorn"
    else if (choices(user) == "Sax"  && choices(computer) == "Sten") "Datorn"
    else if (choices(user) == "Sax"  && choices(computer) == "Påse") "Du"
    else "Ingen"
\end{CodeSmall}


En klurigare lösning (och svårbegripligare?) med hjälp av modulo-räkning:

\begin{Code}
  def winner(user: Int, computer: Int): String = {
     val result = (user - computer + 3) % 3
     if (user == computer) "Ingen"
     else if (result == 1) "Du"
     else "Datorn"
  }
\end{Code}
Moduloräkningen kräver att elementen i \code{choices} är i \emph{förlorar-över}-ordning, alltså Sten, Påse, Sax. Addition med 3 görs för att undvika negativa tal, som beter sig annorlunda i moduloräkning.

\QUESTEND



\WHAT{Jämför exekveringstiden för storleksförändring mellan \code{Array} och \code{Vector}.}

\QUESTBEGIN

\Task\Uberkurs \what~\\
Klistra in nedan kod i REPL:
\begin{Code}
def time(block: => Unit): Double = {
  val t = System.nanoTime
  block
  (System.nanoTime-t)/1e6  // ger millisekunder
}
\end{Code}

\Subtask Skriv kod som gör detta i tur och ordning:
\begin{enumerate}[nolistsep,noitemsep]
  \item deklarerar en \code{val as} som är en \code{Array} fylld med en miljon heltalsnollor,
  \item deklarerar en \code{val vs} som är en \code{Vector} fylld med en miljon heltalsnollor,
  \item kör \code{time(as :+ 0)} 10 gånger och räknar ut medelvärdet av tidmätningarna,
  \item kör \code{time(vs :+ 0)} 10 gånger och räknar ut medelvärdet av tidmätningarna.
\end{enumerate}

\Subtask Vilken av \code{Array} och \code{Vector} är snabbast vid tillägg av element? Varför är det så?

\SOLUTION

\TaskSolved \what~

\SubtaskSolved Med en dator som har en \code{i7-4790K CPU @ 4.00GHz} blev det så här:
\begin{REPL}
scala> def time(block: => Unit): Double = {
     |   val t = System.nanoTime
     |   block
     |   (System.nanoTime - t)/1e6  // ger millisekunder
     | }

scala> val as = Array.fill(1e6.toInt)(0)
as: Array[Int] = Array(0, 0, 0, 0, 0, 0, 0, 0, 0, 0, 0, 0, 0, 0, 0, ...

scala> val vs = Vector.fill(1e6.toInt)(0)
vs: scala.collection.immutable.Vector[Int] = Vector(0, 0, 0, 0, 0, ...

scala> val ast = (for (i <- 1 to 10) yield time(as :+ 0)).sum / 10.0
res1: Double = 1.8719819999999998

scala> val vst = (for (i <- 1 to 10) yield time(vs :+ 0)).sum / 10.0
res2: Double = 0.006485099999999999

scala> ast / vst
res3: Double = 288.6589258453995

\end{REPL}

\SubtaskSolved \code{Vector} är två tiopotenser snabbare i detta exempel. Anledningen är att varje storleksförändring av en \code{Array} kräver allokering och elementvis kopiering av en helt ny \code{Array} medan den oföränderliga \code{Vector} kan återanvända hela datastrukturen med redan allokerade element när nya element läggs till.

\QUESTEND




\WHAT{Minnesåtgång för \code+Range+.}

\QUESTBEGIN

\Task\Uberkurs \what~Datastrukturen \code{Range} håller reda på start- och slutvärde, samt stegstorleken för en uppräkning, men alla talen i uppräkningen genereras inte förrän på begäran. En \code{Int} tar 4 bytes i minnet. Ungefär hur mycket plats i minnet tar de objekt som variablerna (a) \code{intervall} respektive (b) \code{sekvens} refererar till nedan?

\begin{REPL}
scala> val intervall = (1 to Int.MaxValue by 2)
scala> val sekvens = r.toArray
\end{REPL}
\emph{Tips:} Använd uttrycket \code{ BigInt(Int.MaxValue) * 2 } i dina beräkningar.


\SOLUTION

\TaskSolved  \what~

\SubtaskSolved Variabeln \code{intervall} refererar till objekt som tar upp 12 bytes.

\SubtaskSolved Variabeln \code{sekvens} refererar till objekt som tar upp ca 4 miljarder bytes.

\QUESTEND




\WHAT{Undersök den genererade byte-koden.}

\QUESTBEGIN

\Task\Uberkurs  \what~  Kompilatorn genererar byte-kod, uttalas ''bajtkod'' \Eng{byte code}, som den virtuella maskinen tolkar och översätter till maskinkod medan programmet kör. Med kommandot \code{:javap} i REPL kan du undersöka byte-koden.
\begin{REPL}
scala> def plusxy(x: Int, y: Int) = x + y
scala> :javap plusxy
\end{REPL}

\Subtask Leta upp raden \code{public int plusxy(int, int);} och studera koden efter \code{Code:} och försök gissa vilken instruktion som utför själva additionen.

\Subtask Lägg till en parameter till: \\ \code{def plusxyz(x: Int, y: Int, z: Int) = x + y + z}
\\ och studera byte-koden med \code{:javap plusxyz}. Vad skiljer byte-koden mellan \code{plusxy} och \code{plusxyz}?

\Subtask Läs om byte-kod här: \href{https://en.wikipedia.org/wiki/Java\_bytecode}{en.wikipedia.org/wiki/Java\_bytecode}. Vad betyder den inledande bokstaven i additionsinstruktionen?


\SOLUTION

\TaskSolved \what~

\SubtaskSolved Så här ser funktionen \code{plusxy} ut:
\begin{REPL}
public int plusxy(int, int);
  descriptor: (II)I
  flags: ACC_PUBLIC
  Code:
    stack=2, locals=3, args_size=3
       0: iload_1
       1: iload_2
       2: iadd
       3: ireturn
    LocalVariableTable:
      Start  Length  Slot  Name   Signature
          0       4     0  this   L;
          0       4     1     x   I
          0       4     2     y   I
    LineNumberTable:
      line 11: 0
\end{REPL}
Det är instruktionen \code{iadd} som gör själva additionen.


\SubtaskSolved Det har tillkommit en parameter till i byte-koden. Instruktionen \code{iadd} görs nu två gånger. Instruktionen \code{iadd} adderar exakt två tal i taget.

\begin{REPL}
public int plusxyz(int, int, int);
  descriptor: (III)I
  flags: ACC_PUBLIC
  Code:
    stack=2, locals=4, args_size=4
       0: iload_1
       1: iload_2
       2: iadd
       3: iload_3
       4: iadd
       5: ireturn
    LocalVariableTable:
      Start  Length  Slot  Name   Signature
          0       6     0  this   L;
          0       6     1     x   I
          0       6     2     y   I
          0       6     3     z   I
    LineNumberTable:
      line 11: 0
\end{REPL}


\SubtaskSolved Prefixet \code{i} i instruktionsnamnet \code{iadd} står för ''integer'' och anger att heltalsdivision avses.

\QUESTEND




\WHAT{Skillnaden mellan krullpareneteser och vanliga parenteser}

\QUESTBEGIN

\Task  \what~ Läs om krullparenteser och vanliga parenteser på stack overflow: \\ \href{http://stackoverflow.com/questions/4386127/what-is-the-formal-difference-in-scala-between-braces-and-parentheses-and-when}{stackoverflow.com/questions/4386127/what-is-the-formal-difference-in-scala-between-braces-and-parentheses-and-when} och prova själv i REPL hur du kan blanda dessa olika slags parenteser på olika vis.

\SOLUTION

\TaskSolved \what~

\SubtaskSolved Prova själv i REPL.

\QUESTEND
