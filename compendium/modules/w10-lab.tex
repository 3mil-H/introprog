%!TEX encoding = UTF-8 Unicode

%!TEX root = ../compendium.tex

\Lab{\LabWeekTEN}

\begin{Goals}
\item Implementera sorteringsalgoritmerna insättning- (insertion) och urvalssortering (selection) i Java.
\item Implementera sökalgoritmerna linear-search och binary-search.
\item Implementera sorteringsalgoritmerna selection-sort och insertion-sort.
\item Analysera hur algoritmers komplexitet påverkar exekveringen.
\item Repetera in- och utläsning till fil.
\item Serialisera data för transport och lagring.
\end{Goals}

\begin{Preparations}
\item Fyll i enkäten på länken: % TODO
\item Övning XX vecka 10.
\item Läs om de olika algoritmerna så att du förstår hur de fungerar.
\end{Preparations}

\subsection{Om Labben}

Veckans labb handlar om datahantering, främst sortering, sökning och registrering av data. Rådatan som kommer behandlas är enkätsvaren från förberedelseuppgifterna.

Indatan är av typen CSV (comma-separated values) och ser ut något såhär:
%formattering??
\begin{Code}
rad1kolumn1,rad1kolumn2,rad1kolumn3,rad1kolumn4,rad1kolumn5
rad2kolumn1,rad2kolumn2,rad2kolumn3,rad2kolumn4,rad2kolumn5
rad3kolumn1,rad3kolumn2,rad3kolumn3,rad3kolumn4,rad3kolumn5
...
\end{Code}

Programmet som delvis ska implementeras kommer kunna sortera, söka och registrera indatan med hjälp av algoritmerna nämnda tidigare. Uppgifterna består i att implementera algoritmerna

\subsection{Kodstruktur}
Koden är uppdelad i följande delar:

\begin{itemize}
\item Klassen \code{StringMatrix} är datastrukturen som har hand om datan och tillhandahåller de nödvändiga metoderna som kommer användas i implementeringen av sök-, registrering och sorteringsalgoritmerna.
\item OSV
\end{itemize}

\subsection{Given kod}


\subsection{Obligatoriska uppgifter}

\Task Implementera datastrukturen

\Subtask En underuppgift.

\Subtask En underuppgift.

\subsection{Frivilliga extrauppgifter}
<<<<<<< HEAD

\Task Implementera sorteringsfunktionen generellt
=======
    
\Task En labbuppgiftsbeskrivning.
>>>>>>> upstream/master

\Subtask En underuppgift.

\Subtask En underuppgift.

