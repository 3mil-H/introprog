
%!TEX encoding = UTF-8 Unicode
%!TEX root = ../exercises.tex

\ifPreSolution

\Exercise{\ExeWeekELEVEN}\label{exe:W11}

\begin{Goals}
\item \TODO Fler mål här!
\item Känna till hur implicita sorteringsordningar används för grundtyperna och egendefinierade typer.
\item Känna till existensen av, funktionen hos, och relationen mellan klasserna \code{Ordering} och \code{Comparator}, samt  \code{Ordered} och \code{Comparable}.

\end{Goals}

\begin{Preparations}
\item \StudyTheory{11}
\end{Preparations}

\BasicTasks %%%%%%%%%%%%%%%%

\else

\ExerciseSolution{\ExeWeekELEVEN}

\BasicTasks %%%%%%%%%%%

\fi




\WHAT{Lösning på konfigurationsproblemet med hjälp av givna värden.}

\QUESTBEGIN

\Task  \what~ Antag att vi vill kunna konfigurera beteendet hos en funktion för att göra den mer flexibel. Nedan visas tre principiellt olika sätt att göra detta på för en funktion \code{greet} som skriver ut en hälsning: 1) en globalt åtkomlig variabel, 2) defaultargument, samt 3) kontextuell abstraktion med \code{given} och \code{using}.

\begin{Code}
object GlobalVariable:
  case class GreetConfig(greeting: String, receiver: String)
  object GreetConfig:
    val default = GreetConfig(greeting = "Hello", receiver = "World")
    var config = default
  
  def greet() = 
    s"${GreetConfig.config.greeting} ${GreetConfig.config.receiver}!"

object DefaultArgs:
  case class GreetConfig(greeting: String, receiver: String)
  object GreetConfig:
    val default = GreetConfig(greeting = "Hello", receiver = "World")
  
  def greet(config: GreetConfig = GreetConfig.default) =
    s"${config.greeting} ${config.receiver}!"

object GivenValue:
  case class GreetConfig(greeting: String, receiver: String)
  object GreetConfig:
    given default: GreetConfig = GreetConfig("Hello", "World")
  
  def greet(using g: GreetConfig) = s"${g.greeting} ${g.receiver}"
\end{Code}

\Subtask Skriv kod som testar de olika varianterna ovan. Visa speciellt hur du först kan ge en konfiguration som skiljer sig från \code{default} och hur du sedan kan återgå till \code{default} igen. Notera  vilka lösningar som medger anropa av \code{greet} med eller utan efterföljande \code{()}?

\Subtask Vad är för- och nackdelar med de olika varianterna ovan? Diskutera speciellt vilken/vilka lösningar som medger flera lokala konfigurationer utan att de påverkar varandra.

\Subtask Förklara vad som händer vid anrop av \code{summon[GivenValue.GreetConfig]}. Hur är summon implementerad?

\SOLUTION


\TaskSolved \what

\SubtaskSolved  \TODO ...


\QUESTEND

\AdvancedTasks %%%%%%%%%%%


\WHAT{Varians och typgränser.}

\QUESTBEGIN

\Task  \what~ Koden nedan är en modell av husdjur med följande innebörd: Husdjur kan vara friska eller sjuka och föds i normalfallet friska. Det kan finnas många katter och hundar, vilka alla är olika slags husdjur.

\begin{Code}
trait Pet(var isHealthy: Boolean = true)
class Cat extends Pet()
class Dog extends Pet()

\end{Code}

\Subtask Förändra koden nedan så att efterföljande REPL-sats \emph{inte} ger kompileringsfel?
\begin{Code}
case class Box[A](x: A)
\end{Code}
\begin{REPLnonum}
scala> val b: Box[Any] = Box[Cat](Cat())
\end{REPLnonum}

\Subtask Prova nedan i REPL och förklara vad som händer.      
\begin{REPLnonum}
scala> val v: Vector[Pet] = Vector[Cat](Cat())

scala> val s: Set[Pet] = Set[Cat](Cat())

scala> :settings -explain

scala> val s: Set[Pet] = Set[Cat](Cat())
\end{REPLnonum} 
\emph{Ledtråd:} I Scalas standardbibliotek så är ärver \code{Set[T]} funktionstypen \code{T => Boolean} som är deklarerad kontravariant i sin inparameter.

\Subtask Det ska finnas veterinärer som kan behandla husdjur och göra dem friska. Varför fungerar inte nedan kod? Är det ett kompileringsfel eller körtidsfel?

\begin{Code}
class Vet[-A]:
  def treat(x: A): Unit = x.isHealthy = true
\end{Code}

\Subtask Inför en typgräns i veterinärens typparametern som åtgärdar felet.

\Subtask Skriv valfri kod som visar 1) att kompilatorn tillåter kattveterinärer att behandla katter men 2) förhindrar att kattveterinärer får behandla godtyckliga husdjur och att 3) en veterinär som har komptens att behandla godtyckliga husdjur kan behandla både katter och hundar. Förklara varför kompilatorn tillåter/förhindrar detta.

\SOLUTION


\TaskSolved \what

\SubtaskSolved Gör lådan flexibel i sin typparameter med ett \code{+} före typparametern enligt nedan. 
\begin{Code}
case class Box[+A](x: A)
\end{Code}
Kompilatorn tillämpar reglerna för kovarians eftersom typparametern har ett plustecken framför sig: \code{Box[Cat]} är en suptyp till \code{Box[Any]} om \code{Cat} är en subtyp till \code{Any}, vilket den ju är eftersom alla typer är subtyp till \code{Any}. 

\SubtaskSolved Förklaringen till beteendet har med olika varians att göra:
\begin{itemize}
  \item Samlingen \code{Vector} är kovariant och därmed flexibel i sin typparameter (liksom andra oföränderliga sekvenser i Scalas standardbibliotek). Kompilatorn betraktar därmed \code{Vector[Cat]} som en subtyp till \code{Vector[Pet]} eftersom \code{Cat} är en subtyp till \code{Pet}. På platser i koden där en \code{Vector[Pet]} krävs så anses \code{Vector[Cat]} överensstämma med \Eng{conforms to} \code{Vector[Pet]} och får därmed duga på dessa platser.
  \item En mängd har en apply-metod från elemttypen till \code{Boolean} som ger innehållstest. Av det skälet har man låtit \code{Set[T]} ärva \code{Function1[T, Boolean]} som är deklarerad kontravariant i \code{T}, så att en mängd kan användas där en \code{T => Boolean} förväntas. Även om det skulle vara praktiskt om Set[T] vore kovariant i \code{T}, i likhet med \code{Vector}, \code{List}, \code{Seq} etc, så kan inte \code{T} vara både kovariant och kontravariant på en och samma gång. Man har därför valt att göra \code{Set} invariant och därmed är mängder ej flexibla i sin typparameter. \code{Set[Cat]} är alltså \emph{inte} en subtyp till \code{Set[Pet]} \emph{även} om \code{Cat} är en subtyp till \code{Pet}, vilket ger kompileringsfel i uppgiftens exempel. 
  Se även \url{https://stackoverflow.com/questions/676615/why-is-scalas-immutable-set-not-covariant-in-its-type}  
  \item Med \code{:settings -explain} ger kompilatorn en längre utskrift som förklarar den bevisföring som skedde under kompileringens typkontroll.
\end{itemize}



\SubtaskSolved Det blir kompileringsfel då metoden \code{isHealthy} ej existerar för godtycklig typ.

\SubtaskSolved Lägg till en övre gräns som garanterar att metoden \code{isHealthy} finns för alla typer som kan bindas till typparametern \code{A}:
\begin{Code}
class Vet[-A <: Pet]:
  def treat(x: A): Unit = x.isHealthy = true
\end{Code}
Kompilatorn garanterar alltså att typparametern \code{A} är ''mindre än eller lika med'' \code{Pet}.

\SubtaskSolved Veterinären \code{Vet} är flexibel i sin typparameter och minustecknet anger kontravarians och därmed att \code{Vet[Pet]} är en subtyp till \code{Vet[Cat]} då \code{Cat} är en subtyp till \code{Pet}. Detta kan demonstreras med nedan exempel:
\begin{REPL}
scala> val pinkPanther = Cat()
val pinkPanther: Cat = Cat@33e7ece5

scala> val somePet: Pet = Cat()
val somePet: Pet = Cat@57f1cb96

scala> val catVet = Vet[Cat]()
val catVet: Vet[Cat] = Vet@1060e784

scala> pinkPanther.isHealthy = false

scala> catVet.treat(pinkPanther)

scala> pinkPanther.isHealthy
val res2: Boolean = true

scala> somePet.isHealthy = false

scala> catVet.treat(somePet)
-- [E007] Type Mismatch Error: --------------
1 |catVet.treat(somePet)
  |             ^^^^^^^
  |             Found:    (somePet : Pet)
  |             Required: Cat

scala> val powerVet = Vet[Pet]()
val powerVet: Vet[Pet] = Vet@2eb90ae9

scala> pinkPanther.isHealthy = false

scala> powerVet.treat(pinkPanther)

scala> pinkPanther.isHealthy
val res3: Boolean = true

scala> val pluto = Dog()
val pluto: Dog = Dog@6f27db5d

scala> pluto.isHealthy = false

scala> powerVet.treat(pluto)

scala> pluto.isHealthy
val res4: Boolean = true

\end{REPL}

\QUESTEND




\WHAT{Typklasser och kontextparametrar.}

\QUESTBEGIN

\Task  \what~  I Scala finns möjligheter till avancerad funktionsprogrammering med s.k. \textbf{typklasser}, som definierar generella beteenden som fungerar för befintliga typer utan att implementationen av dessa befintliga typer behöver ändras. Vi nosar i denna uppgift på hur implicita argument kan användas för att skapa typklasser, illustrerat med hjälp av implicita ordningarna, som är en typisk och användbar tillämpning av konceptet typklasser.

\Subtask \emph{Implicit parameter och implicit värde.} Med nyckelordet \code{implicit} framför en parameter öppnar man för möjligheten att låta kompilatorn ge argumentet ''automatiskt'' om den kan hitta ett värde med passande typ som också är deklarerat med \code{implicit}, så som visas nedan.
\begin{REPL}
scala> def add(x: Int)(implicit y: Int) = x + y
scala> add(1)(2)
scala> add(1)
scala> implicit val ngtNamn = 42
scala> add(1)
\end{REPL}
Vad blir felmeddelandet på rad 3 ovan? Varför fungerar det på rad 5 utan fel?

\Subtask \emph{Typklasser.} Genom att kombinera koncepten implicita värden, generiska klasser och implicita parametrar får man möjligheten att göra typklasser, så som \code{CanCompare} nedan, som vi kan få att fungera för befintliga typer utan att de behöver ändras.

Vad händer nedan? Vilka rader ger felmeddelande? Varför?

\begin{REPL}
scala> trait CanCompare[T] { def compare(a: T, b: T): Int }
scala> def sort2[T](a: T, b: T)(implicit cc: CanCompare[T]): (T, T) =
         if (cc.compare(a, b) > 0) (b, a) else (a, b)
scala> sort2(42, 41)
scala> implicit object intComparator extends CanCompare[Int]{
         override def compare(a: Int, b: Int): Int = a - b
       }
scala> sort2(42, 41)
scala> sort2(42.0, 41.0)
\end{REPL}

\Subtask Definiera ett implicit objekt som gör så att \code{sort2} fungerar för värden av typen \code{Double}.

\Subtask Definiera ett implicit objekt som gör så att \code{sort2} fungerar för värden av typen \code{String}.


\SOLUTION


\TaskSolved \what

\SubtaskSolved ---


\SubtaskSolved ---


\SubtaskSolved
Tänk på att det fortfarande måste returneras en Int.


\SubtaskSolved
Undersök i Javas API hur metoden \code{compareTo} är implementerad för strängar.

\QUESTEND





\WHAT{Användning av implicit ordning.}

\QUESTBEGIN

\Task \label{task:implicit-ordering} \what~  Vi ska nu göra \code{isSorted} från uppgift \ref{task:isSorted} mer generellt användbar genom att möjliggöra att implicita ordningsfunktioner finns tillgängliga för olika typer.

\Subtask  Med signaturen  \code{isSorted(xs: Vector[Int]): Boolean} så
fungerar sorteringstestet bara för samlingar av typen \code{Vector[Int]}.

Om vi i stället använder
\code{isSorted(xs: Seq[Int]): Boolean} fungerar den för alla samlingar med heltal, även \code{Array} och \code{List}. Testa nedan funktion i REPL med heltalssekvenser av olika typ.
\begin{Code}
def isSorted(xs: Seq[Int]): Boolean = xs == xs.sorted
\end{Code}

\Subtask Det blir problem med nedan försök att göra \code{isSorted} generisk. Hur lyder felmeddelandet? Vad saknas enligt felmeddelandet?
\begin{REPLnonum}
scala> def isSorted[T](xs: Seq[T]): Boolean = xs == xs.sorted
\end{REPLnonum}

\Subtask Vi vill gärna att \code{isSorted} ska fungera för godtyckliga typer T som har en ordningsdefinition. Detta kan göras med nedan funktion där typparametern \code{[T:Ordering]} betyder att \code{isSorted} är definierad för alla samlingar där typen \code{T} har en implicit ordning \code{Ordering[T]}. Speciellt gäller detta för alla grundtyperna \code{Int}, \code{Double}, \code{String}, etc., som alla har implicit tillgänglig \code{Ordering[Int]} etc.
\begin{Code}
def isSorted[T:Ordering](xs: Seq[T]): Boolean = xs == xs.sorted
\end{Code}
Testa metoden ovan i REPL enligt nedan.
\begin{REPL}
scala> isSorted(Vector(1,2,3))
scala> isSorted(Array(1,2,3,1))
scala> isSorted(Vector("A","B","C"))
scala> isSorted(List("A","B","C","A"))
scala> case class Person(firstName: String, familyName: String)
scala> val persons = Vector(Person("Kim", "Finkodare"), Person("Robin","Fulhack"))
scala> isSorted(persons)
\end{REPL}
Vad ger sista raden för felmeddelande? Varför?


\Subtask \emph{Implicita ordningar.} En typparameter på formen \code{[T:Ordering]} kallas kontextgräns \Eng{context bound} och föranleder kompilatorn att expandera funktionshuvudet för \code{isSorted} med en extra parameterlista som har en implicit parameter. I stället för att använda \code{[T:Ordering]} kan vi själva lägga till den implicita parametern som motsvarar kontextgränsen. Då får vi också tillgång till ett namn på den implicita ordningen och kan använda det namnet i funktionskroppen och anropa metoder som är medlemmar av typen \code{Ordering}.

\begin{CodeSmall}
def isSorted[T](xs: Seq[T])(implicit ord: Ordering[T]): Boolean =
  xs.zip(xs.tail).forall(x => ord.lteq(x._1, x._2))
\end{CodeSmall}

Objekt av typen \code{Ordering} har jämförelsemetoder som t.ex. \code{lteq} (förk. \emph{less than or equal}) och \code{gt} (förk. \emph{greater than}).

Det finns fördefinierade implicita objekt \code{Ordering[T]} för alla grundtyper, alltså t.ex. \code{Ordering[Int]}, \code{Ordering[String]}, etc.
Testa så att kompilatorn hittar ordningen för samlingar med värden av några grundtyper. Kontrollera även enligt nedan att det fortfarande blir problem för egendefinierade klasser, t.ex. \code{Person}  (detta ska vi råda bot på i uppgift \ref{task:custom-ordering}).
\begin{REPL}
scala> isSorted(Vector(1,2,3))
scala> isSorted(Array(1,2,3,1))
scala> isSorted(Vector("A","B","C"))
scala> isSorted(List("A","B","C","A"))
scala> class Person(firstName: String, familyName: String)
scala> val persons = Vector(Person("Kim", "Finkodare"), Person("Robin","Fulhack"))
scala> isSorted(persons)
\end{REPL}

\Subtask \emph{Importera implicita ordningsoperatorer från en \code{Ordering}.} Om man gör import på ett \code{Ordering}-objekt får man tillgång till implicita konverteringar som gör att jämförelseoperatorerna fungerar. Testa nedan variant av \code{isSorted} på olika sekvenstyper och verifiera att \code{<=}, \code{>}, etc., nu fungerar enligt nedan.
\begin{CodeSmall}
def isSorted[T](xs: Seq[T])(implicit ord: Ordering[T]): Boolean = {
  import ord._
  xs.zip(xs.tail).forall(x => x._1 <= x._2)
}
\end{CodeSmall}


\SOLUTION


\TaskSolved \what ---

\QUESTEND






\WHAT{Skapa egen implicit ordning med \code{Ordering}.}

\QUESTBEGIN

\Task \label{task:custom-ordering} \what~

\Subtask Ett sätt att skapa en egen, specialanpassad ordning är att mappa dina objekt till typer som redan har en implicit ordning. Med hjälp av metoden \code{by} i objektet \code{scala.math.Ordering} kan man skapa ordningar genom bifoga en funktion \code{T => S} där \code{T} är typen för de objekt du vill ordna och \code{S} är någon annan typ, t.ex. \code{String} eller \code{Int}, där det redan finns en implicit ordning.
\begin{REPL}
scala> case class Team(name: String, rank: Int)
scala> val xs =
         Vector(Team("fnatic", 1499), Team("nip", 1473), Team("lumi", 1601))
scala> xs.sorted  // Hur lyder felmeddelandet? Varför blir det fel?
scala> val teamNameOrdering = Ordering.by((t: Team) => t.name)
scala> xs.sorted(teamNameOrdering)   //explicit ordning
scala> implicit val teamRankOrdering = Ordering.by((t: Team) => t.rank)
scala> xs.sorted   // Varför funkar det nu?
\end{REPL}

\Subtask Vill man sortera i omvänd ordning kan man använda
\code{Ordering.fromLessThan} som tar en funktion \code{(T, T) => Boolean} vilken ska ge \code{true} om första parametern ska komma före, annars \code{false}. Om vi vill sortera efter \code{rank} i omvänd ordning kan vi göra så här:
\begin{REPL}
scala> val highestRankFirst =
         Ordering.fromLessThan[Team]((t1, t2) => t1.rank > t2.rank)
scala> xs.sorted(highestRankFirst)
\end{REPL}

\Subtask Om du har en case-klass med flera fält och vill ha en fördefinierad implicit sorteringsordning samt även erbjuda en alternativ sorteringsordning kan du placera olika ordningsdefinitioner i ett kompanjonsobjekt; detta är nämligen ett av de ställen där kompilatorn söker efter eventuella implicita värden innan den ger upp att leta.
\begin{Code}
case class Team(name: String, rank: Int)
object Team {
  implicit val highestRankFirst = Ordering.fromLessThan[Team]{
    (t1, t2) => t1.rank > t2.rank
  }
  val nameOrdering = Ordering.by((t: Team) => t.name)
}
\end{Code}
\begin{REPL}
scala> :pa
// Exiting paste mode, now interpreting.
case class Team(name: String, rank: Int)
object Team {
  implicit val highestRankFirst =
    Ordering.fromLessThan[Team]{(t1, t2) => t1.rank > t2.rank}
  val nameOrdering = Ordering.by((t: Team) => t.name)
}
scala> val xs =
         Vector(Team("fnatic", 1499), Team("nip", 1473), Team("lumi", 1601))
scala> xs.sorted
scala> xs.sorted(Team.nameOrdering)
\end{REPL}



\Subtask Det går också med kompanjonsobjektet ovan att få jämförelseoperatorer att fungera med din case-klass, genom att importera medlemmarna i lämpligt ordningsobjekt. Verifiera att så är fallet enligt nedan:
\begin{REPL}
scala> Team("fnatic",1499) < Team("gurka", 2)  // Vilket fel? Varför?
scala> import Team.highestRankFirst._
scala> Team("fnatic",1499) < Team("gurka", 2)  // Inget fel? Varför?
\end{REPL}


\SOLUTION


\TaskSolved \what ---

\QUESTEND






\WHAT{Specialanpassad ordning genom att ärva från \code{Ordered}.}

\QUESTBEGIN

\Task  \what~  Om det finns \emph{en} väldefinierad, specifik ordning som man vill ska gälla för sina case-klass-instanser kan man göra den ordnad genom att låta case-klassen mixa in traiten \code{Ordered} och implementera den abstrakta metoden \code{compare}.

\begin{Background}
En trait som används på detta sätt kallas \textbf{gränssnitt} \Eng{interface}, och om man \emph{implementerar} ett gränssnitt så uppfyller man ett ''kontrakt'', som i detta fall innebär att man implementerar det som krävs av ordnade objekt, nämligen att de har en konkret \code{compare}-metod. Du lär dig mer om gränssnitt i kommande kurser.
\end{Background}

\Subtask Implementera case-klassen \code{Team} så att den är en subtyp till \code{Ordered} enligt nedan skiss. Högre rankade lag ska komma före lägre rankade lag. Metoden \code{compare} ska ge ett heltal som är negativt om \code{this} kommer före \code{that}, noll om de ordnas lika, annars positivt.

\begin{Code}
case class Team(name: String, rank: Int) extends Ordered[Team]{
  override def compare(that: Team): Int = ???
}
\end{Code}
\emph{Tips:} Du kan anropa metoden \code{compare} på alla grundtyper, t.ex. \code{Int}, eftersom de är implicit ordnade. Genom att negera uttrycket blir ordningen den omvända.
\begin{REPL}
scala> -(2.compare(1))
\end{REPL}

\Subtask Testa att  din case-klass nu uppfyller det som krävs för att vara ordnad.
\begin{REPL}
scala> Team("fnatic",1499) < Team("gurka", 2)
\end{REPL}


\SOLUTION


\TaskSolved \what


Tänk på att för att sortering i omvänd ordning (alltså högst rank först) ska fungera så måste jämförelsen returnera \code{false}.

\begin{CodeSmall}
case class  Team(name: String, rank: Int) extends Ordered[Team]{
  override def compare(that: Team): Int = -rank.compare(that.rank)
}
\end{CodeSmall}



\QUESTEND



\WHAT{Sortering med inbyggda sorteringsmetoder.}

\QUESTBEGIN

\Task  \what~  För grundtyperna (\code{Int}, \code{Double}, \code{String}, etc.) finns en fördefinierad ordning som gör så att färdiga sorteringsmetoder fungerar på alla samlingar i \code{scala.collection}. Även jämförelseoperatorerna i uppgift \ref{task:string-order-operators} fungerar enligt den fördefinierade ordningsdefinitionen för alla grundtyper. Denna ordningsdefinition är \textit{implicit tillgänglig} vilket betyder att kompilatorn hittar ordningsdefinitionen utan att vi explicit måste ange den. Detta fungerar i Scala även med primitiva \code{Array}.

\Subtask Testa metoden \code{sorted} på några olika samlingar. Förklara vad som händer. Hur lyder felmeddelandet på sista raden? Varför blir det fel?

\begin{REPL}
scala> Vector(1.1, 4.2, 2.4, 42.0, 9.9).sorted
scala> val xs = (100000 to 1 by -1).toArray
scala> xs.sorted
scala> xs.map(_.toString).sorted
scala> xs.map(_.toByte).sorted.distinct
scala> case class Person(firstName: String, familyName: String)
scala> val ps = Vector(Person("Robin", "Finkodare"), Person("Kim","Fulhack"))
scala> ps.sorted
\end{REPL}

\Subtask Om man har en samling med egendefinierade klasser eller man vill ha en annan sorteringsordning får man definiera ordningen själv. Ett helt generellt sätt att göra detta på  illustreras i uppgift \ref{task:custom-ordering}, men de båda hjälpmetoderna \code{sortWith} och \code{sortBy} räcker i de flesta fall. Hur de används illustreras nedan. Metoden \code{sortBy} kan användas om man tar fram ett värde av grundtyp och är nöjd med den inbyggda sorteringsordningen.

Metoden \code{sortWith} används om man vill skicka med ett eget jämförelsepredikat som ordnar två element; funktionen ska returnera \code{true} om det första elementet ska vara först, annars \code{false}.

\begin{REPL}
scala> case class Person(firstName: String, familyName: String)
scala> val ps = Vector(Person("Robin", "Finkodare"), Person("Kim","Fulhack"))
scala> ps.sortBy(_.firstName)
scala> ps.sortBy(_.familyName)
scala> ps.sortBy  // tryck TAB två gånger för att se signaturen
scala> ps.sortWith((p1, p2) => p1.firstName > p2.firstName)
scala> ps.sortWith  // tryck TAB två gånger för att se signaturen
scala> Vector(9,5,2,6,9).sortWith((x1, x2) => x1 % 2 > x2 % 2)
\end{REPL}
Vad har metoderna \code{sortWith} och \code{sortBy} för signaturer?

\Subtask Lägg till attributet \code{age: Int} i case-klassen \code{Person} ovan och lägg till fler personer med olika namn och ålder i en vektor och sortera den med \code{sortBy} och \code{sortWith} för olika attribut. Välj själv några olika sätt att sortera på.



\SOLUTION


\TaskSolved \what


\SubtaskSolved
\begin{enumerate}
\item Returnerar en sorterad \code{Vector} av \code{double}-värden
\item Skapar en variabel xs och sparar en \code{Array} med \code{Int}-värden mellan 100000 till 1.
\item Sorterar \code{xs = 1,2,3...}
\item Konverterar xs till en \code{Array} av \code{String}-värden och sorterar dem lexikografiskt: \code{xs = "1", "10", "100"} etc.
\item Konverterar xs till en \code{Array} av \code{Byte}-värden (max 127, min -128) och sorterar dem, samt tar bort dubletter: \code{xs = -128, -127, -1...}
\item Skapar en ny klass \code{Person} som tar 2 \code{String}-argument i konstruktorn
\item Sparar en Vector med två \code{Person}-objekt i en variabel ps
\item Försöker kalla på \code{sorted}-metoden för klassen \code{Person}. Eftersom vi skrivit denna klass själva och inte berättat för Scala hur \code{Person}-objekt ska sorteras, resulterar detta i ett felmeddelande.
\end{enumerate}

\SubtaskSolved

\begin{enumerate}
\item ---
\item ---
\item Sorterar \code{Person}-objekten i ps med avseende på värdet i \code{firstName}
\item Sorterar \code{Person}-objekten i ps med avseende på värdet i \code{familyName}
\item \code{sortBy} tar en funktion f som argument. f ska ta ett argument av typen \code{Person} och returnera en generisk typ B.
\item Sortera \code{Person}-objekten i ps med avseende på \code{firstName} i sjunkande ordning (omvänt från tidigare alltså)
\item \code{sortWith} tar en funktion lt som argument. lt ska i sin tur ta två argument av typen \code{Person} och returnera ett booleskt värde.
\item Sorterar en vektor så att värdena som är minst delbara med 2 hamnar först, och de mest delbara med 2 hamnar sist. Detta delar alltså upp udda och jämna tal.
\end{enumerate}

\SubtaskSolved
Klassens signatur blir då:
\begin{REPLnonum}
case class Person(firstName: String, lastName: String, age: Int)
\end{REPLnonum}

Lägg in dem i en vektor:
\begin{REPLnonum}
val ps2 = Vector(Person("a", "asson", 34), Person("asson", "assonson", 1234),
Person("anna", "Book", 2))
\end{REPLnonum}

Sortera dem på olika sätt:
\begin{enumerate}
\item
Vektorn blir sorterad med avseende på personernas ålder i stigande ordning
\begin{REPLnonum}
scala> ps2.sortWith((p1, p2) => p1.age < p2.age)
res40: scala.collection.immutable.Vector[Person] = Vector(Person(anna,Book,2),
Person(a,asson,34), Person(asson,assonson,1234))
\end{REPLnonum}

\item
Sorterar vektorn med avseende på namn, men också med avseende på ålder (i sjunkande ordning). För att komma före någon i ordningen måste alltså både namnet komma före, och åldern vara högre.
\begin{REPLnonum}
scala> ps2.sortWith((p1, p2) => (p1.firstName < p2.firstName) &&
(p1.age > p2.age))
res42: scala.collection.immutable.Vector[Person] = Vector(Person(a,asson,34),
Person(asson,assonson,1234), Person(anna,Book,2))
\end{REPLnonum}
\end{enumerate}



\QUESTEND



