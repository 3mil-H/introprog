%!TEX encoding = UTF-8 Unicode
%!TEX root = ../compendium2.tex

\Lab{\LabWeekTHREE}
\begin{Goals}
%!TEX encoding = UTF-8 Unicode
%!TEX root = ../compendium2.tex

%\item Kunna kompilera Scalaprogram med \texttt{scalac}.
%\item Kunna köra Scalaprogram med \texttt{scala}.
%\item Kunna definiera och anropa funktioner.
%\item Kunna använda och förstå default-argument.
%\item Kunna ange argument med parameternamn.
\item Kunna skapa ett större program med din egen kod efter dina egna idéer.
\item Kunna använda en editor och terminalen för att iterativt editera, kompilera, och testa din kod.
\item Kunna använda variabler i kombination med alternativ och repetetition i flera nivåer.
\item Kunna stegvis förbättra din kod för att underlätta förändring och öka läsbarhet.
\item Kunna skapa och använda abstraktioner för att generalisera och möjliggöra återanvändning av kod.

\end{Goals}

\begin{Preparations}
\item \DoExercise{\ExeWeekTWO}{02}
\item \DoExercise{\ExeWeekTHREE}{03}
\item Utveckla en första, spelbar version av ditt textspel, som du kan jobba vidare på efter återkoppling.
\item Hitta någon som spelar ditt spel och läser din kod och skriv ner den återkoppling du får på kodens läsbarhet.
\item Spela någon annans textspel och ge återkoppling på kodens läsbarhet.
\end{Preparations}


\subsection{Obligatoriska uppgifter}

Utveckla ditt textspel enligt nedan krav:

\begin{itemize}
\item Du ska skapa ett \textit{lagom} irriterande textspel anpassat för terminalen. Vid starten av ditt laborationstillfälle ska du ha en spelbar version av ditt textspel, som du kan jobba vidare på under själva laborationen. 

\item Under redovisningen mot slutet av laborationen ska du redogöra för hur du tränat på olika begrepp under utvecklingen av ditt textspel. Du ska också för handledaren beskriva den återkoppling du fått från någon som spelat ditt textspel och hur den har påverkat utformningen av din kod. 

\item Ditt textspel ska vara \emph{lagom} irriterande om den som spelar har läst koden, medan spelet gärna får vara mer än lagom irriterande för den som \emph{inte} läst koden. Det ska gå att klara spelet (du väljer själv vad det innebär) och därmed avsluta programmet inom rimlig tid med kännedom om koden.

\item Din kod ska vara \textit{lätt att läsa och förstå}, även om själva spelet stundtals kan vara mer eller mindre obegripligt, knasigt, eller besvärligt, för den spelare som inte har tillgång till koden...

\item Din kod ska vara uppdelad i många, korta abstraktioner med väl valda namn för att öka läsbarheten. Du ska också i görligaste mån namnge värden och dela upp koden i delar för att underlätta förändring av koden.

\item Din kod ska använda de viktiga begrepp som kursen hittills har behandlat, med speciellt fokus på det som just du behöver träna mest på. 

\item Slumptal ska ingå i ditt spel och styra valfria delar av exekveringen.

\item Det ska gå att ge ett valfritt slumptalsfrö som argument vid exekveringen av ditt program. Om fröargument ges ska exekveringen bli återupprepningsbar för en given indatasekvens, annars ska utfallet kunna bli olika vid upprepade körningar med samma indata.
\end{itemize}

\subsection{Inspiration}

Här följer en lista med olika förslag på funktioner som du kan välja bland, kombinera och variera på olika vis. Du kan också låta helt andra funktioner ingå i ditt spel.

\begin{itemize}
\item Be användaren logga in. Ge knasiga felmeddelande om användaren inte kan lösenordet.
\item Låt användaren hamna i en irriterande oändlig loop av meningslösa frågor om den gör ''fel''.
\item Beskriv en läskig fantasiplats där användaren befinner sig, till exempel en grotta/en källare/ett rymdskepp/M-huset.
\item Låt användaren välja mellan olika vägar/dörrar/tunnlar i flera steg. Låt valet styra vilka monster som påträffas.
\item Låt användaren välja mellan fåniga vapen, till exempel golvmopp, örontops, förgiftat kexchoklad.
\item Inför någon slags poäng som redovisas under spelets gång och i slutet.
\item Inför olika sorters poäng för hälsa, stridskraft, uppnådd skicklighetsnivå, etc.
\item Fråga användaren om mer eller mindre relevanta detaljer: namn/skonummer/favorithusdjur. Ge knasiga kommentarer där dessa detaljer ingår som delsträngar.
\item Spela sten/sax/påse med användaren.
\item Spela ''gissa talet'' och ge ledtrådar om talet är för litet eller för stort.
\item Mät hur lång tid det tar för användaren att klara ditt spel och ge poäng därefter.
\item Kolla reaktionstiden hos användaren genom att mäta tiden det tar att trycka Enter efter att man fått vänta en slumpmässig tid på att strängen \code{"NU!"} skrivs ut. Om man trycker Enter innan startutskriften ges är det fusk och man förlorar. Mät reaktionstiden upprepade gånger och ge poäng efter medelvärdet.
\item Låt användaren på tid snabbt skriva olika ord baklänges.
\item Be användaren skriva en palindrom. Ge poäng efter längd.
\item Träna användaren i multiplikationstabellen.
\end{itemize}


\subsection{Tips}

\begin{itemize}
\item Du kommer åt första argumentet till ditt program genom att indexera i en array som heter \code{args} på plats noll så här: \code{args(0)}.
\item Du kan kontrollera om det finns minst ett argument med hjälp av det booelska uttrycket \code{args(0).length > 0}.
\item Metoden \code{toInt} kan göra om en sträng till ett heltal. Du kan vid felaktiga heltal ge ett defaultvärde med \code{scala.util.Try(args(0).toInt).getOrElse(42)}.
\item Du läser från \textit{standard in} med \code{scala.io.StdIn.readline(prompt)} där \code{prompt} är en sträng som skrivs till \textit{standard out} innan inläsning sker.
\item Sök upp och studera dokumentationen för klassen \code{scala.util.Random}. 
\item Du kan vänta i t.ex. 3 sekunder med hjälp av Thread.sleep(3000).
\end{itemize}




