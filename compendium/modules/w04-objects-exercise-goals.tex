%!TEX encoding = UTF-8 Unicode
%!TEX root = ../exercises.tex

\item Kunna skapa och använda objekt som moduler.
\item Kunna förklara vad ett block och en lokal variabel är.
\item Kunna skapa och använda lokala funktioner och förklara nyttan med dessa.
\item Kunna förklara hur nästlade block påverkar namnsynlighet och namnöverskuggning.
\item Kunna förklara begreppen synlighet, privat medlem, import, namnrymd och namnskuggning.
%https://it-ord.idg.se/ord/overskuggning/

\item Kunna förklara kopplingen mellan paketstruktur och kodfilstruktur.
\item Kunna skapa en jar-fil.

\item Kunna skapa dokumentation med scaladoc. \TODO ???flytta till projektveckan och kräva att man visar doc för handledaren?

\item Kunna skapa och använda variabler med fördröjd initialisering.
\item Kunna förklara skillnaden mellan värdeanrop och namnanrop.
\item Känna till att det går att skapa egna kontrollstrukturer genom namnanrop.

\item Kunna skapa och använda tupler.
\item Kunna skapa funktioner som har multipla returvärden.
\item Känna till att parenteserna kan skippas runt tupel-argument.

\item \TODO FLER MÅL OM OBJEKT OCH MODULER HÄR
