%!TEX encoding = UTF-8 Unicode
%!TEX root = ../exercises.tex

\item Kunna skapa och använda objekt som moduler.
%\item Kunna förklara vad ett block och en lokal variabel är.
%\item Kunna skapa och använda lokala funktioner och förklara nyttan med dessa.
\item Kunna förklara hur nästlade block påverkar namnsynlighet och namnöverskuggning.
\item Kunna förklara begreppen synlighet, privat medlem, namnrymd och namnskuggning.
%https://it-ord.idg.se/ord/overskuggning/

\item Kunna skapa och använda tupler.
\item Kunna skapa funktioner som har multipla returvärden.
\item Kunna förklara den semantiska relationen mellan funktioner och objekt i Scala.

\item Kunna förklara kopplingen mellan paketstruktur och kodfilstruktur.
\item Kunna använda en jar-fil och classpath.

\item Kunna använda import av medlemmar i objekt och paket.
\item Kunna byta namn vid import.

\item Kunna skapa och använda variabler med fördröjd initialisering.
\item Kunna förklara när parenteserna kan skippas runt tupel-argument.

\item Känna till att det går att skapa egna kontrollstrukturer genom namnanrop.
\item Känna till skillnaden mellan värdeanrop och namnanrop.
