%!TEX encoding = UTF-8 Unicode
%!TEX root = ../compendium.tex

\Assignment{bank}

\subsection{Fokus}
\begin{itemize}[nosep,label={$\square$},leftmargin=*]
\item Kunna implementera ett helt program efter given specifikation
\item Kunna sätta samman olika delar från olika moduler
\item Förstå hur Java-klasser kan användas i Scala
\item Förstå och bedöma när immutable/mutable såväl som var/val bör användas i större sammanhang
\item Kunna använda sig av kompanjons-objekt
\item Kunna läsa och skriva till fil
\item Kunna söka i olika datastrukturer på olika sätt
\end{itemize}

\subsection{Bakgrund}

I detta projekt ska du skriva ett program som håller reda på bankkonton och kunder i en bank. Programmet ska även hålla reda på bankens nuvarande tillstånd, såväl som föregående.
Tillstånden ska vid varje tillståndsförändring skrivas till fil så att utifall banken skulle krascha finns alla transaktioner som genomförts sparade,  banken kan således återställas.

Programmet ska vara helt textbaserat, man ska alltså interagera med programmet via konsollen där en meny skrivs ut och input görs via tangentbordet.

Du ska skriva hela programmet själv, med andra ord ges ingen färdig kod. Programmet ska dock följa de specifikationer som ges i uppgiften, såväl som de objektorienterade principer du lärt dig i kursen.

\subsection{Krav}

Kraven för bankapplikationen återfinns här nedan. För att bli godkänd på denna uppgift måste samtliga krav uppfyllas:

\begin{itemize}
\item Programmet ska ha följande menyval:

\begin{itemize}
\item 1. Hitta konton för en viss kontoinnehavare med angivet ID.
\item 2. Söka efter kontoinnehavare på (del av) namn.
\item 3. Sätta in pengar på ett konto.
\item 4. Ta ut pengar på ett konto.
\item 5. Överföra pengar mellan två olika konton.
\item 6. Skapa ett nytt konto.
\item 7. Ta bort ett befintligt konto.
\item 8. Skriv ut bankens alla konton, sorterade i bokstavsordning efter innehavare.
\item 9. Återställa banken till ett tidigare tillstånd för ett givet datum. För simplictet får alla transaktioner genomförda efter det datum banken återställts till permanent kasseras. 
\item 10. Avsluta.
\end{itemize}

\item Programmet ska skapa ett nytt tillstånd med tidsstämpel och spara gamla tillstånd varje gång då:
\begin{itemize}
\item Pengar sätts in på ett konto
\item Pengar tas ut från ett konto.
\item Pengar överförs mellan två konton.
\item Ett konto skapas.
\item Ett konto tas bort.
\end{itemize}
\item Då bankens tillstånd förändras ska detta skrivas till fil.
\item Då banken startas upp ska transaktionshistoriken läsas in så att banken laddar senaste sparade tillståndet.
\item Inga utskrifter eller inläsningar får göras i klasserna Customer, BankAccount, Bank, State eller Transaction. Allt som berör användargränssnittet ska ske i BankApplication. Det är tillåtet att använda valfritt antalat hjälpmetoder och hjälpklasser i klassen BankApplication.
\item Alla metoder och attribut ska ha lämpliga åtkomsträttigheter.
\item Valet av val/var och immutable/mutable måste vara lämpliga.
\item Din indata måste ge samma resultat som i exemplen (som kommer komma i framtiden) i bilagan.
\item Rimlig felhantering ska finnas, det är alltså önskvärt att programmet inte kraschar då man matar in felaktig input, utan istället säger till användaren att input är ogiltlig.
\item Programdesignen ska följa de specifikationer som är angivna nedan.
\item Det räcker med att banken ska kunna hantera heltal, men dessa ska ske med klassen \texttt{BigInt}.
\item Kontonummer ska genereras i klassen \texttt{BankAccount}, dessa ska vara unika för varje konto. Vid en tillståndförändringar ska dessa återställas, detta betyder att om en återställning tar bort ett konto så ska detta kontonummer återigen bli tillgängligt.
\end{itemize}

\subsection{Design}
Nedan följer specifikationerna för de olika klasserna bankapplikationen måste innehålla:

\begin{ScalaSpec}{Customer}
/**
 * Describes a customer of a bank with provided name and id.
 */
class Customer(val name: String, val id: Int) = {
	override def toString(): String = ???
}

\end{ScalaSpec}


\begin{ScalaSpec}{BankAccount}

/**
 * Creates a new bank account for the customer provided.
 * The account is given a unique account number and initially
 * got a balance of 0 kr.
 */
class BankAccount(val holder: Customer) = {

 /**
   * Deposits provided amount on this account.
   */
  def deposit(amount: Int): Unit = ???

/**
   * Returns the balance of this account.
   */
  def getBalance: Int = ???

/**
   * Withdraws provided amount from this acccount, if there
   * is enough money on the account. Returns true if the 
   * transaction was successfull, otherwise false. 
   */
  def withdraw(amount: Int): Boolean = ???

}
\end{ScalaSpec}


\begin{ScalaSpec}{BankEvent}
/**
 * Describes an event happening in the bank.
 */
abstract class BankEvent {   
  /**
   * Output format for the transaction.
   */
  def write: String
}

\end{ScalaSpec}


\begin{ScalaSpec}{Bank}

/**
 * Creates a new bank with no accounts and no state. 
 */
class Bank() = {

  /**
   * Adds a new account in the bank.
   * The account number generates for the account is returned.
   */
  def addAccount(name: String, id: Int): Int = ???

 /**
   * Removes the bank account with provided account number,
   * returns true if successfull, otherwise false is returned.
   */
  def removeAccount(accountNbr: Int): Boolean = ???

 /**
   * Returns a list with every bank account in the bank.
   * The returned list is sorted in alphabetical order based
   * on customer name.
   */
  def getAllAccounts(): ArrayBuffer[BankAccount] = ???

  /**
   * Returns the account holding provided account number.
   * If no such account exists null is returned.
   */
  def findByNumber(accountNbr: Int): BankAccount = ???

/**
   * Returns a list with every account belonging to the customer
   * with provided id.
   */
  def findAccountsForHolder(id: Int): ArrayBuffer[BankAccount] = ???

/**
   * Returns a list with all customers which names matches
   * with provided name pattern.
   */
  def findByName(namePattern: String): ArrayBuffer[Customer] = ???

 /**
   * Executes a transaction in the bank.
   * Returns a string with information whether the
   * transaction was successful or failed.
   */
  def doEvent(transaction: Transaction): String = ???

  /**
   * Resets the bank to the state with time-stamp corresponding to the
   * provided date. If the date provided doesn't correspond exactly to 
   * any time-stamp then the nearest time-stamp with a date previous 
   *  to the provided date is used instead.
   * Returns a string with information whether the transaction was 
   * successful or failed.
   */
  def returnToState(returnDate: Date): String = ???

}
\end{ScalaSpec}


\begin{ScalaSpec}{State}

/**
 * Describes a bankstate.
 * The queue log consists of a lists with all transactions
 * made paired together with all dates corresponding to
 * thoose transactions.
 */
class State(val log: Queue[(Transaction, Date)]) 

\end{ScalaSpec}


För att använda tidsstämplar ska klassen Date som finns bifogat i kursens workspace användas. Detta är en enkel wrapper av Java.time.


\subsection{Tips} 

\begin{itemize}
\item För att representera tillstånden är det viktigt att alla händelser som förändrar tillståndet representeras av ett \texttt{BankEvent}.

\item För att skriva till fil på ett enkelt sätt kan man t.ex. använda sig av klassen \texttt{Files} som finns tillgänglig i \texttt{java.nio.file}. För att undvika portabilitetsproblem kan man då använda sig av ett bestämt \texttt{Charset}, t.ex. \texttt{UTF\_8}, som finns tillgänglig i {java.nio.charset.StandardCharsets.UTF\_8}.

\item För att läsa ifrån en fil kan man t.ex. använda sig av klassen \texttt{Source} som finns tillgänglig i \texttt{scala.io.Source}. 

\item Var nogrann med att testerna klarar alla tänkbara fall, och tänk på att fler fall än dem som givits i exempel kan förekomma vid rättning.
\end{itemize}

\subsection{Obligatoriska uppgifter}

\Task Implementera klassen \texttt{Customer}.

\Task Implementera klassen \texttt{BankAccount}.

\Task Implementera den \texttt{BankEvent} klass som skapar ett nytt konto.

\Task Skapa en ny klass \texttt{BankApplication}.

\Subtask Klassen \texttt{BankApplication} ska innehålla main-metoden. Det kan vara bra att innan man fortsätter se till att denna klass skriver ut menyn korrekt och kan ta input från tangentbordet som motsvarar de menyval som finns.

\Task Implementera klassen \texttt{Bank}.

\Subtask Implementera menyval 6 och 8. Testa noga.

\Subtask Implementera tillståndsfunktionaliteten. Varje nytt \texttt{BankEvent} ska ge upphov till ett nytt tillstånd och gamla tillstånd ska sparas som historik till det nya tillståndet.

\Subtask Implementera alla andra menyval, förutom menyval 9. Implementera även de klasser som förlängar \texttt{BankEvent} utefter att de behövs för nya menyval.
Testa de nya menyvalen noga efterhand som du implementerar dem, i synnerhet så att tillståndsförändringarna fungerar korrekt. Gör de utökningar du anser behövs. 

\Task Implementera menyval 9. När man försöker återställa banken till ett datum ska den senaste \texttt{BankEvent} genomförd före detta datum hämtas, med andra ord ska alla \texttt{BankEvent} med tidsstämpel efter återställningsdatumet kasseras permanent. Testa noga. Det är viktigt att denna funktionalitet fungerar bra innan man går vidare.

\Task Implementera säkerhetskopiering av tillstånden.

\Subtask Implementera utskriften till fil då ett nytt tillstånd skapas, utskriften ska ske omedelbart. Banken ska ej behöva avslutas för att utskriften ska hamna på fil, om så vore fallet kan information fortfarande gå förlorad om banken kraschar.

I repositoriet för denna projekt uppgift finns en sparfil bifogad, för bekvämlighet finns ett utdrag av denna fil  infogad nedanför. Inläsning och utskrift ska ske med dess format:\\~\\
2016 3 7 10 6 N 850127 Fredrik\newline
2016 3 7 10 28 D 1000 16500\newline
2016 3 9 10 52 W 1000 3900\newline
2016 3 9 11 8 N 900318 Casper\newline
2016 3 9 16 28 D 1001 6500\newline
2016 4 1 10 11 W 1001 1900\newline
2016 4 1 11 19 W 1001 2000\newline
2016 4 2 16 33 N 651002 Björn\newline
2016 4 2 16 46 D 1002 25000\newline
2016 4 3 10 11 T 1002 1000 4000\\~\\
Formen är alltså:\\~\\
\textbf{År  Månad  Dag  Timme  Minut  BankEventTag  Parametrar}
\\~\\
De olika klasserna av \texttt{BankEvent} representeras med följande bokstav:

\begin{itemize}
\item D - \texttt{Deposit}
\item W - \texttt{Withdraw}
\item T - \texttt{Transfer}
\item N - \texttt{NewAccount}
\item E - \texttt{DeletedAccount}
\end{itemize}

\Subtask Implementera inläsningen från fil då banken startas.


\subsection{Frivilliga extrauppgifter}

Gör först klart projektets obligatoriska delar. Därefter kan du, om du vill, utöka ditt
program enligt följande.

\Task Skriv en eller flera av klasserna \texttt{Customer}, \texttt{BankAccount} och \texttt{State} i Java istället och använd istället för din Scala versionen.

\Task	Implementera ett nytt menyalternativ som skriver ut all kontohistorik för en given person. I historiken ska typ av transaktion med tillhörande parametrar, dåvarande saldo vid transaktionen synas såväl som datumet för transaktionen synas.

\subsection{Exempel på körning av programmet}

Nedan visas möjliga exempel på körning av programmet. Data som matas in av användaren är markerad i fetstil.
Ditt program måste inte se identiskt ut, men den övergripande strukturen såväl som resultat av körningen ska vara densamma.
När exemplet börjar förutsätts det att banken inte har några konton.
\newline
- - - - - - - - - - - - - - - - - - - - - - - - - - - - - - - - - - - - - - - - - - - -\newline
1.   Hitta ett konto för en given kund\newline
2.   Sök efter en kund utifrån (del av) angivet namn\newline
3.   Sätt in pengar\newline
4.   Ta ut pengar\newline
5.   Överför pengar mellan konton\newline
6.   Skapa nytt konto\newline
7.   Radera existerande konto\newline
8.   Skriv ut alla konton i banken\newline
9.   Återställ banken till ett tidigare datum\newline
10. Avsluta\newline
Val: \textbf{6}\newline
Namn: \textbf{Adam Asson}\newline
Id: \textbf{6707071234}\newline
Nytt konto skapat med kontonummer: 1001\newline
10:03:0 CET 14 / 5 - 2016\newline
\newline
- - - - - - - - - - - - - - - - - - - - - - - - - - - - - - - - - - - - - - - - - - - -\newline
1.   Hitta ett konto för en given kund\newline
2.   Sök efter en kund utifrån (del av) angivet namn\newline
3.   Sätt in pengar\newline
4.   Ta ut pengar\newline
5.   Överför pengar mellan konton\newline
6.   Skapa nytt konto\newline
7.   Radera existerande konto\newline
8.   Skriv ut alla konton i banken\newline
9.   Återställ banken till ett tidigare datum\newline
10. Avsluta\newline
Val: \textbf{1}\newline
Id: \textbf{6707071234}\newline
Adam Asson, id 6707071234\newline
10:04:0 CET 14 / 5 - 2016\newline
\newline

- - - - - - - - - - - - - - - - - - - - - - - - - - - - - - - - - - - - - - - - - - - -\newline
1.   Hitta ett konto för en given kund\newline
2.   Sök efter en kund utifrån (del av) angivet namn\newline
3.   Sätt in pengar\newline
4.   Ta ut pengar\newline
5.   Överför pengar mellan konton\newline
6.   Skapa nytt konto\newline
7.   Radera existerande konto\newline
8.   Skriv ut alla konton i banken\newline
9.   Återställ banken till ett tidigare datum\newline
10. Avsluta\newline
Val: \textbf{6}\newline
Namn: \textbf{Berit Besson}\newline
Id: \textbf{8505255678}\newline
Nytt konto skapat med kontonummer: 1001\newline
10:12:0 CET 14 / 5 - 2016\newline 
\newline
- - - - - - - - - - - - - - - - - - - - - - - - - - - - - - - - - - - - - - - - - - - -\newline
1.   Hitta ett konto för en given kund\newline
2.   Sök efter en kund utifrån (del av) angivet namn\newline
3.   Sätt in pengar\newline
4.   Ta ut pengar\newline
5.   Överför pengar mellan konton\newline
6.   Skapa nytt konto\newline
7.   Radera existerande konto\newline
8.   Skriv ut alla konton i banken\newline
9.   Återställ banken till ett tidigare datum\newline
10. Avsluta\newline
Val: \textbf{8}\newline
Konto 1000 (Adam Asson, id 850127) 0 kr\newline
Konto 1001 (Berit Besson, id 900318) 0 kr\newline
10:13:0 CET 14 / 5 - 2016\newline
- - - - - - - - - - - - - - - - - - - - - - - - - - -\newline
1.   Hitta ett konto för en given kund\newline
2.   Sök efter en kund utifrån (del av) angivet namn\newline
3.   Sätt in pengar\newline
4.   Ta ut pengar\newline
5.   Överför pengar mellan konton\newline
6.   Skapa nytt konto\newline
7.   Radera existerande konto\newline
8.   Skriv ut alla konton i banken\newline
9.   Återställ banken till ett tidigare datum\newline
10. Avsluta\newline
Val: \textbf{2}\newline
Namn: \textbf{adam}\newline
Adam Asson, id 6707071234\newline
10:15:0 CET 14 / 5 - 2016\newline
\newline
- - - - - - - - - - - - - - - - - - - - - - - - - - - - - - - - - - - - - - - - - - - -\newline
1.   Hitta ett konto för en given kund\newline
2.   Sök efter en kund utifrån (del av) angivet namn\newline
3.   Sätt in pengar\newline
4.   Ta ut pengar\newline
5.   Överför pengar mellan konton\newline
6.   Skapa nytt konto\newline
7.   Radera existerande konto\newline
8.   Skriv ut alla konton i banken\newline
9.   Återställ banken till ett tidigare datum\newline
10. Avsluta\newline
Val: \textbf{6}\newline
Namn: \textbf{Berit Besson}\newline
Id: \textbf{8505255678}\newline
Nytt konto skapat med kontonummer: 1002\newline
13:56:0 CET 14 / 5 - 2016\newline
\newline
- - - - - - - - - - - - - - - - - - - - - - - - - - - - - - - - - - - - - - - - - - - -\newline
1.   Hitta ett konto för en given kund\newline
2.   Sök efter en kund utifrån (del av) angivet namn\newline
3.   Sätt in pengar\newline
4.   Ta ut pengar\newline
5.   Överför pengar mellan konton\newline
6.   Skapa nytt konto\newline
7.   Radera existerande konto\newline
8.   Skriv ut alla konton i banken\newline
9.   Återställ banken till ett tidigare datum\newline
10. Avsluta\newline
Val: \textbf{2}\newline
Namn: \textbf{erit}\newline
Konto 1001 (Berit Besson, id 900318) 0 kr\newline
Konto 1002 (Berit Besson, id 900318) 0 kr\newline
14:01:0 CET 14 / 5 - 2016\newline
- - - - - - - - - - - - - - - - - - - - - - - - - - - - - - - - - - - - - - - - - - - -\newline
1.   Hitta ett konto för en given kund\newline
2.   Sök efter en kund utifrån (del av) angivet namn\newline
3.   Sätt in pengar\newline
4.   Ta ut pengar\newline
5.   Överför pengar mellan konton\newline
6.   Skapa nytt konto\newline
7.   Radera existerande konto\newline
8.   Skriv ut alla konton i banken\newline
9.   Återställ banken till ett tidigare datum\newline
10. Avsluta\newline
Val: \textbf{3}\newline
Kontonummer: \textbf{1000}\newline
Summa: \textbf{5000}\newline
Transaktionen lyckades.\newline
14:36:0 CET 14 / 5 - 2016\newline
\newline
- - - - - - - - - - - - - - - - - - - - - - - - - - - - - - - - - - - - - - - - - - - -\newline
1.   Hitta ett konto för en given kund\newline
2.   Sök efter en kund utifrån (del av) angivet namn\newline
3.   Sätt in pengar\newline
4.   Ta ut pengar\newline
5.   Överför pengar mellan konton\newline
6.   Skapa nytt konto\newline
7.   Radera existerande konto\newline
8.   Skriv ut alla konton i banken\newline
9.   Återställ banken till ett tidigare datum\newline
10. Avsluta\newline
Val: \textbf{5}\newline
Kontonummer att överföra ifrån: \textbf{1000}\newline
Kontonummer att överföra till: \textbf{1001}\newline
Summa: \textbf{1000}\newline
Transaktionen lyckades.\newline
14:37:0 CET 14 / 5 - 2016\newline
\newline
- - - - - - - - - - - - - - - - - - - - - - - - - - - - - - - - - - - - - - - - - - - -\newline
1.   Hitta ett konto för en given kund\newline
2.   Sök efter en kund utifrån (del av) angivet namn\newline
3.   Sätt in pengar\newline
4.   Ta ut pengar\newline
5.   Överför pengar mellan konton\newline
6.   Skapa nytt konto\newline
7.   Radera existerande konto\newline
8.   Skriv ut alla konton i banken\newline
9.   Återställ banken till ett tidigare datum\newline
10. Avsluta\newline
Val: \textbf{8}\newline
Konto 1000 (Adam Asson, id 850127) 4000 kr\newline
Konto 1001 (Berit Besson, id 900318) 1000 kr\newline
Konto 1002 (Berit Besson, id 900318) 0 kr\newline
14:52:0 CET 14 / 5 - 2016\newline
\newline
- - - - - - - - - - - - - - - - - - - - - - - - - - - - - - - - - - - - - - - - - - - -\newline
1.   Hitta ett konto för en given kund\newline
2.   Sök efter en kund utifrån (del av) angivet namn\newline
3.   Sätt in pengar\newline
4.   Ta ut pengar\newline
5.   Överför pengar mellan konton\newline
6.   Skapa nytt konto\newline
7.   Radera existerande konto\newline
8.   Skriv ut alla konton i banken\newline
9.   Återställ banken till ett tidigare datum\newline
10. Avsluta\newline
Val: \textbf{2}\newline
Ange konto att radera: : \textbf{1002}\newline
Transaktionen lyckades.\newline
14:01:0 CET 14 / 5 - 2016\newline
- - - - - - - - - - - - - - - - - - - - - - - - - - - - - - - - - - - - - - - - - - - -\newline
1.   Hitta ett konto för en given kund\newline
2.   Sök efter en kund utifrån (del av) angivet namn\newline
3.   Sätt in pengar\newline
4.   Ta ut pengar\newline
5.   Överför pengar mellan konton\newline
6.   Skapa nytt konto\newline
7.   Radera existerande konto\newline
8.   Skriv ut alla konton i banken\newline
9.   Återställ banken till ett tidigare datum\newline
10. Avsluta\newline
Val: \textbf{7}\newline
Namn: \textbf{erit}\newline
Konto 1001 (Berit Besson, id 900318) 0 kr\newline
14:01:0 CET 14 / 5 - 2016\newline
- - - - - - - - - - - - - - - - - - - - - - - - - - - - - - - - - - - - - - - - - - - -\newline
1.   Hitta ett konto för en given kund\newline
2.   Sök efter en kund utifrån (del av) angivet namn\newline
3.   Sätt in pengar\newline
4.   Ta ut pengar\newline
5.   Överför pengar mellan konton\newline
6.   Skapa nytt konto\newline
7.   Radera existerande konto\newline
8.   Skriv ut alla konton i banken\newline
9.   Återställ banken till ett tidigare datum\newline
10. Avsluta\newline
Val: \textbf{9}\newline
Vilket datum vill du återställa banken till?\newline
År: \textbf{2016}\newline
Månad: \textbf{5}\newline
Datum (dag): \textbf{14}\newline
Timme: \textbf{10}\newline
Minut: \textbf{5}\newline
Banken återställd.\newline
15:00:0 CET 14 / 5 - 2016\newline
- - - - - - - - - - - - - - - - - - - - - - - - - - - - - - - - - - - - - - - - - - - -\newline
1.   Hitta ett konto för en given kund\newline
2.   Sök efter en kund utifrån (del av) angivet namn\newline
3.   Sätt in pengar\newline
4.   Ta ut pengar\newline
5.   Överför pengar mellan konton\newline
6.   Skapa nytt konto\newline
7.   Radera existerande konto\newline
8.   Skriv ut alla konton i banken\newline
9.   Återställ banken till ett tidigare datum\newline
10. Avsluta\newline
Val: \textbf{8}\newline
Konto 1000 (Adam Asson, id 850127) 0 kr\newline
15:01:0 CET 14 / 5 - 2016\newline
- - - - - - - - - - - - - - - - - - - - - - - - - - - - - - - - - - - - - - - - - - - -\newline
1.   Hitta ett konto för en given kund\newline
2.   Sök efter en kund utifrån (del av) angivet namn\newline
3.   Sätt in pengar\newline
4.   Ta ut pengar\newline
5.   Överför pengar mellan konton\newline
6.   Skapa nytt konto\newline
7.   Radera existerande konto\newline
8.   Skriv ut alla konton i banken\newline
9.   Återställ banken till ett tidigare datum\newline
10. Avsluta\newline
Val: \textbf{3}\newline
Kontonummer: \textbf{1001}\newline
Summa: \textbf{5000}\newline
Transaktionen misslyckades. Inget sådant konto hittades.\newline
15:06:0 CET 14 / 5 - 2016\newline
\newline
- - - - - - - - - - - - - - - - - - - - - - - - - - - - - - - - - - - - - - - - - - - -\newline
1.   Hitta ett konto för en given kund\newline
2.   Sök efter en kund utifrån (del av) angivet namn\newline
3.   Sätt in pengar\newline
4.   Ta ut pengar\newline
5.   Överför pengar mellan konton\newline
6.   Skapa nytt konto\newline
7.   Radera existerande konto\newline
8.   Skriv ut alla konton i banken\newline
9.   Återställ banken till ett tidigare datum\newline
10. Avsluta\newline
Val: \textbf{3}\newline
Kontonummer: \textbf{1001}\newline
Summa: \textbf{5000}\newline
Transaktionen misslyckades.  Otillräckligt saldo.\newline
15:23:0 CET 14 / 5 - 2016\newline
\newline
