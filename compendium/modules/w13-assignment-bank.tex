%!TEX encoding = UTF-8 Unicode
%!TEX root = ../compendium2.tex

\Assignment{bank}

\subsection{Fokus}
\begin{itemize}[nosep,label={$\square$},leftmargin=*]
\item Kunna implementera ett helt program efter given specifikation
\item Kunna sätta samman olika delar från olika moduler
\item Förstå hur Java-klasser kan användas i Scala
\item Förstå och bedöma när immutable/mutable såväl som var/val bör användas i större sammanhang
\item Kunna använda sig av kompanjonsobjekt
\item Kunna läsa och skriva till fil
\item Kunna söka i olika datastrukturer på olika sätt
\end{itemize}

\subsection{Bakgrund}

I detta projekt ska du skriva ett program som håller reda på bankkonton och kunder i en bank. Programmet ska utöver att hålla reda på bankens nuvarande tillstånd även föra historik över alla tillståndsändringar. Historiken ska vara så pass detaljerad att det nuvarande tillståndet kan återskapas genom att återuppspela alla ändringar som finns lagrade i historiken.

Programmet ska vara helt textbaserat, man ska alltså interagera med programmet via terminalen där en meny skrivs ut och input görs via tangentbordet.

Du ska skriva större delen av programmet själv, utan någon färdig kod. Programmet ska dock följa de specifikationer som ges i uppgiften, såväl som de objektorienterade principer du lärt dig i kursen.

\subsection{Krav}

Kraven för bankapplikationen återfinns här nedan. För att bli godkänd på denna uppgift måste samtliga krav uppfyllas:

\begin{itemize}
\item Programmet ska ha följande menyval:

\begin{itemize}
\item 1. Hitta konton för en viss kontoinnehavare med angivet ID.
\item 2. Söka efter kunder på (del av) namn.
\item 3. Sätta in pengar på ett konto.
\item 4. Ta ut pengar på ett konto.
\item 5. Överföra pengar mellan två olika konton.
\item 6. Skapa ett nytt konto.
\item 7. Ta bort ett befintligt konto.
\item 8. Skriva ut bankens alla konton, sorterade i bokstavsordning efter innehavare.
\item 9. Skriva ut historiken över alla ändringar av bankens tillstånd.
\item 10. Återställa banken till tillståndet den hade vid ett givet datum. För enkelhetens skull får du permanent kassera all historik som skapades efter det datum banken återställs till.
\item 11. Avsluta.
\end{itemize}

\item När något av följande sker ska programmet notera det i historiken:
\begin{itemize}
\item Pengar sätts in på ett konto.
\item Pengar tas ut från ett konto.
\item Pengar överförs mellan två konton.
\item Ett konto skapas.
\item Ett konto tas bort.
\end{itemize}
\item Historiken ska sparas både i minnet och i en fil.
\item Då programmet startas ska det läsa in historikfilen för att återskapa tillståndet som banken hade tidigare.
\item Formatet för historikfilen ska vara detsamma som formatet för den bifogade exempelfilen.
\item Allt som berör användargränssnittet (såsom utskrifter till terminalen och inläsning från terminalen) ska ske i \code{BankApplication} eller hjälpklasser till \code{BankApplication}, inte i någon annan av klasserna som specificeras i uppgiften.
\item Alla metoder och attribut ska ha lämplig synlighet, så att interna, förändringsbara delar inte i onödan exponeras.
\item Valen av val/var och immutable/mutable måste vara lämpliga.
\item Programmet ska fungera som i de bifogade exemplen på körning av programmet.
\item Rimlig felhantering ska finnas. Det är alltså önskvärt att programmet inte kraschar då användaren matar in felaktig input, utan istället säger till användaren att input är ogiltig. Du kan dock anta att historikfilen alltid är i rätt format.
\item Programdesignen ska följa de specifikationer som är angivna nedan.
\item Det räcker med att banken ska kunna hantera heltal, men detta ska göras med klassen \code{BigInt} för att tillåta stora belopp. Om din bank hanterar decimaltal ska detta ske med \code{BigDecimal} för att undvika avrundningsfel.
\item Klassen \code{BankAccount} ska generera ett unikt kontonummer för varje konto. Dessa ska återställas om bankens tillstånd återställs till ett tidigare datum, d.v.s. att om en återställning av banken tar bort ett konto så ska dess kontonummer återigen bli tillgängligt.
\item Det enda sättet att förändra tillståndet för en \code{Bank} ska vara (förutom att anropa \code{returnToState}) att anropa \code{doEvent} med en \code{BankEvent} som beskriver tillståndsförändringen. Vid en första anblick kan detta kan verka lite väl bökigt, men när ändringshistoriken ska implementeras kommer det vara till stor hjälp att det finns en \code{BankEvent} som representerar varje ändring.
\end{itemize}

\subsection{Design}
Nedan följer specifikationerna för de olika klasserna bankapplikationen måste innehålla:

\begin{ScalaSpec}{Customer}
/**
 * Describes a customer of a bank with provided name and id.
 */
case class Customer(name: String, id: Long) = {
	override def toString(): String = ???
}
\end{ScalaSpec}


\begin{ScalaSpec}{BankAccount}
/**
 * Creates a new bank account for the customer provided.
 * The account is given a unique account number and initially
 * has a balance of 0 kr.
 */
class BankAccount(val holder: Customer) = {
  val accountNumber: Int = ???

  /**
   * Deposits the provided amount in this account.
   */
  def deposit(amount: BigInt): Unit = ???

  /**
   * Returns the balance of this account.
   */
  def balance: BigInt = ???

  /**
   * Withdraws the provided amount from this account,
   * if there is enough money in the account. Returns true
   * if the transaction was successful, otherwise false.
   */
  def withdraw(amount: BigInt): Boolean = ???

  override def toString(): String = ???
}
\end{ScalaSpec}


\begin{ScalaSpec}{Bank}
/**
 * Creates a new bank with no accounts and no history.
 */
class Bank() {
 /**
   * Returns a list of every bank account in the bank.
   * The returned list is sorted in alphabetical order based
   * on customer name.
   */
  def allAccounts(): Vector[BankAccount] = ???

  /**
   * Returns the account holding the provided account number.
   */
  def findByNumber(accountNbr: Int): Option[BankAccount] = ???

  /**
   * Returns a list of every account belonging to
   * the customer with the provided id.
   */
  def findAccountsForHolder(id: Long): Vector[BankAccount] = ???

  /**
   * Returns a list of all customers whose names match
   * the provided name pattern.
   */
  def findByName(namePattern: String): Vector[Customer] = ???

  /**
   * Executes an event in the bank.
   * Returns a string describing whether the
   * event was successful or failed.
   */
  def doEvent(event: BankEvent): String = ???

  /**
   * Returns a log of all changes to the bank's state.
   */
  def history(): Vector[HistoryEntry] = ???

  /**
   * Resets the bank to the state it had at the provided date.
   * Returns a string describing whether the event was
   * successful or failed.
   */
  def returnToState(returnDate: Date): String = ???
}
\end{ScalaSpec}


Till din hjälp innehåller kursens workspace de färdigskrivna klasserna \code{HistoryEntry} och \code{Date} samt \code{BankEvent} med tillhörande subtyper.


\subsection{Tips}

\begin{itemize}
\item Använd ett \code{match}-uttryck för att hantera de olika subtyperna av \code{BankEvent} när du implementerar \code{doEvent}.
\begin{Code}
event match {
  case Deposit(account, amount) => ???
  case Withdraw(account, amount) => ???
  case Transfer(accFrom, accTo, amount) => ???
  case NewAccount(name, id) => ???
  case DeleteAccount(account) => ???
}
\end{Code}

\item För att skriva till fil på ett enkelt sätt kan man t.ex. använda sig av statiska metoder i klassen \code{Files} som finns tillgänglig i \code{java.nio.file}. För att undvika portabilitetsproblem kan man då använda sig av ett bestämt \code{Charset}, t.ex. \code{UTF_8}, som finns tillgänglig i \code{java.nio.charset.StandardCharsets.UTF_8}.

\item För att läsa ifrån en fil kan man t.ex. använda sig av \code{Source} som finns tillgänglig i \code{scala.io.Source}.

\item Var noggrann med att testerna klarar alla tänkbara fall, och tänk på att fler fall än dem som givits i exempel kan förekomma vid rättning.
\end{itemize}

\subsection{Obligatoriska uppgifter}

\Task Implementera klassen \code{Customer}.

\Task Implementera klassen \code{BankAccount}.

\Task Skapa singelobjektet \code{BankApplication}, som ska innehålla \code{main}-metoden. Det kan vara bra att innan man fortsätter se till att denna skriver ut menyn korrekt och kan ta input från tangentbordet som motsvarar de menyval som finns.

\Task Implementera klassen \code{Bank}.

\Subtask Implementera menyval 6. När användaren väljer att skapa ett nytt konto ska \code{BankApplication} skapa ett \code{NewAccount}-objekt som den sedan använder som argument i ett anrop till \code{doEvent} i \code{Bank}. Det är i \code{doEvent} (eller en privat funktion som anropas från \code{doEvent}) som kontot faktiskt ska skapas.

\Subtask Implementera menyval 8. Kontrollera att både menyval 6 och 8 fungerar rätt.

\Subtask Implementera menyval 9. Varje gång \code{doEvent} exekveras utan fel ska dess \code{BankEvent}-argument läggas till i historiken tillsammans med det nuvarande datumet.

\Subtask Implementera alla andra menyval, förutom menyval 10. Testa de nya menyvalen noga efterhand som du implementerar dem, i synnerhet så att ändringshistoriken fungerar korrekt. Gör de utökningar du anser behövs.

\Task Implementera säkerhetskopiering av historiken.

\Subtask När något läggs till i historiken ska det också skrivas till en historikfil omedelbart. Banken ska ej behöva avslutas för att utskriften ska hamna på fil, om så vore fallet kan information gå förlorad om banken kraschar. Använd \code{toLogString}-metoden i \code{HistoryEntry} för att få utskrifter i rätt format.

\Subtask När programmet startar ska det läsa in alla händelser från historikfilen och återuppspela dem en efter en. På så sätt kan bankens tillstånd återställas, fastän vi bara har sparat ändringshistoriken och inte själva tillståndet. Använd \code{fromLogString}-metoden i \code{HistoryEntry} när du läser in strängar från filen.

\Task Implementera menyval 10 genom att först nollställa bankens tillstånd och sedan återuppspela allt i historiken som hände före det givna datumet. Resten av historiken bör tas bort permanent, både i minnet och i historikfilen.


\subsection{Frivilliga extrauppgifter}

Gör först klart projektets obligatoriska delar. Därefter kan du, om du vill, utöka ditt program enligt följande.

\Task Skriv en eller flera av klasserna \code{Customer} och \code{BankAccount} i Java istället och använd dig av dessa i din Scala-kod.

\Task Implementera ett nytt menyalternativ som skriver ut all kontohistorik för en given person. I historiken ska finnas typ av händelse med tillhörande parametrar, dåvarande saldo vid händelsen, såväl som datumet för händelsen.

\subsection{Exempel på historikfil}

I workspace-katalogen för denna projektuppgift medföljer en historikfil. Inläsning och utskrift ska ske med dess format. Varje rad representerar en händelse, och formatet för en rad är: \textbf{År  Månad  Dag  Timme  Minut  BankEventTyp  Argument}. De olika sorternas \code{BankEvent} representeras med följande bokstäver: D för \code{Deposit}, W för \code{Withdraw}, T för \code{Transfer}, N för \code{NewAccount} och E för \code{DeleteAccount}.

\subsection{Exempel på körning av programmet}

Nedan visas möjliga exempel på körning av programmet. Data som matas in av användaren är markerad i fetstil.
Ditt program måste inte se identiskt ut, men den övergripande strukturen såväl som resultat av körningen ska vara densamma.
När det första exemplet börjar förutsätts det att banken inte har några konton.

Listan över val, som är markerad i kursiv stil i det första exemplet, är inte utskriven i senare exempel för att spara plats på pappret. Ditt program ska alltid skriva ut listan över val före användaren ska mata in ett val.

% This environment uses minipage to prevent column breaks from occurring in the middle of an example
\newenvironment{exampleblock}
	{\begin{minipage}{\columnwidth}
	 - - - - - - - - - - - - - - - - - - - - - - - - - - -\\}
	{\end{minipage}}

\begin{multicols}{2}
\noindent
\begin{exampleblock}
\textit{
1.   Hitta konton för en given kund\\
2.   Sök efter kunder på (del av) namn\\
3.   Sätt in pengar\\
4.   Ta ut pengar\\
5.   Överför pengar mellan konton\\
6.   Skapa nytt konto\\
7.   Radera existerande konto\\
8.   Skriv ut alla konton i banken\\
9.   Skriv ut ändringshistoriken\\
10.  Återställ banken till ett tidigare datum\\
11.  Avsluta\\
}
Val: \textbf{6}\\
Namn: \textbf{Adam Asson}\\
Id: \textbf{6707071234}\\
Nytt konto skapat med kontonummer: 1000\\
10:03 14/5-2016\\
\end{exampleblock}
\begin{exampleblock}
Val: \textbf{1}\\
Id: \textbf{6707071234}\\
Konto 1000 (Adam Asson, id 6707071234) 0 kr\\
10:04 14/5-2016\\
\end{exampleblock}
\begin{exampleblock}
Val: \textbf{6}\\
Namn: \textbf{Berit Besson}\\
Id: \textbf{8505255678}\\
Nytt konto skapat med kontonummer: 1001\\
10:12 14/5-2016\\
\end{exampleblock}
\begin{exampleblock}
Val: \textbf{8}\\
Konto 1000 (Adam Asson, id 6707071234) 0 kr\\
Konto 1001 (Berit Besson, id 8505255678) 0 kr\\
10:13 14/5-2016\\
\end{exampleblock}
\begin{exampleblock}
Val: \textbf{2}\\
Namn: \textbf{adam}\\
Adam Asson, id 6707071234\\
10:15 14/5-2016\\
\end{exampleblock}
\begin{exampleblock}
Val: \textbf{6}\\
Namn: \textbf{Berit Besson}\\
Id: \textbf{8505255678}\\
Nytt konto skapat med kontonummer: 1002\\
13:56 14/5-2016\\
\end{exampleblock}
\begin{exampleblock}
Val: \textbf{2}\\
Namn: \textbf{erit}\\
Berit Besson, id 8505255678\\
14:01 14/5-2016\\
\end{exampleblock}
\begin{exampleblock}
Val: \textbf{3}\\
Kontonummer: \textbf{1000}\\
Summa: \textbf{5000}\\
Transaktionen lyckades.\\
14:36 14/5-2016\\
\end{exampleblock}
\begin{exampleblock}
Val: \textbf{5}\\
Kontonummer att överföra ifrån: \textbf{1000}\\
Kontonummer att överföra till: \textbf{1001}\\
Summa: \textbf{1000}\\
Transaktionen lyckades.\\
14:37 14/5-2016\\
\end{exampleblock}
\begin{exampleblock}
Val: \textbf{8}\\
Konto 1000 (Adam Asson, id 6707071234) 4000 kr\\
Konto 1001 (Berit Besson, id 8505255678) 1000 kr\\
Konto 1002 (Berit Besson, id 8505255678) 0 kr\\
14:52 14/5-2016\\
\end{exampleblock}
\begin{exampleblock}
Val: \textbf{7}\\
Ange konto att radera: \textbf{1002}\\
Transaktionen lyckades.\\
14:54 14/5-2016\\
\end{exampleblock}
\begin{exampleblock}
Val: \textbf{8}\\
Konto 1000 (Adam Asson, id 6707071234) 4000 kr\\
Konto 1001 (Berit Besson, id 8505255678) 1000 kr\\
14:55 14/5-2016\\
\end{exampleblock}
\begin{exampleblock}
Val: \textbf{9}\\
10:03 14/5-2016: Skapade ett konto tillhörandes Adam Asson, id 6707071234\\
10:12 14/5-2016: Skapade ett konto tillhörandes Berit Besson, id 8505255678\\
13:56 14/5-2016: Skapade ett konto tillhörandes Berit Besson, id 8505255678\\
14:36 14/5-2016: Satte in 5000 kr i konto 1000\\
14:37 14/5-2016: Överförde 1000 kr från konto 1000 till konto 1001\\
14:54 14/5-2016: Raderade konto 1002\\
14:58 14/5-2016\\
\end{exampleblock}
\begin{exampleblock}
Val: \textbf{10}\\
Vilket datum vill du återställa banken till?\\
År: \textbf{2016}\\
Månad: \textbf{5}\\
Datum (dag): \textbf{14}\\
Timme: \textbf{10}\\
Minut: \textbf{5}\\
Banken återställd.\\
15:00 14/5-2016\\
\end{exampleblock}
\begin{exampleblock}
Val: \textbf{9}\\
10:03 14/5-2016: Skapade ett konto tillhörandes Adam Asson, id 6707071234\\
15:00 14/5-2016\\
\end{exampleblock}
\begin{exampleblock}
Val: \textbf{8}\\
Konto 1000 (Adam Asson, id 6707071234) 0 kr\\
15:01 14/5-2016\\
\end{exampleblock}
\begin{exampleblock}
Val: \textbf{3}\\
Kontonummer: \textbf{1001}\\
Summa: \textbf{5000}\\
Transaktionen misslyckades. Inget sådant konto hittades.\\
15:06 14/5-2016\\
\end{exampleblock}
\begin{exampleblock}
Val: \textbf{4}\\
Kontonummer: \textbf{1000}\\
Summa: \textbf{1000}\\
Transaktionen misslyckades. Otillräckligt saldo.\\
15:23 14/5-2016\\
\end{exampleblock}

\end{multicols}
