%!TEX encoding = UTF-8 Unicode

%!TEX root = ../compendium.tex

\Exercise{\ExeWeekEIGHT}

\begin{Goals}
\item 
\end{Goals}

\begin{Preparations}
\item 
\end{Preparations}

\BasicTasks %%%%%%%%%%%%%%%%

\Task 

\Subtask 

\Task \emph{Byta ut metoden \code{equals}}  Om man byter ut metoden den befintliga \code{equals} kommer metode \code{==} att fungera annorlunda. Vi börjar studera den befintliga equals med referenslikhet.

\begin{REPL}
scala> class Gurka(val vikt: Int)
scala> val g1, g2 = new Gurka(42)
scala> val g3 = new Gurka(42)
scala> g1 == g2
scala> g1 == g3
scala> g1.equals  // tryck TAB två gånger
\end{REPL}

\Subtask Om du trycker TAB två gånger efter ett metodnamn får du se metodens signatur. Vilken signatur har metoden \code{equals}?

\Subtask Byt ut equals enligt nedan och förklara vad som händer.

\begin{REPL}
scala> class Gurka(val vikt: Int) { 
         override def equals(other: Any): Boolean = other match {
           
         } 
scala> val g = new Gurka(42)
scala> g.equals  // tryck TAB två gånger
\end{REPL}



\Task \emph{Klassen \code{Complex} och metoden \code{equals}.} Implementera klassen \code{Complex} som representerar ett komplext tal\footnote{\href{https://sv.wikipedia.org/wiki/Komplexa_tal}{sv.wikipedia.org/wiki/Komplexa\_tal}} med realdel och imaginärdel.

\begin{REPL}
scala> class Complex(val re: Double, im: Double)

\end{REPL}

\ExtraTasks %%%%%%%%%%%%%%%%%%%

\Task 

\AdvancedTasks %%%%%%%%%%%%%%%%%

\Task     
    