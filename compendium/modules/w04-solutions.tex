%!TEX encoding = UTF-8 Unicode

%!TEX root = ../solutions.tex

\ExerciseSolution{\ExeWeekFOUR}

\Task %Uppgift 1

\Subtask  
\begin{REPLnonum}
scala> val pt = (15.9, 28.9)

scala> math.hypot(pt._1, pt._2)
res0: Double = 32.98514817307935
\end{REPLnonum}

\Subtask  \code{val (x, y) = pt}

\Subtask  \code{(String, String, Double, Boolean)}

\Subtask  \code{Vector[(Double, Double)]}

\Subtask  \code{huvudstäder :+ ''Danmark'' -> ''Köpenhamn''}

\Subtask  \code{Vector[Double]}

\Subtask  \code{val antalUdda = (1234 to 3456).map(i => div(i, 2)._2).sum}

\Subtask  0-tupel

\Task %Uppgift 2

\Subtask  \code{mittKonto.saldo = (math.random * 1000000).toInt}

\Subtask  Går ej eftersom val är oförändlig, man får alltså ett Error.

\Task %Uppgift 3

\Subtask  
Tilldelningen på rad 8 \code{k1.nummer = 12345L} ger felmeddelande eftersom variablen är oförändlig.

\Task %Uppgift 4

\Subtask  \code{String = Konto@cd576}, där \code{Konto@cd576} är ett unikt namn som identifierar instansen.

\Subtask  Ja.

\Subtask

\Subtask  Felmeddeland, eftersom variablen är oförändlig.

\Subtask  En fördel med klass är att man kan specificera att variablen ska kunna vara förändlig. En till är att man kan inkludera metoder i klassen som man vill kunna använda på värdena.

\Task %Uppgift 5

\Subtask 
Det går bra att ändra på variablen saldo i instansen av Konto1 men inte av Konto2 där man får ett error på raden ''k2.saldo += 1000''

\Subtask 
Som i deluppgift a ger båda exemplena samma error meddelande.

\Subtask 
\begin{REPLnonum}
def ut(belopp: Int): (Int, Int) = \{
	if(saldo >= belopp) \{
		saldo -= belopp
		(belopp, saldo)
	\} else \{
		val temp = saldo
		saldo = 0
		(temp, 0)
	\}
\}
\end{REPLnonum}

\Subtask 
Lägg till if(belopp > 0) i båda funktionerna och låt if-satsen omsluta all den gamla koden.

\Subtask 
Minskar risken till att det kan bli fel genom ändrar värdena på fel sätt. Genom att endast ha metoder som du speciferar kan du kontrollera hur användare av koden får ändra på värdena.

\Task %Uppgift 6

\Subtask 
''val i: Int = pt.x'' error: type mismatch;
Eftersom typen Int ej är kompatibel med ett värde av typen Double.

''val p: Double = new Punkt(5.0, 5.0)'' error: type mismatch;
Eftersom typen Double ej är kompatibel med ett värde av typen Punkt.

''val p = new Punkt(5.0, 5.0): Double'' error: type mismatch;
Samma som förra.

\Subtask 
true, false, false, true, false

\Task %Uppgift 7

\Subtask 
En variabel med namn pt skapas med typen Punkt.
true
true
String = 1.0
skriver ut: 1.0
error: not found: value a
String = 2.0
error: not found: value a

\Task %Uppgift 8

\Subtask 
''println(pt)'' kallar på pt.toString, och eftersom metoden är överskriven kallas vår version.

\Subtask 
override def toString: String = ''Punkt('' + x + '', '' + y + '').''

\Subtask 
får ett error meddelande.
error: overriding method toString in class OBject of type ()String;

\Task %Uppgift 9

\Subtask 
\begin{REPL}
scala> val pt = Pt(1.0, 2.0)
pt: Pt = Pt(x=1.0,y=2.0)

scala> Pt(4.0, 2.0)
res45: Pt = Pt(x=4.0,y=2.0)

scala> Pt(6.0, 3.0)
res46: Pt = Pt(x=6.0,y=3.0)

scala> Pt(666.0, 1337.0)
res47: Pt = Pt(x=666.0,y=1337.0)
\end{REPL}

\Subtask \code{def apply(): Pt = new Pt(0, 0)}

\Subtask \code{class Rational(val nom: Int, val denom: Int)}

\Subtask 
\begin{REPLnonum}
object Rational { 
def apply(nom: Int, denom: Int): Rational = new Rational(nom, denom)
}
\end{REPLnonum}

\Subtask 
\begin{REPL}
scala> Rational(2, 5)
scala> Rational(2, 7)
scala> Rational(7, 4)
scala> Rational(666, 1337)
\end{REPL}

\Task %Uppgift 10
\Subtask \code{case class Rational(nom: Int, denom: Int)}

\Task %Uppgift 11

\Subtask 
5 är avståndet.

\Subtask 
p3.x = 8
p3.y = 10

\Task %Uppgift 12

\Subtask 
Operatornotation:	4, 6, 10, 12
Punktnotation:		3, 5, 8, 9, 11, 13
FelMeddelande:		9

\Task %Uppgift 14

\Subtask  Ungefär 150 metoder.

\Subtask  Ungefär lika många.

\Task %Uppgift 15

\Subtask 
Instansierar en tom vektor med element av typen int.
Error eftersom ''xs :+ ''42''' ger en Vector[Any] när Vector[Int] krävs.
Vecktorn utökas.
Vektorn skrivs ut.
ger ett resultat av vektorn plus talet 42.
Error: type mismatch
skapa tom vektor
ge variablen en lista som värde.

\Subtask 
Tre metoder skapas, första för att få första elementet i en lista, och eftersom den defineras med special typen T går den att använda med alla Vektorer oavsätt typen av variable i vektorn. Den andra får fram sista elementet och sista hämtar båda två.

En till function defineras längre ner ''wrap'', som tar en lista och lägger till ett element längst fram och ett längst bak.

\Task %Uppgift 16

\Subtask  String = ''ka2''

\Subtask  String = ''abra''

\Subtask  false

\Subtask  false

\Subtask  100000

\Subtask  100000

\Subtask  minsta talet i listan

\Subtask  största talet i listan

\Subtask  1

\Subtask  3

\Subtask  Vektor b fast med ''först'' som första element

\Subtask  Vektor a fast med ''sist'' som sista element.

\Subtask  plats 3 i vektorn xs får värdet 42

\Subtask  En ny vektor fylld med ''!'' från och med plats 4 till 10. Men de andra värdena samma som i a.

\Subtask  b sorterad i bokstävsordning

\Subtask  b baklänges

\Subtask  true

\Subtask  true

\Subtask  en vektor med alla unika element i b.

\Task %Uppgift 17

\Subtask 
Metoden ger tillbaka en ny Vector[String] som nu består av alla element i a plus alla element i b. I samma ordning med elementen i a först.

\Subtask 
Samma som i uppgift a fast vektorn som returnas är av typen Vector[Any]. Det är eftersom Any är den närmsta typen som String och Double delar. Elementen från vektor a är fortfarande först och uppföljt av elementen i stor.

\Subtask 
Variablen ys får värdet av en Vector[Int] som innehåller alla talen från xs fast multiplicerade med 5. Alltså ys = 5, 10, 15..., osv.

\Subtask 
Functionen tar alla värden från en Vektor och sätter in i ett Set (mängd). Eftersom en mängd ej har dubletter så försvinner ett ''sala'' och ett ''bim'', Vector[String] som returneras blir därför (''sim'', ''sala'', ''bim'').

\Subtask 
Metoden head ger första elementet i en samling, och last sista. Därför blir kombinationen av a.head och b.last en ny Vector[String] som består av a:s första element, och b:s första element.

\Subtask 
Ger en Vector[String] som innehåller alla element efter det första. Alltså i detta fallet ''ka'' och ''dabra''.

\Subtask 
True, eftersom head ger första elementet och tail ger resten, sedan sätter metoden +: ihop dem till en vektor med samma värden som a.

\Subtask 
Eftersom ++ sätter ihop alla värden från två vektorer måste vi först omvandla från en sträng till vektor. Resultatet blir en ny vektor av samma typ som innan med a:s första element och b:S sista.

\Subtask 
Samma resultat som i h, metoden take börjar från vänster och tar så många element som man skickar med som parameter och gör till en ny lista. Med 1 som parameter motsvarar det att göra Vector(a.head). Metoden takeRight gör samma sak fast från höger.

\Subtask 
Metoden drop är motsvarigheten till take fast excuderar de speciferade elementen istället för att inkludera dem i vektorn.

\Subtask 
Eftersom a endast innehåller 3 element returnerar drop(100) en tom vektor.

\Subtask 
Returnerar en tom vektor med element typen String

\Subtask 
returnerar Vector(true, false) 

\Subtask 
True, metoden contains kollar om en samling innehåller ett specifikt element.

\Subtask 
True. Eftersom en sträng även kan ses som Vector[Char].

\Subtask 
Filtrerar vektorn a till att endast innehålla strängar som innehåller k.

\Subtask 
Exakt samma som i p

\Subtask 
map(\_.toUpperCase) omvandlar alla strängar i a till stora bokstäver
filterNot(\_.contains(''K'')) tar resultatet vi precis fick och tar bort alla strängar som innehåller stora K.

\Subtask 
filtrerar så att endast jämna tal finns kvar.

\Subtask 
Exakt samma som i s



\Task %Uppgift 18

\Subtask 
Vi instansierar en vektor xs med talen 1, 2 och 3.
sedan definerar vi en metod blandat som ger oss en randomiserad version av xs.
sedan definerar vi en till metod som testar om xs är lika med resultatet från blandat. Om det är så returnerar den strängen ''lika'' annars ''olika''.
Sist kör vi en for-loop där vi 100 gånger kör testet, samtidigt räknas hur många gånger ''lika'' returneras.

Vårt resultat är en siffra på hur många gånger xs var samma som en blandad version av sig själv, eftersom det finns 6 permutationer med 3 variabler så borde det vara ungefär 1/6 chans.

\Subtask 
\code{map(\_.trim)} tar bort alla onödiga mellanrum i början och slutet på varje rad
\code{filterNot(\_.startsWith(''<''))} filtrerar bort alla rader som börjar med strängen ''<''
\code{filterNot(\_.isEmpty)} filtrerar bort alla tomma rader.
\code{foreach(println)} skriver ut alla rader.

\Task %Uppgift 19

\Subtask 
I princip alla metoder delas, en lista har några fler t. ex. ''::'', '':::'', ''mapConserve'' osv.

\Subtask 
Först skapas en lista med 4 sträng värden och instansierar variablen xs med det värdet.
sedan skapar vi en ny lista, som består av ''zero'' + den gamla listan och ger värdet till xs.
Sist instansierar vi en ny variabel ys, som får värdet av xs omvänd plus xs.

\Task %Uppgift 20

\Subtask 
true, Boolean

\Subtask 
En samling av alla värden i s och t, Set[String]

\Subtask 
true, Boolean

\Subtask 
false, Boolean

\Subtask 
false, Boolean

\Subtask 
false, Boolean

\Subtask 
true, Boolean

\Subtask 
Samlingen s utan elementet ''Stockholm'', Set[String]

\Subtask 
Samlingen t utan elementen ''Norge'' och ''Danmark'', Set[String]

\Subtask 
returnerar s, Set[String]

\Subtask 
Samlingen s utan ''Malmö'' och ''Oslo'', Set[String]

\Subtask 
Set(2, 3), Set[Int]

\Subtask 
se deluppgift l

\Subtask 
Set(1, 2, 3 ,4), Set[Int]

\Subtask 
se deluppgift n

\Task %Uppgift 21

\Subtask 
Returnerar strängen ''Malmö'' eftersom det värdet är indexerat på platsen ''Skåne''.

\Subtask 
Returnerar strängen ''Stockholm'' eftersom det värdet är indexerat på platsen ''Sverige''.

\Subtask 
true, eftersom huvudstar innehåller indexet ''Skåne''

\Subtask 
false, eftersom huvudstad ej innehåller indexet ''Malmö''. Notera att det är index och inte värden vi 
kollar om det finns.

\Subtask 
Lägger till indexet ''Danmark'' med värdet ''Köpenhamn'' i samlingen.

\Subtask 
Skriver ut alla 2-tupler.

\Subtask 
Returnerar ''Oslo'', Note: Om indexet ''Norge'' inte hade funnits hade ''???'' returnerats istället.

\Subtask 
Returnerar ''???''

\Subtask 
Returnerar en sorterar vektor med alla index.

\Subtask 
Returnerar en sorterar vektor med alla värden.

\Subtask 
Returnerar en ny mängd men med ''Skåne'' -> ''Malmö'' borttaget. 

\Subtask 
Returnerar huvudstad mängden eftersom det inte finns ett ''Jylland'' index att ta bort.

\Subtask 
Uppdaterar indexet ''Skåne'' till att istället leda till värdet ''Lund''

\Task %Uppgift 22

\Subtask 
\begin{REPLnonum} 
pairs: scala.collection.immutable.Vector[(String, Long)] = 
					Vector((Björn,444), (Maj,441), (Lucy,666))
\end{REPLnonum}

\Subtask 
Map[String, Long]

\Subtask 
\begin{REPLnonum}
scala> telnr(''Maj'')
res0: Long = 441

scala> telnr.get(''Maj'')
res1: Option[Long] = Some(441)

scala> telnr(''Kim'')
java.util.NoSuchElementException: key not found: 'Kim
  at scala.collection.MapLike$class.default(MapLike.scala:228)
  at scala.collection.AbstractMap.default(Map.scala:59)
  at scala.collection.MapLike$class.apply(MapLike.scala:141)
  at scala.collection.AbstractMap.apply(Map.scala:59)
  ... 32 elided

scala> telnr.get(''Kim'')
res2: Option[Long] = None
\end{REPLnunom}

\Subtask 
\begin{REPLnonum} 
scala> telnr.getOrElse(''Maj'', -1L)
res0: Long = 441

scala> telnr.getOrElse(''Kim'', -1L)
res1: Long = -1
\end{REPLnonum} 

\Subtask 
telnr += ''Fröken Ur'' -> 464690510L

\Subtask 
telnr.toVector.map(p => p.\_1 -> (''0'' + p.\_2.toString.substring(2)))

\Subtask 
Använd metoden toMap och apply.



\Task %Uppgift 23

\Subtask  Metoden maxBy hämtar det element som är ''störst'', där storhet är baserat först på första värdet i 2-tuplarna och sedan på det andra värdet.

\Subtask  
\begin{REPLnonum}
scala> xs.maxBy(_._1)
scala> xs.maxBy(_._2)
\end{REPLnonum}

\Subtask  
\begin{REPLnonum}
scala> xs.minBy(_._1)
scala> xs.minBy(_._2)
\end{REPLnonum}

