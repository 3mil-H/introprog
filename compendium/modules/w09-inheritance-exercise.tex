
%!TEX encoding = UTF-8 Unicode
%!TEX root = ../exercises.tex

\ifPreSolution

\Exercise{\ExeWeekNINE}\label{exe:W09}

\begin{Goals}
%!TEX encoding = UTF-8 Unicode

%!TEX root = ../compendium2.tex

\item Förstå följande begrepp: supertyp, subtyp, bastyp, abstrakt typ, polymorfism.
\item Kunna deklarera och använda en arvshierarki i flera nivåer med nyckelordet \code{extends}.
\item Kunna deklarera och använda inmixning med flera traits och nyckelordet \code{with}.
\item Kunna deklarera och känna till nyttan med finala klasser och finala attribut och nyckelordet \code{final}.
\item Känna till synlighetsregler vid arv och nyttan med privata och skyddade attribut.
\item Kunna deklarera och använda skyddade attribut med nyckelordet \code{protected}.
\item Känna till hur typtester och typkonvertering vid arv kan göras med metoderna \code{isInstanceOf} och \code{asInstanceOf} och känna till att detta görs bättre med \code{match}.
\item Känna till begreppet anonym klass.
\item Kunna deklarera och använda överskuggade metoder med nyckelordet \code{override}.
\item Känna till reglerna som gäller vid överskuggning av olika sorters medlemmar.
\item Kunna deklarera och använda hierarkier av klasser där konstruktorparametrar överförs till superklasser.
\item Kunna deklarera och använda uppräknade värden med case-objekt och gemensam bastyp.

\end{Goals}

\begin{Preparations}
\item \StudyTheory{09}
\end{Preparations}

\BasicTasks

\else

\ExerciseSolution{\ExeWeekNINE}

\BasicTasks

\fi



\WHAT{Para ihop begrepp med beskrivning.}

\QUESTBEGIN

\Task \what

\vspace{1em}\noindent Koppla varje begrepp med den (förenklade) beskrivning som passar bäst:

\begin{ConceptConnections}
  bastyp & 1 & & A & körtidstypen avgör vilken metod som körs \\ 
  supertyp & 2 & & B & minneslagring kan optimeras, har supertypen \code|AnyVal| \\ 
  subtyp & 3 & & C & klass utan namn, utvidgad med extra implementation \\ 
  körtidstyp & 4 & & D & är endast synlig i subtyper \\ 
  dynamisk bindning & 5 & & E & kan ej instansieras \\ 
  polymorfism & 6 & & F & abstrakt klass, kan mixas in, kan ej ha parametrar \\ 
  trait & 7 & & G & saknar implementation \\ 
  inmixning & 8 & & H & kan vara mer specifik än den statiska typen \\ 
  överskuggad medlem & 9 & & I & ej värdetyp, har supertypen \code|AnyRef| \\ 
  anonym klass & 10 & & J & kan ha många former, t.ex. en av flera subtyper \\ 
  skyddad medlem & 11 & & K & subtypning utanför denna kodfil är förhindrad \\ 
  abstrakt medlem & 12 & & L & medlem i subtyp ersätter medlem i supertyp \\ 
  abstrakt klass & 13 & & M & en typ som är mer specifik \\ 
  referenstyp & 14 & & N & en typ som är mer generell \\ 
  förseglad typ & 15 & & O & klass får nya egenskaper från trait \\ 
  värdetyp & 16 & & P & den mest generella typen i en arvshierarki \\ 
\end{ConceptConnections}

\SOLUTION

\TaskSolved \what

\begin{ConceptConnections}
  bastyp & 1 & ~~\Large$\leadsto$~~ &  P & den mest generella typen i en arvshierarki \\ 
  sypertyp & 2 & ~~\Large$\leadsto$~~ &  N & en typ som är mer generell \\ 
  subtyp & 3 & ~~\Large$\leadsto$~~ &  M & en typ som är mer specifik \\ 
  körtidstyp & 4 & ~~\Large$\leadsto$~~ &  H & kan vara mer specifik än den statiska typen \\ 
  dynamisk bindning & 5 & ~~\Large$\leadsto$~~ &  A & körtidstypen avgör vilken metod som körs \\ 
  plymorfism & 6 & ~~\Large$\leadsto$~~ &  J & kan ha många former, t.ex. en av flera subtyper \\ 
  trait & 7 & ~~\Large$\leadsto$~~ &  F & abstrakt klass, kan mixas in, kan ej ha parametrar \\ 
  inmixning & 8 & ~~\Large$\leadsto$~~ &  O & klass får nya egenskaper från trait \\ 
  överskuggad medlem & 9 & ~~\Large$\leadsto$~~ &  L & medlem i subtyp ersätter medlem i supertyp \\ 
  anonym klass & 10 & ~~\Large$\leadsto$~~ &  C & den mest generella typen i en arvshierarki \\ 
  skyddad medlem & 11 & ~~\Large$\leadsto$~~ &  D & är endast synlig i subtyper \\ 
  abstrakt medlem & 12 & ~~\Large$\leadsto$~~ &  G & saknar implementation \\ 
  abstrakt klass & 13 & ~~\Large$\leadsto$~~ &  E & kan ej instansieras \\ 
  referenstyp & 14 & ~~\Large$\leadsto$~~ &  I & ej värdetyp, har supertypen \code|AnyRef| \\ 
  förseglad typ & 15 & ~~\Large$\leadsto$~~ &  K & subtypning utanför denna kodfil är förhindrad \\ 
  värdetyp & 16 & ~~\Large$\leadsto$~~ &  B & minneslagring kan optimeras, har supertypen, \code|AnyVal| \\ 
\end{ConceptConnections}

\QUESTEND





\WHAT{Gemensam bastyp.}

\QUESTBEGIN

\Task  \what~  Man vill ofta lägga in objekt av olika typ i samma samling.
\begin{REPL}
scala> class Gurka(val vikt: Int)
scala> class Tomat(val vikt: Int)
scala> val gurkor = Vector(new Gurka(100), new Gurka(200))
scala> val grönsaker = Vector(new Gurka(300), new Tomat(42))
\end{REPL}
\Subtask Om en samling innehåller objekt av flera olika typer försöker kompilatorn härleda den mest specifika typen som objekten har gemensamt. Vad blir det för typ på värdet \code{grönsaker} ovan?

\Subtask Försök ta reda på summan av vikterna enligt nedan. Vad ger andra raden för felmeddelande? Varför?

\begin{REPL}
scala> gurkor.map(_.vikt).sum
scala> grönsaker.map(_.vikt).sum
\end{REPL}

\Subtask Vi kan göra så att vi kan komma åt vikten på alla grönsaker genom att ge gurkor och tomater en gemensam bastyp som de olika konkreta grönsakstyperna utvidgar med nyckelordet \code{extends}. Man säger att subtyperna \code{Gurka} och \code{Tomat} \textbf{ärver} egenskaperna hos supertypen \code{Grönsak}.

Attributet \code{vikt} i traiten \code{Grönsak} nedan initialiseras inte förrän konstruktorerna anropas när vi gör \code{new} på någon av klasserna \code{Gurka} eller \code{Tomat}.

\begin{REPL}
scala> trait Grönsak { val vikt: Int }
scala> class Gurka(val vikt: Int) extends Grönsak
scala> class Tomat(val vikt: Int) extends Grönsak
scala> val gurkor = Vector(new Gurka(100), new Gurka(200))
scala> val grönsaker = Vector(new Gurka(300), new Tomat(42))
\end{REPL}

\Subtask Vad blir det nu för typ på variabeln \code{grönsaker} ovan?

\Subtask Undersök om det nu går att räkna ut summan av vikterna i \code{grönsaker} med \\ \code{grönsaker.map(_.vikt).sum}


\Subtask En trait liknar en klass, men man kan inte instansiera den och den kan inte ha några parametrar. En typ som inte kan instansieras kallas \textbf{abstrakt} \Eng{abstract}. Vad blir det för felmeddelande om du försöker göra \code{new} på en trait enligt nedan?
\begin{REPL}
scala> trait Grönsak { val vikt: Int }
scala> new Grönsak
\end{REPL}


\Subtask Traiten \code{Grönsak} har en abstrakt medlem \code{vikt}. Den sägs vara abstrakt eftersom den saknar definition -- medlemmen har bara ett namn och en typ men inget värde. Du kan instansiera den abstrakta traiten \code{Grönsak} om du fyller i det som ''fattas'', nämligen ett värde på \code{vikt}. Man kan fylla på det som fattas i genom att ''hänga på'' ett block efter typens namn vid instansiering. Man får då vad som kallas en \textbf{anonym} klass, i detta fall en ganska konstig grönsak som inte är någon speciell sorts grönsak med som ändå har en vikt.

Vad får \code{anonymGrönsak} nedan för typ och strängrepresenation?
\begin{REPL}
scala> val anonymGrönsak = new Grönsak { val vikt = 42 }
\end{REPL}



\SOLUTION


\TaskSolved \what


\SubtaskSolved  \code{Vector[Object]}.

\SubtaskSolved  Det beror på att vektorns element är av typen \code{Object}. \code{vikt} är inte definierat för denna typ.

\SubtaskSolved  -.

\SubtaskSolved  \code{Vector[Grönsak]}.

\SubtaskSolved  Ja.

\SubtaskSolved  -.

\SubtaskSolved  \code{Grönsak}. \$anon\$1@88dfbe.


\QUESTEND






\WHAT{Polymorfism i samband med arv.}

\QUESTBEGIN

\Task  \what~  Polymorfism betyder ''många skepnader''. I samband med arv  innebär det att flera subtyper, till exempel \code{Ko} och \code{Gris}, kan hanteras gemensamt som om de vore instanser av samma supertyp, så som \code{Djur}. Subklasser kan implementera en metod med samma namn på olika sätt. Vilken metod som exekveras bestäms vid körtid beroende på vilken subtyp som instansieras. På så sätt kan djur komma i många skepnader.

\Subtask Implementera funktionen \code{skapaDjur} nedan så att den returnerar antingen en ny \code{Ko} eller en ny \code{Gris} med lika sannolikhet.

\begin{REPL}
scala> trait Djur { def väsnas: Unit }
scala> class Ko   extends Djur { def väsnas = println("Muuuuuuu") }
scala> class Gris extends Djur { def väsnas = println("Nöffnöff") }
scala> def skapaDjur: Djur = ???
scala> val bondgård = Vector.fill(42)(skapaDjur)
scala> bondgård.foreach(_.väsnas)
\end{REPL}

\Subtask Lägg till ett djur av typen Häst som väsnas på lämpligt sätt och modifiera \code{skapaDjur} så att det skapas kor, grisar och hästar med lika sannolikhet.



\SOLUTION


\TaskSolved \what


\SubtaskSolved
\begin{Code}
def skapaDjur: Djur =
   {if(math.random > 0.5) new Ko else new Gris}
\end{Code}

\SubtaskSolved
\begin{Code}
class Häst extends Djur{ def väsnas = println("Gnääääägg") }
def skapaDjur: Djur = {val r = math.random;
   if(r < 0.33) new Ko else if(r < 0.67) new Gris else new Häst}
\end{Code}


\QUESTEND







\WHAT{Olika typer av heltalspar till veckans laboration \hyperref[section:lab:\LabWeekNINE]{\texttt{\LabWeekNINE}}.}


\QUESTBEGIN


\Task\label{exe:inheritance:labprep-pair}  \what~Under veckans laboration ska du använda olika typer av par som representerar riktning och position på en tvådimensionell spelplan, samt spelplanens storlek. I stället för att använda en vanlig 2-tupel till dessa tre olika typer av par ska du skapa egna, specifika  typer som alla ärver bastypen \code{Pair[T]}. Dessa typer ska alla ligga i filen \code{pairs.scala} i \code{package snake}.
\begin{Code}
// detta är en skiss på filen pairs.scala
package snake

trait Pair[T] {
  def x: T
  def y: T
  // uppgift a) lägg till den konkreta metoden tuple
}

// efterföljande deluppgifterna implementeras här:
//   kompanjonsobjektet Pair
//   case klass Dim med kompanjonsobjekt
//   case klass Pos med kompanjonsobjekt
//   abstrakt förseglad klass Dir
//   tillhörande caseobjekt North, South, etc. som ärver Dir
\end{Code}
Skillnaden mellan \code{Pair[T]} och en vanlig 2-tupel är att medlemmarna \code{x} och \code{y} garanterat är av \emph{samma} typ, medan en 2-tupel kan innehålla element av olika typ.

I fig. \ref{snake:fig:pairs-uml} visas en bild av klasshierarkin som du steg-för-steg ska utveckla i efterföljande  uppgifter. Fördelen med att ha olika typer av par är att det är mer typsäkert \Eng{type safe}: vi får hjälp av kompilatorn att upptäcka om vi av misstag förväxlar t.ex. en position med en riktning.

\begin{figure}[H]
\begin{center}
\newcommand{\TextBox}[1]{\raisebox{0pt}[1em][0.5em]{#1}}
\tikzstyle{umlclass}=[rectangle, draw=black,  thick, anchor=north, text width=2cm, rectangle split, rectangle split parts = 3]
\begin{tikzpicture}[inner sep=0.5em,scale=1.2, every node/.style={transform shape}]

  \node [umlclass, rectangle split parts = 1, xshift=0cm, yshift=4.5cm] (BaseType1)  {
              \textit{\textbf{\centerline{\TextBox{\code{Pair[T]}}}}}
%              \nodepart[align=left]{second}\code{def x: T} \newline \code{def y: T}
          };


  \node [umlclass, rectangle split parts = 1, xshift=-3cm, yshift=2.5cm] (SubType1)  {
              \textit{\textbf{\centerline{\TextBox{\code{Dim}}}}}
%              \nodepart[align=left]{second}\code{val x: Int} \newline \code{val y: Int}
          };

\node [umlclass, rectangle split parts = 1, xshift=0cm, yshift=2.5cm] (SubType2)  {
            \textit{\textbf{\centerline{\TextBox{\code{Pos}}}}}
%            \nodepart[]{second}\TextBox{\code{val dim: Int}}
        };

\node [umlclass, rectangle split parts = 1, xshift=3cm, yshift=2.5cm] (SubType3)  {
            \textit{\textbf{\centerline{\TextBox{\code{Dir}}}}}
%            \nodepart[]{second}\TextBox{\code{val dim: Int}}
        };


\draw[umlarrow] (SubType1.north) -- ++(0,0.5) -| (BaseType1.south);
\draw[umlarrow] (SubType2.north) -- ++(0,0.5) -| (BaseType1.south);
\draw[umlarrow] (SubType3.north) -- ++(0,0.5) -| (BaseType1.south);

\end{tikzpicture}
\end{center}
\caption{Arvshierarki med \code{Pair[T]} som bastyp.}
\label{snake:fig:pairs-uml}
\end{figure}

\Subtask Öppna en editor och koda \code{trait Pair[T]} i en fil \code{pairs.scala}. Lägg dessutom till en konkret metod \code{tuple} i \code{Pair[T]} som returnerar en 2-tupel med de båda elementen i paret, så att det vid behov går att omvandla \code{Pair}-instanser till 2-tupler. Klistra in i REPL testa att detta fungerar:
\begin{REPLnonum}
scala> val p = new Pair[Int] { override val x = 10; override val y = 20 }
p: Pair[Int]{val x: Int; val y: Int} = $anon$1@784223e9

scala> p.tuple
res0: (Int, Int) = (10,20)
\end{REPLnonum}

\Subtask Skapa en case-klass \code{Dim} som ärver \code{Pair[Int]}. Instanser av denna klass kommer du att använda under veckans laboration för att representera en spelplans storlek genom att låta \code{x} ange antalet horisontella positioner och \code{y} antalet vertikala positioner.

\Subtask Lägg till ett kompanjonsobjekt \code{Dim} med en \code{apply}-metod som kan skapa \code{Dim}-instanser givet en 2-tupel.
Testa i REPL enligt nedan.
\begin{REPLnonum}
scala> Dim(50, 60)
res1: Dim = Dim(50,60)

scala> Dim((60, 50))
res2: Dim = Dim(60,50)

scala> res2.tuple
res3: (Int, Int) = (60,50)
\end{REPLnonum}

\Subtask Lägg till en case-klass \code{Pos} som ärver \code{Pair[Int]} som representerar en position med en \code{x}-koordinat och en \code{y}-koordinat inom en spelplansstorlek \code{dim}. Kordinatpositionerna är heltal och räknas från \code{0} till (men inte med) \code{dim.x} resp. \code{dim.y}.

\Subtask Implementera metoderna \code{+} och \code{-} i case-klassen \code{Pos}. Båda metoderna ska ta en parameter \code{p} av typen \code{Pair[Int]} och returnera en ny \code{Pos}, där \code{p.x} resp. \code{p.y} är adderat resp. subtraherat från aktuell position. Aritmetiken ska ske modulo storleken \code{dim}, d.v.s en position kan aldrig hamna utanför spelplanen; om något beräkningsresultat hamnar utanför spelplanen så börjar man om på andra sidan. \\ \emph{Tips:} Använd  \code|java.lang.Math.floorMod|.

\Subtask Lägg till ett kompanjonsobjekt \code{Pos} med en \code{apply}-metod som skapar en ny \code{Pos}-instans som garanterat har koordinaterna inom \code{dim} med hjälp av \code{floorMod}. Gör dessutom konstruktorn i case-klassen \code{Pos} privat så att det inte går att skapa positioner utanför \code{dim} med \code{new}.
Testa att det fungerar enligt nedan:
\begin{REPLnonum}
scala> Pos(-1,20,Dim(10,20))
res4: Pos = Pos(9,0,Dim(10,20))

scala> new Pos(-1,20,Dim(10,20))
error: constructor Pos in class Pos cannot be accessed
\end{REPLnonum}

\Subtask Lägg även till fabriksmetoden \code{random} som kan skapa nya slumpmässiga positioner inom en viss spelplansstorlek. \emph{Tips:} Använd \code{scala.util.Random.nextInt}.
\begin{REPLnonum}
scala> for (_ <- 1 to 3) yield Pos.random(Dim(10,10))
res5: Vector(Pos(6,1,Dim(10,10)), Pos(7,9,Dim(10,10)), Pos(3,2,Dim(10,10)))
\end{REPLnonum}

\Subtask Lägg till en abstrakt, förseglad klass \code{Dir} som ärver \code{Pair[Int]} och har två \code{val}-parametrar \code{x} och \code{y}. Lägg också till fyra case-object som ärver \code{Dir} och som representerar riktningar i de fyra väderstrecken väderstrecken genom att parametrarna \code{x} resp. \code{y} ges något av värden $1$, $-1$ eller $0$. Norrut ska anges som $-1$ etc. Verifiera i REPL att förseglingen fungerar. Hur lyder felmeddelandet?


\SOLUTION


\TaskSolved \what

\SubtaskSolved
\begin{CodeSmall}
trait Pair[T] {
  def x: T
  def y: T
  def tuple: (T, T) = (x, y)
}
\end{CodeSmall}

\SubtaskSolved
\begin{CodeSmall}
case class Dim(x: Int, y: Int) extends Pair[Int]
\end{CodeSmall}

\SubtaskSolved
\begin{CodeSmall}
object Dim {
  def apply(dim: (Int, Int)): Dim = Dim(dim._1, dim._2)
}
\end{CodeSmall}

\SubtaskSolved
\begin{CodeSmall}
case class Pos(x: Int, y: Int, dim: Dim) extends Pair[Int]
\end{CodeSmall}

\SubtaskSolved
\begin{CodeSmall}
case class Pos(x: Int, y: Int, dim: Dim) extends Pair[Int] {
  import java.lang.Math.{floorMod => mod}

  def +(p: Pair[Int]): Pos =
    copy(x = mod(x + p.x, dim.x), y = mod(y + p.y, dim.y))

  def -(p: Pair[Int]): Pos =
    copy(x = mod(x - p.x, dim.x), y = mod(y - p.y, dim.y))
}
\end{CodeSmall}

\SubtaskSolved \SubtaskSolved
\begin{CodeSmall}
object Pos {
  import java.lang.Math.{floorMod => mod}
  import scala.util.Random

  def apply(x: Int, y: Int, dim: Dim): Pos =
    new Pos(mod(x, dim.x), mod(y, dim.y), dim)

  def random(dim: Dim): Pos = {
    Pos(Random.nextInt(dim.x), Random.nextInt(dim.y), dim)
  }
}
\end{CodeSmall}

\SubtaskSolved
\begin{CodeSmall}
abstract sealed class Dir(val x: Int, val y: Int) extends Pair[Int]

case object North extends Dir( 0, -1)
case object South extends Dir( 0,  1)
case object East  extends Dir( 1,  0)
case object West  extends Dir(-1,  0)
\end{CodeSmall}

\begin{REPLnonum}
scala> case object Gurka extends Dir(0,0)
error: illegal inheritance from sealed class Dir
\end{REPLnonum}

\QUESTEND






\WHAT{Supertyp med parameter.}

\QUESTBEGIN

\Task  \what~  En trait kan inte ha någon parameter. Vill man ha en parameter till supertypen måste man använda en klass istället, enligt nedan exempel.

Utbildningsdepartementet vill i sitt system hålla koll på vissa personer och skapar därför en klasshierarki enligt nedan. Skriv in koden i en editor och klipp sedan in den i REPL.
\begin{Code}
class Person(val namn: String)

class Akademiker(
  namn: String,
  val universitet: String) extends Person(namn)

class Student(
  namn: String,
  universitet: String,
  program: String) extends Akademiker(namn, universitet)

class Forskare(
  namn: String,
  universitet: String,
  titel: String) extends Akademiker(namn, universitet)
\end{Code}


\Subtask Deklarera fyra olika \code{val}-variabler med lämpliga namn som refererar till olika instanser av alla olika klasser ovan och ge attributen valfria initialvärden genom olika parametrar till konstruktorerna.

\Subtask Skriv två satser: en som först stoppar in instanserna i en \code{Vector} och en som sedan loopar igenom vektorn och skriv ut alla instansers \code{toString} och \code{namn}.

\Subtask Utbildningsdepartementet vill att det inte ska gå att instansiera objekt av typerna \code{Person} och \code{Akademiker}. Det kan åstadkommas genom att placera nyckelordet \code{abstract} före \code{class}. Uppdatera koden i enlighet med detta. Vilket blir felmeddelande om man försöker instansiera en \code{abstract class}?

\Subtask Utbildningsdeparetementet vill slippa implementera \code{toString} och slippa skriva \code{new} vid instansiering. Gör därför om typerna \code{Student} och \code{Forskare} till case-klasser. \emph{Tips:} För att undkomma ett kompileringsfel (vilket?) behöver du använda \code{override val} på lämpligt ställe.
Skapa instanser av de nya case-klasserna \code{Student} och \code{Forskare} och skriv ut deras \code{toString}. Hur ser utskriften ut?

\Subtask Eftersom \code{Person} och \code{Akademiker} nu är abstrakta, vill utbildningsdepartementet att du gör om dessa typer till traits med abstrakta attribut istället för klasser. Du kan då undvika \code{override val} i klassparametrarna till de konkreta case-klasserna.
Man inför också en case-klass \code{IckeAkademiker} som man tänker använda i olika statistiska medborgarundersökningar.
Dessutom förser man alla personer med ett personnummer representerat som en \code{Long}.
Hur ser utbildningsdepartementets kod ut nu, efter alla ändringar? Skriv ett testprogram som skapar några instanser och skriver ut deras attribut.

\SOLUTION


\TaskSolved \what


\SubtaskSolved
\begin{Code}
val person = new Person("Person1")
val akademiker = new Akademiker("Person2", "LTH")
val student = new Student("Person3", "LTH", "D")
val forskare = new Forskare("Person4", "LTH", "Doktorand")
\end{Code}

\SubtaskSolved
\begin{Code}
val vec = Vector(person, akademiker, student, forskare)
for(i <- vec){ print(i.toString + i.namn) }
\end{Code}

\SubtaskSolved  error: class Person is abstract; cannot be instantiated.

\SubtaskSolved  error: overriding value namn in class Person of type String; value namn needs `override' modifier.\\
toString för Student: Student(Person3,LTH,D). \\
toString för Forskare: Student(Person4,LTH,Doktorand).

\SubtaskSolved
\begin{Code}
trait Person {val namn: String; val nbr: Long}
trait Akademiker extends Person {val universitet: String}
case class Student(
  namn: String,
  nbr: Long,
  universitet: String,
  program: String) extends Akademiker
case class Forskare(
  namn: String,
  nbr: Long,
  universitet: String,
  titel: String) extends Akademiker
case class IckeAkademiker(
    namn: String,
    nbr: Long) extends Person
\end{Code}



\QUESTEND




%\clearpage




\ExtraTasks %%%%%%%%%%%%%%%%%





\WHAT{Uppräknade värden.}

\QUESTBEGIN

\Task  \what~  Ett sätt att hålla reda på uppräknade värden, t.ex. färgen på olika kort i en kortlek, är att använda olika heltal som får representera de olika värdena, till exempel så här:\footnote{Om namnkonventioner för konstanter i Scala: läs under rubriken ''Constants, Values, Variable and Methods'' här \href{http://docs.scala-lang.org/style/naming-conventions.html}{docs.scala-lang.org/style/naming-conventions.html}}
\begin{Code}
object Färg {
  val Spader = 1
  val Hjärter = 2
  val Ruter = 3
  val Klöver = 4
}
\end{Code}
Dessa kan sedan användas som parametrar till denna case-klass vid skapande av kortobjekt:
\begin{lstlisting}[language=,keywords={case,class}]
case class Kort(färg: Int, valör: Int)
\end{lstlisting}
Man kan hålla reda på färgen med en variabel av typen \code{Int} och tilldela den en viss färg med ovan konstanter. Och när du skapar ett kort kan du använda färgnamnet och du slipper därmed att behöva komma ihåg vilket heltal som representerar färgen.
\begin{REPL}
scala> val f = Färg.Spader
scala> import Färg._
scala> Kort(Ruter, 7)
\end{REPL}
En annan fördelen med detta är att man lätt kan iterera över alla färger:
\begin{REPL}
scala> val kortlek = for (f <- 1 to 4; v <- 1 to 13) yield Kort(f, v)
\end{REPL}
Men den stora nackdelen med detta är att kompilatorn vid kompileringstid inte kollar om variablerna av misstag råkar ges något värde utanför det giltiga intervallet, eftersom alla heltal är möjliga. Detta får vi själv hålla koll på vid körtid, till exempel med funktionen \code{require} eller \code{if}-satser, etc.

Istället kan man använda uppräknade värden med hjälp av case-objekt enligt nedan deluppgifter och därmed få hjälp av kompilatorn att hitta eventuella fel vid kompileringstid.  Ett case-objekt är som ett vanligt singelton-objekt men det får bl.a. automatiskt en \code{toString} som är samma som namnet. Case-objekt kan dessutom användas som värden i mönstermatchningar (mer om detta i kapitel \ref{chapter:W10}).

\Subtask Deklarera följande uppräknade värden som singelton-objekt med gemensam bastyp. Med nyckelordet \code{sealed} så ''förseglas'' klassen och inga andra direkta subtyper tillåts förutom de som finns i samma kod-fil eller block. I detta exempel  med kortfärger vet vi ju att det inte finns fler än dessa fyra färger.
\begin{Code}
sealed trait Färg
case object Spader extends Färg
case object Hjärter extends Färg
case object Ruter extends Färg
case object Klöver extends Färg
\end{Code}
Dessa kan sedan användas som parametrar till denna case-klass vid skapande av kortobjekt:
\begin{lstlisting}[language=,keywords={case,class}]
case class Kort(färg: Färg, valör: Int)
\end{lstlisting}
Skapa därefter några exempelinstanser av klassen \code{Kort}. Vad är fördelen med ovan angreppssätt jämfört med att använda heltalskonstanter?

\Subtask Om man vill kunna iterera över alla värden är det bra om de finns i en samling med alla värden. Vi kan lägga en sådan i ett kompanjonsobjekt till bastypen enligt nedan. Skriv ut alla färgvärden med en \code{for}-sats.

\begin{Code}
sealed trait Färg
object Färg {
  val values = Vector(Spader, Hjärter, Ruter, Klöver)
}
case object Spader extends Färg
case object Hjärter extends Färg
case object Ruter extends Färg
case object Klöver extends Färg
\end{Code}
Skapa en kortlek med 52 olika kort och blanda den sedan med \code{Random.shuffle} enligt nedan. Använd en dubbel \code{for}-sats och \code{yield}.
\begin{REPL}
scala> val kortlek: Vector[Kort] = ???
scala> val blandad = scala.util.Random.shuffle(kortlek)
\end{REPL}

\Subtask Skriv en funktion \code{ def blandadKortlek: Vector[Kort] = ???} som ger en ny blandad kortlek varje gång den anropas med metoden i föregående uppgift.

\Subtask Om man även vill ha en heltalsrepresentation med en medlem \code{toInt} för alla värden, kan man ge bastypen en parameter och i stället för en trait (som inte kan ha några parametrar) använda en abstrakt klass.

\begin{Code}
sealed abstract class Färg(final val toInt: Int)
object Färg {
  val values = Vector(Spader, Hjärter, Ruter, Klöver)
}
case object Spader  extends Färg(0)
case object Hjärter extends Färg(1)
case object Ruter   extends Färg(2)
case object Klöver  extends Färg(3)
\end{Code}
Skapa en funktion \code{färgPoäng} som räknar ut summan av heltalsrepresentationen av alla färger hos en vektor med kort, och använd den sedan för att räkna ut \code{färgPoäng} för de första fem korten.
\begin{REPL}
scala> def blandadKortlek: Vector[Kort] = ???
scala> def färgPoäng(xs: Vector[Kort]): Int = ???
scala> färgPoäng(blandadKortlek.take(5))
\end{REPL}


\SOLUTION

\TaskSolved \what

\SubtaskSolved  Sättet är säkrare då man inte kan tilldela korten en färg som inte finns. Med heltalskonstanterna kan man till exempel ge ett kort färgen 5, vilken inte korresponderar till någon riktig färg.

\SubtaskSolved  \code{for (f <- Färg.values; v <- 1 to 13) yield Kort(f,v)}

\SubtaskSolved
\begin{Code}
def blandadKortlek: Vector[Kort] = {
  val kortlek =
    for (f <- Färg.values; v <- 1 to 13) yield Kort(f,v)
  scala.util.Random.shuffle(kortlek)
}
\end{Code}

\SubtaskSolved  \code{def färgPoäng(xs: Vector[Kort]): Int = xs.map(_.färg.toInt).sum}

\QUESTEND







\WHAT{Bastypen \code{Shape} och subtyperna \code{Rectangle} och \code{Circle}.}

\QUESTBEGIN

\Task  \what~  Du ska i denna uppgift skapa ett litet bibliotek för geometriska former med oföränderliga objekt implementerade med hjälp av case-klasser. De geometriska formerna har en gemensam bastyp kallad \code{Shape}. Utgå från koden nedan.

\begin{Code}
case class Point(x: Double, y: Double) {
  def move(dx: Double, dy: Double): Point = Point(x + dx, y + dy)
}

trait Shape {
  def pos: Point
  def move(dx: Double, dy: Double): Shape
}

case class Rectangle(
  pos: Point,
  dx: Double,
  dy: Double
) extends Shape {
  override def move(dx: Double, dy: Double): Rectangle =
    Rectangle(pos.move(dx, dy), this.dx, this.dy)
}

case class Circle(pos: Point, radius: Double) extends Shape {
  override def move(dx: Double, dy: Double): Circle =
    Circle(pos.move(dx, dy), radius)
}
\end{Code}

\Subtask Instansiera några cirklar och rektanglar och gör några relativa förflyttningar av dina instanser genom att anropa \code{move}.

\Subtask Lägg till metoden \code{moveTo} i \code{Point}, \code{Shape}, \code{Rectangle} och \code{Circle} som gör en absolut förflyttning till koordinaterna \code{x} och \code{y}. Testa med REPL på några instanser av \code{Rectangle} och \code{Circle}.

\Subtask Lägg till metoden \code{distanceTo(that: Point): Double } i case-klassen \code{Point} som räknar ut avståndet till en annan punkt med hjälp av \code{math.hypot}. Klistra in i REPL och testa på några instanser av \code{Point}.

\Subtask Lägg till en konkret metod \code{distanceTo(that: Shape): Double } i traiten \code{Shape} som räknar ut avståndet till positionen för en annan Shape. Testa i REPL på några instanser av \code{Rectangle} och \code{Circle}.

\SOLUTION


\TaskSolved \what


\SubtaskSolved
\begin{Code}
val c1 = Circle(Point(1, 1), 42)
val r1 = Rectangle(Point(3, 3), 20, 30)
c1.move(2, 3)
r1.move(3, 2)
\end{Code}

\SubtaskSolved  För \code{Point}: \code{def moveTo(dx: Double, dy: Double): Point = Point(dx, dy)}. \\
För \code{Shape}: \code{def moveTo(dx: Double, dy: Double): Shape}. \\
För \code{Rectangle}: \code{override def moveTo(dx: Double, dy: Double): Rectangle = } \\
\code{Rectangle(pos.moveTo(dx, dy), this.dx, this.dy)}. \\
För \code{Circle}: \code{override def moveTo(dx: Double, dy: Double): Circle =} \\
\code{Circle(pos.moveTo(dx, dy), radius)}.

\SubtaskSolved  \code{def distanceTo(that: Point): Double = math.hypot(that.x - x, that.y - y)}.

\SubtaskSolved  \code{def distanceTo(that: Shape): Double = pos.distanceTo(that.pos)}.


\QUESTEND






\WHAT{Regler för \code{override}, \code{private} och \code{final}.}

\QUESTBEGIN

\Task  \what~

\Subtask \label{subtask:overriderules} Undersök överskuggningning av abstrakta, konkreta, privata och finala medlemmar genom att skriva in raderna nedan en i taget i REPL. Vilka rader ger felmeddelande? Varför? Vid felmeddelande: notera hur felmeddelandet lyder och ändra deklarationen av den felande medlemmen så att koden blir kompilerbar (eller om det är enda rimliga lösningen: ta bort den felaktiga medlemmen), innan du provar efterkommande rad.

\begin{REPL}
trait Super1 { def a: Int; def b = 42; private def c = "hemlis" }
class Sub2 extends Super1 { def a = 43; def b = 43; def c = 43 }
class Sub3 extends Super1 { def a = 43; override def b = 43 }
class Sub4 extends Super1 { def a = 43; override def c = "43" }
trait Super5 { final def a: Int; final def b = 42 }
class Sub6 extends Super5 { override def a = 43; def b = 43 }
class Sub7 extends Super5 { def a = 43; override def b = 43 }
class Sub8 extends Super5 { def a = 43; override def c = "43" }
trait Super9 { val a: Int; val b = 42; lazy val c: String = "lazy" }
class Sub10 extends Super9 { override def a = 43; override val b = 43 }
class Sub11 extends Super9 { val a = 43; override lazy val b = 43 }
class Sub12 extends Super9 { val a = 43; override var b = 43 }
class Sub13 extends Super9 { val a = 43; override lazy val c = "still lazy" }
class SubSub extends Sub13 { override val a = 44}
trait Super14 { var a: Int; var b = 42; var c: String }
class Sub15 extends Super14 { def a = 43; override var b = 43; val c = "?" }
\end{REPL}

\Subtask Skapa instanser av klasserna \code{Sub3}, \code{Sub13} och \code{SubSub} från ovan deluppgift och undersök alla medlemmarnas värden för respektive instans. Förklara varför de har dessa värden.

%\Subtask Läs igenom reglerna i kapitel  \ref{slideW07:overriderules} om vad som gäller vid arv och överskuggning av medlemmar. Gör några egna undersökningar i REPL som försöker bryta mot någon regel som inte testades i deluppgift \ref{subtask:overriderules}.

\SOLUTION


\TaskSolved \what


\SubtaskSolved  2. Måste ha \code{override} framför \code{b} för att kunna ändra på metoden. \\
4. \code{c} är \code{private}, vilket betyder att den är gömd för subklasserna. Därför kan den inte överskuggas. Genom att ta bort \code{override} fungerar klassen. \\
5. En \code{final}-medlem måste ha ett bestämt värde. Kan lösas genom att tilldela \code{final a} ett värde eller ta bort \code{final}. \\
6. En \code{final}-medlem kan inte överskuggas, varken med eller utan \code{override}. Här får konflikterna tas bort.  \\
7. Se 6. \\
8. Eftersom \code{c} inte finns i \code{Super5} kan den inte överskuggas. Genom att ta bort \code{override} fungerar klassen. \\
10. Överskuggningen av \code{val} måste vara oföränderlig (immutable); detta är inte nödvändigtvis \code{def}. Löses genom att byta ut \code{def a} mot \code{val a} hos \code{Sub10}.  \\
11. Samma problem som i 10.; \code{lazy val} kan vara föränderlig. Löses genom att ta bort \code{lazy}. \\
12. Samma problem igen! \code{var} är föränderlig, vilket bryter mot typsäkerheten när man försöker överskugga en \code{val}. Löses genom att ändra \code{var} till \code{val}. \\
15.\code{def a = 43} och \code{val c = "?"} täcker inte allt som \code{var} kräver. Det behövs en setter för att kunna uppfylla kraven för överskuggning för en \code{var}. Dessutom finns det ingen anledning för en \code{val} att överskuggas; man kan ju ändra på den lite hur man vill!

\SubtaskSolved  Sub3: a = 43, b = 43 eftersom medlemmen är överskuggad. c hittas inte eftersom den är \code{private}.

Sub13: a = 43, b = 42, c = "still lazy" eftersom medlemmen överskuggas.

SubSub: a = 44 eftersom medlemmen överskuggas, b = 42, c = "still lazy".

\SubtaskSolved  -.


\QUESTEND





%\clearpage





\AdvancedTasks %%%%%%%%%%%%%%%%%

\WHAT{Använda \code{trait} eller \code{class}?}

\QUESTBEGIN

\Task \what~ I vilka sammanhang är det nödvändigt att använda en \code{trait} respektive en \code{class}? Läs här för fördjupning:\\  \href{http://www.artima.com/pins1ed/traits.html\#12.7}{http://www.artima.com/pins1ed/traits.html\#12.7}.


\SOLUTION


\TaskSolved \what~Man måste använda en klass om man behöver klassparametrar. Man måste använda en trait om man vill göra in-mixning med \code{with}. \\

 \QUESTEND



\WHAT{Inmixning.}

\QUESTBEGIN

\Task \label{task:fyle} \what~   Man kan utvidga en klass med multipla traits med nyckelordet \code{with}. På så sätt kan man fördela medlemmar i olika traits och återanvända gemensamma koddelar genom så kallad \textbf{inmixning}, så som nedan exempel visar.

En alternativ fågeltaxonomi, speciellt populär i Skåne, delar in alla fåglar i två specifika kategorier: Kråga respektive Ånka. Krågor kan flyga men inte simma, medan Ånkor kan simma och oftast även flyga. Fågel i generell, kollektiv bemärkelse kallas på gammal skånska för Fyle.%
\footnote{\href{http://www.klangfix.se/ordlista.htm}{www.klangfix.se/ordlista.htm}}

\begin{CodeSmall}
trait Fyle {
  val läte: String
  def väsnas: Unit = print(läte * 2)
  val ärSimkunnig: Boolean
  val ärFlygkunnig: Boolean
}

trait KanSimma       { val ärSimkunnig = true }
trait KanInteSimma   { val ärSimkunnig = false }
trait KanFlyga       { val ärFlygkunnig = true }
trait KanKanskeFlyga { val ärFlygkunnig = math.random < 0.8 }

class Kråga extends Fyle with KanFlyga with KanInteSimma {
  val läte = "krax"
}

class Ånka extends Fyle with KanSimma with KanKanskeFlyga {
  val läte = "kvack"
  override def väsnas = print(läte * 4)
}
\end{CodeSmall}

\Subtask En flitig ornitolog hittar 42 fåglar i en perfekt skog där alla fågelsorter är lika sannolika, representerat av vektorn \code{fyle} nedan. Skriv i REPL ett uttryck som undersöker hur många av dessa som är flygkunniga Ånkor, genom att använda metoderna \code{filter}, \code{isInstanceOf}, \code{ärFlygkunnig} och \code{size}.

\begin{REPL}
scala> val fyle =
         Vector.fill(42)(if (math.random > 0.5) new Kråga else new Ånka)
scala> fyle.foreach(_.väsnas)
scala> val antalFlygånkor: Int = ???
\end{REPL}

\Subtask \label{subtask:fyle:sound} Om alla de fåglar som ornitologen hittade skulle väsnas exakt en gång var, hur många krax och hur många kvack skulle då höras? Använd metoderna \code{filter} och \code{size}, samt predikatet \code{ärSimkunnig} för att beräkna antalet krax respektive kvack.
\begin{REPL}
scala> val antalKrax: Int = ???
scala> val antalKvack: Int = ???
\end{REPL}

\SOLUTION


\TaskSolved \what


\SubtaskSolved
\begin{Code}
fyle.filter(f => f.isInstanceOf[Ånka] && f.ärFlygkunnig).size
\end{Code}

\SubtaskSolved
\begin{Code}
val antalKrax: Int = fyle.filter(f => !f.ärSimkunnig).size * 2
val antalKvack: Int = fyle.filter(f => f.ärSimkunnig).size * 4
\end{Code}


\QUESTEND











\WHAT{Finala klasser.}

\QUESTBEGIN

\Task  \what~  Om man vill förhindra att man kan göra \code{extends} på en klass kan man göra den final genom att placera nyckelordet \code{final} före nyckelordet \code{class}.

\Subtask Eftersom klassificeringen av fåglar i uppgiften ovan i antingen Ånkor eller Krågor är fullständig och det inte finns några subtyper till varken Ånkor eller Krågor är det lämpligt att göra dessa finala. Ändra detta i din kod.

\Subtask Testa att ändå försöka göra en subklass \code{Simkråga extends Kråga}. Vad ger kompilatorn för felmeddelande om man försöker utvidga en final klass?


\SOLUTION


\TaskSolved \what


\SubtaskSolved  Sätt \code{final} framför \code{class} i klasserna.

\SubtaskSolved  error: illegal inheritance from final class Kråga.


\QUESTEND






\WHAT{Accessregler vid arv och nyckelordet \code{protected}.}

\QUESTBEGIN

\Task  \what~  Om en medlem i en supertyp är privat så kan man inte komma åt den i en subklass. Ibland vill man att subklassen ska kunna komma åt en medlem även om den ska vara otillgänglig i annan kod.

\begin{REPL}
trait Super {
  private val minHemlis = 42
  protected val vårHemlis = 42
}
class Sub extends Super {
  def avslöja = minHemlis
  def kryptisk = vårHemlis * math.Pi
}
\end{REPL}

\Subtask Vad blir felmeddelandet när klassen \code{Sub} försöker komma åt \code{minHemlis}?

\Subtask Deklarera \code{Sub} på nytt, men nu utan den förbjudna metoden \code{avslöja}. Vad blir felmeddelandet om du försöker komma åt \code{vårHemlis} via en instans av klassen \code{Sub}? Prova till exempel med \code{(new Sub).vårHemlis}

\Subtask Fungerar det att anropa metoden \code{kryptisk} på instanser av klassen \code{Sub}?

\SOLUTION


\TaskSolved \what


\SubtaskSolved  error: not found: value minHemlis.

\SubtaskSolved  error: value vårHemlis in class Super\$class cannot be accessed in Sub.

\SubtaskSolved  Ja.


\QUESTEND






\WHAT{Använding av \code{protected}.}

\QUESTBEGIN

\Task  \what~  Den flitige ornitologen från uppgift \ref{task:fyle} ska ringmärka alla 42 fåglar hen hittat i skogen. När hen ändå håller på bestämmer hen att även försöka ta reda på hur mycket oväsen som skapas av respektive fågelsort. För detta ändamål apterar den flitige ornitologen en Linuxdator på allt infångat fyle. Du ska hjälpa ornitologen att skriva programmet.

\Subtask Inför en \code{protected var räknaLäte} i traiten \code{Fyle} och skriv kod på lämpliga ställen för att räkna hur många läten som respektive fågelinstans yttrar.

\Subtask Inför en metod \code{antalLäten} som returnerar antalet krax respektive kvack som en viss fågel yttrat sedan dess skapelse.

\Subtask Varför inte använda \code{private} i stället for \code{protected}?

\Subtask Varför är det bra att göra räknar-variabeln oåtkomlig från ''utsidan''?



\SOLUTION


\TaskSolved \what


\SubtaskSolved  I Fyle:
\begin{Code}
protected var räknaLäte: Int = 0
def väsnas: Unit = { print(läte * 2); räknaLäte += 2 }
\end{Code}

I Ånka: \code| override def väsnas = { print(läte * 4); räknaLäte += 4 }|

\SubtaskSolved  \code{ def antalLäten: Int = räknaLäte }

\SubtaskSolved  Om en klass som representerar en fågel som skulle ge ifrån sig fler/färre läten än en vanlig \code{Fyle}, behöver \code{väsnas} ändras. Denna metod behöver tillgång till \code{räknaLäte}, vilken inte får vara \code{private}.

\SubtaskSolved  Räknar-variabeln ska inte kunna påverkas i någon annan del av programmet.


\QUESTEND





\WHAT{Inmixning av egenskaper.}

\QUESTBEGIN

\Task  \what~ Det visar sig att vår flitige ornitolog från uppgift \ref{task:fyle} på sidan \pageref{task:fyle} sov på en av föreläsningarna i zoologi när hen var nolla på Natfak, och därför helt missat fylekategorin \code{Pjodd}. Hjälp vår stackars ornitolog så att fylehierarkin nu även omfattar Pjoddar. En Pjodd kan flyga som en Kråga men den \code{ÄrLiten} medan en Kråga \code{ÄrStor}. En Pjodd kvittrar dubbelt så många gånger som en Ånka kvackar. En Pjodd \code{KanKanskeSimma} där simkunnighetssannolikheten är $0.2$. Låt ornitologen ånyo finna 42 slumpmässiga fåglar i skogen och filtrera fram lämpliga arter. Undersök sedan hur dessa väsnas, i likhet med deluppgift \ref{task:fyle}\ref{subtask:fyle:sound}.


\SOLUTION

\TaskSolved \what


\begin{Code}
trait Fyle {
  val läte: String
  def väsnas: Unit = { print(läte * 2); räknaLäte += 2 }
  protected var räknaLäte: Int = 0
  val ärSimkunnig: Boolean
  val ärFlygkunnig: Boolean
  val ärStor : Boolean
  def antalLäten: Int = räknaLäte
}
trait KanSimma { val ärSimkunnig = true }
trait KanInteSimma { val ärSimkunnig = false }
trait KanFlyga { val ärFlygkunnig = true }
trait KanKanskeFlyga { val ärFlygkunnig = math.random < 0.8 }
trait KanKanskeSimma { val ärSimkunnig = math.random < 0.2 }
trait ÄrStor { val ärStor = true }
trait ÄrLiten { val ärStor = false }

final class Kråga
  extends Fyle
  with KanFlyga
  with KanInteSimma
  with ÄrStor{
  val läte = "krax"
}

final class Ånka
  extends Fyle
  with KanSimma
  with KanKanskeFlyga
  with ÄrStor{
  val läte = "kvack"
  override def väsnas = { print(läte * 4); räknaLäte += 4 }
}

final class Pjodd
  extends Fyle
  with KanFlyga
  with KanKanskeSimma
  with ÄrLiten{
  val läte = "kvitter"
  override def väsnas = { print(läte * 8); räknaLäte += 8 }
}
\end{Code}

I REPL:
\begin{REPL}
val fyle = Vector.fill(42)(
  if(math.random < 0.33) new Kråga else
  if (math.random < 0.5) new Ånka else
  new Pjodd)
fyle.filter(f => f.isInstanceOf[Kråga]).size*2
fyle.filter(f => f.isInstanceOf[Ånka]).size*4
fyle.filter(f => f.isInstanceOf[Pjodd]).size*8
\end{REPL}

\QUESTEND





\WHAT{Typtest och typkonvertering.}

\QUESTBEGIN

\Task  \what~I Scala kan man testa körtidstyp och samtidigt konvertera till en mer specifik typ på ett typsäkert sätt med hjälp av \emph{mönstermatchning} i \code{match}-uttryck som vi ska se i kommande övning \texttt{\ExeWeekTEN}. För att underlätta interoperabilitet med Java finns  Scala-metoderna \code{isInstanceOf} och \code{asInstanceOf}, som motsvarar hur typtest och typkonvertering görs i Java.\footnote{\code{isInstanceOf} och \code{asInstanceOf} används sällan i Scala eftersom \code{match} är kraftfullare och säkrare.}

Gör nedan deklarationer.
\begin{REPL}
scala> trait A; trait B extends A; class C extends B; class D extends B
scala> val (c, d) = (new C, new D)
scala> val a: A = c
scala> val b: B = d
\end{REPL}

\Subtask Rita en bild över vilka typer som ärver vilka.

\Subtask Vilket resultat ger dessa typtester? Varför?
\begin{REPL}
scala> c.isInstanceOf[C]
scala> c.isInstanceOf[D]
scala> d.isInstanceOf[B]
scala> c.isInstanceOf[A]
scala> b.isInstanceOf[A]
scala> b.isInstanceOf[D]
scala> a.isInstanceOf[B]
scala> c.isInstanceOf[AnyRef]
scala> c.isInstanceOf[Any]
scala> c.isInstanceOf[AnyVal]
scala> c.isInstanceOf[Object]
scala> 42.isInstanceOf[Object]
scala> 42.isInstanceOf[Any]
\end{REPL}

\Subtask Vilka av dessa typkonverteringar ger felmeddelande? Vilket och varför?
\begin{REPL}
scala> c.asInstanceOf[B]
scala> c.asInstanceOf[A]
scala> d.asInstanceOf[C]
scala> a.asInstanceOf[B]
scala> a.asInstanceOf[C]
scala> a.asInstanceOf[D]
scala> a.asInstanceOf[E]
scala> b.asInstanceOf[A]
\end{REPL}



\SOLUTION


\TaskSolved \what


\SubtaskSolved  B ärver A. C och D ärver B.

\SubtaskSolved  1. True eftersom c är av typen C. \\
2. False eftersom c inte är av typen D. \\
3. True eftersom d är av typen D som är en subtyp av B. \\
4. True eftersom c är av typen C som är en subtyp av B, som i sin tur är en subtyp av A. \\
5. True eftersom b är av typen D, som är en subtyp av B, som i sin tur är en subtyp av A. \\
6. True eftersom b är av typen D. \\
7. True eftersom a är av typen C som är en subtyp av B. \\
8. True eftersom c är av typen C som är en subtyp av AnyRef. \\
9. True eftersom c är av typen C som är en subtyp av Any. \\
10. Error eftersom \code{isInstanceOf} inte kan använda sig av \code{AnyVal}.  \\
11. True eftersom c är av typen C som är en subtyp av Object (Object är java-representationen av AnyRef). \\
12. Error eftersom \code{isInstanceOf} inte kan testa om värdetyper (i detta fallet \code{42}) är referenstyper. \\
13. True eftersom \code{42} är av typen \code{Int} som är en subtyp av Any. \\

\SubtaskSolved  3. Går inte eftersom c inte är av typen D, utan typen C. \\
6. Går inte eftersom a inte är av typen D, utan typen C. \\
7. Går inte eftersom typen E inte finns. \\


\QUESTEND













\WHAT{Saknad referens med \texttt{null} och bottentypen \texttt{Nothing}.}

\QUESTBEGIN

\Task  \what~ Hitta på en egen fördjupningsuppgift inspirerat av denna artikel på Stackoverflow: \url{http://stackoverflow.com/questions/16173477/usages-of-null-nothing-unit-in-scala}

\SOLUTION


\QUESTEND






\WHAT{Arvshierarki med matematiska tal.}

\QUESTBEGIN

\Task  \what~ Studera den djupa arvshierarkin i paketet \code{numbers} i koden på efterföljande sidor. Paketet  \code{numbers} modellerar olika sorters tal i matematiken, med syftet att erbjuda ett s.k. DSL \footnote{\url{https://en.wikipedia.org/wiki/Domain-specific_language}}, alltså ett specialspråk för en viss applikationsdomän\footnote{\url{https://stackoverflow.com/questions/49216312/what-is-dsl-in-scala}}, här: domänen matematiska tal.

Du kan ladda ner koden för \code{numbers} här: \\
\href{https://github.com/lunduniversity/introprog/blob/master/compendium/examples/numbers.scala}{github.com/lunduniversity/introprog/blob/master/compendium/examples/numbers.scala}
\\ Notera speciellt metoden \code{reduce} som reducerar ett tal till sin enklaste form. Metoden \code{reduce} överskuggas på lämpliga ställen med relevant reduktion.

\Subtask Rita en bild över typhierarkin, t.ex. som ett upp-och-nedvänt träd med bastypen  \code{Number} som rot.

\Subtask Skriv kod som använder de olika konkreta klasserna i \code{package numbers}. Om du kompilerar koden i samma bibliotek som du kör igång REPL är det bara att använda paketet direkt, som ett litet ''matetmatiklaboratorium'':
\begin{REPL}
$ scalac numbers.scala
$ scala

scala> numbers.  // Tryck Tab
AbstractComplex   AbstractNatural    AbstractReal   Frac    Nat      Polar
AbstractInteger   AbstractRational   Complex        Integ   Number   Real

scala> numbers.Integ(12)
res0: numbers.Integ = Integ(12)

scala> import numbers.Syntax._
import numbers.Syntax._

scala> 42.j
res1: numbers.Complex = Complex(Real(0),Real(42))

scala> 42.42.j
res2: numbers.Complex = Complex(Real(0),Real(42.42))

\end{REPL}

\Subtask Ändra på metoden \code{+} i \code{trait Number} så att den blir abstrakt och implementera den i alla konkreta klasser.

\Subtask Implementera fler räknesätt och bygg vidare på koden så som du finner intressant.

\Subtask Gör så att metoden \code{reduce} i klassen \code{AbstractRational} använder algoritmen Greatest Common Divisor (GCD)\footnote{\url{https://sv.wikipedia.org/wiki/St\%C3\%B6rsta\_gemensamma\_delare}} så som beskrivs här: \\ \href{http://www.artima.com/pins1ed/functional-objects.html#6.8}{www.artima.com/pins1ed/functional-objects.html\#6.8} \\ så att täljare och nämnare blir så små som möjligt.

%\clearpage

\scalainputlisting[numbers=left, basicstyle=\ttfamily\fontsize{9.1}{12.2}\selectfont]{examples/numbers.scala}\SOLUTION


\QUESTEND
