%!TEX encoding = UTF-8 Unicode

%!TEX root = ../compendium2.tex

\Exercise{\ExeWeekNINE}\label{exe:W09}

\begin{Goals}
%!TEX encoding = UTF-8 Unicode

%!TEX root = ../compendium2.tex

\item Förstå följande begrepp: supertyp, subtyp, bastyp, abstrakt typ, polymorfism.
\item Kunna deklarera och använda en arvshierarki i flera nivåer med nyckelordet \code{extends}.
\item Kunna deklarera och använda inmixning med flera traits och nyckelordet \code{with}.
\item Kunna deklarera och känna till nyttan med finala klasser och finala attribut och nyckelordet \code{final}.
\item Känna till synlighetsregler vid arv och nyttan med privata och skyddade attribut.
\item Kunna deklarera och använda skyddade attribut med nyckelordet \code{protected}.
\item Känna till hur typtester och typkonvertering vid arv kan göras med metoderna \code{isInstanceOf} och \code{asInstanceOf} och känna till att detta görs bättre med \code{match}.
\item Känna till begreppet anonym klass.
\item Kunna deklarera och använda överskuggade metoder med nyckelordet \code{override}.
\item Känna till reglerna som gäller vid överskuggning av olika sorters medlemmar.
\item Kunna deklarera och använda hierarkier av klasser där konstruktorparametrar överförs till superklasser.
\item Kunna deklarera och använda uppräknade värden med case-objekt och gemensam bastyp.

\end{Goals}

\begin{Preparations}
\item \StudyTheory{09}
\end{Preparations}

\BasicTasks %%%%%%%%%%%%%%%%


\Task \emph{Gemensam bastyp.} Man vill ofta lägga in objekt av olika typ i samma samling.
\begin{REPL}
class Gurka(val vikt: Int)
class Tomat(val vikt: Int)
val gurkor = Vector(new Gurka(100), new Gurka(200))
val grönsaker = Vector(new Gurka(300), new Tomat(42))
\end{REPL}
\Subtask Om en samling innehåller objekt av flera olika typer försöker kompilatorn härleda den mest specifika typen som objekten har gemensamt. Vad blir det för typ på värdet \code{grönsaker} ovan?

\Subtask Försök ta reda på summan av vikterna enligt nedan. Vad ger andra raden för felmeddelande? Varför?

\begin{REPL}
gurkor.map(_.vikt).sum
grönsaker.map(_.vikt).sum
\end{REPL}

\Subtask Vi kan göra så att vi kan komma åt vikten på alla grönsaker genom att ge gurkor och tomater en gemensam bastyp som de olika konkreta grönsakstyperna utvidgar med nyckelordet \code{extends}. Man säger att subtyperna \code{Gurka} och \code{Tomat} \textbf{ärver} egenskaperna hos supertypen \code{Grönsak}.

Attributet \code{vikt} i traiten \code{Grönsak} nedan initialiseras inte förrän konstruktorerna anropas när vi gör \code{new} på någon av klasserna \code{Gurka} eller \code{Tomat}.

\begin{REPL}
trait Grönsak { val vikt: Int }
class Gurka(val vikt: Int) extends Grönsak
class Tomat(val vikt: Int) extends Grönsak
val gurkor = Vector(new Gurka(100), new Gurka(200))
val grönsaker = Vector(new Gurka(300), new Tomat(42))
\end{REPL}

\Subtask Vad blir det nu för typ på variabeln \code{grönsaker} ovan?

\Subtask Fungerar det nu att räkna ut summan av vikterna i \code{grönsaker} med \code{grönsaker.map(_.vikt).sum}?


\Subtask En trait liknar en klass, men man kan inte instansiera den och den kan inte ha några parametrar. En typ som inte kan instansieras kallas \textbf{abstrakt} \Eng{abstract}. Vad blir det för felmeddelande om du försöker göra \code{new} på en trait enligt nedan?
\begin{REPL}
trait Grönsak { val vikt: Int }
new Grönsak
\end{REPL}


\Subtask Traiten \code{Grönsak} har en abstrakt medlem \code{vikt}. Den sägs vara abstrakt eftersom den saknar definition -- medlemmen har bara ett namn och en typ men inget värde. Du kan instansiera den abstrakta traiten \code{Grönsak} om du fyller i det som ''fattas'', nämligen ett värde på \code{vikt}. Man kan fylla på det som fattas i genom att ''hänga på'' ett block efter typens namn vid instansiering. Man får då vad som kallas en \textbf{anonym} klass, i detta fall en ganska konstig grönsak som inte är någon speciell sorts grönsak med som ändå har en vikt.

Vad får \code{anonymGrönsak} nedan för typ och strängrepresenation?
\begin{REPL}
val anonymGrönsak = new Grönsak { val vikt = 42 }
\end{REPL}



\Task \emph{Polymorfism i samband med arv.} Polymorfism betyder ''många skepnader''. I samband med arv  innebär det att flera subtyper, till exempel \code{Ko} och \code{Gris}, kan hanteras gemensamt som om de vore instanser av samma supertyp, så som \code{Djur}. Subklasser kan implementera en metod med samma namn på olika sätt. Vilken metod som exekveras bestäms vid körtid beroende på vilken subtyp som instansieras. På så sätt kan djur komma i många skepnader.

\Subtask Implementera funktionen \code{skapaDjur} nedan så att den returnerar antingen en ny Ko eller en ny Gris med lika sannolikhet.

\begin{REPL}
scala> trait Djur { def väsnas: Unit }
scala> class Ko   extends Djur { def väsnas = println("Muuuuuuu") }
scala> class Gris extends Djur { def väsnas = println("Nöffnöff") }
scala> def skapaDjur: Djur = ???
scala> val bondgård = Vector.fill(42)(skapaDjur)
scala> bondgård.foreach(_.väsnas)
\end{REPL}

\Subtask Lägg till ett djur av typen Häst som väsnas på lämpligt sätt och modifiera \code{skapaDjur} så att det skapas kor, grisar och hästar med lika sannolikhet.



\Task \emph{Bastypen \code{Shape} och subtyperna \code{Rectangle} och \code{Circle}.} Du ska nu skapa ett litet bibliotek för geometriska former med oföränderliga objekt implementerade med hjälp av case-klasser. De geometriska formerna har en gemensam bastyp kallad \code{Shape}. Skriv nedan kod i en editor och klistra sedan in den i REPL med kommandot \code{:paste}.
\begin{Code}
case class Point(x: Double, y: Double) {
  def move(dx: Double, dy: Double): Point = Point(x + dx, y + dy)
}

trait Shape {
  def pos: Point
  def move(dx: Double, dy: Double): Shape
}

case class Rectangle(
  pos: Point,
  dx: Double,
  dy: Double
) extends Shape {
  override def move(dx: Double, dy: Double): Rectangle =
    Rectangle(pos.move(dx, dy), this.dx, this.dy)
}

case class Circle(pos: Point, radius: Double) extends Shape {
  override def move(dx: Double, dy: Double): Circle =
    Circle(pos.move(dx, dy), radius)
}
\end{Code}

\Subtask Instansiera några cirklar och rektanglar och gör några relativa förflyttningar av dina instanser genom att anropa \code{move}.

\Subtask Lägg till metoden \code{moveTo} i \code{Point}, \code{Shape}, \code{Rectangle} och \code{Circle} som gör en absolut förflyttning till koordinaterna \code{x} och \code{y}. Klistra in i REPL och testa på några instanser av \code{Rectangle} och \code{Circle}.

\Subtask Lägg till metoden \code{distanceTo(that: Point): Double } i case-klassen \code{Point} som räknar ut avståndet till en annan punkt med hjälp av \code{math.hypot}. Klistra in i REPL och testa på några instanser av \code{Point}.

\Subtask Lägg till en konkret metod \code{distanceTo(that: Shape): Double } i traiten \code{Shape} som räknar ut avståndet till positionen för en annan Shape. Klistra in i REPL och testa på några instanser av \code{Rectangle} och \code{Circle}.







\Task \label{task:fyle} \emph{Inmixning.} Man kan utvidga en klass med multipla traits med nyckelordet \code{with}. På så sätt kan man fördela medlemmar i olika traits och återanvända gemensamma koddelar genom så kallad \textbf{inmixning}, så som nedan exempel visar.

En alternativ fågeltaxonomi, speciellt populär i Skåne, delar in alla fåglar i två specifika kategorier: Kråga respektive Ånka. Krågor kan flyga men inte simma, medan Ånkor kan simma och oftast även flyga. Fågel i generell, kollektiv bemärkelse kallas på gammal skånska för Fyle.%
\footnote{\href{http://www.klangfix.se/ordlista.htm}{www.klangfix.se/ordlista.htm}}
Skriv in nedan kod i en editor och spara den för kommande uppgifter. Klistra in koden i REPL med kommandot \code{:paste}.

\begin{Code}
trait Fyle {
  val läte: String
  def väsnas: Unit = print(läte * 2)
  val ärSimkunnig: Boolean
  val ärFlygkunnig: Boolean
}

trait KanSimma       { val ärSimkunnig = true }
trait KanInteSimma   { val ärSimkunnig = false }
trait KanFlyga       { val ärFlygkunnig = true }
trait KanKanskeFlyga { val ärFlygkunnig = math.random < 0.8 }

class Kråga extends Fyle with KanFlyga with KanInteSimma {
  val läte = "krax"
}

class Ånka extends Fyle with KanSimma with KanKanskeFlyga {
  val läte = "kvack"
  override def väsnas = print(läte * 4)
}
\end{Code}

\Subtask En flitig ornitolog hittar 42 fåglar i en perfekt skog där alla fågelsorter är lika sannolika, representerat av vektorn \code{fyle} nedan. Skriv i REPL ett uttryck som undersöker hur många av dessa som är flygkunniga Ånkor, genom att använda metoderna \code{filter}, \code{isInstanceOf}, \code{ärFlygkunnig} och \code{size}.

\begin{REPL}
scala> val fyle =
         Vector.fill(42)(if (math.random > 0.5) new Kråga else new Ånka)
scala> fyle.foreach(_.väsnas)
scala> val antalFlygånkor: Int = ???
\end{REPL}

\Subtask \label{subtask:fyle:sound} Om alla de fåglar som ornitologen hittade skulle väsnas exakt en gång var, hur många krax och hur många kvack skulle då höras? Använd metoderna \code{filter} och \code{size}, samt predikatet \code{ärSimkunnig} för att beräkna antalet krax respektive kvack.
\begin{REPL}
scala> val antalKrax: Int = ???
scala> val antalKvack: Int = ???
\end{REPL}

\Task \emph{Finala klasser.} Om man vill förhindra att man kan göra \code{extends} på en klass kan man göra den final genom att placera nyckelordet \code{final} före nyckelordet \code{class}.

\Subtask Eftersom klassificeringen av fåglar i uppgiften ovan i antingen Ånkor eller Krågor är fullständig och det inte finns några subtyper till varken Ånkor eller Krågor är det lämpligt att göra dessa finala. Ändra detta i din kod.

\Subtask Testa att ändå försöka göra en subklass \code{Simkråga extends Kråga}. Vad ger kompilatorn för felmeddelande om man försöker utvidga en final klass?


\Task \emph{Accessregler vid arv och nyckelordet \code{protected}.} Om en medlem i en supertyp är privat så kan man inte komma åt den i en subklass. Ibland vill man att subklassen ska kunna komma åt en medlem även om den ska vara otillgänglig i annan kod.

\begin{REPL}
trait Super {
  private val minHemlis = 42
  protected val vårHemlis = 42
}
class Sub extends Super {
  def avslöja = minHemlis
  def kryptisk = vårHemlis * math.Pi
}
\end{REPL}

\Subtask Vad blir felmeddelandet när klassen \code{Sub} försöker komma åt \code{minHemlis}?

\Subtask Deklarera \code{Sub} på nytt, men nu utan den förbjudna metoden \code{avslöja}. Vad blir felmeddelandet om du försöker komma åt \code{vårHemlis} via en instans av klassen \code{Sub}? Prova till exempel med \code{(new Sub).vårHemlis}

\Subtask Fungerar det att anropa metoden \code{kryptisk} på instanser av klassen \code{Sub}?

\Task \emph{Använding av \code{protected}.} Den flitige ornitologen från uppgift \ref{task:fyle} ska ringmärka alla 42 fåglar hen hittat i skogen. När hen ändå håller på bestämmer hen att även försöka ta reda på hur mycket oväsen som skapas av respektive fågelsort. För detta ändamål apterar den flitige ornitologen en linuxdator på allt infångat fyle. Du ska hjälpa ornitologen att skriva programmet.

\Subtask Inför en \code{protected var räknaLäte} i traiten \code{Fyle} och skriv kod på lämpliga ställen för att räkna hur många läten som respektive fågelinstans yttrar.

\Subtask Inför en metod \code{antalLäten} som returnerar antalet krax respektive kvack som en viss fågel yttrat sedan dess skapelse.

\Subtask\Pen Varför inte använda \code{private} i stället for \code{protected}?

\Subtask\Pen Varför är det bra att göra räknar-variabeln oåtkomlig från ''utsidan''?



\Task \emph{Typtester med \code{isInstanceOf} och typkonvertering med \code{asInstanceOf}.} Gör nedan deklarationer.
\begin{REPL}
scala> trait A; trait B extends A; class C extends B; class D extends B
scala> val (c, d) = (new C, new D)
scala> val a: A = c
scala> val b: B = d
\end{REPL}

\Subtask Rita en bild över vilka typer som ärver vilka.

\Subtask Vilket resultat ger dessa typtester? Varför?
\begin{REPL}
scala> c.isInstanceOf[C]
scala> c.isInstanceOf[D]
scala> d.isInstanceOf[B]
scala> c.isInstanceOf[A]
scala> b.isInstanceOf[A]
scala> b.isInstanceOf[D]
scala> a.isInstanceOf[B]
scala> c.isInstanceOf[AnyRef]
scala> c.isInstanceOf[Any]
scala> c.isInstanceOf[AnyVal]
scala> c.isInstanceOf[Object]
scala> 42.isInstanceOf[Object]
scala> 42.isInstanceOf[Any]
\end{REPL}

\Subtask Vilka av dessa typkonverteringar ger felmeddelande? Vilket och varför?
\begin{REPL}
scala> c.asInstanceOf[B]
scala> c.asInstanceOf[A]
scala> d.asInstanceOf[C]
scala> a.asInstanceOf[B]
scala> a.asInstanceOf[C]
scala> a.asInstanceOf[D]
scala> a.asInstanceOf[E]
scala> b.asInstanceOf[A]
\end{REPL}



\Task \emph{Regler för \code{override}, \code{private} och \code{final}.}

\Subtask \label{subtask:overriderules} Undersök överskuggningning av abstrakta, konkreta, privata och finala medlemmar genom att skriva in raderna nedan en i taget i REPL. Vilka rader ger felmeddelande? Varför? Vid felmeddelande: notera hur felmeddelandet lyder och ändra deklarationen av den felande medlemmen så att koden blir kompilerbar (eller om det är enda rimliga lösningen: ta bort den felaktiga medlemmen), innan du provar efterkommande rad.

\begin{REPL}
trait Super1 { def a: Int; def b = 42; private def c = "hemlis" }
class Sub2 extends Super1 { def a = 43; def b = 43; def c = 43 }
class Sub3 extends Super1 { def a = 43; override def b = 43 }
class Sub4 extends Super1 { def a = 43; override def c = "43" }
trait Super5 { final def a: Int; final def b = 42 }
class Sub6 extends Super5 { override def a = 43; def b = 43 }
class Sub7 extends Super5 { def a = 43; override def b = 43 }
class Sub8 extends Super5 { def a = 43; override def c = "43" }
trait Super9 { val a: Int; val b = 42; lazy val c: String = "lazy" }
class Sub10 extends Super9 { override def a = 43; override val b = 43 }
class Sub11 extends Super9 { val a = 43; override lazy val b = 43 }
class Sub12 extends Super9 { val a = 43; override var b = 43 }
class Sub13 extends Super9 { val a = 43; override lazy val c = "still lazy" }
class SubSub extends Sub13 { override val a = 44}
trait Super14 { var a: Int; var b = 42; var c: String }
class Sub15 extends Super14 { def a = 43; override var b = 43; val c = "?" }
\end{REPL}

\Subtask Skapa instanser av klasserna \code{Sub3}, \code{Sub13} och \code{SubSub} från ovan deluppgift och undersök alla medlemmarnas värden för respektive instans. Förklara varför de har dessa värden.

\Subtask Läs igenom reglerna i kapitel  \ref{slideW07:overriderules} om vad som gäller vid arv och överskuggning av medlemmar. Gör några egna undersökningar i REPL som försöker bryta mot någon regel som inte testades i deluppgift \ref{subtask:overriderules}.

\Task \emph{Supertyp med parameter.} En trait kan inte ha någon parameter. Vill man ha en parameter till supertypen måste man använda en klass istället, enligt nedan exempel.

Utbildningsdepartementet vill i sitt system hålla koll på vissa personer och skapar därför en klasshierarki enligt nedan. Skriv in koden i en editor och klipp sedan in den i REPL.
\begin{Code}
class Person(val namn: String)

class Akademiker(
  namn: String,
  val universitet: String) extends Person(namn)

class Student(
  namn: String,
  universitet: String,
  program: String) extends Akademiker(namn, universitet)

class Forskare(
  namn: String,
  universitet: String,
  titel: String) extends Akademiker(namn, universitet)
\end{Code}


\Subtask Deklarera fyra olika \code{val}-variabler med lämpliga namn som refererar till olika instanser av alla olika klasser ovan och ge attributen valfria initialvärden genom olika parametrar till konstruktorerna.

\Subtask Skriv två satser: en som först stoppar in instanserna i en \code{Vector} och en som sedan loopar igenom vektorn och skriv ut alla instansers \code{toString} och \code{namn}.


\Subtask Utbildningsdepartementet vill att det inte ska gå att instansiera objekt av typerna \code{Person} och \code{Akademiker}. Det kan åstadkommas genom att placera nyckelordet \code{abstract} före \code{class}. Uppdatera koden i enlighet med detta. Vilket blir felmeddelande om man försöker instansiera en \code{abstract class}?

\Subtask Utbildningsdeparetementet vill slippa implementera \code{toString} och slippa skriva \code{new} vid instansiering. Gör därför om typerna \code{Student} och \code{Forskare} till case-klasser. \emph{Tips:} För att undkomma ett kompileringsfel (vilket?) behöver du använda \code{override val} på lämpligt ställe.

Skapa instanser av de nya case-klasserna \code{Student} och \code{Forskare} och skriv ut deras \code{toString}. Hur ser utskriften ut?

\Subtask Eftersom \code{Person} och \code{Akademiker} nu är abstrakta, vill utbildningsdepartementet att du gör om dessa typer till traits med abstrakta attribut istället för klasser. Du kan då undvika \code{override val} i klassparametrarna till de konkreta case-klasserna.

Man inför också en case-klass \code{IckeAkademiker} som man tänker använda i olika statistiska medborgarundersökningar.

Dessutom förser man alla personer med ett personnummer representerat som en \code{Int}.

Hur ser utbildningsdepartementets kod ut nu, efter alla ändringar? Skriv ett testprogram som skapar några instanser och skriver ut deras attribut.

\Subtask\Pen I vilka sammanhang är det nödvändigt att använda en \code{trait} respektive en \code{class}?




\Task \emph{Uppräknade värden.} Ett sätt att hålla reda på uppräknade värden, t.ex. färgen på olika kort i en kortlek, är att använda olika heltal som får representera de olika värdena, till exempel så här:\footnote{Om namnkonventioner för konstanter i Scala: läs under rubriken ''Constants, Values, Variable and Methods'' här \href{http://docs.scala-lang.org/style/naming-conventions.html}{docs.scala-lang.org/style/naming-conventions.html}}
\begin{Code}
object Färg {
  val Spader = 1
  val Hjärter = 2
  val Ruter = 3
  val Klöver = 4
}
\end{Code}
Dessa kan sedan användas som parametrar till denna case-klass vid skapande av kortobjekt:
\begin{lstlisting}[language=,keywords={case,class}]
case class Kort(färg: Int, valör: Int)
\end{lstlisting}
Man kan hålla reda på färgen med en variabel av typen \code{Int} och tilldela den en viss färg med ovan konstanter. Och när man skapar ett kort behöver man inte komma ihåg vilket numret är.
\begin{REPL}
scala> val f = Färg.Spader
scala> import Färg._
scala> Kort(Ruter, 7)
\end{REPL}
En annan fördelen med detta är att man lätt kan loopa från 1 till 4 för att gå igenom alla färger.
\begin{REPL}
scala> val kortlek = for (f <- 1 to 4; v <- 1 to 13) yield Kort(f, v)
\end{REPL}
Nackdelen är att kompilatorn vid kompileringstid inte kollar om variablerna av misstag råkar ges något värde utanför det giltiga intervallet, t.ex. 42. Detta får vi själv hålla koll på vid körtid, till exempel med funktionen \code{require} eller \code{if}-satser, etc.

Istället kan man använda case-objekt enligt nedan deluppgifter och få hjälp av kompilatorn att hitta eventuella fel vid kompileringstid.  Ett case-objekt är som ett vanligt singelton-objekt men det får automatiskt en \code{toString} samma som namnet och kan användas i matchningar etc. (mer om match i kapitel \ref{chapter:W08}).

\Subtask Deklarera följande uppräknade värden som singelton objekt med gemensam bastyp i en editor och klistra in i REPL med kommandot \code{:paste}. Med nyckelordet \code{sealed} så ''förseglas'' klassen och inga andra direkta subtyper tillåts förutom de som finns i samma kod-fil eller block. I detta exempel  med kortfärger vet vi ju att det inte finns fler än dessa fyra färger.
\begin{Code}
sealed trait Färg
case object Spader extends Färg
case object Hjärter extends Färg
case object Ruter extends Färg
case object Klöver extends Färg
\end{Code}
Dessa kan sedan användas som parametrar till denna case-klass vid skapande av kortobjekt:
\begin{lstlisting}[language=,keywords={case,class}]
case class Kort(färg: Färg, valör: Int)
\end{lstlisting}
Skapa därefter några exempelinstanser av klassen \code{Kort}. Vad är fördelen med ovan angreppssätt jämfört med att använda heltalskonstanter?

\Subtask Om man vill kunna iterera över alla värden är det bra om de finns i en samling med alla värden. Vi kan lägga en sådan i ett kompanjonsobjekt till bastypen. Uppdatera koden enligt nedan och klistra in på nytt i REPL med kommandot \code{:paste}. Skriv ut alla färgvärden med en \code{for}-sats.

\begin{Code}
sealed trait Färg
object Färg {
  val values = Vector(Spader, Hjärter, Ruter, Klöver)
}
case object Spader extends Färg
case object Hjärter extends Färg
case object Ruter extends Färg
case object Klöver extends Färg
\end{Code}
Skapa en kortlek med 52 olika kort och blanda den sedan med \code{Random.shuffle} enligt nedan. Använd en dubbel \code{for}-sats och \code{yield}.
\begin{REPL}
scala> val kortlek: Vector[Kort] = ???
scala> val blandad = scala.util.Random.shuffle(kortlek)
\end{REPL}

\Subtask Skriv en funktion \code{ def blandadKortlek: Vector[Kort] = ???} som ger en ny blandad kortlek varje gång den anropas med metoden i föregående uppgift.

%%%%%%%%%%%%%%%%%%% FEEEEEELLL \end{Code}



\Subtask Om man även vill ha en heltalsrepresentation med en medlem \code{toInt} för alla värden, kan man ge bastypen en parameter och i stället för en trait (som inte kan ha några parametrar) använda en abstrakt klass.

\begin{Code}
sealed abstract class Färg(final val toInt: Int)
object Färg {
  val values = Vector(Spader, Hjärter, Ruter, Klöver)
}
case object Spader  extends Färg(0)
case object Hjärter extends Färg(1)
case object Ruter   extends Färg(2)
case object Klöver  extends Färg(3)
\end{Code}
Skapa en funktion \code{färgPoäng} som räknar ut summan av heltalsrepresentationen av alla färger hos en vektor med kort, och använd den sedan för att räkna ut \code{färgPoäng} för de första fem korten.
\begin{REPL}
scala> def blandadKortlek: Vector[Kort] = ???
scala> def färgPoäng(xs: Vector[Kort]): Int = ???
scala> färgPoäng(blandadKortlek.take(5))
\end{REPL}


\ExtraTasks %%%%%%%%%%%%%%%%%%%

\Task Det visar sig att vår flitige ornitolog från uppgift \ref{task:fyle} på sidan \pageref{task:fyle} sov på en av föreläsningarna i zoologi när hen var nolla på Natfak, och därför helt missat fylekategorin \code{Pjodd}. Hjälp vår stackars ornitolog så att fylehierarkin nu även omfattar Pjoddar. En Pjodd kan flyga som en Kråga men den \code{ÄrLiten} medan en Kråga \code{ÄrStor}. En Pjodd kvittrar dubbelt så många gånger som en Ånka kvackar. En Pjodd \code{KanKanskeSimma} där simkunnighetssannolikheten är $0.2$. Låt ornitologen ånyo finna 42 slumpmässiga fåglar i skogen och filtrera fram lämpliga arter. Undersök sedan hur dessa väsnas, i likhet med deluppgift \ref{task:fyle}\ref{subtask:fyle:sound}.


\clearpage

\AdvancedTasks %%%%%%%%%%%%%%%%%

\Task Hitta på en egen fördjupningsuppgift inspirerat av denna artikel på Stackoverflow: \url{http://stackoverflow.com/questions/16173477/usages-of-null-nothing-unit-in-scala}

\Task Studera den djupa arvshierarkin i paketet \code{numbers} nedan som modellerar olika sorters tal i matematiken. Du kan även ladda ner koden här: \\
\href{https://github.com/lunduniversity/introprog/blob/master/compendium/examples/numbers.scala}{github.com/lunduniversity/introprog/blob/master/compendium/examples/numbers.scala}
\\ Notera metoden \code{reduce} som reducerar ett tal till sin enklaste form och dess implementation överskuggas på lämpliga ställen med relevant reduktion.

\Subtask Skriv kod som använder de olika konkreta klasserna i \code{package numbers}. Om du kompilerar koden i samma bibliotek som du kör igång REPL är det bara att använda paketet direkt:
\begin{REPL}
$ scalac numbers.scala
$ scala
scala> numbers.  // Tryck Tab
AbstractComplex   AbstractNatural    AbstractReal   Frac    Nat      Polar
AbstractInteger   AbstractRational   Complex        Integ   Number   Real

scala> numbers.Integ(12)
res0: numbers.Integ = Integ(12)

scala> import numbers.Syntax._
import numbers.Syntax._

scala> 42.j
res1: numbers.Complex = Complex(Real(0),Real(42))

scala> 42.42.j
res2: numbers.Complex = Complex(Real(0),Real(42.42))

\end{REPL}

\Subtask Ändra på metoden \code{+} i \code{trait Number} så att den blir abstrakt och implementera den i alla konkreta klasser.

\Subtask Implementera fler räknesätt och bygg vidare på koden så som du finner intressant.

\Subtask Gör så att metoden \code{reduce} i klassen \code{AbstractRational} använder algoritmen Greatest Common Divisor (GCD)\footnote{\href{https://sv.wikipedia.org/wiki/St\%C3\%B6rsta_gemensamma_delare}{https://sv.wikipedia.org/wiki/St\%C3\%B6rsta\_gemensamma\_delare}} så som beskrivs här: \\ \href{http://www.artima.com/pins1ed/functional-objects.html#6.8}{www.artima.com/pins1ed/functional-objects.html\#6.8} \\ så att täljare och nämnare blir så små som möjligt.

\scalainputlisting[numbers=left, basicstyle=\ttfamily\fontsize{9}{11}\selectfont]{examples/numbers.scala}
