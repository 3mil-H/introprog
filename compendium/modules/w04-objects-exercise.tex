
%!TEX encoding = UTF-8 Unicode
%!TEX root = ../exercises.tex

\ifPreSolution


\Exercise{\ExeWeekFOUR}\label{exe:W04}
\begin{Goals}
%!TEX encoding = UTF-8 Unicode
%!TEX root = ../exercises.tex

\item Kunna skapa och använda objekt som moduler.
\item Känna till att funktioner är objekt med en \code{apply}-metod.
\item Förstå begreppen synlighet, \code{private}, \code{import}, namnrymd och skuggning.
\item Förstå hur nästlade block påverkar namnsynlighet och namnöverskuggning.
%https://it-ord.idg.se/ord/overskuggning/
\item Förstå kopplingen mellan paketstruktur och kodfilstruktur.
\item Kunna skapa en jar-fil.
\item Kunna skapa dokumentation med scaladoc. \TODO ???flytta till projektveckan och kräva att man visar doc för handledaren?
\item \TODO FLER MÅL OM OBJEKT OCH MODULER HÄR

\end{Goals}

\begin{Preparations}
\item \StudyTheory{04}
\end{Preparations}

\else

\ExerciseSolution{\ExeWeekFOUR}

\fi


\BasicTasks %%%%%%%%%%%%%%%%



\WHAT{Para ihop begrepp med beskrivning.}

\QUESTBEGIN

\Task \what

\vspace{1em}\noindent Koppla varje begrepp med den (förenklade) beskrivning som passar bäst:

\begin{ConceptConnections}
  modul & 1 & & A & tillhör ett objekt; nås med punktnotation om synlig \\ 
  singelobjekt & 2 & & B & modul som kan ha tillstånd; finns i en enda upplaga \\ 
  paket & 3 & & C & modul som skapar namnrymd; maskinkod får egen katalog \\ 
  import & 4 & & D & används för att komma åt icke-privata delar \\ 
  lat initialisering & 5 & & E & variabel som utgör (del av) ett objekts tillstånd \\ 
  medlem & 6 & & F & omgivning där är alla namn är unika \\ 
  attribut & 7 & & G & gör namn tillgängligt utan att hela sökvägen behövs \\ 
  metod & 8 & & H & funktion som är medlem av ett objekt \\ 
  privat & 9 & & I & alternativt namn på typ som ofta ökar läsbarheten \\ 
  överlagring & 10 & & J & allokering sker först när namnet refereras \\ 
  namnskuggning & 11 & & K & kodenhet med abstraktioner som kan återanvändas \\ 
  namnrymd & 12 & & L & modifierar synligheten av en objektmedlem \\ 
  uniform access & 13 & & M & metoder med samma namn men olika parametertyper \\ 
  punktnotation & 14 & & N & ändring mellan def och val påverkar ej användning \\ 
  typalias & 15 & & O & lokalt namn döljer samma namn i omgivande block \\ 
\end{ConceptConnections}

\SOLUTION

\TaskSolved \what

\begin{ConceptConnections}
  modul & 1 & ~~\Large$\leadsto$~~ &  N & kodenhet med abstraktioner som kan återanvändas \\ 
  singelobjekt & 2 & ~~\Large$\leadsto$~~ &  L & modul som kan ha tillstånd; finns i en enda upplaga \\ 
  paket & 3 & ~~\Large$\leadsto$~~ &  H & modul som skapar namnrymd; maskinkod får egen katalog \\ 
  import & 4 & ~~\Large$\leadsto$~~ &  O & gör namn tillgängligt utan att hela sökvägen behövs \\ 
  lat initialisering & 5 & ~~\Large$\leadsto$~~ &  I & allokering sker först när namnet refereras \\ 
  medlem & 6 & ~~\Large$\leadsto$~~ &  G & tillhör ett objekt; nås med punktnotation om synlig \\ 
  attribut & 7 & ~~\Large$\leadsto$~~ &  J & variabel som utgör (del av) ett objekts tillstånd \\ 
  metod & 8 & ~~\Large$\leadsto$~~ &  K & funktion som är medlem av ett objekt \\ 
  privat & 9 & ~~\Large$\leadsto$~~ &  E & modifierar synligheten av en objektmedlem \\ 
  överlagring & 10 & ~~\Large$\leadsto$~~ &  F & metoder med samma namn men olika parametertyper \\ 
  namnskuggning & 11 & ~~\Large$\leadsto$~~ &  M & lokalt namn döljer samma namn i omgivande block \\ 
  namnrymd & 12 & ~~\Large$\leadsto$~~ &  C & omgivning där är alla namn är unika \\ 
  uniform access & 13 & ~~\Large$\leadsto$~~ &  D & ändring mellan def och val påverkar ej användning \\ 
  punktnotation & 14 & ~~\Large$\leadsto$~~ &  B & används för att komma åt icke-privata delar \\ 
  typalias & 15 & ~~\Large$\leadsto$~~ &  A & alternativt namn på typ som ofta ökar läsbarheten \\ 
\end{ConceptConnections}

\QUESTEND








\WHAT{Objekt som moduler.}

\QUESTBEGIN

\Task  \what~  \TODO{FLYTTA ALLT OM OBJEKT TILL W04}

\Subtask Lär dig följande terminologi utantill:

\begin{itemize}[noitemsep, nolistsep]
\item Ett singelobjekt som samlar funktioner och variabler kallas även en \textbf{modul}.
\item Funktioner i objekt kallas även \textbf{metoder}.
\item Variabler och metoder i objekt kallas \textbf{medlemmar}.
%\item Moduler kan i sin tur innehålla moduler, i godtyckligt \textbf{nästlingsdjup}.
\item Man kommer åt innehållet i en modul med \textbf{punktnotation}.
\item Med \textbf{import} slipper man punktnotation.
\item Ett objekt med variabler sägs ha ett \textbf{tillstånd}.
\end{itemize}

\Subtask Deklarera modulerna \code{stringstat} och \code{Test} nedan i REPL eller i Kojo.

\begin{Code}
object stringstat {
  object stringfun {
    def sentences(s: String): Array[String] = s.split('.')
    def words(s: String): Array[String] = s.split(' ')
    def countWords(s: String): Int = words(s).size
    def countSentences(s: String): Int = sentences(s).size
  }

  object statistics {
    var history = ""
    def printFreq(s: String): Unit = {
      println("\n---- Frekvenser ----")
      println("Antal tecken:   " + s.size)
      println("Antal ord:      " + stringfun.countWords(s))
      println("Antal meningar: " + stringfun.countSentences(s))
      history = history + " " + s
    }
    def printTotal: Unit = printFreq(history)
  }
}

object Test {
  import stringstat._
  def apply(n: Int = 42): Unit = {
    val s1 = "Fem myror är fler än fyra elefanter. Ät gurka."
    val s2 = "Galaxer i mina braxer. Tomat är gott. Hejsan."
    statistics.printFreq(s1 * n)
    statistics.printFreq(s2 * n)
    statistics.printTotal
  }
}
\end{Code}

\Subtask Anropa \code{Test()} och förklara vad som händer. Vad skrivs ut?

\Subtask Vilket av objekten i modulen \code{stringstat} har tillstånd och vilket av objekten är tillståndslöst? Vad består tillståndet av?


\SOLUTION


\TaskSolved \what


\SubtaskSolved  -

\SubtaskSolved  -

\SubtaskSolved
\begin{REPLnonum}

---- Frekvenser ----
Antal tecken: 1932
Antal ord: 337
Antal meningar: 84

---- Frekvenser ----
Antal tecken: 1890
Antal ord: 295
Antal meningar: 126

---- Frekvenser ----
Antal tecken: 3824
Antal ord: 633
Antal meningar: 210

\end{REPLnonum}

Först och främst, vi har default argument 42 så det är det som används vid anropet. Detta används för att multiplicera \code{s1} och \code{s2} i \code{Test} när man anropar \code{printFreq}. \code{statistics} objektet använder i sin tur funktionerna i \code{stringfun} för att räkna ord och meningar. Resultatet presenteras av \code{statistics}, vilket är det vi ser.

\SubtaskSolved  Vi ser att \code{stringfun} enbart innehåller funktioner och därmed inte har något tillstånd. \code{statistics} har däremot ett tillstånd eftersom variabeln \code{history} sparar vad som skett, så när man anropar funktioner i objektet så kan objektet ändra beteende beroende på dess tillstånd, tillståndet består alltså av \code{history}.



\QUESTEND
