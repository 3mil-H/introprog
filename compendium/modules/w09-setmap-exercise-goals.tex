%!TEX encoding = UTF-8 Unicode
%!TEX root = ../compendium2.tex

%\item Kunna skapa och använda tupler, som variabelvärden, parametrar och returvärden.

%\item Förstå skillnaden mellan ett objekt och en klass och kunna förklara betydelsen av begreppet instans.

%\item Kunna skapa och använda attribut som medlemmar i objekt och klasser och som som klassparametrar.

%\item Kunna beskriva den praktiska nyttan med att ett attribut är privat.

%\item Kunna byta ut implementationen av metoden \code{toString}.

%\item Kunna skapa och använda en objektfabrik med metoden \code{apply}.

%\item Kunna skapa och använda en enkel case-klass.

%\item Kunna använda operatornotation och förklara relationen till punktnotation.

%\item Förstå konsekvensen av uppdatering av föränderlig data i samband med multipla referenser.

%\item Kunna förklara den principiella skillnaderna mellan olika typer av samlingar.
\item Kunna skapa och använda tupler som parametrar och returvärden.
\item Känna till och kunna använda grundläggande metoder på samlingar.
\item Kunna skapa och använda både oföränderliga och föränderliga mängder.
\item Förstå skillnader och likheter mellan en mängd och en sekvens.
\item Kunna beskriva hur algoritmen linjärsökning fungerar.
\item Kunna skapa och använda både oföränderliga och föränderliga nyckel-värde-tabeller.
\item Kunna använda nyckel-värde-tabeller för att implementera registrering.
\item Förstå likheter och skillnader mellan en nyckel-värde-tabell och en sekvens.
\item Kunna spara och läsa data till/från textfiler på disk.
 