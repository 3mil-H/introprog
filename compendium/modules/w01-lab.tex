%!TEX root = ../compendium.tex

\Lab{\LabWeekONE}

\subsubsection{Mål}
\begin{itemize}[nosep]
\item Använda sekvens, alternativ, repetition, abstraktion.
\end{itemize}

\subsubsection{Förberedelser}
\begin{itemize}[nosep]
\item Gör övning {\tt \ExeWeekONE} i kapitel \ref{exe:W01}.
\item Läs igenom ''Kojo - An Introduction'' (25 sidor) som du kan ladda ner i pdf  här: \href{http://www.kogics.net/kojo-ebooks}{http://www.kogics.net/kojo-ebooks}
\item Du behöver en dator med Kojo installerad, se appendix \ref{appendix:kojo}.
\end{itemize}

\subsection{Obligatoriska uppgifter}


\Task \textit{Sekvens}. 

\Subtask Starta Kojo. Använd procedurerna \code{clear}, \code{forward} och \code{right} för att instruera sköldpaddan att rita en kvadrat.

\Subtask Ge \code{forward} en parameter så att kvadraten blir större.

\Subtask Rita en triangel.

\Task \textit{Repetition}. 

\Subtask Använd proceduren \code+repeat(4){ ??? }+ och rita en kvadrat.

\Subtask Använd proceduren \code{setAnimationDelay} med lämplig parameter och öka antalet repetitioner så att sköldpaddan går ungefär hundra varv på 2 sekunder.

\subsection{Frivilliga extrauppgifter}

\Task Ladda ner och gör uppgifterna i dessa pdf-kompendier:

\Subtask ''Uppdrag med Kojo'' som kan laddas ner här: \href{http://fileadmin.cs.lth.se/cs/Personal/Bjorn_Regnell/uppdrag.pdf}{http://www.lth.se/programmera}

\Subtask ''Programming Fundamentals with Kojo'' som kan laddas ner här:\\
 \href{http://wiki.kogics.net/kojo-codeactive-books}{wiki.kogics.net/kojo-codeactive-books}
