%!TEX root = ../compendium.tex

\Lab{\LabWeekONE}

\subsubsection{Mål}
\begin{itemize}[nosep]
\item Kunna kombinera principerna sekvens, alternativ, repetition, och abstraktion i skapandet av egna program om minst 20 rader kod.
\item Kunna förklara vad ett program gör i termer av sekvens, alternativ, repetition, och abstraktion.
\item Kunna formatera egna program så att de blir lätta att läsa och förstå.
\item Kunna genomföra upprepade varv i cykeln \emph{editera-exekvera-felsöka/förbättra} för att succesivt bygga upp allt mer utvecklade program. 
\item Kunna identifiera liknande satser och utifrån dessa skapa återanvändbara abstraktioner.
\end{itemize}

\subsubsection{Förberedelser}
\begin{itemize}[nosep]
\item Gör övning {\tt \ExeWeekONE} i kapitel \ref{exe:W01}.
\item Läs igenom ''Kojo - An Introduction'' (25 sidor) som du kan ladda ner i pdf  här: \href{http://www.kogics.net/kojo-ebooks}{http://www.kogics.net/kojo-ebooks}
\item Du behöver en dator med Kojo installerad, se appendix \ref{appendix:kojo}.
\end{itemize}

\subsection{Obligatoriska uppgifter}


\Task \textit{Sekvens}. 

\Subtask Starta Kojo. Skriv in nedan program och tryck på den gröna play-knappen. Välj antingen svenska eller engelska. 

\begin{multicols}{2}
\noindent Svenska:
\begin{Code}
sudda

fram; höger
fram; vänster
\end{Code}

\columnbreak
\noindent Engelska:
\begin{Code}
clear

forward; right
forward; left
\end{Code}


\end{multicols}

\Subtask Prova att ändra på ordningen mellan satserna och använd den gula play-knappen  (programspårning) för att studera vad som händer.

\Subtask Rita en kvadrat.

\Subtask Ge \code{fram} (eller \code{forward}) en parameter så att kvadraten blir större.

\Subtask Rita en triangel genom att ge lämplig vinkel som parameter till det kommando som orsakar vridningen av sköldpaddan.

\Task \textit{Repetition}. 

\Subtask Använd proceduren \code+repeat(4){ ??? }+ och rita en kvadrat.

\Subtask Använd proceduren \code{sakta} (eller \code{setAnimationDelay}) med lämplig parameter och öka antalet repetitioner så att sköldpaddan går ungefär hundra varv i kvadraten på 2 sekunder.

\Task \textit{Abstraktion}. 

\Task \textit{Alternativ}.

\Task \textit{Tidmätning}. Hur snabb är din dator?

\Subtask \label{task:timer} Skriv in koden nedan i Kojos editor och kör med den gröna play-knappen. Hur långt tid tar det för din dator att räkna till 4.4 miljarder?

\begin{Code}
object timer {
  def now: Long = System.currentTimeMillis
  var saved: Long = now
  def elapsedMillis: Long = now - saved
  def elapsedSeconds: Double = elapsedMillis / 1000.0
  def reset: Unit = { saved = now }
}

// HUVUDPROGRAM:
timer.reset
var i = 0L
while (i < 1e8.toLong) { i += 1 }
val t = timer.elapsedSeconds
println("Räknade till " + i + " på " + t + " sekunder.")
\end{Code}

\Subtask  Kör nedan Linux-kommandot upprepade gånger i ett terminalfönster. Med hur många MHz kör din dators klocka för tillfället? Prova medan du kör tidmätningen i Kojo.
\begin{REPL}
> lscpu | grep MHz
\end{REPL}

\Subtask Ändra i koden i uppgift \ref{task:timer}) så att \code{while}-loopen bara kör 5 gånger. Kör programmet med den \emph{gula} play-kappen. Scrolla i programspårningen och förklara vad som händer. Klicka på \code{CALL}-rutorna och se vilken rad som markeras i ditt program.

\Subtask Lägg till koden nedan i ditt program och försök ta reda på ungefär hur långt din dator hinner räkna till på en sekund för \code{Long}- respektive \code{Int}-variabler. Använd den gröna play-knappen.
\begin{Code}
def timeLong(n: Long): Double = {
  timer.reset
  var i = 0L
  while (i < n) { i += 1 }
  timer.elapsedSeconds
}

def timeInt(n: Int): Double = {
  timer.reset
  var i = 0
  while (i < n) { i += 1 }
  timer.elapsedSeconds
}

def show(msg: String, sec: Double): Unit = {
  print(msg + ": ")
  println(sec + " seconds")
}

def report(n: Long): Unit = {
  show("Long " + n, timeLong(n))
  if (n <= Int.MaxValue) show("Int  " + n, timeInt(n.toInt))
}

// HUVUDPROGRAM, mätningar:

report(Int.MaxValue)

for (i <- 1 to 10) {
  report(4.26e9.toLong)
}
\end{Code}

\Subtask Hur mycket snabbare går det att räkna med \code{Int}-variabler jämfört med \code{Long}-variabler?

 


\subsection{Frivilliga extrauppgifter}

\Task Ladda ner dessa pdf-kompendier och gör några uppgifter som du tycker verkar intressanta:

\Subtask ''Uppdrag med Kojo'' som kan laddas ner här:\\ \href{http://fileadmin.cs.lth.se/cs/Personal/Bjorn_Regnell/uppdrag.pdf}{fileadmin.cs.lth.se/cs/Personal/Bjorn\_Regnell/uppdrag.pdf}

\Subtask ''Programming Fundamentals with Kojo'' som kan laddas ner här:\\
 \href{http://wiki.kogics.net/kojo-codeactive-books}{wiki.kogics.net/kojo-codeactive-books}
