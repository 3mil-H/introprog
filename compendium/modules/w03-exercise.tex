%!TEX root = ../compendium.tex

\Exercise{\ExeWeekTHREE}

\begin{Goals}
\item 
\end{Goals}

\begin{Preparations}
\item 
\end{Preparations}

\BasicTasks %%%%%%%%%%%%%%%%

\Task \label{task:funcall}\emph{Definiera och anropa funktioner.} En funktion med två parametrar definieras med följande syntax i Scala: \vspace{0.5em} \\  \texttt{\code{def} \textit{namn}(\textit{parameter1}: \textit{Typ1}, \textit{parameter2}: \textit{Typ2}): \textit{Returtyp} = \textit{returvärde}}

\Subtask Definiera en funktion med namnet \code{öka} som har en heltalsparameter \code{x} och som returnerar \code{x + 1}. Ange returtypen explicit. Testa funktionen i REPL med argumentet 42.

\begin{REPL}
scala> ???  // definiera funktionen öka
scala> öka(42)
43
\end{REPL}

\Subtask\Pen Vad har funktionen \code{öka} i föregående uppgift för returtyp?

\Subtask\Pen Vad gör kompilatorn om du utelämnar returtypen?

\Subtask\Pen Varför kan det vara bra att ange returtypen explicit?

\Subtask\Pen Vad är det för skillnad mellan parameter och argument?
 
\Subtask Vad har uttrycket \code{öka(öka(öka(öka(42))))} för värde?

\Subtask Definera funktionen \code{minska(x: Int): Int} med returvärdet \code{x - 1}.

\Subtask Vad är värdet av uttrycket \code{öka(minska(öka(öka(minska(minska(42))))))}


\Task \emph{Funktion med flera parametrar.} Definiera i REPL två funktioner \code{sum} och \code{diff} med två heltalsparametrar som returnerar summan respektive differensen av argumenten: \\
\code{def sum(x: Int, y: Int): Int = x + y} \\
\code{def diff(x: Int, y: Int): Int = x - y} \\
Vad har nedan uttryck för värden?

\Subtask \code{diff(0, 100)}

\Subtask \code{diff(100, add(42, 43))}

\Subtask \code{sum(sum(42, 43), diff(100, sum(0, 0))}

\Subtask \code{sum(diff(Byte.MaxValue, Byte.MinValue),1)}

\Task \emph{Funktion med default-värden.} Förklara vad som händer här?
\begin{REPL}
scala> def inc(i: Int, j: Int = 1) = i + j
scala> inc(42, 2)
scala> inc(42, 1)
scaka> inc(42)
\end{REPL}

\Task \emph{Funktionsanrop med namngivna argument.} 
\begin{REPL}
scala> def skrivNamn(förnamn: String, efternamn: String) = 
         println("Namn: " + efternamn + ", " + förnamn)
scala> skrivNamn("Kim", "Robinson")
scala> skrivNamn(förnamn = "Viktor", efternamn = "Oval")
scaka> skrivNamn(efternamn = "Triangelsson", förnamn = "Stina")
\end{REPL}

\Subtask Förklara vad som händer ovan?

\Subtask\Pen Vad är fördelen med namngivna argument?



\Task \emph{Applicera en funktion på elementen i en samling.} Använd dina funktioner \code{öka} och \code{minska} från uppgift \ref{task:funcall}. Vad har nedan uttryck för värde?

\Subtask \code{for (i <- 0 to 4) yield öka(i)}

\Subtask \code{for (i <- 1 to 5) yield minska(i)}

\Subtask \code{(0 to 4).map(i => öka(i))}

\Subtask \code{(1 to 5).map(i => minska(i))}

\Subtask \code{(0 to 4).map(öka)}

\Subtask \code{(1 to 5).map(minska)}

\Subtask \code{Vector(12, 3, 41, -8).map(öka)}

\Subtask \code{Vector(12, 3, 41, -8).map(öka).map(minska).map(minska)}



\Task En funktion som inte returnerar ngt intressant värde, men som anropas för det den \emph{gör} kallas \textbf{procedur}. Definiera följande procedur i REPL: \\ 
\code{def tUvirks(msg: String) = println(msg.reverse)} \\
Vad skriver nedan satser ut? Förklara vad som händer.

\Subtask \code{println("sallad".reverse)}

\Subtask \code{tUvirks("sallad")}

\Subtask \code{val x = tUvirks("sallad"); println(x)}

\Subtask \code{def enhetsvärdet = (); println(enhetsvärdet)}

\Subtask \code{def bortkastad: Unit = 1 + 1; println(bortkastad)}

\Subtask \code|def bortkastad2 = {val x = 1 + 1}; println(bortkastad2)|

\Subtask\Pen Varför är det bra att explicit ange \code{Unit} som returtyp för procedurer?



\Task \emph{Funktion som värde.} Funktioner är äkta värden i Scala.

\Subtask \label{subtask:funcval} Förklara vad som händer nedan. Notera understrecket på rad 4:

\begin{REPL}[numbers=left, numberstyle=\color{black}\ttfamily\scriptsize\selectfont]
scala> def inc(x: Int): Int = x + 1
scala> inc(42)
scala> Vector(12, 3, 41, -8).map(inc)
scala> val f = inc _
scala> Vector(12, 3, 41, -8).map(f)
\end{REPL}

\Subtask Vad händer om du bara skriver \code{val f = inc} utan understreck?

\Subtask På liknande sätt som i uppgift \ref{subtask:funcval}: definiera en funktion \code{dec} som i stället \emph{minskar} med 1. Deklarera ett funktionsvärde \code{g} som tilldelas funktionen \code{dec} och kör sedan \code{g} på varje element i \code{Vector(12, 3, 41, -8)} med metoden \code{map}.

\Subtask\Pen Vad har variablerna \code{f} och \code{g} ovan för typ?

\Subtask Förklara vad som händer nedan. Vad får \code{d} och \code{h} för värde?

\begin{REPL}
scala> def räkna(x: Int, f: Int => Int) = f(x,y)
scala> def dubbla(x: Int, y: Int) = 2 * x
scala> def halva(x: Int, y: Int) = x / 2
scala> val d = räkna(42, dubbla)
scala> val h = räkna(42, halva)
\end{REPL}

\Task \emph{Objekt som moduler.} Ett objekt kan samla användbara funktioner och variabler. Ett objekt som kapslar in användbara koddelar kallas en \textbf{modul}. Moduler kan i sin tur innehålla moduler. Man kommer åt innehållet i en modul med punktnotation. Deklarera modulen nedan i Scala REPL eller i Kojo. 

\begin{Code}
object stringstat {
  object stringfun {
    def sentences(s: String): Array[String] = s.split('.')
    def words(s: String): Array[String] = s.split(' ')
    def countWords(s: String): Int = words(s).size
    def countSentences(s: String): Int = sentences(s).size
  }
  
  object statistics {
    var history = ""
    def printFreq(s: String): Unit = {
      println("\n---- Frekvenser ----")
      println("Antal tecken:   " + s.size)
      println("Antal ord:      " + stringfun.countWords(s))
      println("Antal meningar: " + stringfun.countSentences(s))
      history = history + " " + s
    }
    def printTotal: Unit = printFreq(history)
  }
}
  
object stringstatTest {
  import stringstat._
  def apply(n: Int = 42): Unit = {
    val s1 = "Fem myror är fler än fyra elefanter. Ät gurka."
    val s2 = "Galaxer i mina braxer. Tomat är gott. Hejsan."
    statistics.printFreq(s1 * n)
    statistics.printFreq(s2 * n)
    statistics.printTotal
  }
}
\end{Code}

\Subtask Anropa \code{stringstatTest()} och förklara vad som händer. Vad skrivs ut?

\Subtask Vilket av objekten i modulen \code{stringstat} har tillstånd vilket är tillståndslöst? Vad består tillståndet av?


\Task \emph{Äkta funktioner.} En \textbf{äkta funktion} ger alltid samma resultat med samma argument. Vilka funktioner i objektet \code{findPurity} nedan är äkta funktioner? Anropa de funktioner som inte är äkta i REPL och visa att de kan ge olika resultat för samma argument.

\begin{Code}
object findPurity {
  var x = 0
  val y = x
  def inc(i: Int) = i + 1
  def oink(i: Int) = {x = x + i; "Pig says oink " + x}
  def addX(i: Int): Int = x + i
  def addY(i: Int): Int = y + i
  def isPalindrome(s: String): Boolean = s == s.reverse
  def rnd(min: Int, max: Int) = math.random * max + min
}
\end{Code}




\Task Föränderlighet och oföränderlighet.

\Subtask Innan du kör nedan kod: Försök lista ut vad som kommer skrivas ut. Rita minnessituationen efter varje tilldelning.

\begin{Code}
println("\n--- Example 1: mutable value assigmnent")
var x1 = 42
var y1 = x1
x1 = x1 + 42
println(x1)
println(y1)
\end{Code}

\Subtask Innan du kör nedan kod: Försök lista ut vad som kommer skrivas ut. Rita minnessituationen efter varje tilldelning.

\begin{Code}
println("\n--- Example 2: mutable object reference assignment")
class MutableInt(private var i: Int) {
  def +(a: Int): MutableInt = { i = i + a; this }
  override def toString = i.toString
}
var x2 = new MutableInt(42)
var y2 = x2
x2 = x2 + 42
println(x2)
println(y2)
\end{Code}

\Subtask Innan du kör nedan kod: Försök lista ut vad som kommer skrivas ut. Rita minnessituationen efter varje tilldelning.

\begin{Code}
println("\n--- Example 3: immutable object reference assignment")
class ImmutableInt(val i: Int) {
  def +(a: Int): ImmutableInt = new ImmutableInt(i + a) 
  override def toString = i.toString
}
var x3 = new ImmutableInt(42)
var y3 = x3
x3 = x3 + 42
println(x3)
println(y3)
\end{Code}

\Subtask\Pen Vad finns det för fördelar med oföränderliga objekt?

\ExtraTasks %%%%%%%%%%%%%%%%%%%

\Task 

\AdvancedTasks %%%%%%%%%%%%%%%%%

\Task 