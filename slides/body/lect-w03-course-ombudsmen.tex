%!TEX encoding = UTF-8 Unicode
%!TEX root = ../lect-w03.tex

\ifkompendium\else
\begin{SlideExtra}{Hämta beställda bokpaket!}
  \begin{itemize}
    \item Du som beställt \Emph{bokpaket} men \Alert{ännu inte} hämtat ut det: \\
    \item[] Gå till datavetenskaps \Emph{expedition} på andra våningen i E-huset trapphus A på expeditionstid eller mejla \url{birger.swahn@cs.lth.se} \\ för överenskommelse om annan tid. 
  \end{itemize}
\end{SlideExtra}


% \begin{SlideExtra}{Schema-ändring}
%   \begin{itemize}
%     \item Pga Corona så behöver vi mer plats och därför flyttas \Emph{kontrollskrivningen} till den \Alert{28} oktober kl 14-19.
%     \item Se \url{http://cs.lth.se/pgk/schema/}
%   \end{itemize}
% \end{SlideExtra}
  
\begin{SlideExtra}{Kursombud}
\begin{itemize}
%\item Glädjande nog är det många intresserade!
\item Om du är intresserad: fyll i enkät om kursombud i Canvas.
\item Kursombud träffar vid behov kursansvarig på rast mellan föreläsningar eller pratar på Discord.
\item Kursombud träffar kursansvarig och programledning efter kursen och diskuterar kursutvärderingen.
\item Det vore bra med minst 2 D:are och minst 2 C:are och gärna en W:are och en fristående.
%\item \url{https://www.dsek.se/sektionen/srd/kursombud/}
\item {\SlideFontSmall\url{https://www.dsek.se/sektionen/srd/verksamhet.php}}
%\item Vi lottar med lite lajvkodning inspirerat av:
\end{itemize}
% \begin{REPL}
% scala> val kursombud = Vector("Kim Finkodare", "Robin Schnellhacker")
% scala> scala.util.Random.shuffle(kursombud).take(1)
% \end{REPL}
\end{SlideExtra}
\fi
