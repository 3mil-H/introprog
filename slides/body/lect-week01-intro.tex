%!TEX root = ../lect-week01.tex

%%%%%%%%%%%%%%%%%%%%%%%%%%%%%%%%%%%%%%
\Subsection{Om programmering}

%%%

\begin{Slide}{Att skapa koden som styr världen}
\begin{multicols}{2}\footnotesize
I stort sett alla delar av samhället är beroende av programkod:
\begin{itemize}\scriptsize
\item kommunikation
\item transport
\item byggsektorn
\item statsförvaltning
\item finanssektorn
\item media \& underhållning
\item sjukvård
\item övervakning
\item integritet
\item upphovsrätt
\item miljö \& energi
\item sociala relationer
\item utbildning 
\item ...
\end{itemize}
\columnbreak %---------
Hur blir ditt framtida yrkesliv som systemutvecklare?
\begin{itemize}
\item  Redan nu är det en skriande brist på utvecklare och bristen blir bara värre och värre... \\
  \href{http://computersweden.idg.se/2.2683/1.634770/rekrytera-utvecklare}{CS 2015-08-17}
\item Störst brist är det på kvinnliga utvecklare: \\
\href{http://www.dn.se/ekonomi/it-branschen-hotas-av-brist-pa-kvinnor/}{DN 2015-04-02}
\item Global kompetensmarknad \\ 
  \href{http://computersweden.idg.se/2.2683/1.630901/det-finns-programmerare-och-sa-finns-det-programmerare}{CS 2015-06-14}\\
   \href{http://computersweden.idg.se/2.2683/1.634700/7-satt-att-bli-en-battre-programmerare}{CS 2015-08-15}
\end{itemize}
\end{multicols}
\end{Slide}


\ifkompendium\noindent
{\scriptsize
\url{http://computersweden.idg.se/2.2683/1.634770/rekrytera-utvecklare}\\
\url{http://www.dn.se/ekonomi/it-branschen-hotas-av-brist-pa-kvinnor}\\
\url{http://computersweden.idg.se/2.2683/1.630901/det-finns-programmerare-och-sa-finns-det-programmerare}
\url{http://computersweden.idg.se/2.2683/1.634700/7-satt-att-bli-en-battre-programmerare}
}
\fi

\begin{Slide}{Utveckling av mjukvara i praktiken}
\begin{itemize}
\item \Emph{Inte bara kodning:} kravbeslut, releaseplanering, design, test, versionshantering, kontinuerlig integration, driftsättning, återkoppling från dagens användare, ekonomi \& investering, gissa om morgondagens användare, ... 
\item \Emph{Teamwork:} Inte ensamma hjältar utan autonoma team i decentraliserade organisationer med innovationsuppdrag
\item \Emph{Snabbhet:} Att koda innebär att hela tiden uppfinna nya ''byggstenar'' som ökar organisationens förmåga att snabbt skapa värde med hjälp av mjukvara. Öppen källkod. Skapa kraftfulla API:er.
\item \Emph{Livslångt lärande:} Lär nytt och dela med dig hela tiden. Exempel på pedagogisk utmaning: hjälp andra förstå och använda ditt API $\implies$ \textit{Samarbetskultur}
\end{itemize}
\end{Slide}

\ifkompendium\else
\SlideImg{Programming unplugged: Två frivilliga?}{../img/unplugged}
\SlideImg{Editera och exekvera ett program}{../img/kojo}
\fi


%%%%%%%%%%%%%%%%%%%%%%%%%%%%%%%%%%%%%%
\ifkompendium\else
\Subsection{Meddelande från \href{http://lth.se/code}{Code@LTH}} 
\fi