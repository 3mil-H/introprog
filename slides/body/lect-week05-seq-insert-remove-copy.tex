%!TEX encoding = UTF-8 Unicode
%!TEX root = ../lect-week05.tex

%%%

\Subsection{Vad är en sekvensalgoritm?}

\begin{Slide}{Vad är en sekvensalgoritm?}
\begin{itemize} 
\item En algoritm är en stegvis beskrivning av hur man löser ett problem. 
\item En sekvensalgoritm är en algoritm där dataelement i sekvens utgör en viktig del av problembeskrivningen och/eller lösningen.   

\item Exempel: sortera en sekvens av personer efter deras ålder.

\item Två olika principer:
\begin{itemize} 
\item Skapa \Emph{ny sekvens} utan att förändra indatasekvensen
\item Ändra \Emph{på plats} \Eng{in place} i den \Alert{förändringsbara} indatasekvensen
\end{itemize}
\end{itemize}

\end{Slide}

\begin{Slide}{Skapa ny sekvenssamling eller ändra på plats?}
\begin{itemize}
\item Ofta är det \Emph{lättast att skapa ny samling} och kopiera över elementen medan man loopar.
\item Om man har mycket stora samlingar kan man behöva ändra på plats för att spara tid/minne.
\item Det är bra att själv kunna implementera sekvensalgortimer även om många av dem finns färdiga, för att bättre förstå vad som händer ''under huven'', och för att i enstaka fall kunna optimera om det verkligen behövs.
\item Vi illustrerar därför hur man kan implementera några sekvensalgoritmer med primitiva arrayer även om man sällan gör så i praktiken (i Scala).
\end{itemize}
\end{Slide}


\Subsection{SEQ-COPY}

\begin{Slide}{Algoritm: SEQ-COPY}
\Emph{Pseudokod} för algoritmen SEQ-COPY som kopierar en sekvens, här en Array med heltal:\\
\noindent\hrulefill
\begin{algorithm}[H]
 \SetKwInOut{Input}{Indata}\SetKwInOut{Output}{Resultat}
 \Input{Heltalsarray $xs$} 
 \Output{En ny heltalsarray som är en kopia av $xs$. \\ \vspace{1em}}
 $result \leftarrow$ en ny array med plats för $xs.length$ element\\
 $i \leftarrow 0$  \\
 \While{$i < xs.length$}{
  $result(i) \leftarrow xs(i)$\\
  $i \leftarrow i + 1$\\
 }
 \Return $result$
\end{algorithm}
\noindent\hrulefill
\end{Slide}

\ifkompendium\else

\begin{Slide}{Implementation av SEQ-COPY med \texttt{while}}
\lstinputlisting[numbers=left]{../compendium/examples/workspace/w05-seqalg/src/seqCopy.scala}
\end{Slide}

\begin{Slide}{Implementation av SEQ-COPY med \texttt{for}}
\lstinputlisting[numbers=left]{../compendium/examples/workspace/w05-seqalg/src/seqCopyFor.scala}
\end{Slide}

\begin{Slide}{Implementation av SEQ-COPY med \texttt{for-yield}}
\lstinputlisting[numbers=left]{../compendium/examples/workspace/w05-seqalg/src/seqCopyForYield.scala}
\end{Slide}

\Subsection{For-satser och arrayer i Java}

\begin{Slide}{For-satser och arrayer i Java}
En for-sats i Java har följande struktur:
\begin{Code}[language=Java, basicstyle=\fontsize{10}{12}\ttfamily\selectfont]
for (initialisering; slutvillkor; inkrementering) {
    sats1;
    sats2;
    ...
}
\end{Code}
En primitiv heltals-array deklareras så här i Java:
\begin{Code}[language=Java, basicstyle=\fontsize{10}{12}\ttfamily\selectfont]
int[] xs = new int[42];  // 42 st heltal, init 0:or
int[] ys = {1, 2, 3};    // init 3 st heltal  
\end{Code}
Exempel: fyll en array med 1:or 
\begin{Code}[language=Java, basicstyle=\fontsize{9}{11}\ttfamily\selectfont]
for (int i = 0; i < xs.length; i = i + 1){ // vanligare: i++
  xs[i] = 1;    // indexering sker med hakparenteser
}
\end{Code}

\end{Slide}

\begin{Slide}{Implementation av SEQ-COPY i Java med \texttt{for}-sats}
\begin{minipage}{0.55\textwidth}
\vspace{-0.5em}
\javainputlisting[numbers=left,numberstyle=,basicstyle=\fontsize{6.5}{8}\ttfamily\selectfont]{../compendium/examples/workspace/w05-seqalg/src/SeqCopyForJava.java}
\end{minipage}
\begin{minipage}{0.44\textwidth}\SlideFontTiny\vspace{-1.5em}
~~~Lite syntax och semantik för Java:
\begin{itemize}
\item En Java-klass med enbart statiska medlemmar motsvarar ett singelobjekt i Scala. 

\item Typen kommer \Alert{före} namnet.

\item Man \Alert{måste} skriva \code{return}.

\item Man \Alert{måste} ha semikolon efter varje sats.

\item Metodnamn \Alert{måste} följas av parenteser; om inga parametrar finns används \code{()}

\item En array i Java är inget vanligt objekt, men har ett ''attribut'' \code{length} som ger antal element.

\item \Emph{Övning}: skriv om med \code{while}-sats i stället; har samma syntax i Scala \& Java.

\end{itemize}
\end{minipage}

\end{Slide}


\Subsection{Exempel: PolygonWindow}

\begin{Slide}{Exempel: PolygonWindow}
\vspace{-0.6em}\scalainputlisting[numbers=left,numberstyle=,basicstyle=\fontsize{6.5}{8}\ttfamily\selectfont]{../compendium/examples/workspace/w05-seqalg/src/PolygonWindow.scala}
\pause
\vspace{1em}\scalainputlisting[numbers=left,numberstyle=,basicstyle=\fontsize{6.5}{8}\ttfamily\selectfont]{../compendium/examples/workspace/w05-seqalg/src/polygonTest1.scala}
\end{Slide}

\begin{Slide}{Typ-alias för att abstrahera typnamn}\SlideFontSmall
Med hjälp av nyckelordet \code{type} kan man deklarera ett \Emph{typ-alias} för att ge ett \Alert{alternativt} namn till en viss typ. Exempel:
\begin{REPL}
scala> type Pt = (Int, Int)

scala> def distToOrigo(pt: Pt): Int = math.hypot(pt._1, pt._2)

scala> type Pts = Vector[Pt]

scala> def firstPt(pts: Pts): Pt = pts.head

scala> val xs: Pts = Vector((1,1),(2,2),(3,3))

scala> firstPt(xs)
res0: Pt = (1,1)
\end{REPL}

Detta är bra om:
\begin{itemize}
\item man har en lång och krånglig typ och vill använda ett kortare namn,

\item om man vill abstrahera en typ och öppna för möjligheten att byta implementation senare (t.ex. till en egen klass), medan man ändå kan fortsätta att använda befintligt namn.
\end{itemize}
\end{Slide}


\Subsection{SEQ-INSERT/REMOVE-COPY}

\begin{Slide}{Exempel: SEQ-INSERT/REMOVE-COPY}
Nu ska vi ''uppfinna hjulet'' och som träning implementera \Emph{insättning} och \Emph{borttagning} till en \Alert{ny} sekvens: 
\begin{Code}
object pointSeqUtils {
  type Pt = (Int, Int)  // a type alias to make the code more concise

  def primitiveInsertCopy(pts: Array[Pt], pos: Int, pt: Pt): Array[Pt] = ???

  def primitiveRemoveCopy(pts: Array[Pt], pos: Int): Array[Pt] = ???
}
\end{Code}
\end{Slide}




\begin{Slide}{Pseudo-kod för SEQ-INSERT-COPY}\SlideFontSmall
\begin{algorithm}[H]
 \SetKwInOut{Input}{Indata}\SetKwInOut{Output}{Resultat}
 
 \Input{\texttt{pts: Array[Pt],}\\\texttt{pt: Pt,}\\\texttt{pos: Int}} ~\\
 \Output{En ny sekvens av typen \texttt{Array[Pt]} som är en kopia av $pts$ men där $pt$ är infogat på plats $pos$} 
 
 \noindent\hrulefill\\
 $result \leftarrow$ en ny \texttt{Array[Pt]} med plats för $pts.length + 1$ element \\
 \For{$i \leftarrow 0$ \KwTo $pos - 1$}{
  $result(i) \leftarrow pts(i)$
 }
 $result(pos) \leftarrow pt$ \\
 \For{$i \leftarrow pos + 1$ \KwTo $xs.length$}{
  $result(i) \leftarrow xs(i - 1)$
 }
 
 \Return $result$
 
  \noindent\hrulefill\\ 
\end{algorithm}
\pause\vspace{0.5em}\Emph{Övning}: Skriv pseudo-kod för SEQ-REMOVE-COPY
\end{Slide}

\begin{Slide}{Insättning/borttagning i kopia av primitiv Array}
Man gör mycket lätt \Alert{fel} på gränser/specialfall (t.ex. tom sekv.):
\vspace{-0.0em}\scalainputlisting[numbers=left,numberstyle=,basicstyle=\fontsize{6}{7.2}\ttfamily\selectfont]{../compendium/examples/workspace/w05-seqalg/src/pointSeqUtils.scala}
\end{Slide}

\begin{Slide}{Exempel: Test av SEQ-INSERT/REMOVE-COPY}
\vspace{-0.6em}\scalainputlisting[numbers=left,numberstyle=,basicstyle=\fontsize{6.5}{8}\ttfamily\selectfont]{../compendium/examples/workspace/w05-seqalg/src/polygonTest2.scala}
\end{Slide}

\begin{Slide}{Exempel: Göra insättning med take/drop}\SlideFontSmall
Om du inte vill ''uppfinna hjulet'' och inte använda \code{patch} kan du göra så här: \\Använd \code{take} och \code{drop} tillsammans med \code{:+} och \code{++} \\Du kan också göra insättningen generiskt användbar för alla sekvenser:
\begin{REPLnonum}
scala> val xs = Vector(1,2,3)
xs: scala.collection.immutable.Vector[Int] = 
  Vector(1, 2, 3)

scala> val ys = (xs.take(2) :+ 42) ++ xs.drop(2)
ys: scala.collection.immutable.Vector[Int] = 
  Vector(1, 2, 42, 3)
  
scala> def insertCopy[T](xs: Seq[T], elem: T, pos: Int) = 
        (xs.take(pos) :+ elem) ++ xs.drop(pos)

scala> insertCopy(xs, 42, 2)
res0: Seq[Int] = Vector(1, 2, 42, 3)
  
\end{REPLnonum}
\Emph{Övning}: Implementera \code{insertCopy[T]} med \code{patch} istället.
\end{Slide}


\fi







