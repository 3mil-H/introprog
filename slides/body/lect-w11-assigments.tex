%!TEX encoding = UTF-8 Unicode
%!TEX root = ../lect-w11.tex

%%%


\Subsection{Veckans uppgifter: \texttt{scalajava} och \texttt{javatext}}
\begin{Slide}{Veckans uppgifter: \texttt{scalajava} och \texttt{javatext}}\SlideFontSmall
%\Emph{Labbförberedelse:}
\begin{itemize}
\item Övning \texttt{scalajava}:
\begin{itemize}\SlideFontTiny
\item Översätta Java till Scala och från Scala till Java
\item Undersöka autoboxning \Eng{autoboxing}
\item Använda \code{import scala.jdk.CollectionConverters._}
\item[] Utfasad \Eng{deprecated} sedan Scala 2.13.0: \code{import scala.collection.JavaConverters._}
\end{itemize}
\item Laboration \code{javatext}:
\begin{itemize}\SlideFontTiny
  \item Gör klart ett (lagom) intressant/roligt textspel för terminalen huvudsakligen i Java men vissa delar i Scala, enligt krav, tips och inspiration i labb-instruktionerna.
  \item Labben är \Alert{individuell} men du ska \Emph{spela en tidig version av någon annans spel} och ge återkoppling på kodens \Alert{läsbarhet} och vice versa.
  \item Om du har flera ''kompletteras'' efter dig eller tycker det är alldeles övermäktigt att få ihop de obligatoriska kraven: diskutera din situation med handledare på resurstid; kontakta kursansvarig vid behov. 
\end{itemize}
\item Tips: sbt kan blanda både \code{.scala} och \code{.java} i samma projekt.
\end{itemize}
\end{Slide}
