%!TEX encoding = UTF-8 Unicode
%!TEX root = ../lect-week04.tex

\ifkompendium\else

\Subsection{Vad är en datastruktur?}
\begin{Slide}{Vad är en datastruktur?}\SlideFontSmall
\begin{itemize}
\item En \href{https://sv.wikipedia.org/wiki/Datastruktur}{datastruktur} är en struktur för organisering av data som...
\begin{itemize}\SlideFontTiny
\item kan innehålla \Alert{många} element,
\item kan refereras till med \Alert{ett} enda namn, och
\item ger möjlighet att komma åt de enskilda elementen.
\end{itemize}

\item En \Emph{samling} \Eng{collection} är en datastruktur som kan innehålla många element av \Alert{samma typ}.

\item Exempel på \Emph{färdiga samlingar} i Scalas standardbibliotek där elementen är organiserade på olika vis så att samlingen får olika egenskaper som passar \Alert{olika användningsområden}: 
\begin{itemize}\SlideFontTiny
\item scala.collection.immutable.Vector
\item Array
\item List
\item Set
\item Map
\end{itemize}

\end{itemize}

\end{Slide} 

\fi


\begin{Slide}{Olika sätt att skapa datastrukturer}
\begin{itemize}
\item Tupler
  \begin{itemize}
  \item samla $n$ st datavärden i element \Emph{\code{_1}}, \Emph{\code{_2}}, ...  \code{_}$n$
  \item elementen kan vara av \Alert{olika} typ
  \end{itemize}
\item Klasser   
  \begin{itemize}
  \item samlar data i \Emph{attribut} med (väl valda!) namn
  \item attributen kan vara av \Alert{olika} typ
  \item definierar även \Emph{metoder} som använder attributen (operationer på data)
  \end{itemize}

\item Färdiga samlingar 
  \begin{itemize}
  \item speciella klasser som samlar data i element av \Alert{samma} typ
  \item finns ofta \emph{många} färdiga \Emph{bra-att-ha-metoder} 
  \item Exempel: \code{scala.collection.immutable.Vector}
  \end{itemize}
  
\item Egenutvecklade samlingar
  \begin{itemize}
  \item $\rightarrow$ Fortsättningskurs
  \end{itemize}
  
\end{itemize}
\end{Slide}

%%%


\ifkompendium\else
\begin{Slide}{Denna vecka: Förstå datastrukturer}
\begin{itemize}
\item Läs teori
\item Gör övning \code{data}
\item Gör lab \code{???}
\end{itemize}
\end{Slide}
\fi