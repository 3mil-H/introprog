%!TEX encoding = UTF-8 Unicode
%!TEX root = ../lect-w07.tex


\Subsection{Kontrollskrivning}

\begin{Slide}{Kontrollskrivning}
Ta med \Alert{legitimation} och \Emph{snabbreferens}, blyertspenna, suddgummi, \Alert{röd} penna till rättningen, förtäring. Ingen mobil. Jackor och väskor vid väggen.
\begin{itemize}
  \item \textbf{Diagnostisk}: vad har du lärt dig hittills?
  \item \textbf{Kamraträttad}: träna på att läsa och bedöma kod
  \item \textbf{Obligatorisk}: sjukdom \Alert{måste} meddelas i förväg
  \item Kan ge \Emph{samarbetsbonus} \Alert{om man har visat samarbetskontrakt}
  %\SlideFontSmall
  \begin{itemize}
    \item[] Max 5p i samarbetsbonus på första ordinarie tentamen, som adderas i slutet av kursen till din tentamenspoäng (max 100) och baseras på \Emph{medelvärdet} av resultaten på kontrollskrivningen i din samarbetsgrupp (se kap ''Anvisningar'' i kompendiet).
  \end{itemize}
\end{itemize}
\end{Slide}

\begin{Slide}{Kontrollskriving -- upplägg}\SlideFontSmall
\begin{itemize}
  \item \textbf{Moment 1}: ca 2 h 15 min: Du löser uppgifterna individuellt
  \item \textbf{Moment 2}: ca 1 h 15 min: Parvis kamratbedömning
  \item \textbf{Moment 3}: ca 30 min: Inspektera din egen skrivning

\item Läs \Alert{noga} igenom instruktionerna på tidigare kontrollskrivning här: 
\url{http://fileadmin.cs.lth.se/pgk/kontroll2018okt30.pdf}
%\url{http://fileadmin.cs.lth.se/pgk/kontroll2017okt24.pdf}

\end{itemize}


\end{Slide}

\begin{Slide}{Plugga själv OCH i din samarbetsgrupp}
\begin{itemize}
  \item Träffas och prata i din samarbetsgrupp om hur ni bäst pluggar \Emph{individuellt} och \Alert{tillsammans} inför kontrollskrivningen för att maximera lärandet i gruppen!

  \item Repetera teorin i läsperiod 1.
  
  \item Repetera lösningar till övningarna och labbarna.

  \item Träna på att koda med papper och penna!



\item Använd tidigare års kontrollskrivningar: \\ \url{http://cs.lth.se/pgk/examination/}

\pause

\begin{itemize}\SlideFontTiny\vspace{1em}
    \item Observera: 2016 ingick \Emph{arv} i lp1 men sedan 2017 kommer detaljerna om \code{extends} och \code{trait} och \code{abstract} i lp2.
\item Snabbkurs om arv:
\begin{itemize}\SlideFontTiny
\item Med \code{trait Grönsak} skapas en typ med namnet \code{Grönsak} som kan användas som \Emph{bastyp} i en hierarki av typer.
\item Med \code{extends Grönsak} anges att en typ \Alert{är en} \code{Grönsak}:
\end{itemize}

\end{itemize}

\end{itemize}
\begin{Code}
  trait Grönsak { var vikt: Int }   // alla grönsaker har en vikt
  class Gurka(var vikt: Int) extends Grönsak  // Gurka är en Grönsak
  class Tomat(var vikt: Int) extends Grönsak  // Tomat är en Grönsak
\end{Code}
\end{Slide}


% \begin{Slide}{Beställning av tryck av kompendium lp 2}\SlideFontSmall

% \begin{itemize}
% \item Nu är det dags att beställa tryck av kompendiet del 2 som innehåller
% övningar och laborationer inför kommande läsperiod.

% \item Beställ här: \url{http://cs.lth.se/pgk/tryck2}

% \item Priset är till självkostnad och beror på hur många som beställer.
% \item Priset blir max 270kr om färre än 100st beställer
% och ca 165kr om minst 100st beställer.

% \item Svara *snarast* dock senast 19 Oktober kl 0900.

% \item \Alert{Det är himla bra att ha kompendiet på papper, bredvid skärmen speciellt när du jobbar med en IDE med massor av fönster!!}

% \end{itemize}
% \end{Slide}


\begin{Slide}{Grumligt-lådan}
\begin{itemize}
\item Jag skickar runt \Emph{Grumligt}-lådan.
\item Skriv lappar, \Alert{en lapp per begrepp}, som du tycker är \Emph{''grumligt''} och  önskar förstå bättre.
\item Skicka vidare lådan så fort du är klar.
\item Sista person i salen lämnar tillbaka lådan till mig på rasten.
\item Jag kommer att försöka reda ut några högfrekventa grumligheter på kommande föreläsning.
\end{itemize}
\end{Slide}
