%!TEX encoding = UTF-8 Unicode
%!TEX root = ../lect-w07.tex


\Subsection{Kontrollskrivning}

\begin{Slide}{Kontrollskrivning}
Ta med \Alert{legitimation} och snabbreferens, blyertspenna, suddgummi, röd penna, förtäring.
\begin{itemize}
  \item Diagnostisk: vad har du lärt dig hittills?
  \item Kamraträttad: träna på att läsa och bedöma kod
  \item Obligatorisk: sjukdom \Alert{måste} meddelas i förväg
  \item Kan ge samarbetsbonus: kontrollskrivningsresultatet kan ge dig max 5p i samarbetsbonus, som adderas i slutet av kursen till din tentamenspoäng (max 100) och baseras på medelvärdet av resultaten i din samarbetsgrupp (se anvisningar i kompendiet)
\end{itemize}
\end{Slide}

\begin{Slide}{Plugga i din samarbetsgrupp}
\begin{itemize}
  \item Träffas och prata om hur ni kan plugga ihop
  \item Förra årets kontrollskrivning: \\ \url{http://cs.lth.se/pgk/examination/}

\end{itemize}
\end{Slide}
