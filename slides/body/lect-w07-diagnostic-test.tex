%!TEX encoding = UTF-8 Unicode
%!TEX root = ../lect-w07.tex


\Subsection{Kontrollskrivning}

\begin{Slide}{Kontrollskrivning}
Ta med \Alert{legitimation} och snabbreferens, blyertspenna, suddgummi, röd penna, förtäring.
\begin{itemize}
  \item Diagnostisk: vad har du lärt dig hittills?
  \item Kamraträttad: träna på att läsa och bedöma kod
  \item Obligatorisk: sjukdom \Alert{måste} meddelas i förväg
  \item Kan ge samarbetsbonus: kontrollskrivningsresultatet kan ge dig max 5p i samarbetsbonus, som adderas i slutet av kursen till din tentamenspoäng (max 100) och baseras på medelvärdet av resultaten i din samarbetsgrupp (se anvisningar i kompendiet)
\end{itemize}
\end{Slide}

\begin{Slide}{Plugga i din samarbetsgrupp}
\begin{itemize}
\item Träffas och prata \Alert{på rasten} eller enl. ök. om hur ni kan plugga ihop för att maximera lärandet i gruppen!
\pause
\item Förra årets kontrollskrivning: \\ \url{http://cs.lth.se/pgk/examination/}
\item Förra året ingick \Emph{arv} och nyckelorden \code{extends} och \code{trait} men i år väntar vi med det till lp2.
\pause
\item Snabbkurs: (så du förstår lite om uppgifterna med arv)
\begin{itemize}\SlideFontSmall
\item Med \code{trait Grönsak} skapas en slags klass med namnet \code{Grönsak} som kan användas som bastyp i en hierarki av typer.
\item Med \code{extends Grönsak} anges att en klass \Alert{är en} \code{Grönsak}:
\begin{Code}
trait Grönsak { var vikt: Int }   // alla grönsaker har en vikt
class Gurka(var vikt: Int) extends Grönsak
class Tomat(var vikt: Int) extends Grönsak
\end{Code}
\end{itemize}


\end{itemize}
\end{Slide}
