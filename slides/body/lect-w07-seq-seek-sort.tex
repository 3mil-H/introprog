%!TEX encoding = UTF-8 Unicode
%!TEX root = ../lect-w07.tex

%%%

\Subsection{Sökning och sortering}


\begin{Slide}{Sökning}\SlideFontSmall

\begin{itemize}
\item \Emph{Sökning} återkommer i många skepnader: \\ i en datastruktur, vilken det än må vara, vill man ofta kunna \\ \Emph{hitta ett element med en viss egenskap}.

\pause
Några färdiga linjärsökningar i Scalas standardbibliotek:

\begin{REPL}
scala> Vector("gurka","tomat","broccoli").indexOf("tomat")
res0: Int = 1

scala> Vector("gurka","tomat","broccoli").indexWhere(_.contains("o"))
res1: Int = 1

scala> Vector("gurka","tomat","broccoli").find(_.contains("o"))
res2: Option[String] = Some(tomat)
\end{REPL}

\pause
\item Sökning efter ett visst index i en sekvens:

\begin{itemize}\SlideFontTiny
\item Indata: en sekvens och ett \Emph{sökkriterium}
\item Utdata: index för första eftersökta element, annars -1
\end{itemize}

\pause
\item Två typiska varianter av sökning i en sekvens:
\begin{itemize}\SlideFontTiny
\item Linjärsökning: börja från början och sök tills ett eftersökt element är funnet
\item Binärsökning: antag sorterad sekvensen; börja i mitten, välj rätt halva ... 
\end{itemize}
\end{itemize}
\end{Slide}

%\Subsection{Linjärsökning}

% \begin{Slide}{Linjärsökning: hitta index för elementet 42}
% Skriv pseudokod för:\\ \code{def indexOf42(xs: Vector[Int]): Int = ???}
% \pause
% \begin{Code}
% def indexOf42(xs: Vector[Int]): Int = {
%   var i = /* index för första elementet */
%   var found = false
%   while (!found && /* index inom indexgräns */) {
%     if (/* element på plats i är 42 */) found = true
%     else i = /* nästa index */
%   }
%   if (/* hittat */) i else -1
% }
% \end{Code}
% \end{Slide}
%
% \begin{Slide}{Linjärsökning: hitta index för elementet 42}
% Implementera:\\ \code{def indexOf42(xs: Vector[Int]): Int = ???}
% \pause
% \begin{Code}
% def indexOf42(xs: Vector[Int]): Int = {
%   var i = 0
%   var found = false
%   while (!found && i < xs.length) {
%     if (xs(i) == 42) found = true
%     else i += 1
%   }
%   if (found) i else -1
% }
% \end{Code}
% \end{Slide}

\begin{Slide}{Linjärsökning: hitta index för elementet x}
Implementera \code{indexOf}:
\begin{Code}
def indexOf(xs: Vector[Int], x: Int): Int = ???
\end{Code}
Utdata: index \code{i} där \code{xs(i) == x}\\Om värde saknas. returnera \code{-1}
\pause
\begin{Code}
def indexOf(xs: Vector[Int], x: Int): Int = 
  var i = 0
  var found = false
  while (!found && i < xs.length) {
    if (xs(i) == x) found = true
    else i += 1
  }
  if (found) i else -1
\end{Code}
(Är du nyfiken på binärsökning, se kapitel 12: Valfri fördjupning.)
\end{Slide}

% \begin{Slide}{Linjärsökning: hitta index för elementet p(x)}\SlideFontSmall
% Implementera:\\ \code{def indexWhere(xs: Vector[Int], p: Int => Boolean): Int = ???}
% \pause
% \begin{Code}
% def indexWhere(xs: Vector[Int], p: Int => Boolean): Int = {
%   var i = 0
%   var found = false
%   while (!found && i < xs.length) {
%     if (p(xs(i))) found = true
%     else i += 1
%   }
%   if (found) i else -1
% }
% \end{Code}
% \end{Slide}
%
% \begin{Slide}{Linjärsökning: generalisera till godtycklig typ}\SlideFontSmall
% Implementera:\\ \code{def indexWhere[T](xs: Vector[T], p: T => Boolean): Int = ???}
% \pause
% \begin{Code}
% def indexWhere[T](xs: Vector[T], p: T => Boolean): Int = {
%   var i = 0
%   var found = false
%   while (!found && i < xs.length) {
%     if (p(xs(i))) found = true
%     else i += 1
%   }
%   if (found) i else -1
% }
% \end{Code}
% \end{Slide}
%
% \begin{Slide}{Linjärsökning: generalisera till godtycklig typ}\SlideFontSmall
% Typinferensen fungerar bättre om stegad funktion:\\
% \code{def indexWhere[T](xs: Vector[T])(p: T => Boolean): Int}
% \begin{Code}
% def indexWhere[T](xs: Vector[T])(p: T => Boolean): Int = {
%   var i = 0
%   var found = false
%   while (!found && i < xs.length) {
%     if (p(xs(i))) found = true
%     else i += 1
%   }
%   if (found) i else -1
% }
% \end{Code}
% \end{Slide}
%
%
% \begin{Slide}{Linjärsökning: generalisera till godtycklig sekvens}\SlideFontSmall
% Implementera:\\ \code{def indexWhere[T](xs: Seq[T])(p: T => Boolean): Int = ???}
% \pause
% \begin{Code}
% def indexWhere[T](xs: Seq[T])(p: T => Boolean): Int = {
%   var i = 0
%   var found = false
%   while (!found && i < xs.length) {
%     if (p(xs(i))) found = true
%     else i += 1
%   }
%   if (found) i else -1
% }
% \end{Code}
% \end{Slide}
%
% \begin{Slide}{Linjärsökning: generalisera till godtycklig sekvens}\SlideFontSmall
% Implementera:\\ \code{def find[T](xs: Seq[T])(p: T => Boolean): Option[T] = ???}
% \pause
% \begin{Code}
% def find[T](xs: Seq[T])(p: T => Boolean): Option[T] = {
%   var i = 0
%   var found = false
%   while (!found && i < xs.length) {
%     if (p(xs(i))) found = true
%     else i += 1
%   }
%   if (found) Some(xs(i)) else None
% }
% \end{Code}
% \end{Slide}

\begin{Slide}{Sortering}
\Emph{Problem}: Vi har en osorterad sekvens med heltal. Vi vill ordna denna osorterade sekvens i en sorterad sekvens från minst till störst.
\pause

\vspace{1em}\noindent
En \emph{generalisering} av problement: \\ \vspace{1em}

\noindent Vi har många element av godtycklig typ och en \Emph{ordningsrelation} som säger vad vi menar med att ett element är \emph{mindre än} eller \emph{större än eller lika med} ett annat element. \\ 

\vspace{1em}\noindent Vi vill lösa problemet att ordna elementen i sekvens så att för varje element på plats $i$ så är efterföljande element på plats $i + 1$ större eller lika med elementet på plats $i$.

% \end{Slide}

% \begin{Slide}{Insättningssortering \& Urvalssortering}
\begin{itemize}
\item Insättningssortering \Emph{lösningsidé}: Ta ett element i taget från den osorterade listan och \Alert{sätt in} det på \Alert{rätt plats} i den sorterade listan och upprepa till det inte finns fler osorterade element.
% \pause
% \item Urvalsssortering \Emph{lösningsidé}: \Alert{Välj ut} det minsta kvarvarande elementet i den osorterade listan och placera det \Alert{sist} i den sorterade listan och upprepa till det inte finns fler osorterade element.
\end{itemize}
\end{Slide}

\begin{Slide}{Det finns många olika sorteringsalgoritmer}
\begin{itemize}
\item Visualisering av 15 olika sorteringsalgoritmer på 6 min:\\{\SlideFontSmall\url{https://www.youtube.com/watch?v=kPRA0W1kECg}}
\item Olika sorteringsalgoritmer har olika tids- \& minneskomplexitet: i bästa fall, i värsta fall, i medeltal, för nästan sorterad, etc.
\\{\SlideFontSmall\url{https://en.wikipedia.org/wiki/Sorting_algorithm}}
\item Olika sorteringsalgoritmer lämpar sig olika väl för parallellisering på många kärnor.
\end{itemize}
\end{Slide}

\begin{Slide}{Bogo sort}
\begin{Code}
def bogoSort(xs: Vector[Int]) = 
  var result = xs
  while result != result.sorted do
    result = scala.util.Random.shuffle(result)
  result
\end{Code}
När blir denna färdig? \pause \\~\\
Antal jämförelser i medeltal vid $n$ element: $ n \cdot n!$ \\~\\
\url{https://en.wikipedia.org/wiki/Bogosort}

\end{Slide}


\begin{Slide}{Sortera till ny vektor med insättningssortering: pseudo-kod}

{\SlideFontSmall Det är nog lättare att förstå \Emph{insertion sort} om man sorterar till en ny vektor. \\ Vi ska sedan se hur man sorterar ''på plats'' \Eng{in place} i en  array.\\} \vspace{0.5em}

\noindent \Emph{Indata}: en osorterad vektor med heltal \\
\Emph{Utdata}: en ny, sorterad vektor med heltal
\begin{Code}
def insertionSort(xs: Vector[Int]): Vector[Int] = 
  val sorted = /* tom ArrayBuffer */
  for /* alla element i xs */ do
     /* linjärsök rätt position i sorted */
     /* sätt in element på rätt plats i sorted */
  end for
  sorted.toVector
\end{Code}
\end{Slide}


\begin{Slide}{Sortera till ny vektor med insättningssortering: implementation} % i Scala}
\begin{Code}
def insertionSort(xs: Vector[Int]): Vector[Int] = 
  val sorted = scala.collection.mutable.ArrayBuffer.empty[Int]
  for elem <- xs do
     // linjärsök rätt position i sorted:
     var pos = 0
     while pos < sorted.length && sorted(pos) < elem do
       pos += 1
     end while
     // sätt in element på rätt plats i sorted:
     sorted.insert(pos, elem)
  end for
  sorted.toVector
end insertionSort
\end{Code}
\end{Slide}

\begin{Slide}{Sorter till ny samling med godtyckligt ordningspredikat}
\begin{CodeSmall}
def sortWith(xs: Vector[Int])(lt: (Int, Int) => Boolean ): Vector[Int] = 
  val sorted = scala.collection.mutable.ArrayBuffer.empty[Int]
  for elem <- xs do  // insertion sort using lt as "less than"
     var pos = 0
     while pos < sorted.length && lt(sorted(pos), elem) do
       pos += 1
     end while
     sorted.insert(pos, elem)
  end for
  sorted.toVector
end sortWith
\end{CodeSmall}
\pause
\begin{REPL}
scala> val xs = Vector(1,2,1,2,12,42,1)

scala> sortWith(xs)(_ < _)
val res0: Vector[Int] = Vector(1, 1, 1, 2, 2, 12, 42)

scala> sortWith(xs)(_ > _)
val res1: Vector[Int] = Vector(42, 12, 2, 2, 1, 1, 1)
\end{REPL}
\end{Slide}


\begin{Slide}{Insättningssortering på plats -- pseudo-kod}
\Emph{Indata:} en array med heltal\\
\Emph{Utdata:} samma array, men nu sorterad\\
\begin{Code}
def insertionSortInPlace(xs: Array[Int]): Unit = 
  for i <- 1 until xs.length do  //från ANDRA till sista
    var j = i
    while j > 0 && xs(j - 1) > xs(j) do
      /* byt plats på xs(j) och xs(j - 1) */
      j -= 1;  // stega bakåt
\end{Code}
\pause
Se animering här: \href{https://sv.wikipedia.org/wiki/Ins\%C3\%A4ttningssortering}{Insättningssortering på wikipedia}\\
Gå igenom alla specialfall och kolla så att detta fungerar!
\end{Slide}

\begin{Slide}{Insättningssortering på plats -- implementation} %, Scala}
\begin{Code}
def insertionSortInPlaceSwap(xs: Array[Int]): Unit = 
  def swap(i: Int, j: Int): Unit = 
    val temp = xs(i)
    xs(i) = xs(j)
    xs(j) = temp
  end swap 

  for i <- 1 until xs.length do  //från ANDRA till sista
    var j = i
    while j > 0 && xs(j - 1) > xs(j) do
      swap(j, j - 1)
      j -= 1;  // stega bakåt
    end while
  end for
end insertionSortInPlaceSwap
\end{Code}
\end{Slide}

\begin{Slide}{Checklista vid granskning av sekvensalgoritmer}
\begin{itemize}
\item Fungerar algoritmen för en tom sekvens?
\item Fungerar algoritmen för en sekvens med endast ett element?
\item Fungerar algoritmen för för osorterad sekvens med (minst) två element?
\item Vad händer om sekvensen redan är sorterad?
\end{itemize}
\end{Slide}