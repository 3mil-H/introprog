%!TEX encoding = UTF-8 Unicode
%!TEX root = ../lect-w05.tex

%%%

\begin{Slide}{Case-klasser ger innehållslikhet}
Förutom annat ''godis på köpet'' får du med \code{case class} även detta:
\begin{itemize}
\item Metoden \code{==} ger \Emph{innehållslikhet} (och inte referenslikhet).
\end{itemize}
\end{Slide}



\begin{Slide}{Likhet och case-klasser}
Metoden \code{equals} är i case-klasser automatiskt överskuggad så att metoden \code{==} ger test av strukturlikhet.
\begin{REPL}
scala> case class Gurka(vikt: Int)

scala> val g1 = Gurka(42)
g1: Gurka = Gurka(42)

scala> val g2 = Gurka(42)
g2: Gurka = Gurka(42)

scala> g1 eq g2          // olika instanser
res0: Boolean = false

scala> g1 == g2          // samma innehåll!
res1: Boolean = true
\end{REPL}
\end{Slide}



\begin{Slide}{Sammanfattning case-klass-godis}
Kom-ihåg-lista med ''godis'' i \code{case}-klasser så här långt:
\begin{enumerate}
\item klassparametrar blir \code{val}-attribut
\item najs toString
\item slipper skriva \code{new}
\item == ger innehållslikhet \Eng{structural equality}
\pause~\\...
\end{enumerate}

\vspace{1em}Men vi har inte sett allt godis än: \\Mönstermatchning (mer om det senare).
\end{Slide}
