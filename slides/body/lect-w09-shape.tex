%!TEX encoding = UTF-8 Unicode
%!TEX root = ../lect-w09.tex


\Subsection{Exempel: Shape}


\begin{Slide}{Exempel: shapes1.scala}
Typisk Scala-kod: En trait som bastyp åt flera case-klasser.
\scalainputlisting[numbers=left,numberstyle=,basicstyle=\fontsize{6.4}{7.7}\ttfamily\selectfont]{../compendium/examples/workspace/w07-inherit/src/shapes1.scala}
\end{Slide}

\begin{Slide}{Exempel: shapesTest1.scala}
Test av konkreta subklasser till bastypen \code{Shape}.
\scalainputlisting[numbers=left,numberstyle=,basicstyle=\fontsize{6.4}{7.7}\ttfamily\selectfont]{../compendium/examples/workspace/w07-inherit/src/shapesTest1.scala}

\begin{REPL}
Rectangle((100.0,100.0),(75.0,120.0))
Rectangle((141.0,183.0),(75.0,120.0))
Triangle((0.0,0.0),(4.0,0.0),(4.0,3.0))
Triangle((1.0,1.0),(4.0,0.0),(4.0,3.0))
Triangle((0.0,0.0),(4.0,0.0),(4.0,3.0))
\end{REPL}
\end{Slide}


\begin{Slide}{Fördjupningsexempel gränssnitt: draw.scala}
Två traits som kan användas för att ''koppla ihop'' kod och ge flexibilitet i implementationen:
\scalainputlisting[%numbers=left,numberstyle=,
basicstyle=\fontsize{9}{11}\ttfamily\selectfont]{../compendium/examples/workspace/w07-inherit/src/draw.scala}

\pause
\setlength{\leftmargini}{0pt}
\begin{itemize}\SlideFontSmall
\item  Traits som använda för att abstrahera implementation och möjliggöra uppfyllandet av ett slags ''kontrakt'' om vad som ska finnas kallas \Emph{gränssnitt} \Eng{interface} och används ofta för att skapa av ett flexibelt api.

%\item Implementationen av de delar vi vill kunna ändra senare placeras i subtyper som inte används direkt av klientkoden.

\item Vi visar bara information om vad som erbjuds men inte hur det ser ut ''inuti''.

\end{itemize}
\end{Slide}

\begin{Slide}{Exempel: shapes2.scala}\SlideFontTiny
Genom att \Emph{mixa in} vår \code{trait CanDraw} kan en rektangel nu även ritas ut:

\scalainputlisting[numbers=left,numberstyle=,basicstyle=\fontsize{6.5}{7.8}\ttfamily\selectfont]{../compendium/examples/workspace/w07-inherit/src/shapes2.scala}

\pause
\vspace{-0.5em}Notera: ingen ändring i \code {Shape}! Vi behöver nu bara ett \Emph{\code{DrawingWindow}}...
\end{Slide}


\begin{Slide}{Exempel: SimpleDrawingWindow.scala}\SlideFontTiny
\vspace{-0.35em}
Vi skapar en ny klass som ärver \code{SimpleWindow}, som \Alert{dessutom} även \Emph{är ett} \code{DrawingWindow}, tack vare \Emph{inmixning} med nyckelordet \code{with}.\\
Observera att vi måste skicka vidare klassparametrarna till superklassens konstruktor.

\scalainputlisting[numbers=left,numberstyle=,basicstyle=\fontsize{6.3}{7.5}\ttfamily\selectfont]{../compendium/examples/workspace/w07-inherit/src/SimpleDrawingWindow.scala}

\pause
\vspace{-0.35em}
\scalainputlisting[numbers=left,numberstyle=,basicstyle=\fontsize{6.3}{7.5}\ttfamily\selectfont]{../compendium/examples/workspace/w07-inherit/src/shapesTest2.scala}
\end{Slide}
