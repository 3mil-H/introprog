%!TEX encoding = UTF-8 Unicode
%!TEX root = ../lect-w06.tex

%%%

\Subsection{Scanna filer och strängar med \texttt{java.util.Scanner}}

\begin{Slide}{Scanna filer och strängar med \texttt{java.util.Scanner}}\SlideFontTiny
\setlength{\leftmargini}{0pt}
\begin{itemize}
\item I Scala kan man läsa från fil så här (se quickref sid 3 längst ner):

\begin{Code}
val names = scala.io.Source.fromFile("src/names.txt").getLines.toVector
\end{Code}

\item Klassen \code{java.util.Scanner} kan också läsa från fil (se Java Snabbref sid 4):


\begin{Code}
def readFromFile(fileName: String): Vector[String] = {
  val file = new java.io.File(fileName)
  val scan = new java.util.Scanner(file)
  val buffer = scala.collection.mutable.ArrayBuffer.empty[String]
  while (scan.hasNext) {
    buffer += scan.next
  }
  scan.close
  buffer.toVector
}
\end{Code}

\item Med \code{new java.util.Scanner(System.in)} kan man även scanna tangentbordet.

\item Med \code{new java.util.Scanner("hej 42")} kan man även scanna en sträng.

\item Scanna \code{Int} och \code{Double} med metoderna \code{nextInt} och \code{nextDouble}. Se doc: \href{https://docs.oracle.com/javase/8/docs/api/java/util/Scanner.html}{\SlideFontTiny docs.oracle.com/javase/8/docs/api/java/util/Scanner.html}
\end{itemize}
\end{Slide}


\begin{Slide}{Exempel: Scanner}
\begin{REPL}
scala> val scan = new java.util.Scanner("hej 42 42.0   42 slut")

scala> scan.hasNext
res0: Boolean = true

scala> scan.hasNextInt
res1: Boolean = false

scala> scan.next
res2: String = hej

scala> scan.hasNextInt
res3: Boolean = true

scala> scan.nextInt
res4: Int = 42

scala> while (scan.hasNext) println(scan.next)
42.0
42
slut
\end{REPL}
\end{Slide}








