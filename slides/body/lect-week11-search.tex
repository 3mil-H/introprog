%!TEX encoding = UTF-8 Unicode
%!TEX root = ../lect-week11.tex

%%%

\ifkompendium\else

\Subsection{Sökning}
\begin{Slide}{Sökning}\SlideFontSmall

\begin{itemize}
\item \Emph{Sökning} återkommer i många skepnader: \\ i en datastruktur, vilken det än må vara, vill man ofta kunna \\ \Emph{hitta ett element med en viss egenskap}. 

\pause
Några färdiga linjärsökningar i Scalas standardbibliotek:

\begin{REPL}
scala> Vector("gurka","tomat","broccoli").indexOf("tomat")
res0: Int = 1

scala> Vector("gurka","tomat","broccoli").indexWhere(_.contains("o"))
res1: Int = 1

scala> Vector("gurka","tomat","broccoli").find(_.contains("o"))
res2: Option[String] = Some(tomat)
\end{REPL}

\pause
\item Sökning efter ett visst index i en sekvens:

\begin{itemize}\SlideFontTiny
\item Indata: en sekvens och ett \Emph{sökkriterium}
\item Utdata: index för första eftersökta element, annars -1 
\end{itemize}

\pause
\item Två typiska varianter av sökning i en sekvens:
\begin{itemize}\SlideFontTiny
\item Linjärsökning: börja från början och sök tills ett eftersökt element är funnet
\item Binärsökning: antag sorterad sekvensen; börja i mitten, välj rätt halva ...
\end{itemize}
\end{itemize}
\end{Slide}

\Subsection{Linjärsökning}

\begin{Slide}{Linjärsökning: hitta index för elementet 42}
Skriv pseudokod för:\\ \code{def indexOf42(xs: Vector[Int]): Int = ???}
\pause
\begin{Code}
def indexOf42(xs: Vector[Int]): Int = {
  var i = /* index för första elementet */
  var found = false
  while (!found && /* index inom indexgräns */) {
    if (/* element på plats i är 42 */) found = true 
    else i = /* nästa index */
  }
  if (/* hittat */) i else -1
} 
\end{Code}
\end{Slide}

\begin{Slide}{Linjärsökning: hitta index för elementet 42}
Implementera:\\ \code{def indexOf42(xs: Vector[Int]): Int = ???}
\pause
\begin{Code}
def indexOf42(xs: Vector[Int]): Int = {
  var i = 0
  var found = false
  while (!found && i < xs.length) {
    if (xs(i) == 42) found = true 
    else i += 1
  }
  if (found) i else -1
} 
\end{Code}
\end{Slide}

\begin{Slide}{Linjärsökning: hitta index för elementet x}
Implementera:\\ \code{def indexOf(xs: Vector[Int], x: Int): Int = ???}
\pause
\begin{Code}
def indexOf(xs: Vector[Int], x: Int): Int = {
  var i = 0
  var found = false
  while (!found && i < xs.length) {
    if (xs(i) == x) found = true 
    else i += 1
  }
  if (found) i else -1
} 
\end{Code}
\end{Slide}

\begin{Slide}{Linjärsökning: hitta index för elementet p(x)}\SlideFontSmall
Implementera:\\ \code{def indexWhere(xs: Vector[Int], p: Int => Boolean): Int = ???}
\pause
\begin{Code}
def indexWhere(xs: Vector[Int], p: Int => Boolean): Int = {
  var i = 0
  var found = false
  while (!found && i < xs.length) {
    if (p(xs(i))) found = true 
    else i += 1
  }
  if (found) i else -1
} 
\end{Code}
\end{Slide}

\begin{Slide}{Linjärsökning: generalisera till godtycklig typ}\SlideFontSmall
Implementera:\\ \code{def indexWhere[T](xs: Vector[T], p: T => Boolean): Int = ???}
\pause
\begin{Code}
def indexWhere[T](xs: Vector[T], p: T => Boolean): Int = {
  var i = 0
  var found = false
  while (!found && i < xs.length) {
    if (p(xs(i))) found = true 
    else i += 1
  }
  if (found) i else -1
} 
\end{Code}
\end{Slide}

\begin{Slide}{Linjärsökning: generalisera till godtycklig typ}\SlideFontSmall
Typinferensen fungerar bättre om stegad funktion:\\ 
\code{def indexWhere[T](xs: Vector[T])(p: T => Boolean): Int}
\begin{Code}
def indexWhere[T](xs: Vector[T])(p: T => Boolean): Int = {
  var i = 0
  var found = false
  while (!found && i < xs.length) {
    if (p(xs(i))) found = true 
    else i += 1
  }
  if (found) i else -1
} 
\end{Code}
\end{Slide}


\begin{Slide}{Linjärsökning: generalisera till godtycklig sekvens}\SlideFontSmall
Implementera:\\ \code{def indexWhere[T](xs: Seq[T])(p: T => Boolean): Int = ???}
\pause
\begin{Code}
def indexWhere[T](xs: Seq[T])(p: T => Boolean): Int = {
  var i = 0
  var found = false
  while (!found && i < xs.length) {
    if (p(xs(i))) found = true 
    else i += 1
  }
  if (found) i else -1
} 
\end{Code}
\end{Slide}

\begin{Slide}{Linjärsökning: generalisera till godtycklig sekvens}\SlideFontSmall
Implementera:\\ \code{def find[T](xs: Seq[T])(p: T => Boolean): Option[T] = ???}
\pause
\begin{Code}
def find[T](xs: Seq[T])(p: T => Boolean): Option[T] = {
  var i = 0
  var found = false
  while (!found && i < xs.length) {
    if (p(xs(i))) found = true 
    else i += 1
  }
  if (found) Some(xs(i)) else None
} 
\end{Code}
\end{Slide}

\Subsection{Binärsökning}

\begin{Slide}{Binärsökning: snabbare men kräver sorterad sekvens}
\begin{REPL}[basicstyle=\color{white}\ttfamily\SlideFontSize{6.5}{8}]
scala> val enMiljon = 1000000

scala> val xs = Array.tabulate(enMiljon)(i => i + 1)   // sorterad

scala> xs(enMiljon - 1)
res0: Int = 1000000

scala> timed { xs.indexOf(enMiljon) }        // linjärsökning
res42: (Double, Int) = (0.016622994,999999)

scala> import scala.collection.Searching._  // ger tillgång till metoden search
import scala.collection.Searching._

scala> timed { xs.search(enMiljon) }        // binärsökning
res54: (Double, collection.Searching.SearchResult) = (2.45691E-4,Found(999999))

\end{REPL}
\pause
Citat från scaladoc för \code{search}:\\
''The sequence \Alert{should be sorted} before calling; otherwise, the results are \Alert{undefined.}''
\end{Slide}

\begin{Slide}{Binärsökning: lösningsidé}
\Emph{Problemlösningsidé}: Om sekvensen är sorterad kan vi utnyttja detta för en mer effektiv sökning, genom att jämföra med mittersta värdet och se om det vi söker finns före eller efter detta, och upprepa med ''halverad'' sekvens tills funnet.
\pause
\begin{itemize}\SlideFontSmall
\item \Emph{Indata}: sorterad sekvens av heltal och det eftersökta elementet
\item \Emph{Utdata}: \code{Found(index)} för det eftersökta elementet
\\ om saknas: \code{InsertionPoint(index)}
\pause
\begin{Code}
sealed trait SearchResult {
  def insertionPoint: Int
}

case class Found(foundIndex: Int) extends SearchResult {
  override def insertionPoint = foundIndex
}
  
case class InsertionPoint(insertionPoint: Int) extends SearchResult
\end{Code}
\end{itemize} 
\end{Slide}

\begin{Slide}{Binärsökning: pseudokod, iterativ lösning}
\Emph{Pseudo-kod}: iterativ lösning
\begin{Code}
def binarySearch(xs: Vector[Int])(elem: Int): SearchResult = {
  var found = false
  var (low, high) = (/* lägsta index */, /* högsta index */)
  var mid = /* något startvärde */ 
  while (!found && /* finns fler element kvar */) {
    mid = /* mittpunkten i intervallet (low, high) */
    if (xs(mid == elem) found = true
    else if (xs(mid) < elem) /* flytta intervallets undre gräns */
    else /* flytta intervallets övre gräns" */
  }
  if (found) Found(mid)
  else InsertionPoint(low)
}
\end{Code}
\end{Slide}



\begin{Slide}{Binärsökning: implementation, iterativ lösning}
\Emph{Implementation}: iterativ lösning
\begin{Code}
def binarySearch(xs: Vector[Int])(elem: Int): SearchResult = {
  var found = false
  var (low, high) = (0, xs.length - 1)
  var mid = -1 
  while (!found && low <= high) {
    mid = (low + high) / 2
    if (xs(mid) == elem) found = true
    else if (xs(mid) < elem) low = mid + 1
    else high = mid - 1 
  }
  if (found) Found(mid) 
  else InsertionPoint(low)
}
\end{Code}
\end{Slide}

\begin{Slide}{Binärsökning: instrumentering av iterativ lösning}
\vspace{-0.5em}
\begin{CodeSmall}
def waitForEnter: Unit = scala.io.StdIn.readLine("")
def show(msg: String): Unit = {println(msg); waitForEnter}

def binarySearch(xs: Vector[Int])(elem: Int): SearchResult = {
  var found = false                         ; show(s"found = $found")
  var (low, high) = (0, xs.length - 1)      ; show(s"(low, high) = ($low, $high)")
  var mid = -1                              ; show(s"mid = $mid")
  while (!found && low <= high) {           ; show(s"while ${!found && low <= high}")
    mid = (low + high) / 2                  ; show(s"mid = $mid")
    if (xs(mid) == elem) {found = true      ; show(s"found = $found")}
    else if (xs(mid) < elem) {low = mid + 1 ; show(s"low = $low")}
    else {high = mid - 1                    ; show(s"high = $high")}
  }
  if (found) Found(mid) 
  else InsertionPoint(low)
}
\end{CodeSmall}
\vspace{-0.5em}
\begin{REPL}[basicstyle=\color{white}\ttfamily\SlideFontSize{6}{7}]
scala> binarySearch(Vector(0,1,2,3,42,5))(42)
found = false
(low, high) = (0, 5)
mid = -1
while true
mid = 2
low = 3
while true
mid = 4
found = true
res0: collection.Searching.SearchResult = Found(4)
\end{REPL}
\end{Slide}

\begin{Slide}{Binärsökning: rekursiv lösning}
\Emph{Fördjupning}: rekursiv lösning
\begin{Code}
def binarySearch(xs: Vector[Int])(elem: Int): SearchResult = {
  def loop(low: Int, high: Int): SearchResult = 
    if (low > high) InsertionPoint(low) 
    else (low + high) / 2 match {
      case mid if xs(mid) == elem => Found(mid)
      case mid if xs(mid) < elem  => loop(mid + 1, high)
      case mid                    => loop(low, mid - 1)
    }
    
  loop(0, xs.length - 1) 
}
\end{Code}
\end{Slide}


\begin{Slide}{Binärsökning: generisk rekursiv lösning}
\Emph{Fördjupning}: iterativ generisk lösning med implicit ordning
\begin{CodeSmall}
def binarySearch[T](xs: Seq[T])(elem: T)(implicit ord: Ordering[T]): SearchResult = {
  import ord._
  def loop(low: Int, high: Int): SearchResult = 
    if (low > high) InsertionPoint(low) 
    else (low + high) / 2 match {
      case mid if xs(mid) == elem => Found(mid)
      case mid if xs(mid) < elem  => loop(mid + 1, high)
      case mid                    => loop(low, mid - 1)
    }
    
  loop(0, xs.length - 1) 
}
\end{CodeSmall}
{\SlideFontSmall För den intresserade:\\Se fördjupningsuppgifter om implicita ordningar i veckans övning.}
\end{Slide}

\begin{Slide}{Tidskomplexitet, sökning}\SlideFontSmall

\Emph{Fördjupning:}\\Algoritmteoretisk analys av sökalgoritmerna ger:
\begin{itemize}
\item Linjärsökning: tiden är proportionell mot $n$, skrivs: $O(n)$ 
\item Binärsökning:  tiden är proportionell mot $\log_2 n$, skrivs: $O(\log n)$
\end{itemize}
\vspace{1em}\pause
Empirisk analys: Vi har en vektor med 1000 element. Vi har mätt tiden för att söka upp ett element många gånger och funnit att det tar ungefär 1 $\mu$s både med linjärsökning och binärsökning. 
\\\vspace{1em}Hur lång tid tar det om vi har fler element i vektorn?\\
\vspace{1em}
\begin{tabular}{rccccc}
       & 1000 & 10 000 & 100000 & 1 000 000 & 10 000 000 \\ \hline
linjärsökning & 1     & 10     & 100     & 1000     & 10 000 \\
binärsökning  & 1     & 1.33   & 1.67    & 2.00     & 2.33
\end{tabular}
\vspace{1em} 

{\SlideFontTiny 

Kurserna: \\
''\Emph{Utvärdering av programvarusystem}'', obl. för D1, studerar detta \Alert{empiriskt}\\
''\Emph{Algoritmer, datastrukturer och komplexitet}'', obl. för D2, studerar detta \Alert{analytiskt}
}
\end{Slide} 


\fi










