%!TEX encoding = UTF-8 Unicode
%!TEX root = ../lect-w09.tex


\ifkompendium\else
\begin{Slide}{Omkontroll}
\begin{itemize}
\item För de som ej deltog i kontrollskrivningen erbjuds \Alert{omkontrollskrivning} den 15/11 kl 10:15-15:00 i MA8. 
\item Kontrollskrivningen är obligatorisk och ett krav för att bli godkänd på kursen.
\item Alla som frånvarande på ordinarie skrivning, även de som inte avser delta, ska snarast svara på enkät i Canvas om omkontrollskrivning. Svarsalternativ: Ja, Förhinder, Avbrott.
\end{itemize}  
\end{Slide}
\fi

\Subsection{Veckans uppgifter}

\begin{Slide}{Övning: \texttt{lookup}}
\begin{itemize}
  \item Uppgifterna innehåller delar som är \Alert{nödvändiga} för laborationen.
  \item På övningen tränar du på \Emph{mängder} och \Emph{nyckel-värde-tabeller}.
  \item Du ska skapa en klass \code{FreqMapBuilder} som bygger upp en tabell med ordfrekvenser, som behövs på labben.
\end{itemize}
\end{Slide}

\begin{Slide}{Laboration: \texttt{words}}
\begin{itemize}
  \item Denna uppgift handlar om analys av naurligt språk \Eng{Natural Language Processing, NLP}.
  \item Svara på frågorna:
  \begin{itemize}%[noitemsep]
  \item Hur vanligt är ett visst ord i en given text?
  \item Vilket är det vanligaste ordet som följer efter ett visst ord?
  \item Hur kan man generera ordsekvenser som liknar ordföljden i en given text?
  \end{itemize}
\item Använda mängd för unika ord.
\item Använda nyckel-värde-tabell för att för varje ord i en lång text räkna antalet förekomster av detta ord.
\end{itemize}
\end{Slide}


\begin{Slide}{Laboration: \texttt{words}}
Lärandemål:
\begin{itemize}\SlideFontSmall
%!TEX encoding = UTF-8 Unicode
%!TEX root = ../compendium2.tex

%\item Kunna använda en integrerad utvecklingsmiljö (IDE).
%\item Kunna använda färdiga funktioner för att läsa till, och skriva från, textfil.
%\item Kunna använda enkla case-klasser.
%\item Kunna skapa och använda enkla klasser med föränderlig data.
\item Kunna skapa och använda nyckel-värde-tabeller med samlingstypen \code{Map}.
\item Kunna skapa och använda mängder med samlingstypen \code{Set}.
\item Förstå skillnaden mellan en ordnad sekvens och en mängd.
\item Förstå likheter och skillnader mellan en sekvens av par och en nyckel-värde-tabell. 
\item Kunna implementera algoritmer som använder nästlade strukturer. 
%\item Kunna skapa en ny samling från en befintlig samling.
%\item Förstå skillnaden mellan kompileringsfel och exekveringsfel.
%\item Kunna felsöka i små program med hjälp av utskrifter.
%\item Kunna felsöka i små program med hjälp av en debugger i en IDE.

\end{itemize}
Uppgifter:
\begin{itemize}\SlideFontSmall
  \item Dela upp en sträng i ord.
  \item Skapa ordfrekvenstabeller för böcker som ditt program laddar ner från nätet via projektet Gutenberg med fria böcker.
  \item Skapa frekvenstabeller för ordföljder, s.k. \emph{n-gram}.
  \item Skriv ut intressant statistik om ordvalen i olika böcker, t.ex. ur könsrollsperspektiv.
  \item Valfri uppgift: Gör en bot som genererar slumpvisa, artificiella meningar som liknar mänskligt språk.
\end{itemize}
\end{Slide}



