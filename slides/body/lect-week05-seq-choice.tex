%!TEX encoding = UTF-8 Unicode
%!TEX root = ../lect-week05.tex

%%%

\Subsection{Att välja sekvenssamling}


\begin{Slide}{Oföränderlig eller förändringsbar?}
\begin{itemize}
\item \Emph{Oföränderlig}:  Kan ej ändra elementreferenserna, men effektiv på att skapa kopia som är (delvis) förändrad\\(vanliga i Scala, men inte i Java): \Emph{Vector} eller \Emph{List}

\item \Alert{Förändringsbar}: kan ändra elemententreferenserna
  \begin{itemize} 
  \item Kan \Alert{ej ändra storlek} efter allokering: \\ Scala+Java: \Emph{Array}: indexera och uppdatera varsomhelst
  \item Kan ändra storlek efter allokering: 
  \\ Scala: \Emph{ArrayBuffer} eller \Emph{ListBuffer}
  \\ Java: \Emph{ArrayList} eller \Emph{LinkedList}
  \end{itemize}
\end{itemize}
\end{Slide}




\begin{Slide}{Egenskaper hos några sekvenssamlingar}
\vspace{-0.5em}
\begin{itemize}\SlideFontSmall

\item \code{Vector} 
  \begin{itemize}\SlideFontSmall
  \item \Emph{Oföränderlig}. Snabb på att skapa kopior med små förändringar.
  \item Allsidig prestanda: \Emph{bra till det mesta}.
  \end{itemize}

\item \code{List}   
  \begin{itemize}\SlideFontSmall
  \item \Emph{Oföränderlig}. Snabb på att skapa kopior med små förändringar.
  \item Snabb vid bearbetning \Emph{i början}. 
  \item Smidig \& snabb vid \Emph{rekursiva} algoritmer.
  \item Långsam vid upprepad \Alert{indexering} på godtyckliga ställen.
  \end{itemize}

\item \code{Array} 
  \begin{itemize}\SlideFontSmall
  \item \Alert{Föränderlig}: \Emph{snabb indexering \& uppdatering}.
  \item Kan \Alert{ej ändra storlek}; storlek anges vid allokering.
  \item Har särställning i JVM: ger snabbaste minnesaccessen.
  \end{itemize}

\item \code{ArrayBuffer}  
  \begin{itemize}\SlideFontSmall
  \item \Alert{Föränderlig}: \Emph{snabb indexering \& uppdatering}.
  \item Kan \Emph{ändra storlek} efter allokering. Snabb att indexera överallt.
  \end{itemize}

\item \code{ListBuffer}  
  \begin{itemize}\SlideFontSmall
  \item \Alert{Föränderlig}: snabb indexering \& uppdatering \Emph{i början}.
  \item Snabb om du bygger upp sekvens genom många tillägg i början.
  \end{itemize}

\end{itemize}
\end{Slide}



\begin{Slide}{Vilken sekvenssamling ska jag välja?}\SlideFontSmall
\vspace{-0.5em}
\begin{itemize}
\item \code{Vector} 
  \begin{itemize}\SlideFontTiny
  \item Om du vill ha oföränderlighet: \code{val xs = Vector[Int](1,2,3)}
  \item Om du behöver ändra (men ej prestandakritiskt):\\ \code{var xs = Vector.empty[Int]}
  \item Om du ännu inte vet vilken sekvenssamling som är bäst; du kan alltid ändra efter att du mätt prestanda och kollat flaskhalsar.
  \end{itemize}

\item \code{List} 
  \begin{itemize}\SlideFontTiny
  \item Om du har en rekursiv sekvensalgoritm och/eller bara lägger till i början.
  \end{itemize}


\item \code{Array} 
  \begin{itemize}\SlideFontTiny
  \item Om det behövs av prestandaskäl och du \Alert{vet} storlek vid allokering:\\ \code{val xs = Array.fill(initSize)(initValue)}
  \end{itemize}

\item \code{ArrayBuffer}  
  \begin{itemize}\SlideFontTiny
  \item Om det behövs av prestandaskäl och du \Alert{inte} vet storlek vid allokering:\\ \code{val xs = scala.collection.mutable.empty[Int]} 
  \end{itemize}

\item \code{ListBuffer}  
  \begin{itemize}\SlideFontTiny
  \item om det behövs av prestandaskäl och du bara behöver lägga till i början:\\ \code{val xs = scala.collection.mutable.ListBuffer.empty[Int]} 
  \end{itemize}

\end{itemize}
\end{Slide}

\ifkompendium\else
\begin{Slide}{Lämna det öppet: använd \texttt{Seq[T]}}
\begin{Code}[basicstyle=\ttfamily]
def varannanBaklänges[T](xs: Seq[T]): Seq[T] = 
  for (i <- xs.indices.reverse by -2) yield xs(i) 
\end{Code}
Fungerar med alla sekvenssamlingar:
\begin{REPLnonum}
scala> varannanBaklänges(Vector(1,2,3,4,5))
res0: Seq[Int] = Vector(5, 3, 1)

scala> varannanBaklänges(List(1,2,3,4,5))
res1: Seq[Int] = List(5, 3, 1)

scala> varannanBaklänges(collection.mutable.ListBuffer(1,2))
res2: Seq[Int] = Vector(2)
\end{REPLnonum}
Scalas standardbibliotek returnerar ofta lämpligaste specifika sekvenssamlingen som är subtyp till \texttt{Seq[T]}.
\end{Slide}
\fi







