%!TEX encoding = UTF-8 Unicode
%!TEX root = ../lect-w03.tex


\Subsection{Funktioner}


\begin{Slide}{Funktion}
\SlideOnly{\setlength{\leftmargini}{0pt}}
\begin{itemize}
  \item En funktion kan ha parametrar som binds till argument.
  \item En funktion har ett \Emph{huvud} och efter \code{=} kommer dess \Emph{kropp}.
  \item En \Alert{namngiven} funktion \Emph{deklareras} med nyckelordet \code{def}
\end{itemize}


\code{def} funktionsnamn(parameterdeklarationer): returtyp = uttryck

\vspace{1em}

\begin{Code}
def namn(param1: Typ1, param2: Typ2): Returtyp = uttryck
\end{Code}

\pause

\begin{Code}
def öka(a: Int, b: Int): Int = a + b
\end{Code}

\SlideOnly{\setlength{\leftmargini}{0pt}}
\begin{itemize}
  \item Funktionskroppen exekveras först vid \Emph{anrop} då \Emph{argument} binds till \Emph{parametrar}.
\end{itemize}

\begin{REPL}
scala> öka(42, 1)
res0: Int = 43
\end{REPL}


\end{Slide}


\begin{Slide}{Deklarera funktioner, överlagring}
\begin{itemize}
\item En parameter, och sedan två parametrar:
\begin{REPL}
scala> :paste
  def öka(a: Int): Int = a + 1
  def öka(a: Int, b: Int): Int = a + b

scala> öka(1)
res0: Int = 2

scala> öka(1,1)
res1: Int = 2

\end{REPL}
\item Båda funktionerna ovan kan finnas samtidigt! Trots att de har \Emph{samma namn} är de \Alert{olika funktioner}; kompilatorn kan skilja dem åt med hjälp av de \Alert{olika parameterlistorna}.

\item Detta kallas \Emph{överlagring} \Eng{overloading} av funktioner.
\item Överlagring ger flexibilitet i användningen; vi slipper hitta på nytt namn så som \code{öka2} vid 2 parametrar.
\end{itemize}
\end{Slide}



\begin{Slide}{Funktioner med defaultargument}\SlideFontSmall

\begin{itemize}
\item Vi kan ofta åstadkomma samma flexibilitet som vid överlagring, men med \Alert{en enda} funktion, om vi i stället använder \Emph{defaultargument}:
\begin{REPLnonum}
scala> def inc(a: Int, b: Int = 1) = a + b
inc: (a: Int, b: Int)Int

scala> inc(42, 2)
res0: Int = 44

scala> inc(42, 1)
res1: Int = 43

scala> inc(42)
res2: Int = 43

\end{REPLnonum}
\item Om ett argument utelämnas och parametern deklarerats med defaultargument så appliceras detta. Kompilatorn fyller alltså i argumentet åt oss, om det är entydigt vilken parameter som avses.
\end{itemize}
\end{Slide}


\begin{Slide}{Funktioner med namngivna argument}
\begin{itemize}
\item Genom att använda \Emph{namngivna argument} behöver man inte hålla reda på ordningen på parametrarna, bara man känner till parameternamnen.
\item Namngivna argument går fint att \Alert{kombinera} med defaultargument.
\begin{REPL}
scala> def namn(förnamn: String,
                efternamn: String,
                förnamnFörst: Boolean = true,
                ledtext: String = ""): String =
         if (förnamnFörst) s"$ledtext: $förnamn $efternamn"
         else s"$ledtext: $efternamn, $förnamn"

scala> namn(ledtext = "Name", efternamn = "Coder", förnamn = "Kim")
res0: String = Name: Kim Coder
\end{REPL}
\end{itemize}
\end{Slide}


\begin{Slide}{Tom parameterlista och inga parametrar}\SlideFontSmall
\begin{itemize}
\item Om en funktion deklareras med tom parameterlista \code{()} kan den anropas på två sätt: med och utan tomma parenteser.
\begin{REPL}
scala> def tomParameterLista() = 42

scala> tomParameterLista()
res2: Int = 42

scala> tomParameterLista
res3: Int = 42
\end{REPL}

Denna flexibilitet är grunden för \Emph{enhetlig access}: namnet kan användas enhetligt oavsett om det är en funktion eller en variabel.
\item Om parameterlista saknas får man \Alert{inte} använda \code{()} vid anrop:

\begin{REPL}
scala> def ingenParameterLista = 42

scala> ingenParameterLista
res4: Int = 42

scala> ingenParameterLista()
<console>:13: error: Int does not take parameters
       ingenParameterLista()
\end{REPL}

\end{itemize}
\end{Slide}



\begin{Slide}{Anropsstacken och objektheapen}\SlideFontSmall
Minnet är uppdelat i två delar:
\begin{itemize}
\item \Emph{Anropsstacken}: På stackminnet läggs en \Emph{aktiveringspost} \Eng{stack frame\footnote{\href{https://en.wikipedia.org/wiki/Call_stack}{en.wikipedia.org/wiki/Call\_stack}}, activation record} för varje funktionsanrop med plats för \Alert{parametrar} och \Alert{lokala variabler}.
\item Aktiveringsposten \Alert{raderas} när \Emph{returvärdet} har levererats.
\item Stacken \Alert{växer} vid \Emph{nästlade funktionsanrop}, då en funktion i sin tur anropar en annan funktion.

\item \Emph{Objektheapen}: I heapminnet\footnote{\href{https://en.wikipedia.org/wiki/Memory_management}{en.wikipedia.org/wiki/Memory\_management}}$^{,}$\footnote{Ej att förväxlas med datastrukturen heap  \href{https://sv.wikipedia.org/wiki/Heap}{sv.wikipedia.org/wiki/Heap}} sparas alla objekt (data) som allokeras under körning. Heapen städas då och då av \Emph{skräpsamlaren} \Eng{garbage collector}, och minne som inte används längre frigörs. \\\vspace{0.5em}
\href{http://stackoverflow.com/questions/1565388/increase-heap-size-in-java}{stackoverflow.com/questions/1565388/increase-heap-size-in-java}
\end{itemize}
\end{Slide}

% \begin{Slide}{Aktiveringspost}\SlideFontSmall
% Nästlade anrop ger växande anropsstack.
% \begin{REPL}
% scala> def f(x: Int, y: Int): Int = { val z = x + y; println(z); z}
% scala> def g(a: Int, b: Int): Int = { val c = a + b; println(c); f(c, 2 * c) }
% scala> def h(i: Int): Int = { val n = 5; g(i, i * n) }
% scala> h(2)
% \end{REPL}
%
% \pause
% \Alert{Stacken}
%
% \begin{tabular}{|r | l | l |} \hline
%
% variabel & värde & Aktiveringspost för anrop av... \\ \hline \hline
% \pause
%  i & 2 & h \\
%  n & 5 & \\ \hline
%  \pause
%  a & 2 & g \\
%  b & 10 &  \\
%  c & 12  &  \\  \hline
%  \pause
%  x & 12  & f \\
%  y & 24 &  \\
%  z & 36 & \\ \hline
% \end{tabular}
% \end{Slide}

\begin{Slide}{Aktiveringspost}\SlideFontSmall
Nästlade anrop ger växande anropsstack.
\begin{REPL}
scala> :paste
def h(x: Int, y: Int): Unit = { val z = x + y; println(z) }
def g(a: Int, b: Int): Unit = { val x = 1; h(x + 1, a + b) }
def f(): Unit = { val n = 5; g(n, 2 * n) }

scala> f()

\end{REPL}

\pause
\Alert{Stacken}

\begin{tabular}{|r | l | l |} \hline

variabel & värde & Aktiveringspost för anrop av... \\ \hline \hline
\pause
 n & 5 & f \\ \hline
 \pause
 a & 5 & g \\
 b & 10 &  \\
 x & 1  &  \\  \hline
 \pause
 x & 2  & h \\
 y & 15 &  \\
 z & 17 & \\ \hline
\end{tabular}
\end{Slide}


\begin{Slide}{Lokala funktioner}\SlideFontSmall
Med lokala funktioner kan delproblem lösas med nästlade abstraktioner.

\begin{CodeSmall}
def gissaTalet(max: Int, min: Int = 1): Unit = {
  def gissat = io.StdIn.readLine(s"Gissa talet mellan $min och $max:").toInt

  val hemlis = (math.random * (max - min) + min).toInt

  def skrivLedtrådOmEjRätt(gissning: Int): Unit =
    if (gissning > hemlis) println(s"$gissning är för stort :(")
    else if (gissning < hemlis) println(s"$gissning är för litet :(")

  def inteRätt(gissning: Int): Boolean = {
    skrivLedtrådOmEjRätt(gissning)
    gissning != hemlis
  }

  def loop: Int = { var i = 1; while (inteRätt(gissat)){ i += 1 }; i }

  println(s"Du hittade talet $hemlis på $loop gissningar :)")
}
\end{CodeSmall}

Lokala, nästlade funktionsdeklarationer är tyvärr inte tillåtna i Java.\footnote{\href{http://stackoverflow.com/questions/5388584/does-java-support-inner-local-sub-methods}{\SlideFontSize{8}{9}stackoverflow.com/questions/5388584/does-java-support-inner-local-sub-methods}}

\end{Slide}




\begin{Slide}{Funktioner är äkta värden i Scala}\SlideFontSmall
\begin{itemize}
\item En funktion är ett \Alert{äkta värde}.
\item Vi kan till exempel tilldela en variabel ett funktionsvärde.
\pause
\item Med hjälp av blank+understreck efter funktionsnamnet får vi funktionen som ett \Alert{värde} (inga argument appliceras än):
\begin{REPLnonum}
scala> def add(a: Int, b: Int) = a + b

scala> val f = add _

scala> f
f: (Int, Int) => Int = <function2>

scala> f(21, 21)
res0: Int = 42
\end{REPLnonum}

\item Ett funktionsvärde har en \Alert{typ} precis som alla värden: \\
\code{f: (Int, Int) => Int}
\end{itemize}
\end{Slide}

\begin{Slide}{Funktionsvärden kan vara argument}
En funktion kan ha en annan funktion som parameter:
\begin{REPL}
scala> def tvåGånger(x: Int, f: Int => Int) = f(f(x))

scala> def öka(x: Int) = x + 1

scala> def minska(x: Int) = x - 1

scala> tvåGånger(42, öka _)
res1: Int = 44

scala> tvåGånger(42, minska _)
res1: Int = 40
\end{REPL}
Om argumentets funktionstyp \Alert{kan härledas} av kompilatorn och \Alert{passar} med parametertypen så behövs ej understreck: \\
\begin{REPL}
scala> tvåGånger(42, öka)
res1: Int = 44
\end{REPL}
\end{Slide}



\begin{Slide}{Applicera funktioner på element i samlingar med \texttt{map}}\SlideFontSmall
\begin{Code}
def öka(x: Int) = x + 1

def minska(x: Int) = x - 1

val xs = Vector(1, 2, 3)
\end{Code}
\pause
Metoden \Emph{\texttt{map}} fungerar på alla Scala-samlingar och tar \Emph{en funktion som argument} och applicerar denna funktion på alla element och \Alert{skapar en ny samling} med resultaten:
\begin{REPL}
scala> xs.map(öka)
res0: ???

scala> xs.map(minska)
res1: ???
\end{REPL}
Funktioner som tar andra funktioner som parametrar kallas \\ \Emph{högre ordningens funktioner}.
\end{Slide}


\begin{Slide}{Applicera funktioner på element i samlingar med \texttt{map}}\SlideFontSmall
\begin{Code}
def öka(x: Int) = x + 1

def minska(x: Int) = x - 1

val xs = Vector(1, 2, 3)
\end{Code}
Metoden \Emph{\texttt{map}} fungerar på alla Scala-samlingar och tar \Emph{en funktion som argument} och applicerar denna funktion på alla element och \Alert{skapar en ny samling} med resultaten:
\begin{REPL}
scala> xs.map(öka)
res0: scala.collection.immutable.Vector[Int] = Vector(2, 3, 4)

scala> xs.map(minska)
res1: scala.collection.immutable.Vector[Int] = Vector(0, 1, 2)
\end{REPL}
Funktioner som tar andra funktioner som parametrar kallas \\ \Emph{högre ordningens funktioner}.
\end{Slide}




\begin{Slide}{Anonyma funktioner}
\begin{itemize}
\item  Man behöver inte ge funktioner namn. De kan i stället skapas med hjälp av \Emph{funktionsliteraler}.\footnote{Även kallat ''lambda-värde'' eller bara ''lambda'' efter den s.k. lambdakalkylen. \href{https://en.wikipedia.org/wiki/Anonymous_function}{en.wikipedia.org/wiki/Anonymous\_function}}

\item En funktionsliteral har ...
\begin{enumerate}
\item en parameterlista (utan funktionsnamn, utan returtyp),
\item sedan den reserverade teckenkombinationen \code{=>}
\item och sedan ett uttryck (eller ett block).
\end{enumerate}
\pause
\item Exempel:
\begin{Code}[basicstyle=\ttfamily\SlideFontSize{9}{11}]
(x: Int, y: Int) => x + y             // vilken typ?
\end{Code}
\pause
\item Om kompilatorn kan gissa typerna från sammanhanget så behöver typerna inte anges i själva  funktionsliteralen:
\begin{Code}[basicstyle=\ttfamily\SlideFontSize{9}{11}]
val f: (Int, Int) => Int = (x, y) => x + y
\end{Code}
\end{itemize}
\end{Slide}


\begin{Slide}{Applicera anonyma funktioner på element i samlingar}\SlideFontSmall
Anonym funktion skapad med funktionsliteral direkt i anropet:
\begin{REPL}
scala> val xs = Vector(1, 2, 3)

scala> xs.map((x: Int) => x + 1)
res0: scala.collection.immutable.Vector[Int] = Vector(2, 3, 4)
\end{REPL}
\pause
Eftersom kompilatorn här kan härleda typerna så behövs de inte:
\begin{REPL}
scala> xs.map(x => x + 1)
res1: scala.collection.immutable.Vector[Int] = Vector(2, 3, 4)
\end{REPL}
\pause
Om man bara använder parametern en enda gång i funktionen så kan man byta ut parameternamnet mot ett understreck.
\begin{REPL}
scala> xs.map(_ + 1)
res2: scala.collection.immutable.Vector[Int] = Vector(2, 3, 4)
\end{REPL}
\end{Slide}



\begin{Slide}{Platshållarsyntax för anonyma funktioner}\SlideFontSmall
Understreck i funktionsliteraler kallas \Emph{platshållare} \Eng{placeholder} och medger ett förkortat skrivsätt \Alert{om} den parameter som understrecket representerar används \Alert{endast en gång}.
\begin{Code}[basicstyle=\ttfamily\fontsize{10}{12}\selectfont]
_ + 1
\end{Code}
Ovan expanderas av kompilatorn till följande funktionsliteral \\(där namnet på parametern är godtyckligt):
\begin{Code}[basicstyle=\ttfamily\fontsize{10}{12}\selectfont]
x => x + 1
\end{Code}
\pause
Det kan förekomma flera understreck; det första avser första parametern, det andra avser andra parametern etc.
\begin{Code}[basicstyle=\ttfamily\fontsize{10}{12}\selectfont]
_ + _
\end{Code}
\pause
... expanderas till:
\begin{Code}[basicstyle=\ttfamily\fontsize{10}{12}\selectfont]
(x, y) => x + y
\end{Code}
\end{Slide}


\begin{Slide}{Exempel på platshållarsyntax med samlingsmetoden \texttt{reduceLeft}}\SlideFontSmall
Metoden \code{reduceLeft} applerar en funktion på de två första elementen och tar sedan på resultatet som första argument och nästa element som andra argument och upprepar detta genom hela samlingen.
\begin{REPL}
scala> def summa(x: Int, y: Int) = x + y

scala> val xs = Vector(1, 2, 3, 4, 5)

scala> xs.reduceLeft(summa)
res20: Int = 15

scala> xs.reduceLeft((x, y) => x + y)
res21: Int = 15

scala> xs.reduceLeft(_ + _)
res22: Int = 15

scala> xs.reduceLeft(_ * _)
res23: Int = 120
\end{REPL}
\end{Slide}




\begin{Slide}{Hur fungerar egentligen \code{upprepa} i Kojo?}
\begin{Code}[basicstyle=\ttfamily\SlideFontSize{14}{16}]
upprepa(10) {
  println("hej")
}
\end{Code}

\pause
Vi ska nu se hur vi, genom att kombinera ett antal koncept, kan skapa egna kontrollstrukturer likt upprepa ovan:
\begin{itemize}
\item klammerparentes vid ensam paramenter
\item uppdelad parameterlista
\item namnanrop (fördröjd evaluering)
\end{itemize}
\end{Slide}



\begin{Slide}{Uppdelad parameterlista}
Vi har tidigare sett att man kan ha mer än en parameter:
\begin{REPLnonum}
scala> def add(a: Int, b: Int) = a + b

scala> add(21, 21)
res0: Int = 42
\end{REPLnonum}
Man kan även ha \Alert{mer än en} \Emph{parameterlista}:
\begin{REPLnonum}
scala> def add(a: Int)(b: Int) = a + b

scala> add(21)(21)
res1: Int = 42
\end{REPLnonum}
Detta kallas även \Emph{multipla parameterlistor} \Eng{multiple parameter lists}
\\\href{http://docs.scala-lang.org/style/declarations.html#multiple-parameter-lists}{\SlideFontTiny docs.scala-lang.org/style/declarations.html\#multiple-parameter-lists}
\end{Slide}



\begin{Slide}{Värdeanrop och namnanrop}\SlideFontSmall
Det vi sett hittills är \Emph{värdeanrop}: argumentet evalueras \Alert{först} innan dess \Alert{värde} \emph{sedan} appliceras:
\begin{REPL}
scala> def byValue(n: Int): Unit = for (i <- 1 to n) print(" " + n)

scala> byValue(21 + 21)
 42 42 42 42 42 42 42 42 42 42 42 42 42 42 42 42 42 42 42 42 42 42 42 42 42 42 42 42 42 42 42 42 42 42 42 42 42 42 42 42 42 42

scala> byValue({print(" hej"); 21 + 21})
 hej 42 42 42 42 42 42 42 42 42 42 42 42 42 42 42 42 42 42 42 42 42 42 42 42 42 42 42 42 42 42 42 42 42 42 42 42 42 42 42 42 42 42
\end{REPL}
\pause
Men man kan med \code{=>} före parametertypen åstadkomma \Emph{namnanrop}: argumentet \Alert{''klistras in''} i stället för \Alert{namnet} och evalueras \Alert{varje gång} (kallas även \Emph{fördröjd evaluering}):
\begin{REPL}
scala> def byName(n: => Int): Unit = for (i <- 1 to n) print(" " + n)

scala> byName({print(" hej"); 21 + 21})
 hej hej 42 hej 42 hej 42 hej 42 hej 42 hej 42 hej 42 hej 42 hej 42 hej 42 hej 42 hej 42 hej 42 hej 42 hej 42 hej 42 hej 42 hej 42 hej 42 hej 42 hej 42 hej 42 hej 42 hej 42 hej 42 hej 42 hej 42 hej 42 hej 42 hej 42 hej 42 hej 42 hej 42 hej 42 hej 42 hej 42 hej 42 hej 42 hej 42 hej 42 hej 42 hej 42
\end{REPL}
\Alert{Kluring}: Varför skrivs ''hej'' ut en extra gång i början? \pause ledtråd: \texttt{1 to \Alert{n}}
%evalueringen av n i 1 to n ger ett extra hej
\end{Slide}

\begin{Slide}{Klammerparenteser vid ensam parameter}
Så här har vi sett nyss att man man göra:
\begin{REPL}
scala> def twice(action: => Unit): Unit = { action; action }

scala> twice( { print("hej"); print("san ") } )
hejsan hejsan
\end{REPL}

Det ser rätt klyddigt ut med \code+{(+  och \code+)}+ eller vad tycker du? \pause Men...
För alla funktioner \code{f} gäller att: \\ det är helt ok att byta ut vanliga parenteser: \hfill\code{f(uttryck)} \\ mot krullparenteser: \hfill\code|f{uttryck}| \\ \Alert{om} parameterlistan har \Alert{exakt en} parameter.

\vspace{0.5em}Man kan alltså skippa det yttre parentesparet för bättre läsbarhet:
\begin{REPLnonum}
scala> twice { print("hej"); print("san ") }
\end{REPLnonum}
\end{Slide}



\begin{Slide}{Skapa din egen kontrollstruktur}
\begin{itemize}
\item Genom att \Alert{kombinera} \Emph{uppdelad parameterlista} med \Emph{namnanrop} med \Emph{klammerparentes vid ensam parameter} kan vi skapa vår egen kontrollstruktur: \code{upprepa} \pause
\begin{Code}
upprepa(42){
  if (math.random < 0.5) print(" gurka")
  else print(" tomat")
}
\end{Code}
Hur då?
\pause
 Till exempel så här:
\begin{Code}
def upprepa(n: Int)(block: => Unit) = {
  for(i <- 0 until n) {
    block
  }
}


\end{Code}

\pause

\begin{REPLnonum}
gurka gurka gurka tomat tomat gurka gurka gurka gurka tomat tomat tomat tomat tomat
\end{REPLnonum}
\end{itemize}
\end{Slide}


\begin{Slide}{Stegade funktioner, ''Curry-funktioner''}
Om en funktion har en uppdelad parameterlista kan man skapa \Emph{stegade funktioner}, även kallat \Emph{partiellt applicerade} funktioner \Eng{partially applied functions} eller \Emph{''Curry''-funktioner}.
\begin{REPLnonum}
scala> def add(x: Int)(y: Int) = x + y

scala> val öka = add(1) _
öka: Int => Int = <function1>

scala> Vector(1,2,3).map(öka)
res0:scala.collection.immutable.Vector[Int]= Vector(2, 3, 4)

scala> Vector(1,2,3).map(add(2))
res1:scala.collection.immutable.Vector[Int]= Vector(3, 4, 5)
\end{REPLnonum}
\end{Slide}


\begin{Slide}{Funktion med fångad variabelrymd: \textit{closure}}
\begin{Code}
def f(x: Int): Int => Int = {
  val a = 42 + x
  def g(y: Int): Int = y + a
  g _
}
\end{Code}
Funktionen \code{g} \Alert{fångar} den lokala variabeln \code{a} i ett s.k. \Emph{closure}.
\pause
\begin{REPLnonum}
scala> val funkis = f(1)
funkis: Int => Int = $$Lambda$$1061/726226084@498f1f63

scala> funkis(2)
res0: Int = 45
\end{REPLnonum}
\pause
Ett \Emph{closure} sparar variabler i ett \Alert{persistent funktionsobjekt}. \\
(Mer om funktioner som objekt senare.)
\end{Slide}

\ifkompendium\else
\begin{SlideExtra}{Översikt av begrepp vi gått igenom hittills}
\begin{itemize}
\item överlagring
\item utelämna tom parameterlista (enhetlig access)
\item defaultargument
\item namngivna argument
\item lokala funktioner
\item funktioner som äkta värden
\item anonyma funktioner
\item klammerparentes vid ensam paramenter
\item uppdelad parameterlista
\item namnanrop (fördröjd evaluering)
\item egendefinierade kontrollstrukturer
\item stegade funktioner (''Curry-funktioner'')
\item fångad variablelrymd (''closure'')
\end{itemize}
\end{SlideExtra}
\fi

\begin{Slide}{Begränsningar i Java}\SlideFontTiny
\begin{itemize}
\item[] Av alla dessa funktionsprogrammeringskoncept...
\begin{itemize}\SlideFontTiny
\item överlagring
\item utelämna tom parameterlista (enhetlig access)
\item defaultargument
\item namngivna argument
\item lokala funktioner
\item funktioner som äkta värden
\item anonyma funktioner
\item klammerparentes vid ensam paramenter
\item uppdelad parameterlista
\item egendefinierade kontrollstrukturer
\item namnanrop (fördröjd evaluering)
\item stegade funktioner (''Curry-funktioner'')
\item fångad variablelrymd (''closure'')
\end{itemize}
...kan man i Java 8 endast göra:
\item[] \Emph{överlagring} ("overloading") och \Emph{anonyma funktioner} (''lambda'') \\ där set senare har starka begränsningar. \footnote{\SlideFontTiny\href{https://en.wikipedia.org/wiki/Anonymous_function\#Java_Limitations}{en.wikipedia.org/wiki/Anonymous\_function\#Java\_Limitations}}
%\item \vspace{0.5em} En av de saker jag saknar mest i Java: \Alert{lokala funktioner}!
\item[] Det är \Alert{kombinationen} av alla koncept som \Alert{skapar uttryckskraften} i Scala.
\end{itemize}


\end{Slide}


\Subsection{Kort om rekursion}

\begin{Slide}{Rekursiva funktioner}
\begin{itemize}
\item Funktioner som \Alert{anropar sig själv} kallas \Emph{rekursiva}.


\begin{REPLnonum}
scala> def fakultet(n: Int): Int =
         if (n < 2) 1 else n * fakultet(n - 1)

scala> fakultet(5)
res0: Int = 120
\end{REPLnonum}

\item För varje nytt anrop läggs en ny aktiveringspost på stacken.

\item I aktiveringsposten sparas varje returvärde som gör att \code{5 * (4 * (3 * (2 * 1)))} kan beräknas.

\item Rekrusionen avbryts när man når \Emph{basfallet}, här \code{n < 2}

\item En rekursiv funktion \Alert{måste} ha en returtyp.

\end{itemize}

\end{Slide}

\begin{Slide}{Loopa med rekursion}
\begin{Code}
def gissaTalet(max: Int, min: Int = 1): Unit = {
  def gissat = 
    io.StdIn.readLine(s"Gissa talet mellan [$min, $max]: ").toInt

  val hemlis = (math.random * (max - min) + min).toInt

  def skrivLedtrådOmEjRätt(gissning: Int): Unit =
    if (gissning > hemlis) println(s"$gissning är för stort :(")
    else if (gissning < hemlis) println(s"$gissning är för litet :(")

  def ärRätt(gissning: Int): Boolean = {
    skrivLedtrådOmEjRätt(gissning)
    gissning == hemlis
  }

  def loop(n: Int = 1): Int = if (ärRätt(gissat)) n else loop(n + 1)

  println(s"Du hittade talet $hemlis på ${loop()} gissningar :)")
}
\end{Code}

\end{Slide}


\begin{Slide}{Rekursiva datastrukturer}

\begin{itemize}
\item Datastrukturena Lista och Träd är exempel på datastrukturer som passar bra ihop med rekursion.
\item Båda dessa datastrukturer kan beskrivas rekursivt:
\begin{itemize}
\item En lista består av ett huvud och en lista, som i sin tur består av ett huvud och en lista, som i sin tur...
\item Ett träd består av grenar till träd som i sin tur består av grenar till träd som i sin tur, ...
\end{itemize}
\item Dessa datastrukturer bearbetas med fördel med rekursiva algoritmer.
\item I denna kursen ingår rekursion endast ''för kännedom'': \\ du ska veta vad det är och kunna skapa en enkel rekursiv funktion, t.ex. fakultets-beräkning. Du kommer jobba mer med rekursion och rekursiva datastrukturer i fortsättningskursen.
\end{itemize}
\end{Slide}

\Subsection{Automatisera kompilering: byggverktyg}

\begin{Slide}{Bygga applikationer}
\begin{itemize}
  \item Den kreativa programmeringsprocessen innehåller många korta cykler av koda, ändra, testa.
  \item Det blir många omkompileringar och då vill man gärna slippa skriva samma kommando om och om igen. En lösning är att skapa ett skript, t.ex. i språket \Emph{bash}, som kör kompileringen.
  \item  Om man bara gör en liten ändring vill man bara kompilera om det som ändrats och inte kompilera om rubbet varje gång. En lösning på detta problem är att använda ett \Emph{byggverktyg}, t.ex. Scala Build Tool (\code{sbt}), se Appendix F.

\end{itemize}
\end{Slide}

\begin{Slide}{Bash-skript för kompilering}\SlideFontSmall
\begin{itemize}
  \item Det gamla skriptspråket \Emph{bash} funkar i Linux och MacOS.
  \item Bash är smidigt för enkla program som använder terminalkommando, men syntaxen är knepig och det finns många fallgropar.
  \item I ett bash-skript kan du t.ex. kompilera och köra ett program. Exempel i filen \code{build.sh} nedan:
\begin{Code}
scalac mitt-program.scala && scala MinMain
\end{Code}
Med pil-upp kan du enkelt kompilera om efter varje ändring:
\begin{REPLnonum}
> sh build.sh
\end{REPLnonum}
  \item Det går att få \Emph{bash} och ubuntu-terminalen att funka i Windows 10 med WSL (Windows Linux Subsystem) där du kan välja Ubuntu 18.04 LTS: \\
  {\SlideFontTiny\url{https://docs.microsoft.com/en-us/windows/wsl/install-win10}}
\end{itemize}
{\noindent   Det finns dock stora begränsningar med WSL och om du vill ha Linux ''på riktigt'' rekommenderas att du installera Ubuntu med dual-boot: \SlideFontTiny\url{https://linoxide.com/distros/install-ubuntu-18-04-dual-boot-windows-10/}}
\end{Slide}

\begin{Slide}{Scala Build Tool: \texttt{sbt}}
\begin{itemize}
  \item Ett byggverktyg, t.ex. \code{sbt}, kan användas för att kompilera, testköra, ladda ner, paketer, distribuera programbibliotek och applikationer.
  \item Det är mycket enkelt att använda \code{sbt} för att kompilera och köra om ditt program varje gång du sparar din fil, tex. med \code{Ctrl+S} så här:
\begin{REPLnonum}
> sbt
sbt> ~run   // tecknet ~ ger omkörning vid ändring
\end{REPLnonum}
Tecknet \code{~} kallas \emph{tilde} och skrivs med högra Alt-tangenten nere och två tryck på tangenten bredvid Enter.
  \item LTH:s datorer har \code{sbt} förinstallerat. Ladda ner till din dator: \\
  \url{https://www.scala-sbt.org/download.html}
  \item Läs mer om byggverktyg i Appendix F.

\end{itemize}

\end{Slide}


\ifkompendium\else
\Subsection{Kommande vecka}

\begin{SlideExtra}{Mål med övning \ExeWeekTHREE}
\begin{itemize}\SlideFontSmall
  %!TEX encoding = UTF-8 Unicode
%!TEX root = ../exercises.tex

\item Kunna skapa och använda funktioner med en eller flera parametrar, default-argument, och namngivna argument.
\item Kunna förklara nästlade funktionsanrop med aktiveringsposter på stacken.
\item Kunna förklara skillnaden mellan äkta och ''oäkta'' funktioner.
\item Kunna applicera en funktion på alla element i en samling.

\item Kunna använda funktioner som äkta värden.
\item Kunna skapa och använda anonyma funktioner (ä.k. lambda-funktioner).

\item Känna till att funktioner kan ha uppdelad parameterlista.
\item Känna till att det går att partiellt applicera argument på funktioner med uppdelad parameterlista för att skapa s.k. stegade funktioner (ä.k. curry-funktioner).

\item Känna till rekursion och kunna beskriva vad som kännetecknar en rekursiv funktion.
%\item Känna till att man kan loopa med rekursion och att svansrekursiva funktioner kan optimeras till while-loopar.

\item Känna till att det går att skapa egna kontrollstrukturer med hjälp av namnanrop.
\item Känna till skillnaden mellan värdeanrop och namnanrop.
\item Kunna tolka en stack trace.

\end{itemize}
\end{SlideExtra}

\begin{SlideExtra}{Mål med laboration \LabWeekTHREE}
\begin{itemize}
  %!TEX encoding = UTF-8 Unicode
%!TEX root = ../compendium2.tex

%\item Kunna kompilera Scalaprogram med \texttt{scalac}.
%\item Kunna köra Scalaprogram med \texttt{scala}.
%\item Kunna definiera och anropa funktioner.
%\item Kunna använda och förstå default-argument.
%\item Kunna ange argument med parameternamn.
\item Kunna skapa ett större program med din egen kod efter dina egna idéer.
\item Kunna använda en editor och terminalen för att iterativt editera, kompilera, och testa din kod.
\item Kunna använda variabler i kombination med alternativ och repetetition i flera nivåer.
\item Kunna stegvis förbättra din kod för att underlätta förändring och öka läsbarhet.
\item Kunna skapa och använda abstraktioner för att generalisera och möjliggöra återanvändning av kod.

\end{itemize}
Ni ska spela \Emph{varandras} textspel i din \Alert{samarbetsgrupp}.\\
Läs labbinstruktioner:\url{http://cs.lth.se/pgk/compendium/}
\end{SlideExtra}


\begin{SlideExtra}{Tips inför ditt textspel.}
\begin{Code}
"Yes".toLowerCase.startsWith("y")    // true
"hejsan".contains("ejsa")            // true
"42".toInt                           // 42

val i = 42
s"Livets mening är $i!" // dollar $ före namn vid stränginterpolering med s""
s"Livets mening är inte ${i-1}!"  // klamrar ${} vid evaluering av uttryck

"""|en sträng som spänner över
   |flera rader där marginalen fram till vertikalstreck
   |är bortplockad med stripMargin (kan kombineras med s-interpolatorn)
""".stripMargin

math.random < 0.8                  // true i 80% av fallen
scala.util.Random.nextInt(42)      // ger slumptal mellan 0 och 41
scala.io.StdIn.readLine("prompt>") // ger sträng som användaren skriver

try { "?".toInt } catch { case e: Exception => 42 }  // förhindrar krasch
Thread.sleep(1000)    // sova i 1000 milliskeunder
\end{Code}
Kolla snabbreferensen vad mer du kan göra med strängar!
\end{SlideExtra}

\begin{SlideExtra}{Exempel på en början till ett textspel}
  Här finns en exempel på en enkel \emph{början} på ett textspel som du stegvis kan ändra och bygga ut till något du själv vill göra:
  \url{https://github.com/lunduniversity/introprog/tree/master/workspace/w03_irritext}

\begin{itemize}
  \item Vilka begrepp och principer ger koden träning i?
\end{itemize}

\end{SlideExtra}
\fi
