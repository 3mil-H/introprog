%!TEX encoding = UTF-8 Unicode
%!TEX root = ../lect-w07.tex

%%%


\Subsection{Implementera insert, remove, append}


\begin{Slide}{Exempel: PolygonWindow}\SlideFontTiny
\setlength{\leftmargini}{0pt}
\begin{itemize}
\item En polygon kan representeras som en punktsekvens, där varje punkt är ett heltalspar.

\item \code{PolygonWindow} nedan är ett fönster som kan rita en polygon.
\end{itemize}

\vspace{-0.5em}\scalainputlisting[numbers=left,numberstyle=,basicstyle=\fontsize{6.5}{8}\ttfamily\selectfont]{../compendium/examples/sequences/PolygonWindow.scala}
\pause
\vspace{-0em}\scalainputlisting[numbers=left,numberstyle=,basicstyle=\fontsize{6.5}{8}\ttfamily\selectfont]{../compendium/examples/sequences/PolygonTest.scala}
\end{Slide}

\begin{Slide}{Typ-alias för att abstrahera typnamn}\SlideFontSmall
Med hjälp av nyckelordet \code{type} kan man deklarera ett \Emph{typ-alias} för att ge ett \Alert{alternativt} namn till en viss typ. Exempel:
\begin{REPL}
scala> type Pt = (Int, Int)

scala> def distToOrigo(pt: Pt): Int = math.hypot(pt._1, pt._2)

scala> type Pts = Vector[Pt]

scala> def firstPt(pts: Pts): Pt = pts.head

scala> val xs: Pde3,3))

scala> firstPt(xde
res0: Pt = (1,1)de
\end{REPL}

Typ-alias kan vara bra när:
\begin{itemize}
\item man har en lång och krånglig typ och vill använda ett kortare namn,

\item man vill kunna lätt byta implementation senare\\(t.ex. om man vill använda en case-klass i stället för en tupel).
\end{itemize}
\end{Slide}


\begin{Slide}{Exempel: SEQ-INSERT/REMOVE-COPY}
Nu ska vi ''uppfinna hjulet'' och som träning implementera \Emph{insättning} och \Emph{borttagning} till en \Alert{ny} sekvens utan användning av sekvenssamlingsmetoder (förutom \code{length} och \code{apply}):
\begin{Code}
object PointSeqUtils:
  type Pt = (Int, Int)  // a type alias to make the code more concise

  def primitiveInsertCopy(pts: Array[Pt], pos: Int, pt: Pt): Array[Pt] = ???

  def primitiveRemoveCopy(pts: Array[Pt], pos: Int): Array[Pt] = ???
\end{Code}
\end{Slide}




\begin{Slide}{Pseudo-kod för SEQ-INSERT-COPY}\SlideFontSmall
\begin{algorithm}[H]
 \SetKwInOut{Input}{Indata}\SetKwInOut{Output}{Resultat}

 \Input{\texttt{pts: Array[Pt], pt: Pt, pos: Int}} ~\\

 \Output{En kopia av $pts$ men där $pt$ är infogat på plats $pos$}~\\


 \noindent\hrulefill

  $result \leftarrow$ en ny \texttt{Array[Pt]} med plats för $pts.length + 1$ element \\
 \For{$i \leftarrow 0$ \KwTo $pos - 1$}{
  $result(i) \leftarrow pts(i)$
 }
 $result(pos) \leftarrow pt$ \\
 \For{$i \leftarrow pos + 1$ \KwTo $xs.length$}{
  $result(i) \leftarrow xs(i - 1)$
 }

 \Return $result$

  \noindent\hrulefill\\
\end{algorithm}
\pause\vspace{0.5em}\Emph{Övning}: Skriv pseudo-kod för SEQ-REMOVE-COPY
\end{Slide}

\begin{Slide}{Insättning/borttagning i kopia av primitiv Array}
\vspace{-0.6em}\scalainputlisting[numbers=left,numberstyle=,basicstyle=\fontsize{6}{7.2}\ttfamily\selectfont]{../compendium/examples/sequences/PointSeqUtils.scala}

\pause
\SlideFontSmall Man gör \Alert{mycket lätt fel} på gränser/specialfall: +-1, to/until, tom sekvens etc.
\end{Slide}

% \begin{Slide}{Exempel: Test av SEQ-INSERT/REMOVE-COPY}
% \vspace{-0.6em}\scalainputlisting[numbers=left,numberstyle=,basicstyle=\fontsize{6.5}{8}\ttfamily\selectfont]{../compendium/examples/workspace/w05-seqalg/src/polygonTest2.scala}
% \end{Slide}

% \begin{Slide}{Exempel: Göra insättning med take/drop}\SlideFontSmall
% Om du inte vill ''uppfinna hjulet'' och inte använda \code{patch} kan du göra så här: \\Använd \code{take} och \code{drop} tillsammans med \code{:+} och \code{++} \\Du kan också göra insättningen generiskt användbar för alla sekvenser:
% \begin{REPLnonum}
% scala> val xs = Vector(1,2,3)
% xs: scala.collection.immutable.Vector[Int] =
%   Vector(1, 2, 3)
%
% scala> val ys = (xs.take(2) :+ 42) ++ xs.drop(2)
% ys: scala.collection.immutable.Vector[Int] =
%   Vector(1, 2, 42, 3)
%
% scala> def insertCopy[T](xs: Seq[T], elem: T, pos: Int) =
%         (xs.take(pos) :+ elem) ++ xs.drop(pos)
%
% scala> insertCopy(xs, 42, 2)
% res0: Seq[Int] = Vector(1, 2, 42, 3)
%
% \end{REPLnonum}
% \Emph{Övning}: Implementera \code{insertCopy[T]} med \code{patch} istället.
% \end{Slide}
