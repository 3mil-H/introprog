%!TEX encoding = UTF-8 Unicode
%!TEX root = ../lect-week07.tex

%%%

\ifkompendium\else



\Subsection{Nästa vecka: kontrollskrivning}
\begin{Slide}{Obl. kontrollskrivning: 25/10 kl 14:00-19 Gasquesalen}\SlideFontSmall
Kontrollskrivningen motsvarar i omfång en \Alert{halv} ordinarie tentamen och är uppdelad i två delar; del A och del B. 
\begin{itemize}\SlideFontTiny

\item Del A omfattar $20\%$ av den maximala poängsumman och innehåller uppgifter med korta svar (likt övningarna): ''ange typ och värde''.
\item Del B omfattar $80\%$ av den maximala poängsumman och innehåller uppgifter med svar i form av kod.

\item Maximal poäng på kontrollskrivningen är 50p. (Ordinarie tenta 100p)

\item Om du erhåller \code{p} poäng på kontrollskrivningen bidrar du med \code{(p / 10.0).round} i individuell bonuspoäng inför sammanräkningen av samarbetsbonus. 


\item Din samarbetsbonus är medelvärdet av poängen från dig och de av dina gruppmedlemmar som skriver kontrollskrivningen enligt denna beräkning:
\begin{Code}
  def collaborationBonus(points: Seq[Int]): Int =
    (points.sum / points.size.toDouble).round.toInt
\end{Code}

\item Samarbetsbonus motsvarar max 5\% av totala ordinarie tentapoäng.


\item Samarbetsbonus påverkar inte om du blir godkänd på tentan, men kan påverka vilket betyg du får. 


\end{itemize}
\end{Slide}


\begin{Slide}{Obligatorisk kontrollskrivning: instruktioner}
\Emph{Medtag}: \\ legitimation, penna blyerts, penna i avvikande färg helst röd, \\ ev. förtäring och \Alert{Scala Quickref/Java Snabbref}.

\vspace{2em}\begin{itemize}
\item Moment 1, ca 2,5h: \Emph{Lösning av uppgifterna}. 
\begin{itemize} 
\item Du löser uppgifterna individuellt med blyertspenna. 
\item När du är klar lämnar du in alla dina svar. 
\end{itemize}
\item Moment  2: \Emph{Parvis kamraträttning}.
\item Moment 3: \Emph{Bedömning av rättning}.
\end{itemize}
\end{Slide}

\begin{Slide}{Obligatorisk kontrollskrivning: instruktioner}
\begin{itemize}
\item Moment 1, ca 2,5h: \Emph{Lösning av uppgifterna}. 
 
\item Moment  2: \Emph{Parvis kamraträttning}. 
\begin{itemize}

\item
Ni  sätter  er  parvis  och  får  ut  rättningsmallen  som  ni läser. 

\item Efter  ca  10  minuter  får  ni  ut  två andra  personers  skrivningar  som  ni  rättar enligt  anvisningarna  i  rättningsmallen.  

\item Medtag och använd penna med avvikande färg, helst röd.

\item När  rättningstiden  är  slut  samlar  vi in alla rättade skrivningarna.
\end{itemize}
\pause
\item Moment 3: \Emph{Bedömning av rättning}. 
\begin{itemize}
\item Du får hämta din egen skrivning och titta på rättningen. 
\item Är  du  nöjd  med  rättningen  lämnar  du  bara tillbaka  skrivningen igen. 
\item Är du inte nöjd med rättningen kontaktar du skrivningsansvarig genom handuppräckning.
\end{itemize}
\end{itemize}
\end{Slide}

\begin{Slide}{Plugga på kontrollskrivning}
\begin{itemize}
\item Träffas och plugga i samarbetsgrupperna.
\item Hjälp varandra med det som är svårt.
\item Träna på att skriva kod på papper.
\item Gör övningarna.
\item Repetera laborationerna.
\item Läs föreläsningsanteckningarna.
\item Studera Scala Quickref MYCKET NOGA så att du vet vad som är givet och var det står så att du kan hitta det du behöver snabbt.
\item Se sidan 329 i kompendiet (tips inför ordinarie tenta)
\end{itemize}
\end{Slide}







\Subsection{Veckans uppgifter}


\begin{Slide}{Övning: \texttt{traits}, ev. prioritera uppg. 1, 2, 3, 6, 10, 11}
\begin{itemize}\SlideFontTiny
%!TEX encoding = UTF-8 Unicode

%!TEX root = ../compendium.tex

\item Förstå följande begrepp: supertyp, subtyp, bastyp, abstrakt typ, polymorfism. 
\item Kunna deklarera och använda en arvshierarki i flera nivåer med nyckelordet \code{extends}.
\item Kunna deklarera och använda inmixning med flera traits och nyckelordet \code{with}.
\item Kunna deklarera och känna till nyttan med finala klasser och finala attribut och nyckelordet \code{final}.  
\item Känna till synlighetsregler vid arv och nyttan med privata och skyddade attribut.
\item Kunna deklarera och använda skyddade attribute med nyckelordet \code{protected}.
\item Känna till hur typtester och typkonvertering vid arv kan göras med metoderna \code{isInstanceOf} och \code{asInstanceOf} och känna till att detta görs bättre med \code{match}.
\item Känna till begreppet anonym klass.
\item Kunna deklarera och använda överskuggade metoder med nyckelordet \code{override}.
\item Känna till reglerna som gäller vid överskuggning av olika sorters medlemmar.
\item Kunna deklarera och använda hierarkier av klasser där konstruktorparametrar överförs till superklasser. 
\item Kunna deklarera och använda uppräknade värden med case-objekt och gemensam bastyp.
    
\end{itemize}
\end{Slide}

\begin{Slide}{Instruktioner Grupplaboration}
\begin{itemize}\SlideFontSmall
%!TEX encoding = UTF-8 Unicode
%!TEX root = compendium.tex
\item 
Diskutera i din samarbetsgrupp hur ni ska dela upp koden mellan er i flera olika delar, som ni kan arbeta med var för sig. En sådan del kan vara en klass, en trait, ett objekt, ett paket, eller en funktion. 
\item
Varje del ska ha en \emph{huvudansvarig} individ. 
\item
Arbetsfördelningen ska vara någorlunda jämnt fördelad mellan gruppmedlemmarna.
\item
När ni redovisar er lösning ska ni börja med att redogöra för handledaren hur ni delat upp koden och vem som är huvudansvarig för vad. 
\item
Den som är huvudansvarig för en viss del redovisar den delen.
\item 
Grupplaborationer görs i huvudsak som hemuppgift. Salstiden används primärt för redovisning.
\end{itemize}
\end{Slide}

\begin{Slide}{Grupplaboration: \texttt{turtlerace-team}}
\begin{itemize}
%!TEX encoding = UTF-8 Unicode

%!TEX root = ../compendium.tex

\item Kunna skapa och använda arvshierarkier och förstå dynamisk bindning.
\item Kunna skapa använda en trait som bastyp i en arvshierarki.

\end{itemize}
\end{Slide}



\fi

