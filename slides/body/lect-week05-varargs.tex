%!TEX encoding = UTF-8 Unicode
%!TEX root = ../lect-week05.tex

%%%

\ifkompendium\else

\Subsection{Variabelt antal argument, ''varargs''}\SlideFontSmall

\begin{Slide}{Parameter med variabelt antal argument, ''varargs''}
Med en asterisk efter parametertypen kan antalet argument variera:
\begin{Code}[basicstyle=\fontsize{10}{12}\selectfont\ttfamily]
def sumSizes(xs: String*): Int = xs.map(_.size).sum
\end{Code}
\begin{REPLnonum}
scala> sumSizes("Zaphod")
res0: Int = 6

scala> sumSizes("Zaphod","Beeblebrox")
res1: Int = 16

scala> sumSizes("Zaphod","Beeblebrox","Ford","Prefect")
res3: Int = 27

scala> sumSizes()
res4: Int = 0
\end{REPLnonum}
Typen på \code{xs} blir en \code{Seq[String]}, egentligen en \code{WrappedArray[String]} som kapslar in en array så den beter sig mer som en ''vanlig'' Scala-samling.
\end{Slide}

\begin{Slide}{Sekvenssamling som argument till varargs-parameter}
\begin{Code}[basicstyle=\fontsize{10}{12}\selectfont\ttfamily]
def sumSizes(xs: String*): Int = xs.map(_.size).sum

val veg = Vector("gurka","tomat")
\end{Code}
Om du \emph{redan har} en sekvenssamling så kan du applicera den på en parameter som accepterar variabelt antal argument med typannoteringen \\ {\vspace{1em}\Large\code{: _* }} \\ \vspace{1em}direkt \Alert{efter} sekvenssamlingen.
\begin{REPLnonum}
scala> sumSizes(veg: _*)
res5: Int = 10
\end{REPLnonum}

\end{Slide}

\fi







