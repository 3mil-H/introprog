%!TEX encoding = UTF-8 Unicode
%!TEX root = ../lect-week13.tex

%%%

\ifkompendium\else

\Subsection{Grumligt-lådan}
\begin{Slide}{Översikt av innehållet i Grumligt-lådan}\SlideFontSmall
Ämnen (antal)
\begin{multicols}{2}
\begin{itemize}\SlideFontTiny
\item arv (4)
\item getter, setter (4)
\item kompanjonsobjekt (4)
\item case-objekt (4)
\item try (4)
\item java (3)
\item matriser (3)
\item sortering (3)
\item ArrayBuffer (2)
\item loopar (2)
\item Map och map (2)
\item option (2)
\item problemlösning (2)
\item (1) \\
funktionsvärden; 
generiska funktioner;
groupBy;
in-mixning;
klasser och case-klasser;
konstanter;
konstruktor;
läsa från textfil;
läsa kod;
match case;
objektfabriksmetod;
pirateslabben;
private[this];
sortBy;
static;
type;
typparameter;

\end{itemize}
\end{multicols}
\end{Slide}

\Subsection{Nyfiken-på-lådan}
\begin{Slide}{Översikt av innehållet i Nyfiken-på-lådan}\SlideFontSmall
Ämnen (antal)
\begin{multicols}{2}
\begin{itemize}\SlideFontTiny
\item gränssnitt (3)
\item Java (3)
\item rekursion (3)
\item funktionsprogrammering (2)
\item generiska typer (2)
\item implicit (2)
\item trådar, Future (2)
\item webb, html (2)
\item (1) \\
bilder, ljud och spara filer; 
enkel AI; 
minneshantering i olika språk (GC eller manuell);
gå igenom och förklara QuickRef mer noggrant;
hacka andras kod i låsta applikationer;
hashcode;
kryptering;
prestanda och minnesåtgång Scala vs Java;
Stream[T];
teorin bakom neurala nätverk;
\end{itemize}
\end{multicols}
\end{Slide}

\Subsection{Repetition diverse}

\begin{Slide}{Oföränderlig punkt  i Scala och Java}\SlideFontSmall
I Scala: (utan case-klass ingen najs \code{toString} och måste skriva \code{new}, etc.)
\begin{Code}
class Point(val x: Int, val y: Int)
\end{Code}

I Java:
\begin{Code}[language=Java,basicstyle=\ttfamily\SlideFontSize{5.8}{7}]
public class JPoint {
    private int x;
    private int y;

    public JPoint(int x, int y){
        this.x = x;
        this.y = y;
    }  
    
    public int getX(){
        return x;
    }
    
    public int getY(){
        return y;
    }
}
\end{Code}
\end{Slide}

\begin{Slide}{Föränderlig punkt  i Scala och Java}\SlideFontSmall
I Scala: 
\begin{Code}
class Point(var x: Int, var y: Int)
\end{Code}

I Java:
\begin{Code}[language=Java,basicstyle=\ttfamily\SlideFontSize{5.2}{6}]
public class JPoint {
    private int x;
    private int y;

    public JPoint(int x, int y){
        this.x = x;
        this.y = y;
    }  
    
    public int getX(){
        return x;
    }
    
    public int getY(){
        return y;
    }
    
    public void setX(int x){
        this.x = x;
    }
    
    public void setY(int y){
        this.y = y;
    }    
}
\end{Code}
\end{Slide}

\begin{Slide}{Punkt med räknare och setter i Scala}
Övning: lägg till getter och setter för y-koordinaten.
\begin{Code}
class Point(private var myX: Int, private var myY: Int){
  import Point._
  def x = myX
  def x_=(newX: Int): Unit = {
    myX = newX
  }
  myCount += 1   // kod i klasskroppen körs vid konstruktion
}

object Point {
  private var myCount = 0
  def count = myCount
}
\end{Code}
\SlideFontSmall
Man brukar kalla privata attribut som har getter (och ev. setter) för något i stil med \code{myX} eller vanligare \code{_x} för att namnet inte ska krocka med getter/setter.
\end{Slide}



\begin{Slide}{Punkt med räknare i Java}

\begin{Code}[language=Java,basicstyle=\ttfamily\SlideFontSize{5.2}{6}]
public class JPoint {
    private int x;
    private int y;
    
    static private int count = 0;  // static: finns bara en upplaga av detta attribut
    
    public JPoint(int x, int y){
        this.x = x;
        this.y = y;
        count++;
    }  
    
    public int getX(){
        return x;
    }
    
    public int getY(){
        return y;
    }
    
    public void setX(int x){
        this.x = x;
    }
    
    public void setY(int y){
        this.y = y;
    }
    
    static public int getCount(){
       return count;
    }     
}
\end{Code}


\end{Slide}

\begin{Slide}{Robot}
Följ Robot-exemplet från förra årets Java-kurs här: \\\vspace{1em}
\url{https://github.com/bjornregnell/lth-eda016-2015/blob/master/lectures/notes/week05.pdf}
\\\vspace{2em}


Övning: översätt till Scala.
\end{Slide}

\fi










