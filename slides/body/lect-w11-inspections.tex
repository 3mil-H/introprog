%!TEX encoding = UTF-8 Unicode
%!TEX root = ../lect-w11.tex

\Subsection{Kodgranskning}

\begin{Slide}{Kodgranskning}
\begin{itemize}
\item Ett effektivt sätt att upptäcka fel är att människor \Emph{noga läser igenom} sin egen och andras kod, och försöker hitta relevanta \Alert{problem och förbättringsmöjligheter}. 
\item Man blir ofta ''hemmablind'' när det gäller ens egen kod. Därför kan någon annans, oberoende granskning med ''nya, friska'' ögon vara mycket fruktbar. 
\item I samband med kodgranskning kan man med fördel försöka bedöma  huruvida koden är:
\begin{itemize}
\item lätt att läsa, 
\item lätt att ändra i,  
\item annat som är viktigt för den framtida utvecklingen.
\end{itemize}
\item Ofta hittar man vid granskning även enkla programmeringsmisstag, så som felaktiga villkor och loop-räknare som inte räknas upp på rätt sätt etc.
\end{itemize}
\end{Slide}

\begin{Slide}{Nyttan med kodgranskningar}
Väl genomförda kodgranskningar är effektiva och nyttiga:
\begin{itemize}
\item Bra på att upptäcka problem, även sådana som varken kompilator eller testning hittar.
\item Sprider kunskap inom en arbetsgrupp, speciellt mellan erfarna och juniora utvecklare.
\item En god kodgranskningsprocess främjar en kultur av gemensamt ägarskap och respektfull samverkan.
\end{itemize}
\end{Slide}


\begin{Slide}{Kodgranskning under grupplaborationen}
Under grupplaborationen ska du:
\begin{itemize}
\item Delta i framtagande av en gemensam checklista för kodgranskning
\item Granska minst en annan gruppmedlems kod 
\item Bjuda in minst en annan gruppmedlem att granska din kod
\item Ge konstruktiv feedback
\item Ta emot konstruktiv feedback och försöka förbättra din kod
\item På redovisning redogöra för nyttan och utmaningarna med kodgranskningar utifrån dina egna erfarenheter
\end{itemize}
Mer om kodgranskning i efterföljande kurser, t.ex. ''Programvaruutveckling i grupp'' och ''Programvaruutveckling för stora system''
\end{Slide}
