%!TEX encoding = UTF-8 Unicode
%!TEX root = ../lect-w10.tex

%%%

\Subsection{Mer om \texttt{equals}}
\begin{Slide}{Fördjupning: Implementera \texttt{equals} med \texttt{match}}
Det visar sig att innehållslikhet är förvånansvärt komplicerat att implementera i samband med arv.
\begin{itemize}
\item Det enklare fallet: Gör övning \code{matching:12} och implementera \code{equals} för innehållslikhet utan arv. \\ En bra träning på att använda \code{match}!

\item Svårare: Gör fördjupningsövning \code{matching:19} om du vill se hur en \emph{komplett} \code{equals} ska se ut som funkar i alla lägen.

\end{itemize}

Det ingår inte på tentan att själv kunna implementera en generellt fungerande \code{equals}. Men du ska förstå skillnaden mellan referenslikhet och innehållslikhet. Mer om \code{equals} i fortsättningkursen.
\end{Slide}











