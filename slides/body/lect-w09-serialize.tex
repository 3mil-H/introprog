%!TEX encoding = UTF-8 Unicode
%!TEX root = ../lect-w09.tex


\Subsection{Serialisering och deserialisering}

\begin{Slide}{Serialisering och deserialisering}
\begin{itemize}
  \item Att \Emph{serialisera} innebär att \Alert{koda objekt} i minnet till en avkodningsbar \Alert{sekvens av symboler}, som kan lagras t.ex. i en fil på din hårddisk.
  \item Att \Emph{de-serialisera} innebär att \Alert{avkoda en sekvens av symboler}, t.ex. från en fil, och \Alert{återskapa objekt} i minnet.
\end{itemize}
\end{Slide}


\begin{Slide}{Läsa text från fil och URL}\SlideFontSmall
I paketet \code{scala.io} finns singelobjektet \code{Source} med metoderna \code{fromFile} och \code{fromUrl} för läsning från fil resp. från  URL, alltså Universal Resource Locator, som börjar t.ex. med \texttt{http://}
\begin{Code}
def läsFrånFil(filnamn: String): String = 
  val s = scala.io.Source.fromFile(filnamn)
  try s.mkString finally s.close // säkerställ stängning även vid krasch

def läsRaderFrånFil(filnamn: String): Vector[String] =
  val s = scala.io.Source.fromFile(filnamn)
  try s.getLines.toVector finally s.close 

def läsFrånWebbsida(url: String): String = 
  val s = scala.io.Source.fromURL(url)
  try s.mkString finally s.close

def läsRaderWebbsida(url: String, kodning: String = "UTF-8"): Vector[String] =
  val s = scala.io.Source.fromURL(url, kodning)  // läs med given kodning
  try s.getLines.toVector finally s.close 

\end{Code}
Se vidare veckans övning. Exempel på annan kodning: \code{"ISO-8859-1"}
\end{Slide}


\begin{Slide}{Serialisering i modulen \texttt{introprog.IO}}
  \begin{itemize}
    \item I kursens kodbibliotek \code{introprog} finns ett singelobjekt \code{IO} som samlar smidiga funktioner för serialisering och de-serialisering. 
    \item Se api-dokumentation här: \\ \url{http://cs.lth.se/pgk/api/} \\ Sök på IO och klicka på singelobjektet.
    \item Se koden här:\\
    \url{https://github.com/lunduniversity/introprog-scalalib/blob/master/src/main/scala/introprog/IO.scala}
    \item Om du vill får du gärna använda \code{introprog.IO} istället för \code{scala.io.Source} på labben.  
  \end{itemize}
\end{Slide}
