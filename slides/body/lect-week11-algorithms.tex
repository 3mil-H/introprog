%!TEX encoding = UTF-8 Unicode
%!TEX root = ../lect-week11.tex

%%%

\ifkompendium\else
\subsection{Repetition: Vad är en algoritm?}
\begin{Slide}{Repetition: Vad är en algoritm? }\SlideFontSmall
En \href{https://sv.wikipedia.org/wiki/Algoritm}{algoritm} är en stegvis beskrivning av hur man löser ett problem. \\ 
Exempel: Min/Max, Registrering, Sökning, Sortering \\
\vspace{1em}
Problemlösningsprocessens olika steg (inte nödvändigtvis i denna ordning): 
\begin{enumerate}
\item identifiera (del)\Emph{problemet}: \\ exempel: hitta minsta talet
\item Kom på en \Emph{lösningsidé}: (kan  vara mycket klurigt och svårt) \\ exempel: iterera över talen och håll reda på ''minst hittills''
\item Formulera en \Emph{stegvis beskrivning} som löser problemet: \\ exempel: pseudo-kod med sekvens av instruktioner
\item Implementera en \Emph{körbar lösning} i ''riktig'' kod: \\ exempel: en Java-metod i en klass
\end{enumerate}
Övning: Ge exempel per steg ovan för linjärsökning och registrering. \\
Det krävs ofta \Emph{kreativitiet} i stegen ovan  -- även i att \Emph{känna igen} problemet: \\ Exempel: skapa highscore-lista kräver dellösningen att hitta \emph{största} talet som är en variant av problemet ''hitta minsta talet'' som jag vet hur man kan lösa.
\end{Slide}


\fi











