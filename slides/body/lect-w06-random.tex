%!TEX encoding = UTF-8 Unicode
%!TEX root = ../lect-w06.tex

%%%

\Subsection{Slumptalssekvenser och slumptalsfrö}

\begin{Slide}{Klassen java.util.Random}\SlideFontTiny
\begin{itemize}
\item Om man använder slumptal kan det vara svårt att leta buggar, efter som det blir \Alert{olika varje gång} man kör programmet och buggen kanske bara uppstår ibland.

\item Med klassen \code{java.util.Random} kan man skapa \Emph{pseudo}-slumptalssekvenser.
\pause
\item Om man ger ett \Emph{frö} \Eng{seed} av typen \code{Long} som argument till konstruktorn när man skapar en instans av klassen \code{Random}, får man samma ''slumpmässiga'' sekvens \Alert{varje gång} man kör programmet.

\begin{Code}
  val seed = 42
  val rnd = new java.util.Random(seed)  // SAMMA sekvens varje körning
  val r = rnd.nextInt(6) // ger slumptal mellan 0 till och med 5
\end{Code}
\pause
\item Om man \Alert{inte} ger ett \Emph{frö} så sätts fröet till ''\emph{a value very likely to be distinct from any other invocation of this constructor}''. Då vet vi inte vilket fröet blir och det blir olika varje gång man kör programmet.
\begin{Code}
  val rnd = new java.util.Random  // OLIKA sekvens varje körning
  val r = rnd.nextInt(6) // ger slumptal mellan 0 till och med 5
\end{Code}
\pause
\item Studera dokumentationen för klassen \code{java.util.Random} här: \href{https://docs.oracle.com/javase/8/docs/api/java/util/Random.html}{\SlideFontSmall docs.oracle.com/javase/8/docs/api/java/util/Random.html}

\end{itemize}
\end{Slide}

\begin{Slide}{Syresättning av hjärnan vid sövande föreläsning}
Prova nedan kod som finns här:\\
%\href{https://github.com/lunduniversity/introprog/blob/master/compendium/examples/workspace/w05-seqalg/src/NanananananananaNanananananananaBatman.scala}{\SlideFontTiny github.com/lunduniversity/introprog/.../NanananananananaNanananananananaBatman.scala} \\



\vspace{0.65em}\scalainputlisting[numbers=left,numberstyle=,basicstyle=\fontsize{6.5}{8}\ttfamily\selectfont]{../compendium/examples/workspace/w05-seqalg/src/FixSleepyBrain.scala}

\pause
Medan du lyssnar till: \href{https://www.youtube.com/watch?v=zUwEIt9ez7M}{\SlideFontSmall www.youtube.com/watch?v=zUwEIt9ez7M}\\
Eller: \href{https://www.youtube.com/watch?v=rvXxlXg_V-k}{\SlideFontSmall www.youtube.com/watch?v=rvXxlXg\_V-k}
\end{Slide}
