%!TEX encoding = UTF-8 Unicode
%!TEX root = ../lect-week04.tex

\ifkompendium\else

\Subsection{Klasser}

\begin{Slide}{Vad är en klass?}\SlideFontSmall
Vi har tidigare deklarerat \Emph{singelobjekt} som bara finns i \Alert{en} \Emph{instans}:
\begin{REPLnonum}
scala> object Björn { var ålder = 49; val längd = 178 }
\end{REPLnonum}

Med en \Emph{klass} kan man skapa \Alert{godtyckligt många} \Emph{instanser av klassen} med hjälp av nyckelordet \code{new} följt av klassens namn:

\begin{REPLnonum}
scala> class Person { var ålder = 0; var längd = 0 }

scala> val björn = new Person
björn: Person = Person@7ae75ba6

scala> björn.ålder = 49

scala> björn.längd = 178
\end{REPLnonum}

\begin{itemize}

\item En klass kan ha \Emph{medlemmar} (i likhet med singelobjekt). 

\item Funktioner som är medlemmar kallas \Emph{metoder}.

\item Variabler som är medlemmar kallas \Emph{attribut}.


\end{itemize}

\end{Slide}




\begin{Slide}{En klass kan ha parametrar som initialiserar attribut}
\begin{itemize}
\item Med en parameterlista efter klassnamnet får man en så kallad \Emph{primärkonstruktor} för initialisering av attribut. 
\item Argumenten för initialiseringen ges vid \code{new}.
\begin{REPLnonum}
scala> class Person(var ålder: Int, var längd: Int)

scala> val björn = new Person(49, 178)
björn: Person = Person@354baab2

scala> println(s"Björn är ${björn.ålder} år gammal.")
Björn är 49 år gammal.

scala> björn.ålder = 18

scala> println(s"Björn är ${björn.ålder} år gammal.")
Björn är 18 år gammal.
\end{REPLnonum}
\end{itemize}
\end{Slide}




\begin{Slide}{En klass kan ha privata medlemmar}
Med nyckelordet \code{private} blir medlemmen privat och syns inte.

\vspace{0.1em}
\begin{REPL}
scala> class Person(private var minÅlder: Int, private var minLängd: Int){
         def ålder = minÅlder
       }

scala> val björn = new Person(42, 178)
björn: Person = Person@4b682e71

scala> println(s"Björn är ${björn.ålder} år gammal.")
Björn är 42 år gammal.

scala> björn.ålder = 18
<console>: error: value ålder_= is not a member of Person
       björn.ålder = 18

scala> björn.längd
<console>: error: value längd is not a member of Person
       björn.längd
\end{REPL}
Med \code{private} kan man förhindra tokiga förändringar.
\end{Slide}


\begin{Slide}{Default synlighet för klassparametrar: \texttt{private val}}
\end{Slide}


\begin{Slide}{Privata förändringsbara attribut och publika metoder}
\begin{Code}
class Människa(val födelseLängd: Double, val födelseVikt: Double){
  private var minLängd = födelseLängd
  private var minVikt  = födelseVikt
  private var ålder    = 0
    
  def längd = minLängd  // en sådan här metod kallas "getter"
  def vikt  = minVikt   // vi förhindrar attributändring "utanför" klassen
    
  val slutaVäxaÅlder      = 18
  val tillväxtfaktorLängd = 0.00001
  val tillväxtfaktorVikt  = 0.0002

  def ät(mat: Double): Unit = {
    if (ålder < slutaVäxaÅlder) minLängd += tillväxtfaktorLängd * mat
    minVikt += tillväxtfaktorVikt * mat
  }
  
  def fyllÅr: Unit = ålder += 1
  
  def tillstånd = s"Tillstånd: $minVikt kg, $minLängd cm, $ålder år"
}
\end{Code}
\end{Slide}

\begin{Slide}{Tillstånd förändras indirekt genom metodanrop}
\begin{REPL}
scala> val björn = new Människa(födelseVikt=3.5, födelseLängd=52.1)
björn: Människa = Människa3e52

scala> björn.tillstånd
res0: String = Tillstånd: 3.5 kg, 52.1 cm, 0 år

scala> for (i <- 1 to 42) björn.fyllÅr

scala> björn.tillstånd
res2: String = Tillstånd: 3.5 kg, 52.1 cm, 42 år

scala> björn.ät(mat=5000)

scala> björn.tillstånd
res3: String = Tillstånd: 4.5 kg, 52.1 cm, 42 år
\end{REPL}
\end{Slide}

\begin{Slide}{Metoden \texttt{isInstanceOf} och topptypen \texttt{Any}}
Any har metoden toString
\end{Slide}

\begin{Slide}{Överskugga \texttt{toString}}
\end{Slide}

\begin{Slide}{Objektfabrik i kompanjonsobjekt}
\end{Slide}


\begin{Slide}{Förändringsbara och oföränderliga objekt}
\begin{Code}
class Point(val x: Int, val y: Int) {
  def move(dx: Int, dy: Int): Point = new Point(x += dx, y += dy)
  
  override def toString: String = s"Point($x, $y)"
}

class MutablePoint(private var x: Int, private var y: Int) {
  def move(dx: Int, dy: Int): Unit = {x += dx; y += dy}
  
  override def toString: String = s"MutablePoint($x, $y)"
}

\end{Code}
\end{Slide}


\Subsection{Case-klasser}

\begin{Slide}{Vad är en case-klass?}\SlideFontSmall
\setlength{\leftmargini}{0pt}
\begin{itemize}
\item En \code{case}-klass är ett smidigt sätt att skapa \Emph{oföränderliga objekt}.
\item Kompilatorn ger dig \Alert{en massa ''godis''} på köpet (ca 50-100 rader kod), inkl.:
\begin{itemize}\SlideFontTiny
\item klassparametrar blir automatiskt \code{val}-attribut, alltså \Emph{publika} och \Emph{oföränderliga},
\item en automatisk \Emph{\texttt{toString}} som visar klassparametrarnas värde, 
\item ett automatiskt \Emph{kompanjonsobjekt} med \Emph{fabriksmetod} så du slipper skriva \code{new},
\item automatiska metoden \Emph{\texttt{copy}} för att skapa kopior med andra attributvärden, m.m...
\item[] (Mer om detta i w11, men är du nyfiken kolla på uppgift 2d) på sid 261.)
\end{itemize}

\item Det \Alert{enda} du behöver göra är att lägga till nyckelordet \code{case} före \code{class}...
\end{itemize}

\vspace{-0.5em}\begin{REPLnonum}
scala> case class Point(x: Int, y: Int)

scala> val p = Point(3, 5)
p: Point = Point(3,5)

scala> p.  // tryck TAB och se lite av allt case-klass-godis
scala> Point.  // tryck TAB och se ännu mer godis

scala> val p2 = p.copy(y=30)
p2: Point = Point(3,30)
\end{REPLnonum}


\end{Slide}


\fi




