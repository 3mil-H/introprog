%!TEX encoding = UTF-8 Unicode
%!TEX root = ../lect-w07.tex

%%%

\Subsection{Enumerationer}
\SlideFontSmall
\begin{Slide}{Enumerationer har en ordning}
En uppräkning av färger i en kortlek med \code{enum}:
\begin{Code}
enum Suit:
  case Spade, Heart, Club, Diamond 
\end{Code}
Användbara metoder för att hantera elementens \Emph{ordningen}:
\begin{REPLsmall}
scala> Suit.Heart              // alla element är av typen Suit 
val res5: Suit = Heart
  
scala> Suit.Spade.ordinal      // från element till heltal
val res0: Int = 0

scala> Suit.Club.ordinal
val res1: Int = 2

scala> Suit.fromOrdinal(3)    // från heltal till element
val res2: Suit = Diamond

scala> Suit.values            // alla element i ordning
val res3: Array[Suit] = Array(Spade, Heart, Club, Diamond)

scala> Suit.valueOf("Spade")  // från sträng till element
val res4: Suit = Spade
\end{REPLsmall}
\end{Slide}

\begin{Slide}{Enumerationer kan ha parametrar och medlemmar}
Efter enumerationstypen kan du ha parametrar.\\Använd \code{val} om du vill ha extern synlighet:  
\begin{Code}
enum Color(val consoleColor: String): 
  case Black extends Color(Console.BLUE)
  case Red   extends Color(Console.RED)
\end{Code}
I kroppen kan du ha medlemmar, tex metoder:
\begin{Code}
enum Suit(val color: Color):
  def show(isConsoleColor: Boolean = true): String = 
    if isConsoleColor then color.consoleColor + toString + Console.RESET
    else toString

  case Spade   extends Suit(Color.Black)
  case Heart   extends Suit(Color.Red)
  case Club    extends Suit(Color.Black) 
  case Diamond extends Suit(Color.Red)
\end{Code}
\begin{REPLsmall}
scala> println(Suit.Club.show(isConsoleColor = false)) 
Club
\end{REPLsmall}
  
\end{Slide}