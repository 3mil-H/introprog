%!TEX encoding = UTF-8 Unicode
%!TEX root = ../lect-w07.tex

%%%

\Subsection{Enumerationer}
\SlideFontSmall
\begin{Slide}{Enumerationer har en ordning}
En uppräkning av färger i en kortlek med \code{enum}:
\begin{Code}
enum Suit:
  case Spade, Heart, Club, Diamond 
\end{Code}
Användbara metoder för att hantera elementens \Emph{ordningen}:
\begin{REPLsmall}
scala> Suit.Spade.ordinal      // från element till heltal
val res0: Int = 0

scala> Suit.Club.ordinal
val res1: Int = 2

scala> Suit.fromOrdinal(3)    // från heltal till element
val res2: Suit = Diamond

scala> Suit.values            // alla element i ordning
val res3: Array[Suit] = Array(Spade, Heart, Club, Diamond)

scala> Suit.valueOf("Spade")  // från sträng till element
val res4: Suit = Spade
\end{REPLsmall}
\end{Slide}

\begin{Slide}{Enumerationer kan ha parametrar och medlemmar}
En \code{enum} kan ha parametrar. Använd \code{val} för extern synlighet:  
\begin{Code}
enum Color(val consoleColor: String): 
  case Black extends Color(Console.BLUE) //Blå färg syns på svart bakgrund
  case Red   extends Color(Console.RED)
\end{Code}
I \code{enum}-kroppen kan du ha medlemmar, tex metoder:
\begin{Code}
enum Suit(val color: Color):
  def show(isConsoleColor: Boolean = true): String = 
    if isConsoleColor then color.consoleColor + toString + Console.RESET
    else toString

  case Spade   extends Suit(Color.Black)
  case Heart   extends Suit(Color.Red)
  case Club    extends Suit(Color.Black) 
  case Diamond extends Suit(Color.Red)
\end{Code}
\begin{REPLsmall}
scala> println(Suit.Club.show(isConsoleColor = false)) 
Club
\end{REPLsmall}
\end{Slide}

\begin{Slide}{Enum kan bli fullfjädrade case-klasser}
Vill du kunna göra mönster-matching på enum-värden så behövs parametrar på alternativen för att det ska bli motsvarande case-klasser: 
\begin{Code}
enum Veg:
  def taste: String
  case Tomato(taste: String)
  case Banana(taste: String)
\end{Code}
Ovan expanderas automatiskt av kompilatorn till motsvarande detta:
\begin{Code}
sealed trait Veg:
  def taste: String
object Veg:
  case class Tomato(taste: String) extends Veg
  case class Banana(taste: String) extends Veg
\end{Code}
\end{Slide}

\begin{Slide}{Enum och mönster-matchning}
\SlideFontSmall
Med parametrar på varje fall och en abstrakt medlem för varje attribut... 
\begin{Code}
enum Veg:
  def taste: String
  case Tomato(taste: String)
  case Banana(taste: String)
\end{Code}
...så gör den automatiska expansionen till case-klasser att detta fungerar fint: 
\begin{REPLsmall}
scala> val v: Veg = Veg.Tomato("najs") // notera typen : Veg
val v: Veg = Tomato(najs)

scala> v.taste  // funkar eftersom Veg har en taste
val res0: String = najs

scala> v match { case Veg.Tomato(t) => t case _ => "bad!" }
val res1: String = najs
\end{REPLsmall}
Den abstrakta medlemmen \code{def taste: String} behövs för att attributet ska synas via referenser som är av den mindre specifika typen \code{Veg}.\\(Mer om abstrakta medlemmar i veckan om arv.)

\end{Slide}


\begin{Slide}{Fördelar med \texttt{\textbf{enum}} jämfört med uppräkning med heltal}
Varför inte bara så här?
\begin{Code}
val (Spade, Heart, Club, Diamond) = (0, 1, 2, 3)  
\end{Code}  
Alla element har samma specifika typ enligt \code{enum}-deklarationen:  
\begin{REPL}
scala> Suit.Heart              // alla element är av typen Suit 
val res5: Suit = Heart
\end{REPL}

\begin{itemize}
\item Detta är säkrare jämfört med att bara använda heltalsvärden: kompilatorn kan hjälpa dig att skilja på element av olika typ och ge felmeddelande om du använder fel typ oavsiktligt. 
\item Ej tillåtna värden kan inte representeras (jmf alla möjliga heltal, där bara några är relevanta).
\end{itemize}  
Detta får du prova på veckans labb: först använda heltal sedan \code{enum}.
\end{Slide}
