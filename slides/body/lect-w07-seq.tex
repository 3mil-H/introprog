%!TEX encoding = UTF-8 Unicode
%!TEX root = ../lect-w07.tex

\Subsection{Vad är en sekvens?}

\ifkompendium\else
{
  \setbeamercolor{background canvas}{bg=black}
  \frame[plain]{\centering\Huge\textbf{\color{pink}{ORDNINGEN}\\SPELAR\\ROLL}}
}
\fi


\begin{Slide}{Vad är en sekvens?}  
\begin{itemize}
\item En sekvens är en \Emph{följd av element} som
  \begin{itemize}
   \item har \Alert{ordningsnummer} (t.ex. numrerade från noll)
   \item är av en viss \Alert{typ} (t.ex. heltal).
  \end{itemize}
  \pause
\item En sekvens kan innehålla \Alert{dubbletter}.
\item En sekvens kan vara \Alert{tom} och ha längden noll.
\item Exempel på en icke-tom sekvens med dubbletter:
\begin{REPLnonum}
scala> val xs = Vector(42, 0, 42, -9, 0, 13, 7)
xs: scala.collection.immutable.Vector[Int] =
  Vector(42, 0, 42, -9, 0, 13, 7)
\end{REPLnonum}
\pause
\item \Emph{Indexering} ger ett element via dess ordningsnummer:
\begin{REPLnonum}
scala> xs(2)
res0: Int = 42

scala> xs.apply(2)
res1: Int = 42
\end{REPLnonum}
\end{itemize}
\end{Slide}

\begin{Slide}{Exempel: En sträng är en sekvens av tecken}
\begin{REPLnonum}
scala> "haj po daj"
\end{REPLnonum}
Längd? 
Vad ligger på första platsen?
Elementtyp?
Dubbletter?
\pause
\begin{REPLnonum}
scala> "haj po daj".length
res1: Int = 10

scala> "haj po daj".apply(0)
res2: Char = h

scala> "haj po daj"(0)
res3: Char = h

scala> "haj po daj".distinct
res4: String = haj pod
\end{REPLnonum}

\end{Slide}


\begin{Slide}{Lägg till i början och i slutet av en sekvens}
  \begin{itemize}
  \item Med metoderna \code{+:} och \code{:+} kan du skapa en ny sekvens med nya element tillagda i början resp. i slutet.
  \item Minnesregel: ''\Alert{Colon on the collection side}''
  \begin{REPLnonum}
  scala> val xs = Vector(1,2,3)
  scala> 42 +: xs         // ger ny Vector(42, 1, 2, 3)
  scala> xs :+ 42         // ger ny Vector(1, 2, 3, 42)
  \end{REPLnonum}
  \pause
  \item Semantik: operatornotation med operatorer som \Emph{slutar med kolon} är \Alert{högerassociativa}
  \item Anropet \code{42 +: xs} skrivs av kompilatorn om till \code{xs.+:(42)}
  \begin{REPL}
  scala> xs.+:(42)
  res4: scala.collection.immutable.Vector[Int] = Vector(42, 1, 2, 3)
  \end{REPL}
  \pause
  \item Konkatenering (sammanfogning) av sekvenser: \code{xs ++ ys}
  
  \end{itemize}
\end{Slide}


\begin{Slide}{Egenskaper hos några sekvenssamlingar i Scala}
\vspace{-0.5em}
\begin{itemize}\SlideFontSmall

\item \code{Vector}
  \begin{itemize}\SlideFontSmall
  \item \Emph{Oföränderlig}. Snabb på att skapa kopior med små förändringar.
  \item Allsidig prestanda: \Emph{bra till det mesta}.
  \end{itemize}

\item \code{List}
  \begin{itemize}\SlideFontSmall
  \item \Emph{Oföränderlig}. Snabb på att skapa kopior med små förändringar.
  \item Snabb indexering \& uppdatering \Emph{i början}.
  \item Smidig \& snabb vid \Emph{rekursiva} algoritmer.
  \item Långsam vid upprepad \Alert{indexering} på godtyckliga ställen.
  \end{itemize}

\item \code{ArrayBuffer}
  \begin{itemize}\SlideFontSmall
  \item \Alert{Föränderlig}: \Emph{snabb indexering \& uppdatering}.
  \item Kan \Emph{ändra storlek} efter allokering. Snabb att indexera överallt.
  \end{itemize}

\item \code{ListBuffer}
  \begin{itemize}\SlideFontSmall
  \item \Alert{Föränderlig}: snabb indexering \& uppdatering \Emph{i början}.
  \item Snabb om du bygger upp sekvens genom många tillägg i början.
  \end{itemize}

\item \code{Array} (använd fr.o.m. Scala 2.13 hellre \code{ArraySeq})
  \begin{itemize}\SlideFontSmall
  \item \Alert{Föränderlig}: \Emph{snabb indexering \& uppdatering}.
  \item Kan \Alert{ej ändra storlek}; storlek ges vid allokering.
  \item Har särställning i JVM: ger snabb allokering och access.
  \end{itemize}

\end{itemize}
\end{Slide}

\begin{Slide}{I stället för Array: \texttt{ArraySeq} (Scala >2.13)}
Nytt i Scala 2.13:
\begin{itemize}
  \item \code{collection.mutable.ArraySeq} fungerar lika bra som, eller bättre än \code{Array}
  \item \url{https://stackoverflow.com/questions/5028551/array-vs-arrayseq-comparison}
  \item \code{collection.immutable.ArraySeq} fungerar som en Array som inte kan ändras
  \item \code{WrappedArray} är från 2.13 \textit{deprecated} (på väg bort)
\end{itemize}
{\SlideFontTiny Fördjupning: Sammanfattning av prestanda för olika samlingar: \href{https://docs.scala-lang.org/overviews/collections-2.13/performance-characteristics.html}{docs.scala-lang.org/overviews/collections-2.13/performance-characteristics.html}
}
\end{Slide}

\begin{Slide}{Vilken sekvenssamling ska jag välja?}\SlideFontSmall
\vspace{-0.5em}
\begin{itemize}
\item Välj \code{Vector} om ...
  \begin{itemize}\SlideFontTiny
  \item[a)] du vill ha oföränderlighet: \code{val xs = Vector[Int](1,2,3)}
  \item[b)] du behöver föränderlighet (notera \code{var}):\\ \code{var xs = Vector.empty[Int]}
  \item[c)] du ännu inte vet vilken sekvenssamling som är bäst; du kan alltid ändra efter att du mätt prestanda och kollat flaskhalsar vid upprepade körningar.
  \end{itemize}

\item Välj \code{List} om ...
  \begin{itemize}\SlideFontTiny
  \item[] du har en \Alert{rekursiv} sekvensalgoritm och/eller \Alert{mestadels jobbar i början}.
  \end{itemize}

\item Välj \code{ArrayBuffer} om ...
  \begin{itemize}\SlideFontTiny
  \item[] det behövs av prestandaskäl och du \Alert{inte} vet storlek vid allokering:\\
  \code{val xs = scala.collection.mutable.ArrayBuffer.empty[Int]}
  \end{itemize}

\item Välj \code{ListBuffer} om ...
  \begin{itemize}\SlideFontTiny
  \item[] det behövs av prestandaskäl och du bara behöver lägga till i början:\\ \code{val xs = scala.collection.mutable.ListBuffer.empty[Int]}
  \end{itemize}

\item Välj \code{Array} eller \code{ArraySeq} om ...
  \begin{itemize}\SlideFontTiny
  \item[] det verkligen behövs av prestandaskäl och du \Alert{vet} storlek vid allokering:\\
  \code{val xs = Array.fill(initSize)(initValue)}
  \end{itemize}

\end{itemize}
\end{Slide}

\begin{Slide}{Några konstigheter med Array}
\begin{itemize}
\item Referenslikhet istf innehållslikhet: 
\begin{REPLnonum}
scala> Vector(1,2,3) == Vector(1,2,3)
val res0: Boolean = true

scala> Array(1,2,3) == Array(1,2,3)
val res1: Boolean = false  // aaargh!!
\end{REPLnonum}
\item Special-syntax för att skapa utan explicit initialisering: \\
{\SlideFontSmall\code{val xs = new Array[String](1000)  // 1000 null-referenser}}
\item Fungerar inte lika bra med generiska typer:
\begin{REPLnonum}
scala> def box[T](x: T) = Vector[T](x)  //funkar fint

scala> def abox[T](x: T) = Array[T](x)
  error: No ClassTag available for T
\end{REPLnonum}
\end{itemize}
\end{Slide}

\begin{Slide}{Oföränderlig eller förändringsbar?}
\begin{itemize}
\item \Emph{Oföränderlig}:  Kan ej ändra elementreferenserna, men effektiv på att skapa kopia som är (delvis) förändrad \Emph{Vector} eller \Emph{List}

\item \Alert{Förändringsbar}: kan ändra elementreferenserna
  \begin{itemize}
  \item Kan \Alert{ej ändra storlek} efter allokering: \\ \Emph{Array} eller \Emph{ArraySeq}: indexera och uppdatera varsomhelst
  \item Kan även ändra storlek efter allokering:
  \\\Alert{ArrayBuffer} eller \Alert{ListBuffer}
  %\\ Java: \Alert{ArrayList} eller \Alert{LinkedList}
  \end{itemize}
\item \Emph{Ofta funkar oföränderlig sekvenssamling utmärkt}, men om man \Alert{efter prestandamätning} upptäcker en flaskhals kan man ändra från \Emph{Vector} till t.ex. \Emph{ArrayBuffer}.
\end{itemize}
\end{Slide}



\Subsection{Vad är en sekvensalgoritm?}



\begin{Slide}{Vad är en sekvensalgoritm?}\SlideFontTiny
\begin{itemize}
\item En algoritm är en stegvis beskrivning av lösningen på ett problem.
\item En \textbf{sekvensalgoritm} är en algoritm där \Emph{element i sekvens} utgör en viktig del av \Alert{problembeskrivningen} och/eller \Alert{lösningen}.
\item Exempelproblem: sortera en sekvens av personer efter deras ålder.
\pause
\item \Alert{Sju} ofta återkommande programmeringsproblem som löses med en sekvensalgoritm:
\begin{itemize}\SlideFontTiny
\item \Emph{Kopiering} av alla element i en sekvens till en \Alert{ny} sekvens
\item \Emph{Uppdatering} av sekvensen: ta bort, lägga till, ändra \Emph{enskilda} element
\item \Emph{Transformering}: applicera en \Alert{funktion} på \Emph{alla} element   
\item \Emph{Filtrering}: urval av vissa element som uppfyller ett \Alert{villkor}
\item \Emph{Sökning} efter ett element som uppfyller ett \Alert{sökkriterium}
\item \Emph{Sortering} enligt någon \Alert{ordning}
\item \Emph{Registrering} av \Alert{frekvens} av element med vissa \Alert{egenskaper}
\end{itemize}
\end{itemize}
\href{https://youtu.be/0ArlUSVDQIw?t=27s}{\textbf{KUT FSSR}} 
\end{Slide}

\ifkompendium\else
{
  \setbeamercolor{background canvas}{bg=black}
  \frame[plain]{\centering\Huge\textbf{\color{pink}{ORDNINGEN\\SPELAR\\ROLL}\\\color{red}{KUT FSSR}}}
}
\fi



\Subsection{Använda färdiga sekvenssamlingsmetoder}


\begin{Slide}{Använda färdiga sekvenssamlingsmetoder}
\begin{itemize}
\item Standardbiblioteket i Scala innehåller flera olika samlingar som har sekvensegenskaper, t.ex. \code{Vector} och \code{ArrayBuffer}, som erbjuder olika möjligheter och har olika prestanda beroende på vad man vill göra.

\item Ofta kan man implementera sekvensalgoritmer genom anrop av en eller flera \Alert{färdiga} metoder.

\item Dessa färdiga metoder är \Emph{optimerade och vältestade} och är att föredra om möjligt.

\item Studera Scalas api-dokumentation och kursens quickref för att se vad man kan göra med färdiga metoder.

\item Det är \Emph{lärorikt} att ''\Alert{uppfinna hjulet}'' och implementera några sekvensalgoritmer \Emph{själv} för bättre förståelse, även om de redan finns färdiga i Scalas samlingsbibliotek.

\end{itemize}
\end{Slide}



\begin{Slide}{Några användbara samlingsmetoder vid implementation av sekvensalgoritmer}
\SlideFontTiny
\begin{tabular}{@{}l l}
\code|xs map f|           & transformering, motsv. \code|for (x <- xs) yield f(x)| \\
\code|xs map (x => x)|    & kopiering, motsv. \code|for (x <- xs) yield x| \\
\code|xs filter p|        & filtrering, ta med x om p(x)\\
\code|xs filterNot p|     & filtrering, ta med x om !p(x)\\
\code|xs.distinct|        & filtrering, ta bort dubbletter \\
\code|xs take n|          & ny sekvens med de första n elements, resten skippade\\ 
\code|xs drop n|          & ny sekvens där de första n elements är skippade\\ 
\code|xs takeWhile p|     & filtrera, ta med i början så länge p(x)  \\
\code|xs dropWhile p|     & filtrera, skippa i början så länge p(x)  \\
\code|xs indexOf x|       & sökning framifrån efter ett index som är x \\
\code|xs lastIndexOf x|   & sökning bakifrån efter ett index som är x \\
\code|xs.sorted|          & sortering med inbyggd (implicit) ordning \\
\code|xs.sorted.reverse| & sortering i omvänd ordning \\
\code|xs sortBy f|        & sortering i ordning enligt f(x)\\
\code|xs sortWith lt|     & sortering enligt less than \code|lt: (A, A) => Boolean|\\
\code|xs count p|         & räkna antalet element där p(x) är sant
\end{tabular}

\vspace{0.5em}%
\Emph{Lär dig fler smidiga metoder i} \Alert{quickref} och genom denna översikt:\\
{\SlideFontTiny\href{https://docs.scala-lang.org/overviews/collections-2.13/introduction.html}{docs.scala-lang.org/overviews/collections-2.13/introduction.html}}
\end{Slide}

\begin{Slide}{Uppdaterad sekvens med kraftfulla metoden \texttt{patch}}
  Metoden \code{patch} kan användas så: \code{xs.patch(fromPos, ys, nbrReplaced)} \\för att skapa en \Alert{ny} sekvens där \Emph{ett} eller \Emph{flera} element i xs är...  
  \begin{itemize}
    \item utbytta \Eng{replaced}
    \item borttagna \Eng{removed}
    \item tillagda \Eng{inserted}
  \end{itemize}
  .. med nya element ur \code{ys} 
\begin{REPL}
scala> val xs = Vector(1,2,3)

scala> xs.patch(2, Vector(-1), 1)     // replaced one elem
res0: scala.collection.immutable.Vector[Int] = Vector(1, 2, -1)

scala> xs.patch(1, Vector(42), 0)     // inserted one elem
res11: scala.collection.immutable.Vector[Int] = Vector(1, 42, 2, 3)

scala> xs.patch(0, Vector(), 2)       // removed two elems
res2: scala.collection.immutable.Vector[Int] = Vector(3)
 
\end{REPL}
\end{Slide}

\begin{Slide}{Använda for-uttryck för filtrering med hjälp av gard}
I ett for-uttryck kan man ha en \Emph{gard} \Eng{guard} i form av ett booleskt uttryck efter nyckelordet \code{if}. Då kommer uttrycket efter \code{yield} bara göras om gard-uttrycket är sant.

\vspace{1em}

Syntaxen är så här: (parenteser behövs ej runt gard-uttrycket)
\begin{Code}[basicstyle=\ttfamily\SlideFontSize{12}{14}]
for (x <- xs if uttryck1) yield uttryck2
\end{Code}
\pause
Exempel:
\begin{REPLnonum}
scala> val udda = for (x <- 1 to 6 if x % 2 == 1) yield x
\end{REPLnonum}
\pause
\code{udda} blir \code{Vector(1, 3, 5)}
\end{Slide}


\begin{Slide}{Använda samlingsmetoden \texttt{filter} för filtrering}
Alla samlingar i \code{scala.collection} har metoden \code{filter}. Den har ett predikat som parameter \code{p: T => Boolean} och ger en ny samling med de element för vilka predikatet är sant.
\begin{Code}[basicstyle=\ttfamily\SlideFontSize{12}{14}]
xs.filter(p)
\end{Code}
\pause
Exempel: Antag att \code{xs} är \code{(1 to 6).toVector}
\begin{REPLnonum}
xs.filter(_ % 2 == 1)
\end{REPLnonum}
\pause
uttryckets resultat blir \code{Vector(1, 3, 5)}, vilket motsvarar:
\begin{Code}[basicstyle=\ttfamily\SlideFontSize{10}{13}]
for (x <- xs if x % 2 == 1) yield x
\end{Code}
\pause
I själva verket skriver Scala-kompilatorn om for-uttryck med gard till anrop av metoden \code{filter} före kodgenerering sker.
\end{Slide}


\begin{Slide}{Vanliga sekvensproblem som funktionshuvuden}
Indata och utdata för några vanliga sekvensproblem:
\begin{Code}
def copy(xs: Vector[Int]): Vector[Int] = ???

def filter(xs: Vector[Int], p: Int => Boolean): Vector[Int] = ???

def findIndices(xs: Vector[Int], p: Int => Boolean): Vector[Int] = ???

def sort(xs: Vector[Int]): Vector[Int] = ???

def freq(xs: Vector[Int]): Vector[(Int, Int)] = ???  // (heltal, frekvens)
\end{Code}
Övning: Hur implementera dessa med \code{for}-uttryck och/eller färdiga samlingsmetoder?\\
\Emph{Tips:} För \code{sort}\&\code{freq} se \code{sorted}, \code{distinct}, \code{count} i \href{http://cs.lth.se/pgk/quickref/}{quickref}
\end{Slide}


\begin{Slide}{Implementation av sekvensproblem med \texttt{for}-uttryck och/eller färdiga samlingsmetoder}
\begin{Code}
def copy(xs: Vector[Int]): Vector[Int] = for (x <- xs) yield x

def filter(xs: Vector[Int], p: Int => Boolean): Vector[Int] =
  for (x <- xs if p(x)) yield x

def findIndices(xs: Vector[Int], p: Int => Boolean): Vector[Int] =
  (for (i <- xs.indices if p(xs(i))) yield i).toVector

def sort(xs: Vector[Int]): Vector[Int] = xs.sorted // mer om sortering sen

def freq(xs: Vector[Int]): Vector[(Int, Int)] = // mer om registrering snart
  for (x <- xs.distinct) yield (x, xs.count(_ == x))
\end{Code}
Övning: Hur implementera dessa med \code{map} och \code{filter} och/eller andra färdiga samlingsmetoder?
\end{Slide}

\begin{Slide}{Implementation av sekvensproblem med \texttt{map}, \texttt{filter}}
\begin{Code}
def copy(xs: Vector[Int]): Vector[Int] = xs.map(x => x)

def filter(xs: Vector[Int], p: Int => Boolean): Vector[Int] = xs.filter(p)

def findIndices(xs: Vector[Int], p: Int => Boolean): Vector[Int] =
  xs.indices.filter(i => p(xs(i))).toVector

def sort(xs: Vector[Int]): Vector[Int] = xs.sorted // mer om sortering sen

def freq(xs: Vector[Int]): Vector[(Int, Int)] = // mer om registrering snart
  xs.distinct.map(x => (x, xs.count(_ == x)))
\end{Code}
%Fördjupningsövning: Hur göra dessa metoder generiska för alla sekvenssamlingar av typen \code{Seq[T]}?
\end{Slide}


\begin{Slide}{Hierarki av samlingstyper i \texttt{scala.collection} v2.13}

  \begin{multicols}{2}
  \begin{tikzpicture}[sibling distance=5.0em,->,>=stealth', inner sep=3pt, %scale=0.5,
    every node/.style = {shape=rectangle, draw, align=center,font=\small\ttfamily},
    class/.style = {fill=blue!20},
    trait/.style = {rounded corners, fill=red!20}]
    \node[trait] {Iterable}
        child { node[trait] {Seq} }
        child { node[trait] {Set} }
        child { node[trait] {Map} }
      ;
  \end{tikzpicture}
  
  \columnbreak
  
  {\SlideFontTiny
  
  \code{Iterable} har metoder som är implementerade med hjälp av: \\
  \code{def foreach[U](f: Elem => U): Unit}\\
  \code{def iterator: Iterator[A] }
  
}

\begin{itemize}\SlideFontTiny
  \item[] \code{Seq}: ordnade i sekvens
  \item[] \code{Set}: unika element
  \item[] \code{Map}: par av (nyckel, värde)
  \end{itemize}
  
  
  \end{multicols}
  
  {\SlideFontSmall Samlingen \Emph{\texttt{Vector}} är en \code{Seq} som är en \code{Iterable}. \\ \vspace{0.5em}\pause
  De konkreta samlingarna är uppdelade i dessa paket:\\
  \code{scala.collection.immutable} \hfill där flera är \Emph{automatiskt} importerade\\
  \code{scala.collection.mutable}  \hfill som \Alert{måste importeras} explicit\\\pause
  (undantag: primitiva \code{scala.Array})
  }
\end{Slide}
  


\begin{Slide}{Lämna det öppet: använd \texttt{Seq}}\SlideFontSmall
Typen \Emph{\code{collection.immutable.Seq}} är supertyp till alla sekvenssamlingar i \code{collection.immutable}.
\pause Exempel: kopiering av sekvens:
\begin{itemize}
\item Kopiering av \Alert{specifik} heltalssekvens:
\begin{Code}
def copyIntVector(xs: Vector[Int]): Vector[Int] = for (x <- xs) yield x
\end{Code}

\item Kopiering som fungerar för alla oföränderliga heltalssekvenser:
\begin{Code}
def copyIntSeq(xs: Seq[Int]): Seq[Int] = for (x <- xs) yield x
\end{Code}
\end{itemize}
\pause
\begin{REPL}
scala> val xs = Vector(1,2,3)
xs: Vector[Int] = Vector(1, 2, 3)

scala> val ys = copyIntVector(xs)
ys: Vector[Int] = Vector(1, 2, 3)

scala> val zs = copyIntSeq(xs)
val zs: Seq[Int] = Vector(1, 2, 3)
\end{REPL}
%  Någon lämplig specifik samling som är subtyp till \code{Seq[T]} väljs automatiskt. \\
% (Mer om typparametrar och supertyper/subtyper senare i kursen.)
% \begin{Code}[basicstyle=\ttfamily]
% def varannanBaklänges[T](xs: Seq[T]): Seq[T] =
%   for (i <- xs.indices.reverse by -2) yield xs(i)
% \end{Code}
% Fungerar med alla sekvenssamlingar:
% \begin{REPLnonum}
% scala> varannanBaklänges(Vector(1,2,3,4,5))
% res0: Seq[Int] = Vector(5, 3, 1)
%
% scala> varannanBaklänges(List(1,2,3,4,5))
% res1: Seq[Int] = List(5, 3, 1)
%
% scala> varannanBaklänges(collection.mutable.ListBuffer(1,2))
% res2: Seq[Int] = Vector(2)
% \end{REPLnonum}
% Scalas standardbibliotek returnerar ofta lämpligaste specifika sekvenssamlingen som är subtyp till \texttt{Seq[T]}.
\end{Slide}
  
\ifkompendium  

\begin{Slide}{Fördjupning: implementation som generiska funktioner}
Genom att generalisera funktionshuvudena blir våra lösningar användbara för \Alert{alla} sekvenser av typen \code{Seq[T]}.
\begin{Code}
def copy[T](xs: Seq[T]): Seq[T] = xs.map(x => x)

def filter[T](xs: Seq[T], p: T => Boolean): Seq[T] = xs.filter(p)

def findIndices[T](xs: Seq[T], p: T => Boolean): Seq[Int] =
  xs.indices.filter(i => p(xs(i))).toVector

def sort[T: Ordering](xs: Seq[T]): Seq[T] = xs.sorted // mer om Ordering sen

def freq[T](xs: Seq[T]): Seq[(T, Int)] =
  xs.distinct.map((_, xs.count(_ == x)))
\end{Code}
\pause
Standardbibliotekets metoder försöker ordna så att det blir samma konkreta typ in som ut, men ibland väljs annan lämplig konkret samling, t.ex. kan en \code{Array} ofta bli en \code{ArrayBuffer}.
\end{Slide}

\fi