%!TEX encoding = UTF-8 Unicode
%!TEX root = ../lect-w07.tex


\Subsection{Läsa från fil och URL}

\begin{Slide}{Serialisering och deserialisering}
\begin{itemize}
  \item Att \Emph{serialisera} innebär att \Alert{koda objekt} i minnet till en avkodningsbar \Alert{sekvens av symboler}, som kan lagras t.ex. i en fil på din hårddisk.
  \item Att \Emph{de-serialisera} innebär att \Alert{avkoda en sekvens av symboler}, t.ex. från en fil, och \Alert{återskapa objekt} i minnet.
\end{itemize}
\end{Slide}


\begin{Slide}{Läsa från fil och URL}
I paketet \code{scala.io} finns singelobjektet \code{Source} med metoderna \code{fromFile} och \code{fromUrl} för läsning från fil resp. från  URL\footnote{URL = Universal Resource Locator}, som börjar t.ex. med \code{http://}
\begin{Code}
def läsFrånFil(filnamn: String) = scala.io.Source.fromFile(filnamn).mkString

def läsRaderFrånFil(filnamn: String): Vector[String] =
  scala.io.Source.fromFile(filnamn).getLines.toVector

def läsFrånWebbsida(url: String) = scala.io.Source.fromURL(url).mkString

def läsRaderWebbsida(url: String, kodning: String = "UTF-8"): Vector[String] =
  scala.io.Source.fromURL(url, kodning).getLines.toVector

\end{Code}
Se exempel på veckans övning.
\end{Slide}


\begin{Slide}{Fördjupning: singelobjektet Disk}
Se fördjupningsuppgift övning \texttt{lookup}:
\scalainputlisting[basicstyle=\ttfamily\SlideFontSize{6}{7.5}]{../compendium/examples/Disk.scala}
\end{Slide}
