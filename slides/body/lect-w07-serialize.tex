%!TEX encoding = UTF-8 Unicode
%!TEX root = ../lect-w07.tex


\Subsection{Serialisering och deserialisering}

\begin{Slide}{Serialisering och deserialisering}
\begin{itemize}
  \item Att \Emph{serialisera} innebär att \Alert{koda objekt} i minnet till en avkodningsbar \Alert{sekvens av symboler}, som kan lagras t.ex. i en fil på din hårddisk.
  \item Att \Emph{de-serialisera} innebär att \Alert{avkoda en sekvens av symboler}, t.ex. från en fil, och \Alert{återskapa objekt} i minnet.
\end{itemize}
\end{Slide}


\begin{Slide}{Läsa text från fil och URL}
I paketet \code{scala.io} finns singelobjektet \code{Source} med metoderna \code{fromFile} och \code{fromUrl} för läsning från fil resp. från  URL\footnote{URL = Universal Resource Locator}, som börjar t.ex. med \code{http://}
\begin{Code}
def läsFrånFil(filnamn: String) = scala.io.Source.fromFile(filnamn).mkString

def läsRaderFrånFil(filnamn: String): Vector[String] =
  scala.io.Source.fromFile(filnamn).getLines.toVector

def läsFrånWebbsida(url: String) = scala.io.Source.fromURL(url).mkString

def läsRaderWebbsida(url: String, kodning: String = "UTF-8"): Vector[String] =
  scala.io.Source.fromURL(url, kodning).getLines.toVector

\end{Code}
Se exempel på veckans övning.
\end{Slide}


\begin{Slide}{Fördjupning: modulen \texttt{introprog.IO}}
  \begin{itemize}
    \item I kursens kodbibliotek \code{introprog} finns ett singelobjekt \code{IO} som samlar smidiga funktioner för serialisering och de-serialisering. Se api-dokumentation här: \\ \url{http://fileadmin.cs.lth.se/pgk/api/introprog/IO$.html}
    \item
    Fördjupningsuppgift 15 i övning \texttt{lookup} i det tryckta kompendiet på sidorna 145-147 har ändrats sedan tryck från att använda gamla \code{Disk} till att använda \code{introprog.IO} (ungefär samma kod).
    \item Se koden här:\\
    \url{https://github.com/lunduniversity/introprog-scalalib/blob/master/src/main/scala/introprog/IO.scala}
  \end{itemize}
\end{Slide}
