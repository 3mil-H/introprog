%!TEX encoding = UTF-8 Unicode
%!TEX root = ../lect-w07.tex


\Subsection{Veckans uppgifter}

\begin{Slide}{Laboration: \texttt{words}}
\begin{itemize}
  \item Denna uppgift handlar om analys av naurligt språk \Eng{Natural Language Processing, NLP}.
  \item Svara på frågorna:
  \begin{itemize}%[noitemsep]
  \item Hur vanligt är ett visst ord i en given text?
  \item Vilket är det vanligaste ordet som följer efter ett visst ord?
  \item Hur kan man generera ordsekvenser som liknar ordföljden i en given text?
  \end{itemize}
\item Använda mängd för unika ord.
\item Använda nyckel-värde-tabell för att för varje ord i en lång text räkna antalet förekomster av detta ord.
\end{itemize}
\end{Slide}



\begin{Slide}{Övning: \texttt{lookup}}
\begin{itemize}
  \item Övningen innehåller delar som är \Alert{nödvändiga} för laborationen.
  \item På övningen tränar du på \Emph{mängder} och \Emph{nyckel-värde-tabeller}.
  \item Du ska skapa en klass \code{FreqMapBuilder} som bygger upp en tabell med ordfrekvenser, som behövs på labben.
\end{itemize}
\end{Slide}
