%!TEX encoding = UTF-8 Unicode
%!TEX root = ../lect-w02.tex

\Subsection{Om förra veckan och kommande två veckor}

\begin{SlideExtra}{Förra veckan}
Viktiga övergripande mål:
\begin{itemize}
\item Förstå skillnaden mellan värde och typ
\item Börja använda sekvens, alternativ, repetition, abstraktion
\item Förstå variabel och tilldelning
\item Reflektera över ditt lärande
\item Träffas i samarbetsgrupper och lära känna varandra
\item Komma in i kursens arbetsgång: förel. -> övn. -> labb
\end{itemize}
Om du inte hann klart labben, fortsätt på egen hand och på resurstid. \Alert{Avsätt mer tid till labbförberedelser nästa gång}.
\\ 
\Alert{Glöm inte} att du ska få labbhandledarens \Alert{underskrift} i protokollet i kompendiet på sidan iii när du är \Emph{närvarande} (godkänd eller kompletteras).
\end{SlideExtra}

\begin{SlideExtra}{Fråga på resurstider och labbar}
\begin{itemize}
  \item På LTH skiljer vi tydligt på \Emph{undervisning} och \Alert{examination}.
  \item \Emph{Resurstider} är undervisning och du får fråga vad du vill!
  \item \Emph{Labbarna} är i huvudsak undervisning utom redovisningen på slutet som är examination. \Alert{Du får fråga vad du vill!}
  \item \Emph{Redovisningen} är till för att hjälpa dig avgöra om du har den \Alert{förståelse} som krävs för fortsättningen.
  \item Labbarna ger \Emph{godkänd}/\Alert{kompletteras} -- ej graderat betyg. 
  \item Labbarna påverkar inte slutbetyget men det krävs godkänt på \Alert{alla} \Emph{labbar} för att få göra muntligt prov och tentera.
  \item Att hjälpa varandra i samarbetsgrupperna med labbarna är \Emph{inte fusk}. Men att göra labben åt någon annan \Alert{är fusk}. Hjälp varandra att \Emph{förstå} begrepp och att komma vidare när någon kör fast. \Alert{Läs kapitel -1 Anvisningar}.
\end{itemize}  
\end{SlideExtra}


\begin{SlideExtra}{Kommande 2 veckor}
Övergripande mål:
\begin{itemize}
\item Börja skriva dina \Emph{egna program}
\item Träna på att dela upp din kod i \Alert{många små funktioner}
\end{itemize}

Läsvecka 2:
\begin{itemize}
\item Övning \texttt{\ExeWeekTWO}
\item \Alert{OBS!} \Emph{Ingen labb denna vecka}, pga dod.
\end{itemize}
Läsvecka 3:
\begin{itemize}
\item Övning \texttt{\ExeWeekTHREE}
\item Labb \texttt{\LabWeekTHREE}  \\ spela varandras textspel i samarbetsgrupper
\end{itemize}

\vspace{1em}\Alert{OBS!} Notera \Alert{schemaavvikelser} -- veckorna är inte identiska:\\\url{http://cs.lth.se/pgk/schema/timeedit/}

\end{SlideExtra}



\Subsection{Studieteknik}

\begin{SlideExtra}{Hur studerar du?}
\begin{itemize}
\item Vad är bra \Emph{studieteknik}?
\item \Alert{Aktivera dig!} Inte bara passivt läsa utan också aktivt göra.
\item Hur skapa \Emph{struktur}?
Du behöver ett sammanhang, ett \Emph{system av begrepp}, att \Alert{placera in} din nya kunskap i.
\item Hur uppbåda \Emph{koncentration}? Steg 1: Stäng av mobilen!
\item Hur vara \Emph{disciplinerad}? Studier först, nöje sen! Du måste inte vara med på allt under nollningen. Mitt råd: Hoppa över alla nollningsaktiviteter tills du är ikapp med i dina studier.
\item Du måste \Emph{planera och omplanera} för att säkerställa \Alert{tillräckligt mycket egen pluggtid} då du är pigg och koncentrerad för att det ska funka!
\item \textbf{Programmering} \Alert{kräver} \Emph{pigg \&\& koncentrerad hjärna}!
\end{itemize}
\end{SlideExtra}


\begin{SlideExtra}{Hur ska du studera programmering?}
\SlideFontSmall
  \begin{itemize}\SlideFontSmall
\item När du gör \Emph{övningarna}:
\begin{itemize}\SlideFontSmall
\item Ta fram \Emph{föreläsningsbilderna} i pdf och kolla igen \Alert{innan} övningen.
\item Är det något i föreläsningsbilderna du inte förstår: ta upp det i samarbetsgrupperna eller på resurstiderna.
\item Om något är knepigt:
\begin{itemize}\SlideFontSmall
\item Hitta på egna \Emph{REPL-experiment}, undersök hur det funkar.
%\item Följ ev. länkar i föreläsningsbilderna, eller googla själv på wikipedia, stackoverflow, ... 
\end{itemize}
\end{itemize}

\item Innan du gör \Emph{laborationerna}:
\begin{itemize}\SlideFontSmall
\item Gör \Alert{minst} \Emph{grundövningarna} \Alert{innan} du börjar med labben. Dubbelkolla att du har uppnått \Emph{övningsmålen}.
\item \Emph{Läs igenom} \Alert{hela} \Emph{labbinstruktionen} \Alert{innan} labben och gör en bedömning av din förberedelsetid.
\item Gör förberedelserna \Alert{i god tid innan} labben.
\item För varje labb behövs \Emph{allt större del} av koden \Alert{göras klart innan} ditt labbtillfälle. Mot slutet av kursen används nästan hela labbtiden till redovisning.
\end{itemize}
\end{itemize}

\end{SlideExtra}

\begin{SlideExtra}{Det går inte att förstå allt på en gång!}
\begin{itemize}
\item Vi \Emph{nosar} på ett visst begrepp \Alert{lite grand} i en vecka ...

\item ... för att i senare vecka \Emph{återkomma} till det, men \Alert{djupare}.

\item Förståelse kommer efter hand och kräver bearbetning.

\item Vi måste \Emph{iterera begreppen} innan vi kan nå djup.

\item Det är svårt för dig nu att se vad som är \Alert{detaljer} som du inte ska hänga upp dig på, och vad som är \Emph{det viktiga} i detta läget. \Emph{Men det kommer!} \Alert{Ha tålamod!}
\end{itemize}

\end{SlideExtra}


% \begin{SlideExtra}{Nu på rasten: träffa din samarbetsgrupp}
% \begin{itemize}
% \item Träffas i samarbetsgrupperna och bestäm/gör/diskutera:
% \begin{enumerate}
% \item När ska ni träffas nästa gång?
% \item Bläddra igenom föreläsningsbilderna från w01 i pdf.
% \item Vilka \Emph{koncept} är fortfarande (mest) \Alert{grumliga}? \\Alltså: Vilka koncept från förra veckan vill ni på nästa möte jobba mer med i gruppen för att alla ska förstå grunderna?
% \end{enumerate}
% \end{itemize}
% \vspace{1em}Du som ännu inte visat ditt \Alert{samarbetskontrakt}, visa det för handledare på \Emph{veckans resurstid}. Exempel på kontrakt:
% \\\href{https://github.com/lunduniversity/introprog/tree/master/study-groups}{\footnotesize\texttt{github.com/lunduniversity/introprog/tree/master/study-groups}}
% \end{SlideExtra}


%%%
