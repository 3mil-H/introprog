%!TEX encoding = UTF-8 Unicode
%!TEX root = ../lect-w02.tex

\Subsection{Om förra veckan och kommande två veckor}

\begin{SlideExtra}{Förra veckan}
Viktiga övergripande mål:
\begin{itemize}
\item Förstå skillnaden mellan värde och typ
\item Börja använda sekvens, alternativ, repetition, abstraktion
\item Förstå variabel och tilldelning
\item Reflektera över ditt lärande
\item Träffas i samarbetsgrupper och lära känna varandra
\item Komma in i kursens arbtesgång: förel. -> övn. -> labb
\end{itemize}
Om du inte hann klart labben, fortsätt på egen hand och på resurstid. Avsätt mer tid till labbförberedelser nästa gång.
\\ 
\Alert{Glöm inte} att du ska få labbhandledarens \Alert{underskrift} i framstegsprotokollet när du är \Emph{godkänd}.
\end{SlideExtra}

\begin{SlideExtra}{Fråga på resurstider och labbar}
\begin{itemize}
  \item På LTH skiljer vi tydligt på undervisning och examination.
  \item Resurstider är undervisning och du får fråga vad du vill!
  \item Labbarna är i huvudsak undervisning utom själva redovisningen på slutet som är examination. Du får fråga vad du vill!
  \item Redovisningen är till för att hjälpa dig avgöra om du har den förståelse som krävs för fortsättningen.
  \item Labbarna ger \Emph{godkänd}/\Alert{kompletteras} -- ej graderat betyg. 
  \item Labbarna påverkar inte slutbetyget men det krävs godkänt på \Alert{alla} labbar för att få tentera.
  \item Att hjälpa varandra i samarbetsgrupperna med labbarna är \Emph{inte} fusk. Men att göra labben åt någon annan \Alert{är} fusk. Hjälp varandra att förstå begrepp och att komma vidare när någon kör fast. \Alert{Läs kapitel -1 Anvisningar} i kompendiet.
\end{itemize}  
\end{SlideExtra}


\begin{SlideExtra}{Kommande 2 veckor}
Övergripande mål:
\begin{itemize}
\item Börja skriva dina \Emph{egna program}
\item Träna på att dela upp din kod i \Alert{många små funktioner}
\end{itemize}

Läsvecka 2:
\begin{itemize}
\item Övning \texttt{\ExeWeekTWO}
\item OBS! Ingen labb denna vecka, pga dod.
\end{itemize}
Läsvecka 3:
\begin{itemize}
\item Övning \texttt{\ExeWeekTHREE}
\item Labb \texttt{\LabWeekTHREE}  \\ spela varandras textspel i samarbetsgrupper
\end{itemize}

\vspace{1em}\Alert{OBS!} Noter schemaavvikelser -- alla veckor är inte identiska:\\\url{http://cs.lth.se/pgk/schema/timeedit/}

\end{SlideExtra}



\Subsection{Studieteknik}

\begin{SlideExtra}{Hur studerar du?}
\begin{itemize}
\item Vad är bra \Emph{studieteknik}?
\item Aktivera dig! Inte bara passivt läsa utan också aktivt göra.
\item Hur skapa \Emph{struktur}?
Du behöver ett sammanhang, ett \Emph{system av begrepp}, att \Alert{placera in} din nya kunskap i.
\item Hur uppbåda \Emph{koncentration}? Steg 1: Stäng av mobilen!
\item Hur vara \Emph{disciplinerad}? Studier först, nöje sen! Du måste inte vara med på allt under nollningen. Mitt råd: Hoppa över nollningsaktivitet tills du är ikapp med i dina studier.
\item Du måste \Emph{planera och omplanera} för att säkerställa \Alert{tillräckligt mycket egen pluggtid} då du är pigg och koncentrerad för att det ska funka!
\item \textbf{Programmering} \Alert{kräver} \Emph{pigg \&\& koncentrerad hjärna}!
\end{itemize}
\end{SlideExtra}


\begin{SlideExtra}{Hur ska du studera programmering?}
\SlideFontSmall
  \begin{itemize}\SlideFontSmall
\item När du gör \Emph{övningarna}:
\begin{itemize}\SlideFontSmall
\item Ta fram föreläsningsbilderna i pdf och läs igenom dem \Alert{innan} du påbörjar övningen.
\item Är det något i föreläsningsbilderna du inte förstår: ta upp det i samarbetsgrupperna eller på resurstiderna.
\item Om något är knepigt:
\begin{itemize}\SlideFontSmall
\item Hitta på egna REPL-experiment och undersök hur det funkar.
\item Följ ev. länkar i föreläsningsbilderna, eller googla själv på wikipedia, stackoverflow, ... 
\end{itemize}
\end{itemize}

\item Innan du gör \Emph{laborationerna}:
\begin{itemize}\SlideFontSmall
\item Gör minst grundövningarna \Alert{innan} du börjar med labben. Dubbelkolla att du har uppnått övningsmålen.
\item Läs igen \Alert{hela} labbinstruktionen \Alert{innan} labben och gör en bedömning av din förberedelsetid.
\item Gör förberedelserna \Alert{i god tid innan} labben.
\item För varje labb behövs \Emph{allt större del} av koden göras klart \Alert{innan} ditt labbtillfälle. Mot slutet av kursen används nästan hela labbtiden till redovisning.
\end{itemize}
\end{itemize}

\end{SlideExtra}

\begin{SlideExtra}{Det går inte att förstå allt på en gång!}
\begin{itemize}
\item Vi nosar på ett visst begrepp på ytan i en vecka ...

\item ... för att i senare vecka återkomma till det, men djupare.

\item Förståelse kommer efter hand och kräver bearbetning.

\item Vi måste \Emph{iterera} begreppen innan vi kan nå djup.

\item Det är svårt för dig nu att se vad som är \Alert{detaljer} som du inte ska hänga upp dig på, och vad som är \Emph{det viktiga} i detta läget. Men det kommer! Ha tålamod!
\end{itemize}

\end{SlideExtra}


% \begin{SlideExtra}{Nu på rasten: träffa din samarbetsgrupp}
% \begin{itemize}
% \item Träffas i samarbetsgrupperna och bestäm/gör/diskutera:
% \begin{enumerate}
% \item När ska ni träffas nästa gång?
% \item Bläddra igenom föreläsningsbilderna från w01 i pdf.
% \item Vilka \Emph{koncept} är fortfarande (mest) \Alert{grumliga}? \\Alltså: Vilka koncept från förra veckan vill ni på nästa möte jobba mer med i gruppen för att alla ska förstå grunderna?
% \end{enumerate}
% \end{itemize}
% \vspace{1em}Du som ännu inte visat ditt \Alert{samarbetskontrakt}, visa det för handledare på \Emph{veckans resurstid}. Exempel på kontrakt:
% \\\href{https://github.com/lunduniversity/introprog/tree/master/study-groups}{\footnotesize\texttt{github.com/lunduniversity/introprog/tree/master/study-groups}}
% \end{SlideExtra}


%%%
