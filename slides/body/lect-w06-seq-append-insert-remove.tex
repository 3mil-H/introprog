%!TEX encoding = UTF-8 Unicode
%!TEX root = ../lect-w06.tex

%%%

\begin{Slide}{Implementera Polygon}
\begin{itemize}
\item En polygon kan representeras som en sekvens av punkter.
\item Vi vill kunna lägga till punkter, samt ta bort punkter.
\item En polygon kan implementeras på många olika sätt:
\pause
\begin{itemize}
\item \Alert{Förändringsbar} \Eng{mutable}
\begin{itemize}
\item Med punkterna i en \Alert{\texttt{Array}}
\item Med punkterna i en \Alert{\texttt{ArrayBuffer}}
\item Med punkterna i en \Alert{\texttt{ListBuffer}}
\item Med punkterna i en \Emph{\texttt{Vector}}
\item Med punkterna i en \Emph{\texttt{List}}
\end{itemize}
\item \Emph{Oföränderlig} \Eng{immutable}
\begin{itemize}
\item Som en case-klass med en oföränderlig \Emph{\texttt{Vector}} som returnerar nytt objekt vid uppdatering. Vi kan låta datastrukturen vara \Emph{publik} eftersom allt är oföränderligt.
\item Som en ''vanlig'' klass med någon lämplig \Alert{privat} datastruktur där vi \Alert{inte} möjliggör förändring av efter initialisering och där vi returnerar nytt objekt vid uppdatering.
\end{itemize}
\end{itemize}
\end{itemize}
\pause
Val av implementation \Alert{beror på} sammanhang \& användning!
\end{Slide}

\begin{Slide}{Exempel: PolygonArray, ändring på plats}
\vspace{-0.6em}\scalainputlisting[numbers=left,numberstyle=,basicstyle=\fontsize{6.5}{7.7}\ttfamily\selectfont]{../compendium/examples/sequences/PolygonArray.scala}
\end{Slide}

% \begin{Slide}{Test av PolygonArray, ändring på plats}
% \vspace{0em}\scalainputlisting[numbers=left,numberstyle=,basicstyle=\fontsize{6.5}{8}\ttfamily\selectfont]{../compendium/examples/workspace/w05-seqalg/src/polygonTest3.scala}
% \end{Slide}


\begin{Slide}{Exempel: PolygonVector, variabel referens till oföränderlig datastruktur}
\vspace{-0.6em}\scalainputlisting[numbers=left,numberstyle=,basicstyle=\fontsize{6.5}{7.7}\ttfamily\selectfont]{../compendium/examples/workspace/w05-seqalg/src/PolygonVector.scala}
\end{Slide}

% \begin{Slide}{Test av PolygonVector, variabel referens till oföränderlig datastruktur}
% \vspace{0em}\scalainputlisting[numbers=left,numberstyle=,basicstyle=\fontsize{6.5}{8}\ttfamily\selectfont]{../compendium/examples/workspace/w05-seqalg/src/polygonTest4.scala}
% \end{Slide}


\begin{Slide}{Exempel: Polygon som oföränderlig case class}
\vspace{-0.6em}\scalainputlisting[numbers=left,numberstyle=,basicstyle=\fontsize{6.5}{7.7}\ttfamily\selectfont]{../compendium/examples/sequences/Polygon.scala}
% \begin{itemize}\SlideFontTiny
% \item Nu är attributet points en publik \code{val} som vi kan dela med oss av eftersom datastrukturen \code{Vector} är oföränderlig.
%
% \item Vi behöver inte införa ett beroende till \code{PolygonWindow} här då vi ger tillgång till sekvensen av punkter som kan användas vid anrop av \code{PolygonWindow.draw}
%
% \item Att ändra implementationen till något annat än \code{Vector} blir lätt om klientkoden använder typ-alias \code{Polygon.Pts} i stället för \code{Vector[(Int, Int)]}.
% \end{itemize}
\end{Slide}

% \begin{Slide}{Test av Polygon som oföränderlig case class}
% \vspace{0em}\scalainputlisting[numbers=left,numberstyle=,basicstyle=\fontsize{6.5}{8}\ttfamily\selectfont]{../compendium/examples/workspace/w05-seqalg/src/polygonTest5.scala}
% \end{Slide}
