%!TEX encoding = UTF-8 Unicode
%!TEX root = ../lect-w05.tex

%%%


\Subsection{Likhet}
\begin{Slide}{Referenslikhet eller strukturlikhet?}\SlideFontSmall
Det finns två \Alert{principiellt olika} sorters \Emph{likhet}:
\begin{itemize}
\item \Emph{Referenslikhet} \Eng{reference equality} där två referenser anses lika om de refererar till \Emph{samma instans} i minnet.
\item \Emph{Strukturlikhet} \Eng{structural equality} där två referenser anses lika om de refererar till instanser med \Emph{samma innehåll}.

\pause

\item I Scala finns flera metoder som testar likhet:
\begin{itemize}\SlideFontSmall
\item metoden \code{eq} testar referenslikhet och \code{r1.eq(r2)} ger \code{true} om \code{r1} och \code{r2} refererar till \Emph{samma} instans.

\item metoden \code{ne} testar referens\textbf{o}likhet och \code{r1.ne(r2)} ger \code{true} om \code{r1} och \code{r2} refererar till \Alert{olika} instanser.

\item metoden \code{==} som anropar metoden \code{equals} som default testar referenslikhet men som \Alert{kan överskuggas} om man \Emph{själv vill bestämma} om det ska vara referenslikhet eller strukturlikhet.
\end{itemize}

\pause

\item Scalas \Emph{standardbibliotek} och \Emph{grundtyperna} \code{Int}, \code{String} etc. testar \Emph{strukturlikhet} genom metoden \code{==}
\pause
\item I Java är det annorlunda: symbolen \code{==} är ingen metod i Java utan specialsyntax som  testar referenslikhet mellan instanser, medan metoden \code{equals} kan överskuggas med valfri likhetstest.
\end{itemize}
\end{Slide}


\begin{Slide}{Exempel: referenslikhet och strukturlikhet}
I Scalas standardbibliotek har man överskuggat \code{equals} så att metoden \code{==} ger test av \Emph{strukturlikhet} mellan instanser:
\begin{REPL}
scala> val v1 = Vector(1,2,3)
v1: scala.collection.immutable.Vector[Int] = Vector(1, 2, 3)

scala> val v2 = Vector(1,2,3)
v2: scala.collection.immutable.Vector[Int] = Vector(1, 2, 3)

scala> v1 eq v2                //referenslikhetstest: olika instanser
res0: Boolean = false

scala> v1 ne v2
res1: Boolean = true

scala> v1 == v2                //strukturlikhetstest: samma innehåll
res2: Boolean = true

scala> v1 != v2
res3: Boolean = false
\end{REPL}
\end{Slide}


\begin{Slide}{Referenslikhet och egna klasser}
Om du inte gör något speciellt med dina egna klasser så ger metoden \code{==} test av \Emph{referenslikhet} mellan instanser:
\begin{REPLnonum}
scala> class Gurka(val vikt: Int)

scala> val g1 = new Gurka(42)
g1: Gurka = Gurka@2cc61b3b

scala> val g2 = new Gurka(42)
g2: Gurka = Gurka@163df259

scala> g1 == g2       // samma innehåll men olika instanser
res0: Boolean = false

scala> g1.vikt == g2.vikt
res1: Boolean = true
\end{REPLnonum}
\end{Slide}

\ifkompendium\else
\fi
