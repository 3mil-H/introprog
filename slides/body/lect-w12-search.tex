%!TEX encoding = UTF-8 Unicode
%!TEX root = ../lect-w12.tex

%%%


\Subsection{Binärsökning}\label{subsection:binsearch}

\begin{Slide}{Binärsökning}
Binärsökning är mycket snabbare än linjärsökning men kräver sorterad sekvens.
\begin{REPL}[basicstyle=\color{white}\ttfamily\SlideFontSize{6.5}{8}]
scala> val enMiljon = 1000000

scala> val xs = Array.tabulate(enMiljon)(i => i + 1)   // sorterad

scala> xs(enMiljon - 1)
res0: Int = 1000000

scala> timed { xs.indexOf(enMiljon) }        // linjärsökning
res42: (Double, Int) = (0.016622994,999999)

scala> import scala.collection.Searching.*  // ger tillgång till metoden search
import scala.collection.Searching._

scala> timed { xs.search(enMiljon) }        // binärsökning
res54: (Double, collection.Searching.SearchResult) = (2.45691E-4,Found(999999))

\end{REPL}
\pause
Citat från scaladoc för \code{search}:\\
''The sequence \Alert{should be sorted} before calling; otherwise, the results are \Alert{undefined.}''
\end{Slide}

\begin{Slide}{Binärsökning: lösningsidé}
\Emph{Problemlösningsidé}: Om sekvensen är sorterad kan vi utnyttja detta för en mer effektiv sökning, genom att jämföra med mittersta värdet och se om det vi söker finns före eller efter detta, och upprepa med ''halverad'' sekvens tills funnet.
\pause
\begin{itemize}\SlideFontSmall
\item \Emph{Indata}: sorterad sekvens av heltal och det eftersökta elementet
\item \Emph{Utdata}: \code{Found(index)} för det eftersökta elementet
\\ om saknas: \code{InsertionPoint(index)}
\pause
\begin{Code}
sealed trait SearchResult {
  def insertionPoint: Int
}

case class Found(foundIndex: Int) extends SearchResult {
  override def insertionPoint = foundIndex
}

case class InsertionPoint(insertionPoint: Int) extends SearchResult
\end{Code}
\end{itemize}
\end{Slide}

\begin{Slide}{Binärsökning: pseudokod, imperativ lösning}
\Emph{Pseudo-kod}: imperativ lösning
\begin{Code}
def binarySearch(xs: Vector[Int])(elem: Int): SearchResult = {
  var found = false
  var (low, high) = (/* lägsta index */, /* högsta index */)
  var mid = /* något startvärde */
  while (!found && /* finns fler element kvar */) {
    mid = /* mittpunkten i intervallet (low, high) */
    if (xs(mid) == elem) found = true
    else if (xs(mid) < elem) /* flytta intervallets undre gräns */
    else /* flytta intervallets övre gräns" */
  }
  if (found) Found(mid)
  else InsertionPoint(low)
}
\end{Code}
\end{Slide}



\begin{Slide}{Binärsökning: implementation, imperativ lösning}
\Emph{Implementation}: imperativ lösning
\begin{Code}
def binarySearch(xs: Vector[Int])(elem: Int): SearchResult = {
  var found = false
  var (low, high) = (0, xs.length - 1)
  var mid = -1
  while (!found && low <= high) {
    mid = (low + high) / 2
    if (xs(mid) == elem) found = true
    else if (xs(mid) < elem) low = mid + 1
    else high = mid - 1
  }
  if (found) Found(mid)
  else InsertionPoint(low)
}
\end{Code}
\end{Slide}

\begin{Slide}{Binärsökning: instrumentering av imperativ lösning}
\vspace{-0.5em}
\begin{CodeSmall}[backgroundcolor=\color{white},
  frame=none]
def waitForEnter: Unit = scala.io.StdIn.readLine("")
def show(msg: String): Unit = {println(msg); waitForEnter}

def binarySearch(xs: Vector[Int])(elem: Int): SearchResult = {
  var found = false                         ; show(s"found = $found")
  var (low, high) = (0, xs.length - 1)      ; show(s"(low, high) = ($low, $high)")
  var mid = -1                              ; show(s"mid = $mid")
  while (!found && low <= high) {           ; show(s"while ${!found && low <= high}")
    mid = (low + high) / 2                  ; show(s"mid = $mid")
    if (xs(mid) == elem) {found = true      ; show(s"found = $found")}
    else if (xs(mid) < elem) {low = mid + 1 ; show(s"low = $low")}
    else {high = mid - 1                    ; show(s"high = $high")}
  }
  if (found) Found(mid)
  else InsertionPoint(low)
}
\end{CodeSmall}
\vspace{-0.5em}
\begin{REPL}[basicstyle=\color{white}\ttfamily\SlideFontSize{6}{7}]
scala> binarySearch(Vector(0,1,2,3,42,43))(42)
found = false
(low, high) = (0, 5)
mid = -1
while true
mid = 2
low = 3
while true
mid = 4
found = true
res0: collection.Searching.SearchResult = Found(4)
\end{REPL}
\end{Slide}

\begin{Slide}{Binärsökning: rekursiv lösning}
\Emph{Fördjupning}: rekursiv lösning
\begin{Code}[backgroundcolor=\color{white},
  frame=none]
def binarySearch(xs: Vector[Int])(elem: Int): SearchResult = {
  def loop(low: Int, high: Int): SearchResult =
    if (low > high) InsertionPoint(low)
    else (low + high) / 2 match {
      case mid if xs(mid) == elem => Found(mid)
      case mid if xs(mid) < elem  => loop(mid + 1, high)
      case mid                    => loop(low, mid - 1)
    }

  loop(0, xs.length - 1)
}
\end{Code}
\end{Slide}


\begin{Slide}{Binärsökning: generisk rekursiv lösning}
\Emph{Fördjupning}: iterativ generisk lösning med implicit ordning
\begin{CodeSmall}[backgroundcolor=\color{white},
  frame=none]
def binarySearch[T](xs: Seq[T])(elem: T)(implicit ord: Ordering[T]): SearchResult = {
  import ord._
  def loop(low: Int, high: Int): SearchResult =
    if (low > high) InsertionPoint(low)
    else (low + high) / 2 match {
      case mid if xs(mid) == elem => Found(mid)
      case mid if xs(mid) < elem  => loop(mid + 1, high)
      case mid                    => loop(low, mid - 1)
    }

  loop(0, xs.length - 1)
}
\end{CodeSmall}
{\SlideFontSmall För den intresserade:\\Se fördjupningsuppgifter om implicita ordningar i veckans övning.}
\end{Slide}

\begin{Slide}{Tidskomplexitet, sökning}\SlideFontSmall

\Emph{Fördjupning:}\\Algoritmteoretisk analys av sökalgoritmerna ger:
\begin{itemize}
\item Linjärsökning: tiden är proportionell mot $n$, skrivs: $O(n)$
\item Binärsökning:  tiden är proportionell mot $\log_2 n$, skrivs: $O(\log n)$
\end{itemize}
\vspace{1em}\pause
Empirisk analys: Vi har en vektor med 1000 element. Vi har mätt tiden för att söka upp ett element många gånger och funnit att det tar ungefär 1 $\mu$s både med linjärsökning och binärsökning.
\\\vspace{1em}

\noindent Hur lång tid tar det om vi har fler element i vektorn?\\

\vspace{1em}
\begin{tabular}{rccccc}
       & 1000 & 10 000 & 100000 & 1 000 000 & 10 000 000 \\ \hline
linjärsökning & 1     & 10     & 100     & 1000     & 10 000 \\
binärsökning  & 1     & 1.33   & 1.67    & 2.00     & 2.33
\end{tabular}
\vspace{1em}

{\small\SlideFontTiny

\noindent Kurserna: \\
\Emph{Utvärdering av programvarusystem}, obl. för D1, studerar detta \Alert{empiriskt}\\
\Emph{Algoritmer, datastrukturer och komplexitet}, obl. för D2, studerar detta \Alert{analytiskt}
}
\end{Slide}
