%!TEX encoding = UTF-8 Unicode
%!TEX root = ../lect-w06.tex

%%%

\Subsection{Repeterade parametrar}

\begin{Slide}{Repeterade parametrar blir sekvens}\SlideFontSmall
Med en asterisk efter parametertypen kan antalet argument variera:
\begin{Code}[basicstyle=\fontsize{10}{12}\selectfont\ttfamily]
def sumSizes(xs: String*): Int = xs.map(_.size).sum
\end{Code}
\begin{REPLnonum}
scala> sumSizes("Zaphod")
res0: Int = 6

scala> sumSizes("Zaphod","Beeblebrox")
res1: Int = 16

scala> sumSizes("Zaphod","Beeblebrox","Ford","Prefect")
res3: Int = 27

scala> sumSizes()
res4: Int = 0
\end{REPLnonum}
Repeterade parametrar \Eng{repeated parameters} blir en sekvens av typen \code{WrappedArray} som kapslar in en \code{Array} så den blir som en ''vanlig'' samling.
\end{Slide}


\begin{Slide}{Sekvenssamling som argument till repeterade parametrar}
\begin{Code}[basicstyle=\fontsize{10}{12}\selectfont\ttfamily]
def sumSizes(xs: String*): Int = xs.map(_.size).sum

val veg = Vector("gurka","tomat")
\end{Code}
Om du \emph{redan har} en sekvenssamling så kan du applicera den på en funktion
som har repeterade parametrar med hjälp av denna typannotering:\\
{\vspace{0.5em}\Large\code{: _* }} \\
\vspace{1em}Den ska skrivas direkt \Alert{efter} den sekvenssamling, som du vill att kompilatorn ska tolka som en sekvens av argument, så här:
\begin{REPLnonum}
scala> sumSizes(veg: _*)
res5: Int = 10
\end{REPLnonum}

\end{Slide}
