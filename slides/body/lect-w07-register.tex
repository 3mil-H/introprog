%!TEX encoding = UTF-8 Unicode
%!TEX root = ../lect-w07.tex

%%%

\Subsection{Registrering}

\begin{Slide}{Registrering}
\begin{itemize}
\item \Emph{Registrering} innefattar algoritmer för att kategorisera eller räkna antalet förekomster av element med vissa specifika egenskaper.

\item Exempel:
\\\vspace{0.5em}Utfallsfrekvens vid kast med en tärning 1000 gånger:

\vspace{1em}\begin{tabular}{r c l}
utfall & & antal \\ \hline
1 & $\rightarrow$ & 178 \\
2 & $\rightarrow$ & 187 \\
3 & $\rightarrow$ & 167 \\
4 & $\rightarrow$ & 148 \\
5 & $\rightarrow$ & 155 \\
6 & $\rightarrow$ & 165 \\
\end{tabular}
\end{itemize}
\end{Slide}

\begin{Slide}{Registrering av tärningskast i \code{Array}}
Vi låter plats 0 representera antalet ettor, plats 1 representerar antalet tvåor etc. \Alert{Övning}: implementera \code{???}
\begin{REPLnonum}
scala> def rollDice(): Int = ???  //dra slumptal 1-6

scala> val reg = ??? //skapa heltalsarray med 6 platser
reg: Array[Int] = Array(0, 0, 0, 0, 0, 0)

scala> for k <- 1 to 1000 do 
         ??? //kasta tärning, räkna ut rätt index 
         ??? //registrera

scala> for i <- 1 to 6 do println(s"$i: ${reg(i - 1)}")
1: 178
2: 187
3: 167
4: 148
5: 155
6: 165
\end{REPLnonum}
\end{Slide}


\begin{Slide}{Registrering av tärningskast i \code{Array}}
\Emph{Lösning}:\\~
\begin{REPLnonum}
scala> def rollDice() = scala.util.Random.nextInt(6) + 1

scala> val reg = new Array[Int](6)
reg: Array[Int] = Array(0, 0, 0, 0, 0, 0)

scala> for k <- 1 to 1000 do
         val i = rollDice() - 1 
         reg(i) = reg(i) + 1    // eller: reg(i) += 1

scala> for i <- 1 to 6 do println(s"$i: ${reg(i - 1)}")
1: 178
2: 187
3: 167
4: 148
5: 155
6: 165
\end{REPLnonum}
\end{Slide}

% \begin{Slide}{Registrering av tärningskast i \code{Map}, imperativ lösning}
% Vi registrerar antalet i en Map[Int, Int] där nyckeln är antalet tärningsögon och värdet är frekvensen.
% \begin{REPLnonum}
% scala> val rnd = new java.util.Random(42L)
% rnd: java.util.Random = java.util.Random@6d946eee
%
% scala> var reg = (1 to 6).map(i => i -> 0).toMap
% reg: scala.collection.immutable.Map[Int,Int] =
%   Map(5 -> 0, 1 -> 0, 6 -> 0, 2 -> 0, 3 -> 0, 4 -> 0)
%
% scala> for i <- 1 to 1000 do 
%          val t = rnd.nextInt(6) + 1
%          reg = reg + ((t, reg(t) + 1))
%
% scala> reg
% res0:scala.collection.immutable.Map[Int,Int]= Map(5 -> 155,
% 1 -> 178, 6 -> 165, 2 -> 187, 3 -> 167, 4 -> 148)
% \end{REPLnonum}
% \end{Slide}
%
% \begin{Slide}{Registrering av tärningskast i \code{collection.mutable.Map}, imperativ lösning}
% Om vi är bekymrade över prestanda:
% \begin{REPL}
% scala> val rnd = new java.util.Random(42L)
% rnd: java.util.Random = java.util.Random@6d946eee
%
% scala> val initPairs = (1 to 6).map(i => i -> 0)
% initPairs: scala.collection.immutable.IndexedSeq[(Int, Int)] =
% Vector((1,0), (2,0), (3,0), (4,0), (5,0), (6,0))
%
% scala> var reg = scala.collection.mutable.Map(initPairs: _*)
%
% scala> for i <- 1 to 1000 do {
%          val t = rnd.nextInt(6) + 1
%          reg(t) = reg(t) + 1
%
% scala> reg
% res0: scala.collection.mutable.Map[Int,Int] =
% Map(2 -> 187, 5 -> 155, 4 -> 148, 1 -> 178, 3 -> 167, 6 -> 165)
%
% \end{REPL}
% \end{Slide}
%
% \begin{Slide}{Registrering av tärningskast i \code{Map}, funktionell lösning}
% Oföränderlighet: Skapa nya samlingar utan att ändra något.
% \begin{REPLnonum}
% scala> val rnd = new java.util.Random(42L)
% rnd: java.util.Random = java.util.Random@6d946eee
%
% scala> val dice = (1 to 1000).map(i => rnd.nextInt(6) + 1)
%
% scala> dice.groupBy(i => i).map(p => p._1 -> p._2.size)
% res0:scala.collection.immutable.Map[Int,Int]= Map(5 -> 155,
% 1 -> 178, 6 -> 165, 2 -> 187, 3 -> 167, 4 -> 148)
% \end{REPLnonum}
% Övn. för den nyfikne: mät prestanda för de olika lösningarna.
% \end{Slide}

% \ifkompendium\else
% \begin{SlideExtra}{Syresättning av hjärnan med registreringslek}\SlideFontTiny

% \vspace{-0.65em}\scalainputlisting[numbers=left,numberstyle=,basicstyle=\SlideFontSize{6}{8}\ttfamily\selectfont]{../compendium/examples/sequences/RegisterToggleWinner.scala}
% Övning: Vad gör koden?
% % \begin{Code}
% % object StandUpSleepyBrain {
% %   def randomChar: Char = (scala.util.Random.nextInt('Z' - 'A' + 1) + 'A').toChar
% %   val reg = new Array[Int]('Z' - 'A' + 1)
% %   def tab: Seq[(Char, Int)] = reg.indices.map(i => ((i + 'A').toChar, reg(i)))
% %   def winner = "\n ** Toggle Winner: **" + (reg.indexOf(reg.max)+'A').toChar
% %   def report = "Registreringsrapport:\n" + tab.mkString("\n") + winner
% %
% %   def toggle(n: Int = 10): Unit = {
% %     println(s"Toggla (sitt upp/sitt ner) om ditt förnamn börjar på: ")
% %     var ready = false
% %     while(!ready){
% %       val chars = Vector.fill(n)(randomChar).distinct.sorted
% %       println("\n" + chars.mkString(" "))
% %       chars.foreach(ch => reg(ch - 'A') += 1)
% %       ready = scala.io.StdIn.readLine("\n").length > 0
% %     }
% %     println(report)
% %   }
% % }
% % \end{Code}
% \vspace{-0.5em}Kör koden och lyssna på: \href{https://youtu.be/ZVrgj3A0_BY}{https://youtu.be/ZVrgj3A0\_BY}
% \end{SlideExtra}
% \fi