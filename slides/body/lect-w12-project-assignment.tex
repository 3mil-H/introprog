%!TEX encoding = UTF-8 Unicode
%!TEX root = ../lect-w12.tex

%%%



\Subsection{Om projektuppgiften}

\begin{Slide}{Om din avslutande projektuppgift}\SlideFontSmall
\Emph{Läs noga kompendium Del 1, kapitel -1 avsnitt ''Projektuppgift''!} \\
Några citat:
\begin{itemize}
\item Mål: skapa ett stort program med många samverkande klasser/moduler.
\item Du väljer själv projektuppgift. Börja redan idag att planera ditt arbete!
\item Undvik \Alert{för simpel} uppgift och att ta dig \Alert{vatten över huvudet}!

\item I kompendium Del 2, kapitel 13, finns flera förslag att välja bland, men du kan \Emph{med fördel} definiera ett eget projekt som passar just dig.

\item Övning \code{extra} i kapitel 14, där du gör en enkel web-server och experimenterar med trådar, kan vara lämplig som grund för projektuppgift för de som vill fördjupa sig \Alert{bortom} kursens innehåll.

\item Du ska skapa automatiskt genererad dokumentation med \code{scaladoc} enligt beskrivning i appendix E.

\item Redovisning sker i datorsal på labbtid fredagen den 8/12:
\begin{itemize}\SlideFontTiny
  \item Förklara hur din kod fungerar.
  \item Beskriv framväxten av ditt program.
  \item Gå igenom den genererade dokumentationen av din kod.
\end{itemize}
\end{itemize}

\end{Slide}


\begin{Slide}{Avslutning och uppsamling}
\begin{itemize}

\item Föreläsningarna sista läsveckorna baseras på det som återstår av tentaträning, extra fördjupning och ev. något ur grumligt-lådan/nyfiken-på-lådan.
\item Sista föreläsningen 13/12 innehåller \Alert{hemlig} \Emph{överraskning} speciellt för er! Jag hoppas alla kommer på den!

\item Resurstider 13/12 + 14/12 är till för \Emph{uppsamling}, alltså redovisning av återstående delar.
\item Kolla noga i denna daterade ögonblicksbild ur vårt system SAM där du ser vad du har \Emph{klarat} och vad du ev. \Alert{har kvar}: \url{http://cs.lth.se/pgk/sam}
\end{itemize}
\end{Slide}


\Subsection{Grumligt-lådan \& Nyfiken-på-lådan}
\begin{Slide}{Grumligt-lådan och Nyfiken-på-lådan}
\begin{itemize}
\item Skriv lapp i \Alert{GRUMLIGT}-lådan om du har något \Alert{grundläggande begrepp} i kursen som du fortfarande tycker är \Alert{svårt att begripa}.

\item[]

\item Skriv lapp i \Emph{NYFIKEN-PÅ}-lådan om du vill veta mer om något ämne inom programmering och som går \Emph{bortom grunderna}.
\end{itemize}
\end{Slide}
