%!TEX encoding = UTF-8 Unicode
%!TEX root = ../lect-w12.tex

%%%


\ifkompendium\else

\Subsection{Om projektuppgiften}

\begin{SlideExtra}{Om din avslutande projektuppgift}\SlideFontSmall
\Emph{Läs noga kompendium Del 1, kapitel -1 avsnitt ''Projektuppgift''!} \\
Några citat:
\begin{itemize}
\item Mål: skapa ett stort program med många samverkande klasser/moduler.
\item Du väljer själv projektuppgift. Börja redan idag att planera ditt arbete!
%\item Undvik \Alert{för simpel} uppgift och att ta dig \Alert{vatten över huvudet}!

\item I kompendium Del 2, kapitel 13, finns flera förslag att välja bland, men du kan också definiera ett eget projekt som passar just dig.

%\item Övning \code{extra} i kapitel 14, där du gör en enkel web-server och experimenterar med trådar, kan vara lämplig som grund för projektuppgift för de som vill fördjupa sig \Alert{bortom} kursens innehåll.

\item Du ska skapa automatiskt genererad dokumentation med \code{scaladoc} enligt beskrivning i appendix E.

\item Redovisning sker i datorsal på labbtid fredagen den 11/12:
\begin{itemize}\SlideFontTiny
  \item Förklara hur din kod fungerar.
  \item Beskriv framväxten av ditt program.
  \item Gå igenom den genererade dokumentationen av din kod.
\end{itemize}
\end{itemize}

\end{SlideExtra}

\begin{SlideExtra}{Projektuppgifter}\SlideFontTiny

\begin{itemize}\SlideFontTiny
\item \code{bank}
\begin{itemize}\SlideFontTiny
\item känd domän: skapa bank med transaktionshistorik 
\item oföränderlig data tillsammans med tillståndsförändring
\end{itemize}

\item \code{tabular}
\begin{itemize}\SlideFontTiny
\item behandling av data i tabellform 
\item matris, oföränderlig data
\end{itemize}

\item \code{music}
\begin{itemize}\SlideFontTiny
\item skapa ett enkelt kompositionsverktyg som spelar musik
\item sätta sig in i en domänmodell
\end{itemize}

\item \code{photo} 
\begin{itemize}\SlideFontTiny
\item en enkel variant av photoshop, inblick i enkel matrismatematik
\item samverkan Scala + Java
\end{itemize}


\item egendefinierat projekt 
\begin{itemize}\SlideFontTiny
\item Lagom svår: ej \Alert{för simpel} uppgift men ta dig inte \Alert{vatten över huvudet}!
\item diskutera med handledare och dokumentera
\end{itemize}


\end{itemize}

\end{SlideExtra}

\begin{SlideExtra}{Kom på föreläsning om systemutveckling!}
\begin{itemize}
  \item Måndagen den \Alert{6/12 kl 15-17 i E:A} håller Björn Regnell en föreläsning i C:arnas projektkurs om digitalisering: 
\begin{itemize}
  \item Kravhantering för mjukvara: hur bestämma vad vi ska koda?
  \item Hur bäst dra nytta av öppen källkod?
\end{itemize}
  \item D-are, W-are, fristående är också \Emph{hjärtligt välkomna!!!} \\ - det är ju gott om plats i E:A :)
\end{itemize}
  
\end{SlideExtra}

\begin{SlideExtra}{Avslutning och uppsamling}

\begin{itemize}\SlideFontSmall

\item Föreläsningarna nästa vecka består av repetition och tentaträning.

\item Föreläsningarna anpassas efter era önskemål. % (omröstning: grumligt, nyfiken).

\item Inga föreläsningar sista läsveckan. Sista föreläsningen är alltså onsdagen den 8/12 i E:A.

\item Resurstider och labbtider i \Alert{sista läsveckan} är till för \Emph{muntligt prov, projektredovisning, redovisning av restlabbar}.

\item Gör en \Alert{detaljerad} plan dag-för-dag för din kursavslutning.

\item För att få göra den valfria tentan krävs att \Alert{alla} obligatoriska moment (kontrollskrivning, labbar, projekt, muntligt prov) är \Emph{godkända}. 

\item Godkänd grundkurs är krav för att få påbörja efterföljande \code{pfk}.

\item \Emph{Kolla din status i Canvas} på sida med inklippt ögonblicksbild ur vårt system SAM där du ser vad du har \Emph{klarat} och vad du ev. har kvar.

\end{itemize}

\end{SlideExtra}

\begin{SlideExtra}{Gör den valfria tentan!}
\begin{itemize}
  \item \Emph{Alla} som kan \& vill uppmuntras göra den valfria tentan!
  \item Den valfria tentan kan ge högre betyg om du klarar poäng-gränserna för 4:a el. 5:a.
  \item Tentan liknar i formen de extentor som finns på kurshemsidan under examination.
  \item Tentan är på plats och skrivs med papper och penna. Enda hjälpmedel: snabbreferensen.
  \item Snabbreferens kan köpas hos \url{birger.swahn@cs.lth.se}
  \item Tentan ges en gång om året i januari. Det är tillåtet att ''plussa'' nästa år om du inte är nöjd med ditt betyg.
\end{itemize}
\end{SlideExtra}



% \Subsection{Grumligt-lådan \& Nyfiken-på-lådan}
% \begin{Slide}{Grumligt-lådan och Nyfiken-på-lådan}
% \begin{itemize}
% \item Skriv lapp i \Alert{GRUMLIGT}-lådan om du har något \Alert{grundläggande begrepp} i kursen som du fortfarande tycker är \Alert{svårt att begripa}.
%
% \item[]
%
% \item Skriv lapp i \Emph{NYFIKEN-PÅ}-lådan om du vill veta mer om något ämne inom programmering och som går \Emph{bortom grunderna}.
% \end{itemize}
% \end{Slide}

\fi