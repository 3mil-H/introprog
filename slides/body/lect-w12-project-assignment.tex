%!TEX encoding = UTF-8 Unicode
%!TEX root = ../lect-w12.tex

%%%


\ifkompendium\else

\Subsection{Om projektuppgiften}

\begin{SlideExtra}{Om din avslutande projektuppgift}\SlideFontSmall
\Emph{Läs noga kompendium Del 1, kapitel -1 avsnitt ''Projektuppgift''!} \\
Några citat:
\begin{itemize}
\item Mål: skapa ett stort program med många samverkande klasser/moduler.
\item Du väljer själv projektuppgift. Börja redan idag att planera ditt arbete!
%\item Undvik \Alert{för simpel} uppgift och att ta dig \Alert{vatten över huvudet}!

\item I kompendium Del 2, kapitel 13, finns flera förslag att välja bland, men du kan också definiera ett eget projekt som passar just dig.

%\item Övning \code{extra} i kapitel 14, där du gör en enkel web-server och experimenterar med trådar, kan vara lämplig som grund för projektuppgift för de som vill fördjupa sig \Alert{bortom} kursens innehåll.

\item Du ska skapa automatiskt genererad dokumentation med \code{scaladoc} enligt beskrivning i appendix E.

\item Redovisning sker i datorsal på labbtid fredagen den 14/12:
\begin{itemize}\SlideFontTiny
  \item Förklara hur din kod fungerar.
  \item Beskriv framväxten av ditt program.
  \item Gå igenom den genererade dokumentationen av din kod.
\end{itemize}
\end{itemize}

\end{SlideExtra}

\begin{SlideExtra}{Projektuppgifter}\SlideFontTiny

\begin{itemize}\SlideFontTiny
\item \code{bank}
\begin{itemize}\SlideFontTiny
\item känd domän: skapa bank med transaktionshistorik 
\item oföränderlig data tillsammans med tillståndsförändring
\end{itemize}

\item \code{tabular}
\begin{itemize}\SlideFontTiny
\item behandling av data i tabellform 
\item matris, oföränderlig data
\end{itemize}

\item \code{music}
\begin{itemize}\SlideFontTiny
\item skapa ett enkelt kompositionsverktyg som spelar musik
\item sätta sig in i en domänmodell
\end{itemize}

\item \code{imageprocessing} 
\begin{itemize}\SlideFontTiny
\item en enkel variant av photoshop, inblick i enkel matrismatematik
\item samverkan Scala + Java
\end{itemize}


\item egendefinierat projekt 
\begin{itemize}\SlideFontTiny
\item Lagom svår: ej \Alert{för simpel} uppgift men ta dig inte \Alert{vatten över huvudet}!
\item diskutera med handledare och dokumentera
\end{itemize}


\end{itemize}

\end{SlideExtra}


\begin{SlideExtra}{Avslutning och uppsamling}

\begin{itemize}

\item Föreläsningarna sista läsveckorna baseras på det som återstår av tentaträning, extra fördjupning och repetition.

\item Föreläsningarna kommer delvis att anpassas efter önskemål från de som närvarar (grumligt, nyfiken på).

\item Resurstider i sista läsveckan är till för \Emph{uppsamling}, alltså redovisning av återstående delar, se schema i TimeEdit: \url{http://cs.lth.se/pgk/schema/timeedit}

\item Gör en \Alert{detaljerad} plan dag-för-dag för din kursavslutning.

\item För att få tentera krävs att \Alert{alla} obligatoriska moment (kontrollskrivning, labbar, projekt) är \Emph{godkända}. För D1 gäller samma krav för att få påbörja \code{pfk}.

\item Kolla noga i denna daterade ögonblicksbild ur vårt system SAM där du ser vad du har
 \Emph{klarat} och vad du ev. \Alert{har kvar}: \url{http://cs.lth.se/pgk/sam}

\end{itemize}

\end{SlideExtra}


% \Subsection{Grumligt-lådan \& Nyfiken-på-lådan}
% \begin{Slide}{Grumligt-lådan och Nyfiken-på-lådan}
% \begin{itemize}
% \item Skriv lapp i \Alert{GRUMLIGT}-lådan om du har något \Alert{grundläggande begrepp} i kursen som du fortfarande tycker är \Alert{svårt att begripa}.
%
% \item[]
%
% \item Skriv lapp i \Emph{NYFIKEN-PÅ}-lådan om du vill veta mer om något ämne inom programmering och som går \Emph{bortom grunderna}.
% \end{itemize}
% \end{Slide}

\fi