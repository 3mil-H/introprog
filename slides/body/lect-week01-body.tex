%!TEX root = ../lect-week01.tex

%%%%%%%%%%%%%%%%%%%%%%%%%%%%%%%%%%%%%%
\Subsection{Om kursen}

%%%

\begin{Slide}{Veckoöversikt}
\noindent\resizebox{0.9\columnwidth}{!}{
%!TEX encoding = UTF-8 Unicode
\begin{tabular}{l|l|l|l}
\textit{W} & \textit{Modul} & \textit{Övn} & \textit{Lab} \\ \hline \hline
W01 & Introduktion            & expressions & kojo            \\
W02 & Kodstrukturer           & programs    & --              \\
W03 & Funktioner, Objekt      & functions   & bugs            \\
W04 & Datastrukturer          & data        & pirates         \\
W05 & Sekvensalgoritmer       & sequences   & cards           \\
W06 & Klasser, Likhet         & classes     & turtlegraphics  \\
W07 & Arv, Gränssnitt         & traits      & turtlerace-team \\
KS  & KONTROLLSKRIVN.         & --          & --              \\
W08 & Mönster, Undantag       & matching    & chords-team     \\
W09 & Matriser, Typparametrar & matrices    & maze            \\
W10 & Sökning, Sortering      & sorting     & surveydata-team \\
W11 & Scala och Java          & scalajava   & lthopoly-team   \\
W12 & Trådar                  & threads     & life            \\
W13 & Design                  & Uppsamling  & Projekt         \\
W14 & Tentaträning            & Extenta     & --              \\
T   & TENTAMEN                & --          & --              \\
\end{tabular}

}
\end{Slide}

\ifkompendium
\noindent Kursen består av ett antal moduler med tillhörande teori, övningar och laborationer. Genom att göra övningarna bearbetar du teorin och förebereder dig inför laborationerna. När du klarat av  laborationen i varje modul är du redo att gå vidare till efterkommande modul.  
\fi

\begin{Slide}{Vad lär du dig?}
\begin{itemize}
\item Grundläggande principer för programmering:\\ Sekvens, Alternativ, Repetition, Abstraktion (SARA)\\$\implies$Inga förkunskaper i programmering krävs!
\item Konstruktion av algoritmer
\item Tänka i abstraktioner
\item Förståelse för flera olika angreppssätt: 
\begin{itemize}
\item \Emph{imperativ programmering}: satser, föränderlighet
\item \Emph{objektorientering}: inkapsling, återanvändning
\item \Emph{funktionsprogrammering}: uttryck, oföränderlighet
\end{itemize}
\item Programspråken \Emph{Scala} och \Emph{Java}
\item Utvecklingsverktyg (editor, kompilator, utvecklingsmiljö)
\item Implementera, testa, felsöka
\end{itemize}
\end{Slide}

\begin{Slide}{Hur lär du dig?}
\begin{itemize}
\item Genom praktiskt \Alert{eget arbete}: \Emph{Lära genom att göra!}
\item Genom studier av kursens teori: \Emph{Skapa förståelse!}
\item Genom samarbete med dina kurskamrater: \Emph{Gå djupare!}
\end{itemize}
\end{Slide}


\ifkompendium
\subsection{hej}
Denna text hamnar bara i kompediet

Hejsan svejsan

\begin{itemize}
\item some item
\end{itemize}


\begin{Code}
hej kod
\end{Code}
\fi


\ifkompendium\else
\begin{Slide}{TESTSLAJD EJ I KOMPENDIUM}
\begin{itemize}
\item \emph{Hej} på dig
\item blablab
\item blabla
\end{itemize}
\begin{Code}
hej kod
\end{Code}
\end{Slide}
\fi


%%%%%%%%%%%%%%%%%%%%%%%%%%%%%%%%%%%%%%
\ifkompendium\else
\Subsection{Meddelande från \href{http://lth.se/code}{Code@LTH}} 
\fi