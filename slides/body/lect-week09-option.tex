%!TEX encoding = UTF-8 Unicode
%!TEX root = ../lect-week09.tex

%%%

\ifkompendium\else


\Subsection{Option}
\begin{Slide}{Option för hantering av ev. saknade värden}
\end{Slide}


\begin{Slide}{En gemensam bastyp för ett värde som kanske saknas}\SlideFontSmall
\vspace{-0.5em}\begin{center}
\newcommand{\TextBox}[1]{\raisebox{0pt}[1em][0.5em]{#1}}
\tikzstyle{umlclass}=[rectangle, draw=black,  thick, anchor=north, text width=3cm, rectangle split, rectangle split parts = 3]
\begin{tikzpicture}[inner sep=0.5em]
\node [umlclass, rectangle split parts = 2, xshift=0cm, text width=3.5cm] (BaseType)  {
            \textit{\textbf{\centerline{\TextBox{\code{Option[A]}}}}}
            \nodepart[]{second}
            \TextBox{\code{def get: A}}\newline
            \TextBox{\code{def isEmpty: Boolean}}

        };
        
\node [umlclass, rectangle split parts = 1]  at (-2.5cm,-3.7cm) (SubType1) {
            \textbf{\centerline{\TextBox{\code{Some[A]}}}}
            % \nodepart[]{second} \TextBox{\code{val x: A}}
        };  
                
\node [umlclass, rectangle split parts = 1] at (2.5cm,-3.7cm) (SubType2)  {
            \textbf{\centerline{\TextBox{\code{None}}}}
        };        
\draw[umlarrow] (SubType1.north) -- ++(0,0.5) -| (BaseType.south);    
\draw[umlarrow] (SubType2.north) -- ++(0,0.5) -| (BaseType.south);            
\end{tikzpicture}
\end{center}
\end{Slide}

\fi











