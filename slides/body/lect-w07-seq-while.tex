%!TEX encoding = UTF-8 Unicode
%!TEX root = ../lect-w07.tex

\Subsection{Skapa lösningar på sekvensproblem från grunden}

\begin{Slide}{Skapa lösningar på sekvensproblem från grunden}
  \begin{itemize}
    \item Normalt använder man färdiga samlingsmetoder
    \item Det finns ofta en färdig metod som gör det man vill
    \item Annars kan man ofta göra det man vill genom att kombinera flera färdiga samlingsmetoder
    \item[] \pause
    \item Vi ska nu i lärosyfte implementera några egna varianter av uppdatering från grunden.  
  \end{itemize}

{\SlideFontSmall  För problem av typen KUTFSSR ingår det i kursen att kunna 1) lösa dessa med färdiga samlingsmetoder, och 2) implementera egna lösningar med hjälp av sekvens, alternativ, repetition, abstraktion (\textbf{SARA}).
}
\end{Slide}

\begin{Slide}{Skapa ny sekvenssamling eller ändra på plats?}
Två olika principer vid sekvensalgoritmkonstruktion:
\begin{itemize}
\item Skapa \Emph{ny sekvens} utan att förändra insekvensen
\item Ändra \Emph{på plats} \Eng{in-place} i \Alert{förändringsbar} sekvens
\end{itemize}
\pause
\vspace{1em}
Välja mellan att skapa ny sekvens eller ändra på plats?
\begin{itemize}
\item Ofta är det \Emph{lättast att skapa ny samling} och kopiera över elementen efter eventuella förändringar medan man loopar.
\item Om man har mycket stora samlingar kan man behöva ändra på plats för att spara tid/minne.
\end{itemize}
\end{Slide}

\begin{Slide}{Algoritm: SEQ-COPY}
\Emph{Pseudokod} för algoritmen SEQ-COPY som kopierar en sekvens, här en Array med heltal:\\
\noindent\hrulefill
\begin{algorithm}[H]
 \SetKwInOut{Input}{Indata}\SetKwInOut{Output}{Resultat}
 \Input{Heltalsarray $xs$}
 \Output{En ny heltalsarray som är en kopia av $xs$. \\ \vspace{1em}}
 $result \leftarrow$ en ny array med plats för $xs.length$ element\\
 $i \leftarrow 0$  \\
 \While{$i < xs.length$}{
  $result(i) \leftarrow xs(i)$\\
  $i \leftarrow i + 1$\\
 }
 \Return $result$
\end{algorithm}
\noindent\hrulefill
\end{Slide}

\begin{Slide}{Implementation av SEQ-COPY med \texttt{while}}
\lstinputlisting[numbers=left]{../compendium/examples/workspace/w05-seqalg/src/seqCopy.scala}
\end{Slide}

% \begin{Slide}{Implementation av SEQ-COPY med \texttt{for}}
% \lstinputlisting[numbers=left]{../compendium/examples/workspace/w05-seqalg/src/seqCopyFor.scala}
% \end{Slide}
%
% \begin{Slide}{Implementation av SEQ-COPY med \texttt{for-yield}}
% \lstinputlisting[numbers=left]{../compendium/examples/workspace/w05-seqalg/src/seqCopyForYield.scala}
% \end{Slide}
