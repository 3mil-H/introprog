%!TEX encoding = UTF-8 Unicode
\documentclass{simpleslides}

%\usepackage{beamerthemesplit}
\usepackage[orientation=landscape,size=custom,width=16,height=9,scale=0.6,debug]{beamerposter} 


\renewcommand{\vecka}{1}
\newcommand{\veckotema}{Introduktion}

%\title[Föreläsning, EDAA45 pgk, Björn Regnell, senast uppdaterad: \today]{Vecka \vecka. \veckotema}
%\subtitle{Programmering, grundkurs}
\title{Programmering, grundkurs}
\author{Björn Regnell}
%\institute{Datavetenskap, LTH, Lunds universitet}
%\date{EDAA45, Lp1-2, HT \CurrentYear}
\date{}

\begin{document}

\frame{\titlepage}
\setnextsection{\vecka}


\frame{\tableofcontents}

\section{Introduktion}
\Subsection{Vad är programmering?}

\begin{Slide}{Definitioner}
  \begin{itemize}
    \item programmera = \pause att skapa program (koda) \pause 
    \item program = instruktioner till en dator (kod, mjukvara) \pause
    \item dator = maskin som kan exekvera (köra) program \pause
    \item programmerare = en som programmerar (utvecklare) \pause
    \item programmeringsspråk = ett språk skapat speciellt för att underlätta för människor att programmera
  \end{itemize}
\end{Slide}



\begin{Slide}{Exempel} %: del av ett program}
\begin{Code}
  while (w.lastEventType != Event.WindowClosed) {
    w.awaitEvent(10)  // wait for next event for max 10 milliseconds

    if (w.lastEventType != Event.Undefined) {
      println(s"lastEventType: ${w.lastEventType} => ${Event.show(w.lastEventType)}")
    }

    w.lastEventType match {
      case Event.KeyPressed    => println("lastKey == " + w.lastKey)
      case Event.KeyReleased   => println("lastKey == " + w.lastKey)
      case Event.MousePressed  => println("lastMousePos == " + w.lastMousePos)
      case Event.MouseReleased => println("lastMousePos == " + w.lastMousePos)
      case Event.WindowClosed  => println("Goodbye!"); System.exit(0)
      case _ =>
    }

    Thread.sleep(100) // wait for 0.1 seconds
  }
\end{Code}
\end{Slide}

\begin{Slide}{Jämförelse} %: olika typer av språk}
\begin{table}
  \centering\Large
  \begin{tabular}{r | l}
    \multicolumn{2}{c}{\textit{språk}}\\\hline 
    \textbf{naturligt} & \textbf{artificiellt} \\ \pause
    svenska & klingon \\
    engelska & esperanto \\
    italienska & Scala \\
    latin & ALGOL  \\
  \end{tabular}
\end{table}
%https://en.wikipedia.org/wiki/List_of_languages_by_total_number_of_speakers
\end{Slide}


\end{document}
