\documentclass[article, a5paper]{memoir}

\let\footruleskip\undefined\usepackage{fancyhdr}% http://ctan.org/pkg/fancyhdr

\usepackage{pgfpages}
\pgfpagesuselayout{resize to}[a4paper]

% Swedish.
\usepackage[utf8]{inputenc}
\usepackage[T1]{fontenc}
%\usepackage[swedish]{babel}
\usepackage{microtype}

%%% FONT PACKAGES
%\usepackage[sc]{mathpazo}
%\usepackage[varg]{txfonts}
%\usepackage{times}
%\usepackage{tgtermes}% clone of times
%\usepackage[sfdefault,condensed]{cabin}
\usepackage{PTSansNarrow}\renewcommand*\familydefault{\sfdefault}
%\usepackage{tgcursor}
\usepackage[scaled=0.85]{beramono} % inconsolata or beramono ???
%\usepackage{fouriernc} % serif: new century schoolbook
%\usepackage{avant}     % sans serif: Avant Garde


% Typeblock size, margins.
\settypeblocksize{190mm}{127mm}{*}
\setlrmargins{10.5mm}{*}{*}
\setulmargins{10.0mm}{*}{*}
\setheadfoot{0.1pt}{0.1pt}
\checkandfixthelayout

\usepackage{xcolor}

\definecolor{mygreen}{rgb}{0,0.4,0}
\definecolor{grammarcolor}{rgb}{0.3,0.6,0.1}
\definecolor{mylinkcolor}{rgb}{0,0.1,0.5}
\definecolor{myemphcolor}{rgb}{0,0.4,0.1}
\definecolor{myalertcolor}{rgb}{0.4,0.1,0}
\definecolor{eclipsepurple}{rgb}{0.5,0,0.25}
\definecolor{eclipseblue}{rgb}{0.16,0,1.0}
\definecolor{eclipsegreen}{rgb}{0,0.5,0}


\newcommand{\OptL}{\textbf{\textcolor{grammarcolor}{[}}}
\newcommand{\OptR}{\textbf{\textcolor{grammarcolor}{]}}}
\newcommand{\RepL}{\textbf{\textcolor{grammarcolor}{(}}}
\newcommand{\RepR}{\textbf{\textcolor{grammarcolor}{)}}}
\newcommand{\Or}{\textbf{\textcolor{grammarcolor}{|}}}

%---------------------------------------------------------------

\newcommand{\LangColor}{red}

\setlength{\parindent}{0pt}
\raggedright
\raggedbottom
\linespread{0.90}\selectfont
\pagestyle{empty}

\newcommand{\mc}[1]{\multicolumn{2}{l}{\hspace{-0.65em}\parbox[t]{102mm}{\small #1}}}

\newcommand{\ind}{\hspace*{1.5em}}

\newcommand{\head}[1]{{\bfseries {\color{\LangColor}{#1}}\par\vspace{1mm}\hrule\vspace{-2mm}}}

\newenvironment{etab}%
{\begin{ctabular}{@{}>{\raggedright\small}p{25mm} @{}>{\raggedright\small}p{45mm} @{}>{\raggedright\arraybackslash\small}p{57mm}}}
{\end{ctabular}}%


\newcommand{\secend}{\\[1mm]}
\newcommand{\subsecend}{\\ \\[-2mm]}
\renewcommand{\arraystretch}{0.9}


%%%%%%%%%%%%%%%%%%%%%%%%%%%%
%%% lingstings specifics:
\usepackage{listings}
\usepackage{upquote} %http://tex.stackexchange.com/questions/145416/how-to-have-straight-single-quotes-in-lstlistings
\lstdefinelanguage{Scala}{
  morekeywords={abstract,case,catch,class,def,%
    do,else,extends,false,final,finally,%
    for,forSome,if,implicit,import,lazy,match,%
    new,null,object,override,package,%
    private,protected,return,sealed,%
    super,this,throw,trait,true,try,%
    type,val,var,while,with,yield},
  otherkeywords={=>,<-,<\%,<:,>:,@},
  sensitive=true,
  morecomment=[l]{//},
  morecomment=[n]{/*}{*/},
  morestring=[b]",
  morestring=[b]',
  morestring=[b]"""
}


\lstset{
    language=Scala,
    tabsize=2,
    basicstyle=\ttfamily\selectfont,
    keywordstyle=\bfseries\textcolor{eclipsepurple},
    commentstyle=\textcolor{mygreen},
    numberstyle={\footnotesize},
    numbers=none,
    %backgroundcolor=\textcolor{gray!15},
    frame=single,
    rulecolor=\color{black!25},
    %title={\footnotesize\lstname},
    breaklines=false,
    breakatwhitespace=false,
    framextopmargin=2pt,
    framexbottommargin=2pt,
    showstringspaces=false,
    columns=fullflexible,keepspaces
}
\lstset{literate=%
{Å}{{\AA}}1
{Ä}{{\"A}}1
{Ö}{{\"O}}1
{Ü}{{\"U}}1
{ß}{{\ss}}1
{ü}{{\"u}}1
{å}{{\aa}}1
{ä}{{\"a}}1
{ö}{{\"o}}1
{æ}{{\ae}}1
{ø}{{\o}}1
{Æ}{{\AE}}1
{Ø}{{\O}}1
{`}{{\`{}}}1
{─}{{\textemdash}}1
{└}{{|}}1
{├}{{|}}1
{│}{{|}}1
}

\newcommand{\code}{\lstinline[basicstyle=\ttfamily]}
\newcommand{\jcode}{\lstinline[basicstyle=\ttfamily,language=Java]}

%*****************************************************************

\usepackage{tikz}
    \usetikzlibrary{calc}

\newcommand{\LangRect}[4]{\tikz[overlay, remember picture,inner sep=7pt,minimum height=0.65cm] \node[fill=#2,text=white,rotate=90] at #4 (name) {\large\normalfont\textbf{#1} ~~{\small \thepage(4)}}; }  

\newcommand{\LangRectOdd}[3]{\LangRect{#1}{#2}{#3}{($(current page.north east)-(0.35,#3)$)}}  
\newcommand{\LangRectEven}[3]{\LangRect{#1}{#2}{#3}{($(current page.north west)-(-0.35,#3)$)}}  
    

\newcommand{\LangMarker}[2]{
%\fancyhead{} % clear all header fields
\fancyfoot{} % clear all footer fields
%\fancyfoot[RO]{\thepage}
%\fancyfoot[LE]{\thepage}
\fancyhead{
\ifodd\thepage\LangRectOdd{#1}{\LangColor}{#2}
\else\LangRectEven{#1}{\LangColor}{#2}
\fi
}
\renewcommand{\headrulewidth}{0pt}
\renewcommand{\footrulewidth}{0pt}
\pagestyle{fancy}
}


\begin{document}

\LangMarker{Scala}{1.5}

\begin{center}
{\Huge{\bfseries\sffamily \color{red}{Scala Quick Reference}}}
%\vspace{2mm}
\end{center}

blablabla

\vspace{\baselineskip}
\head{Control structures}

\vspace{\baselineskip}
Hello \code{if}  \jcode{if}
\clearpage

\vspace{\baselineskip}
\head{Control structures}

\vspace{\baselineskip}
Hello \code{if}  \jcode{if}
\clearpage

\vspace{\baselineskip}
\head{Control structures}

\vspace{\baselineskip}
Hello \code{if}  \jcode{if}
\clearpage


Hello
\clearpage

\setcounter{page}{1}

\renewcommand{\LangColor}{blue}

\LangMarker{Java}{3.5}


\begin{center}
{\Huge{\bfseries\sffamily \color{blue}{Java snabbreferens}}}
%\vspace{2mm}
\end{center}
{\small Tecknet ~\Or~ står för ''eller''. Vanliga parenteser \RepL \RepR ~används för att gruppera alternativ. Med \OptL \OptR ~markeras sådant som inte alltid finns med. Med \jcode{stmt} avses en sats, \jcode{x}, \jcode{i}, \jcode{s},\jcode{ch} är variabler, \jcode{expr} är ett uttryck, \jcode{cond} är ett logiskt uttryck.}

\vspace{\baselineskip}
\head{Satser}
\begin{etab}
Block   & \jcode|{stmt1; stmt2; ...}| &  fungerar ''utifrån'' som \textbf{en} sats \secend

Tilldelningssats & \jcode|x = expr;|                 &  variabeln och uttrycket av kompatibel typ \secend

Förkortade       & \jcode|x += expr; |                &  x = x + expr; även --=, *=, /= \\
                 & \jcode|x++;  |                     &  x = x + 1; även x\hspace{0.5mm}-- -- \secend

if-sats          & \jcode|if (cond) {stmt; ...}|      &  utförs om cond är true \\
                 & \OptL \jcode|else { stmt; ...}| \OptR&  utförs om false \secend

switch-sats      & \jcode|switch (expr) {|         &  expr är ett heltalsuttryck \\ 
                 & \jcode|    case A: stmt1; break;|  &  utförs om expr $=$ A (A konstant) \\
                 & \jcode|    ...|                 & \\
                 & \jcode|    default: stmtN; break;| &  utförs om inget case passar\\
                 & \jcode|}|                   & \secend

for-sats         & \jcode|for (int i = start; i < stop; i++) {| & \\
                 & \jcode|    stmt;|  & satserna utförs för i = start, start+1, \ldots, stop$-1$ \\
                 & \jcode|    ...;|         &  (ingen gång om start >= stop) \\
                 & \jcode|}|                &  i++ kan ersättas med i = i + step \secend

while-sats       & \jcode|while (cond) {|          & \\
                 & \jcode|    stmt; ...|           &  utförs så länge cond är true \\
                 & \jcode|}|                       & \secend

do-while-sats    & \jcode|do {|                     & \\
                 & \jcode|    stmt; ...|         &  utförs minst en gång, \\
                 & \jcode|} while (cond);|          &  så länge cond är true \secend

return-sats      & \jcode|return expr;|        &  returnerar funktionsresultat
\end{etab}

\head{Uttryck}
\begin{etab}
Aritmetiskt uttryck & (x + 2) * i / 3          &  skrivs som i matematiken, för heltal är / \\
                  &                            &  heltalsdivision, \% ''rest'' \secend

Objektuttryck     & \mc{new Classname(\ldots) | ref-var | null | function-call | this | super} \secend

Logiskt uttryck   & \mc{! log-expr | log-expr \&\& log-expr | log-expr || log-expr | function-call | relation | log-var | true | false} \subsecend

Relation          & \mc{ expr ( < | <= | == | >= | > | != ) expr (för objektuttryck bara == och !=, också expr instanceof Classname)} \subsecend

Funktionsanrop    & obj-expr.method(\ldots)    &  anropa ''vanlig metod'' (utför operation) \\
                  & Classname.method(\ldots)   &  anropa statisk metod \secend

Vektor (array)    & new int[size]              &  skapar int-vektor med size element \\
                  & vname[i]                   &  elementet med index i, 0..length$-1$\\
                  & vname.length               &  antalet element \secend

Typkonvertering   & (newtype) expr             &  konverterar expr till typen newtype \\
                  & (int) real-expr            &  -- avkortar genom att stryka decimaler \\
                  & (Square) aShape            &  -- ger ClassCastException om aShape inte \\
                  &                            &  är ett Square-objekt
\end{etab}

\clearpage 
\head{Deklarationer}
\begin{etab}
Allmänt           & \mc{[<protection>] [static] [final] <type> name1, name2, \ldots;} \secend

<type>            & \mc{byte | short | int | long | float | double | boolean | char | Classname} \secend

<protection>      & public | private | protected &  för attribut och metoder i klasser \\
                  &                             &  (paketskydd om inget anges) \secend

Startvärde        & int x = 5;                  &  startvärde bör alltid anges \secend

Konstant          & final int N = 20;           &  konstantnamn med stora bokstäver \secend

Vektor            & <type>[\hspace{0.4mm}] vname = new <type>[10]; &  deklarerar och skapar vektor
\end{etab}


\head{Klasser}
\begin{etab}
Deklaration       & [public][abstract] class Classname & \\
                  & \mc{\ind [extends Classname1] [implements Interface1, Interface2, \ldots] $\{$} \\
                  & \ind <deklaration av attribut> & \\
                  & \ind <deklaration av konstruktorer> & \\
                  & \ind <deklaration av metoder> & \\
                  & $\}$                          & \secend

Attribut          & \mc{Som vanliga deklarationer. Attribut får implicita startvärden, 0, 0.0, false, null.} \secend

Konstruktor       & <prot> Classname(param, \ldots) $\{$ &  Parametrarna är de parametrar som ges vid \\
                  & \ind stmt; \ldots              &  new Classname(\ldots). Satserna ska ge \\ 
                  & $\}$                           &  attributen startvärden \secend

Metod             & <prot> <type> name(param, \ldots) $\{$ &  om typen inte är void måste en return-\\
                  & \ind stmt; \ldots              &  sats exekveras i metoden \\
                  & $\}$                           & \secend
                  
Huvudprogram      & \mc{public static void main(String[\hspace{0.5mm}] args) $\{$ \ldots $\}$} \secend

Abstrakt metod    & \mc{Som vanlig metod, men abstract före typnamnet och $\{\ldots\}$ ersätts med semikolon. Metoden måste implementeras i subklasserna.}
\end{etab}

\head{Standardklasser, java.lang, behöver inte importeras}
\begin{etab}
Object            & \mc{Superklass till alla klasser.} \subsecend
                  & boolean equals(Object other);   &  ger true om objektet är lika med other \\
                  & int hashCode();                 &  ger objektets hashkod \\
                  & String toString();              &  ger en läsbar representation av objektet \secend
                  
Math              & \mc{Statiska konstanter Math.PI och Math.E. Metoderna är statiska (anropas med t~ex Math.round(x)):} \subsecend
                  & long round(double x);           &  avrundning, även float $\rightarrow$ int \\
                  & int abs(int x);                 &  $|x|$, även double, \ldots \\
                  & double hypot(double x, double y); &  $\sqrt{x^2+y^2}$ \\
                  & double sin(double x);           &  $\sin x$, liknande: cos, tan, asin, acos, atan \\
                  & double exp(double x);           &  $e^x$ \\
                  & double pow(double x, double y); &  $x^y$ \\
                  & double log(double x);           &  $\ln x$ \\
                  & double sqrt(double x);          &  $\sqrt{x}$ \\
                  & double toRadians(double deg);   &  $\mathit{deg} \cdot \pi / 180$ \secend

System            & void System.out.print(String s); &  skriv ut strängen s \\
                  & void System.out.println(String s); &  som print men avsluta med ny rad \\
                  & void System.exit(int status);   &  avsluta exekveringen, status != 0 om fel \\
                  & \mc{Parametern till print och println kan vara av godtycklig typ: int, double, \ldots} \secend
\end{etab}
\clearpage
\begin{etab}
Typklasser        & \mc{Till varje datatyp finns en typklass: char $\rightarrow$ Character, int $\rightarrow$ Integer, double $\rightarrow$ Double, \ldots\ Statiska konstanter MIN\_VALUE och MAX\_VALUE ger minsta respektive största värde. Exempel med klassen Integer:} \subsecend
                  & Integer(int value);             &  skapar ett objekt som innehåller value \\
                  & int intValue();                 &  tar reda på värdet \secend

String            & \mc{Teckensträngar där tecknen inte kan ändras. ''asdf'' är ett String-objekt. s1 + s2 för att konkatenera två strängar. StringIndexOutOfBoundsException om någon position är fel.} \subsecend
                  & int length();                   &  antalet tecken \\
                  & char charAt(int i);             &  tecknet på plats i, 0..length()$-1$ \\
                  & boolean equals(String s);       &  jämför innehållet (s1 == s2 fungerar inte) \\
                  & int compareTo(String s);        &  < 0 om mindre, = 0 om lika, > 0 om större \\
                  & int indexOf(char ch);           &  index för ch, $-1$ om inte finns \\
                  & int indexOf(char ch, int from); &  som indexOf men börjar leta på plats from \\
                  & String substring(int first, int last); &  kopia av tecknen first..last$-1$ \\
                  & String[\hspace{0.5mm}] split(String delim); &  ger vektor med ''ord'' (ord är följder av \\
                  &                                 &  tecken åtskilda med tecknen i delim) \secend

                  & \mc{Konvertering mellan standardtyp och String (exempel med int, liknande för andra typer):} \subsecend
                  & String.valueOf(int x);          &  x = 1234 $\rightarrow$ ''1234'' \\ 
                  & Integer.parseInt(String s);     &  s = ''1234'' $\rightarrow$ 1234, NumberFormat- \\
                  &                                 &  Exception om s innehåller felaktiga tecken \secend

StringBuilder     & \mc{Modifierbara teckensträngar. length och charAt som String, plus:} \subsecend
                  & StringBuilder(String s);        &  StringBuilder med samma innehåll som s \\
                  & void setCharAt(int i, char ch); &  ändrar tecknet på plats i till ch \\
                  & StringBuilder append(String s); &  lägger till s, även andra typer: int, char, \ldots \\
                  & StringBuilder insert(int i, String s); &  lägger in s med början på plats i \\
                  & StringBuilder deleteCharAt(int i); &  tar bort tecknet på plats i \\ 
                  & String toString();              &  skapar kopia som String-objekt
\end{etab}

\head{Standardklasser, import java.util.Classname}
\begin{etab}
List              & \mc{List<E> är ett gränssnitt som beskriver listor med objekt av parameterklassen E. Man kan lägga in värden av standardtyperna genom att kapsla in dem, till exempel int i Integer-objekt. Gränssnittet implementeras av klasserna ArrayList<E> och LinkedList<E>, som har samma operationer. Man ska inte använda operationerna som har en position som parameter på en LinkedList (i stället en iterator). IndexOutOfBoundsException om någon position är fel.} \subsecend
                  & \mc{För att operationerna contains, indexOf och remove(Object) ska fungera måste klassen E överskugga funktionen equals(Object). Integer och de andra typklasserna gör det.} \subsecend
ArrayList         & ArrayList<E>();                 &  skapar tom lista \\ 
LinkedList        & LinkedList<E>();                &  skapar tom lista \\ 
                  & int size();                     &  antalet element \\
                  & boolean isEmpty();              &  ger true om listan är tom \\
                  & E get(int i);                   &  tar reda på elementet på plats i \\
                  & int indexOf(Object obj);        &  index för obj, $-1$ om inte finns \\
                  & boolean contains(Object obj);   &  ger true om obj finns i listan \\

                  & void add(E obj);                &  lägger in obj sist, efter existerande element \\
                  & void add(int i, E obj);         &  lägger in obj på plats i (efterföljande \\
                  &                                 &  element flyttas) \\
                  & \ldots\ forts nästa sida        & 
\end{etab}
\secend
\begin{etab}
                  & E set(int i, E obj);            &  ersätter elementet på plats i med obj \\
                  & E remove(int i);                &  tar bort elementet på plats i (efter- \\
                  &                                 &  följande element flyttas) \\
                  & boolean remove(Object obj);     &  tar bort objektet obj, om det finns \\
                  & void clear();                   &  tar bort alla element i listan \secend

Random            & Random();                       &  skapar ''slumpmässig'' slumptalsgenerator \\
                  & Random(long seed);              &  -- med bestämt slumptalsfrö \\
                  & int nextInt(int n);             &  heltal i intervallet [0, n) \\
                  & double nextDouble();            &  double-tal i intervallet [0.0, 1.0) \secend

Scanner           & Scanner(File f);                &  läser från filen f, ofta System.in \\
                  & Scanner(String s);              &  läser från strängen s \\
                  & String next();                  &  läser nästa sträng fram till whitespace \\
                  & boolean hasNext();              &  ger true om det finns mer att läsa \\
                  & int nextInt();                  &  nästa heltal; också nextDouble(), \ldots \\
                  & boolean hasNextInt();           &  också hasNextDouble(), \ldots \\
                  & String nextLine();              &  läser resten av raden
\end{etab}



\head{Filer, import java.io.File/FileNotFoundException/PrintWriter}
\begin{etab}
                  & \mc{Läsa från fil: skapa en Scanner med new Scanner(new File(filename)). Ger File\-NotFoundException om filen inte finns. Sedan läser man ''som vanligt'' från scannern (nextInt och liknande).} \subsecend
                  
                  & \mc{Skriva på fil: skapa en PrintWriter med new PrintWriter(new File(filename)). Ger FileNotFoundException om filen inte kan skapas. Sedan skriver man ''som vanligt'' på PrintWriter-objektet (println och liknande).}\subsecend

                  & \mc{Så här gör man för att fånga FileNotFoundException:\\[1mm]
\ind Scanner scan = null;\\
\ind try $\{$\\
\ind \ind scan = new Scanner(new File("indata.txt"));\\
\ind $\}$ catch (FileNotFoundException e) $\{$\\
\ind \ind \ldots\ ta hand om felet\\
\ind $\}$}
\end{etab}


\head{Specialtecken}
\begin{etab}
                  & \mc{Några tecken måste skrivas på ett speciellt sätt när de används i teckenkonstanter:}\subsecend

& \textbackslash{}n & radmatning\\
& \textbackslash{}t & tab\\
& \textbackslash{}\textbackslash{} & bakåtsnedstreck (\textbackslash{}, eng. {\em backslash})\\
& \textbackslash{}'' & citationstecken ('')\\
& \textbackslash{}' & apostrof (')\\

\end{etab}

\end{document}