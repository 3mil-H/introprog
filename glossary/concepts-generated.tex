\begin{tabular}{l|l}
\textit{Begrepp} & \textit{Beskrivning} \\ \hline \hline
@main & där exekveringen av kompilerat program startar \\
abstrahera & att införa nya begrepp som förenklar kodningen \\
abstrakt klass & kan ha parametrar, kan ej instansieras, kan ej mixas in \\
abstrakt medlem & saknar implementation \\
algoritm & stegvis beskrivning av en lösning på ett problem \\
anonym funktion & funktion utan namn; kallas även lambda \\
anonym klass & klass utan namn, utvidgad med extra implementation \\
Array & en förändringsbar, indexerbar sekvenssamling \\
attribut & variabel som utgör (del av) ett objekts tillstånd \\
bastyp & den mest generella typen i en arvshierarki \\
block & kan ha lokala namn; sista raden ger värdet \\
boolesk & antingen sann eller falsk \\
case-klass & slipper skriva new; automatisk innehållslikhet \\
datastruktur & många olika element i en helhet; elementvis åtkomst \\
de-serialisera & avkoda symbolsekvens och återskapa objekt i minnet \\
defaultargument & gör att argument kan utelämnas \\
dynamisk bindning & körtidstypen avgör vilken metod som körs \\
element & objekt i en datastruktur \\
exekveringsfel & kan inträffa medan programmet kör \\
fabriksmetod & hjälpfunktion för indirekt konstruktion \\
flyttal & decimaltal med begränsad noggrannhet \\
for-sats & bra då antalet repetitioner är bestämt i förväg \\
funktion & vid anrop beräknas ett returvärde \\
funktionshuvud & har parameterlista och eventuellt en returtyp \\
funktionskropp & koden som exekveras vid funktionsanrop \\
förseglad typ & subtypning utanför denna kodfil är förhindrad \\
generisk & har abstrakt typparameter, typen är generell \\
getter & indirekt åtkomst av attributvärde \\
implementation & en specifik realisering av en algoritm \\
import & gör namn tillgängligt utan att hela sökvägen behövs \\
inmixning & tillföra egenskaper med with och en trait \\
innehållslikhet & instanser anses lika om de har samma tillstånd \\
instans & upplaga av ett objekt med eget tillståndsminne \\
klass & en mall för att skapa flera instanser av samma typ \\
klassparameter & binds till argument som ges vid konstruktion \\
kolonn & annat ord för kolumn \\
kolumnvektor & matris av dimension $m\times{}1$ med $m$ vertikala värden \\
kompanjonsobjekt & ser privata medlemmar i klass med samma namn \\
kompilera & att översätta kod till exekverbar form \\
kompilerad & maskinkod sparad och kan köras igen utan kompilering \\
kompileringsfel & kan inträffa innan exekveringen startat \\
konstruktor & skapar instans, allokerar plats för tillståndsminne \\
körtidstyp & kan vara mer specifik än den statiska typen \\
lat initialisering & allokering sker först när namnet refereras \\
linjärsöka & leta i sekvens tills sökkriteriet är uppfyllt \\
linjärsökning & sökalgoritm som letar i sekvens tills element hittas \\
litteral & anger ett specifikt datavärde \\
map & applicerar en funktion på varje element i en samling \\
mappning & nyckel -> värde \\
matris & indexerbar datastruktur i två dimensioner \\
medlem & tillhör ett objekt; nås med punktnotation om synlig \\
metod & funktion som är medlem av ett objekt \\
minneskomplexitet & hur minnesåtgången växer med problemstorleken \\
modul & kodenhet med abstraktioner som kan återanvändas \\
mängd & oordnad samling med unika element \\
namnanrop & fördröjd evaluering av argument \\
namngivna argument & gör att argument kan ges i valfri ordning \\
namnrymd & omgivning där är alla namn är unika \\
namnskuggning & lokalt namn döljer samma namn i omgivande block \\
new & nyckelord vid direkt instansiering av klass \\
null & ett värde som ej refererar till någon instans \\
nyckel & en unik identifierare \\
nyckel-värde-tabell & oordnad samling av mappningar med unika nycklar \\
objekt & samlar variabler och funktioner \\
ordning & definierar hur element av en viss typ ska ordnas \\
paket & modul som skapar namnrymd; maskinkod får egen katalog \\
parameterlista & beskriver namn och typ på parametrar \\
persistens & egenskapen att finnas kvar efter programmets avslut \\
polymorfism & kan ha många former, t.ex. en av flera subtyper \\
predikat & en funktion som ger ett booleskt värde \\
privat & modifierar synligheten av en objektmedlem \\
procedur & vid anrop sker (sido)effekt; returvärdet är tomt \\
programargument & kan överföras via parametern args till main \\
punktnotation & används för att komma åt icke-privata delar \\
radvektor & matris av dimension $1\times{}m$ med $m$ horisontella värden \\
Range & en samling som representerar ett intervall av heltal \\
referenslikhet & instanser anses olika även om tillstånden är lika \\
referenstyp & har supertypen AnyRef, allokeras i heapen via referens \\
registrering & algoritm som räknar element med vissa egenskaper \\
rekursiv funktion & en funktion som anropar sig själv \\
samling & datastruktur med element av samma typ \\
samlingsbibliotek & många färdiga samlingar med olika egenskaper \\
sats & en kodrad som gör något; kan särskiljas med semikolon \\
sekvens(samling) & noll el. flera element av samma typ i viss ordning \\
sekvensalgoritm & lösning på problem som drar nytta av sekvenssamling \\
sekvenssamling & datastruktur med element i en viss ordning \\
serialisera & koda objekt till avkodningsbar sekvens av symboler \\
setter & indirekt tilldelning av attributvärde \\
singelobjekt & modul som kan ha tillstånd; finns i en enda upplaga \\
skript & maskinkod sparas ej utan skapas vid varje körning \\
skyddad medlem & är endast synlig i subtyper \\
slumptalsfrö & ger återupprepningsbar sekvens av pseudoslumptal \\
sortering & algoritm som ordnar element i en viss ordning \\
sträng & en sekvens av tecken \\
subtyp & en typ som är mer specifik \\
supertyp & en typ som är mer generell \\
sökning & algoritm som letar upp element enligt sökkriterium \\
tidskomplexitet & hur exekveringstiden växer med problemstorleken \\
tilldelning & för att ändra en variabels värde \\
trait & är abstrakt, kan mixas in, kan ej ha parametrar \\
typ & beskriver vad data kan användas till \\
typalias & alternativt namn på typ som ofta ökar läsbarheten \\
typargument & konkret typ, binds till typparameter vid kompilering \\
typhärledning & kompilatorn beräknar typ ur sammanhanget \\
uniform access & ändring mellan def och val påverkar ej användning \\
uttryck & kombinerar värden och funktioner till ett nytt värde \\
Vector & en oföränderlig, indexerbar sekvenssamling \\
värdeanrop & argumentet evalueras innan anrop \\
värdetyp & har supertypen AnyVal, lagras direkt på stacken \\
while-sats & bra då antalet repetitioner ej är bestämt i förväg \\
yield & används i for-uttryck för att skapa ny samling \\
äkta funktion & ger alltid samma resultat om samma argument \\
överlagring & metoder med samma namn men olika parametertyper \\
överskuggad medlem & medlem i subtyp ersätter medlem i supertyp \\
\end{tabular}
